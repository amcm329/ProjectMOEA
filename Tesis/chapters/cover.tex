%Autor: Aarón Martín Castillo Medina.
%Asesora: Dra. Katya Rodríguez Vázquez.
%Contacto: katya.rodriguez@iimas.unam.mx; amcm329@hotmail.com

%Este archivo permite elaborar la portada y por ende el título de 
%la obra; además se incluyen elementos adicionales como información 
%básica del autor,nombre completo de la asesora, incluso se proyectan 
%imágenes con los escudos tanto de la U.N.A.M como de la Facultad de 
%Ciencias.       


%Se indica que el documento es de tipo reporte bajo el paquete standalone.
\documentclass[class=report, crop=false]{standalone}

%Se cargan todos los paquetes que residen en el archivo packages_used_standard.sty
\usepackage{packages_used_standard}

%Comienza el documento.
\begin{document}

%Es menester colocar estas directivas alusivas al título y a la 
%fecha para que éstas se puedan colocar en otras secciones de la 
%portada y no cause errores en la distribución de elementos.
\title{}
\date{}

%Se coloca esta directiva para eliminar encabezado y pie de 
%página.
\thispagestyle{empty}

%Se hace un recorrido horizontal a la izquierda para centrar 
%la portada.
\hskip+0.2cm
%La portada se divide en dos secciones, a continuación se trata
%la primera sección que consiste en los escudos tanto de la U.N.A.M.
%y la Facultad de Ciencias, además de las franjas horizontales.
\begin{minipage}[c][10cm][s]{3cm}
      %Se coloca este espacio para ajustar esta sección adecuadamente
      %de manera vertical para que no haya mucho espacio debajo.
      \vspace{-1.2cm}
      %Todo el contenido de esta sección se pasa a la 
      %izquierda.
      \flushleft
      
      %Se coloca y centra el escudo de la U.N.A.M.
      \center{\includegraphics[width=2.8cm,height=2.9cm]{images/unam.pdf}}
      
      %Se adjunta un espacio entre el escudo y las franjas.
      \vspace{0.7cm}
 
      %Se ponen las 3 franjas verticales.
      \center{
              \vrule width.5pt height15cm\hskip5mm
              \vrule width2pt height15cm\hskip5mm
              \vrule width.5pt height15cm
             } \\

      %Se agrega un espacio entre las franjas y el otro escudo.
      \vspace{0.5cm}

      %Se añade y centra el escudo de la Facultad de Ciencias.
      \center{\includegraphics[width=3.0cm,height=3.1cm]{images/ciencias.pdf}}

\end{minipage}
%Se añade este espacio para separar adecuadamente las dos franjas 
%horizontales con el escudo de la U.N.A.M.
\hspace{0.6cm}
%Termina la primera sección.
%No se debe quitar el espacio entre las secciones porque eso 
%repercute en la portada.
%La segunda sección contiene toda la información adicional de la portada.
\begin{minipage}[c][9.7cm][s]{9.9cm}
      %Se coloca este espacio para ajustar esta sección adecuadamente
      %de manera vertical para que no haya mucho espacio debajo.
      \vspace{-1.2cm}
      %Todo el contenidd de esta sección se mueve a la derecha.
      \flushright   
   
      %Se centran los nombres de la Universidad, las franjas horizontales
      %y la Facultad.
      \center{              
              %Se hace una reducción de espacio vertical para que el nombre
              %de la Universidad esté en el mismo renglón que el inicio del
              %escudo de la U.N.A.M.
              %\vspace{-0.8cm}

              %Se crea y centra el nombre de la Universidad.
              \center{
                      {\fontsize{16pt}{16pt}\selectfont U}{\fontsize{13pt}{13pt}\selectfont NIVERSIDAD} {\fontsize{16pt}{16pt}\selectfont N}{\fontsize{13pt}{13pt}\selectfont ACIONAL} 
                      {\fontsize{16pt}{16pt}\selectfont A}{\fontsize{13pt}{13pt}\selectfont UTÓNOMA} \\[6pt]
                      {\fontsize{13pt}{13pt}\selectfont DE} 
                      {\fontsize{16pt}{16pt}\selectfont M}{\fontsize{13pt}{13pt}\selectfont ÉXICO} 
                     } \\

              %Se crean las dos franjas horizontales.   
              \vspace{0.3cm}
              \hrule height2pt\hspace{15pt}
              \vspace{0.5cm}
              \hrule height.5pt\hspace{15pt}	
              \vspace{0.8cm}
          
              %Se inicia con el nombre de la Facultad.
              {\fontsize{16pt}{16pt}\selectfont F}{\fontsize{13pt}{13pt}\selectfont ACULTAD DE } {\fontsize{16pt}{16pt}\selectfont C}{\fontsize{13pt}{13pt}\selectfont IENCIAS}\\[2.0cm]
   
              %Se agrega el nombre de la tesis.
              %\large{\scshape \rmfamily Algoritmos Evolutivos Multiobjetivo y Evaluadores de Desempeño a Través de un Producto de Software  }\\[1.7cm]
              \uppercase{\large{A}\normalsize{lgoritmos} \large{E}\normalsize{volutivos} \large{M}\normalsize{ultiobjetivo y} \large{E}\normalsize{valuadores de} \large{D}\normalsize{esempeño a} \large{T}\normalsize{ravés de un} \large{P}\normalsize{roducto de} \large{S}\normalsize{oftware}}\\[1.8cm]

              %Se escribe la palabra "TESIS" separada por espacios de 1 cm.
              \Huge{T \hspace{1cm} E \hspace{1cm} S \hspace{1cm} I \hspace{1cm} S  }\\[1.6cm]
 
              %Se pone la frase "QUE PARA OBTENER EL TÍTULO DE:".
              \normalsize{QUE PARA OBTENER EL TÍTULO DE:}\\[1.5cm]

              %Se añade la frase "Licenciado en Ciencias de la Computación".
              \normalsize{\uppercase{Licenciado en Ciencias de la Computación}  }\\[1.1cm]

              %Se adjunta la palabra "PRESENTA:" separada por espacios de 2 ex (cada ex equivale a 0.6cm).
              \normalsize{P \hspace{2ex} R \hspace{2ex} E \hspace{2ex} S \hspace{2ex} E \hspace{2ex} N \hspace{2ex} T \hspace{2ex} A \hspace{2ex} :}\\[1.5cm]

              %Se establece el nombre del autor de la tesis (en mayúsculas).
              \normalsize{\uppercase{Aarón Martín Castillo Medina}  }\\[1.4cm]

              %Se agrega la palabra "TUTORA".
              \normalsize{TUTORA  }\\[0.27cm]

              %Se coloca el nombre de la asesora de la tesis.
              \normalsize{\uppercase{Dra. Katya Rodríguez Vázquez}  }\\[1.4cm]

              %Se adjunta el año de la publicación.
              \normalsize{\the\year}

      }%Termina el primer center

%Termina la segunda sección.
\end{minipage}

%Termina el documento.
\end{document}

%Autor: Aarón Martín Castillo Medina.
%Asesora: Dra. Katya Rodríguez Vázquez
%Contacto: katya.rodriguez@iimas.unam.mx; amcm329@hotmail.com

%Este archivo contiene información relacionada con la capa Vista
%(ó View), la cual representa tanto física como lógicamente a uno de 
%los componentes que conforman el producto de software y por tanto al 
%Manual Técnico. 


%Se indica que el documento es de tipo reporte bajo el paquete standalone.
\documentclass[class=report, crop=false]{standalone}

%Se cargan los paquetes relacionados con los subapéndices (elementos que
%conforman el Apéndice en su totalidad).
\usepackage{packages_used_section}

%Comienza el documento.
\begin{document}

\section{View (sección)}
%Se coloca el vínculo interno procedente de esta misma sección (a_3).
\label{sec:a_3}
La capa View \textbf{(ó Vista)} contiene todos los elementos que 
serán alusivos a la interfaz gráfica. De acuerdo al modelo MVC 
\textbf{(Model-View-Controller)}, opera exclusivamente con la capa 
Controller \textbf{(ó Controlador)}.\medskip\break
A continuación se muestran los elementos que conforman esta sección.

%******* Empieza clase *******
\subsection{MainWindow (clase)}
%Se coloca el vínculo interno procedente de esta misma sección (a_3_1).
\label{sec:a_3_1}

%******* Empieza descripción *******
\begin{fulllineitems}

\begin{DUlineblock}{0em}
\item[] Mezcla todas las estructuras gráficas que forman parte de la 
sección View \textbf{(ó vista)}.\break
Se trata de una Ventana que contendrá todas las opciones necesarias 
para que el usuario pueda ejecutar a voluntad M.O.E.A.'s \textbf{(Multi Objective Evolutionary Algorithm
ó Algoritmo Evolutivo Multi Objetivo)}\medskip\break
El flujo que se suele seguir es el siguiente:

\begin{itemize}
\item El usuario ingresa las características que desea que contenga el M.O.E.A. que será ejecutado.
\item Posteriormente el Controller \textbf{(ó Controlador, véase Controller/Controller.py)} verifica la consistencia de los datos anteriores para que no haya conflicto en el lado del Model \textbf{(ó Modelo)}.
\item Si no existe problema alguno se prosigue con el proceso, en otro caso se arroja un mensaje de error.
\item Siguiendo con el flujo normal se ejecutará una instancia del M.O.E.A. solicitado en la capa de Model \textbf{(ó Modelo)}, la cual tendrá una ventana asociada en View \textbf{(ó Vista)} que indicará el progreso del primero.
\item Cuando una instancia termine de ejecutarse, la ventana del progreso desaparece y en su lugar se muestra otra conteniendo los resultados del M.O.E.A. \textbf{(véase View/Additional/ResultsGrapher/ResultsGrapherToplevel.py)}.
\end{itemize}

Es importante mencionar que esta clase y el proyecto en general están 
diseñados para que se puedan crear varias instancias simultáneamente, 
con ello se espera aprovechar al máximo los recursos computacionales 
en los que el proyecto fuera a ejecutarse.
\end{DUlineblock}

\begin{quote}\begin{description}
\item[{Returns}] \leavevmode
Tkinter.Frame
\item[{Return type}] \leavevmode
Instance
\end{description}\end{quote}

%******* Termina descripción *******

%******* Empieza función *******
\begin{fulllineitems}

\pysiglinewithargsret{\sphinxbfcode{change\_frame}}{\emph{current\_frame\_name}}{}~

\begin{notice}{note}{Note:}
Este método es privado.
\end{notice}

Hace el cambio en la Ventana Principal ocultando un 
Frame y mostrando otro.

\begin{quote}\begin{description}
\item[{Parameters}] \leavevmode\begin{itemize}
\item \textbf{\texttt{current\_frame\_name}} (\emph{\texttt{Tkinter.Frame}}) -- El Frame que se va a mostrar en la Ventana Principal.
\end{itemize}
\end{description}\end{quote}

\end{fulllineitems}
%******* Termina función *******

%******* Empieza función *******
\begin{fulllineitems}

\pysiglinewithargsret{\sphinxbfcode{check\_queues}}{}{}~

\begin{notice}{note}{Note:}
Este método es privado.
\end{notice}

Una vez iniciado un proceso que ejecuta un M.O.E.A., este 
método revisa periódicamente las colas \textbf{(Queues)} 
sobre las cuales los procesos escribirán todo tipo de 
información pertinente.

\end{fulllineitems}
%******* Termina función *******

%******* Empieza función *******
\begin{fulllineitems}

\pysiglinewithargsret{\sphinxbfcode{get\_information}}{}{}~

\begin{notice}{note}{Note:}
Este método es privado.
\end{notice}

Método que obtiene los datos ingresados por el usuario
de cada uno de los Frames asociados a esta Ventana Principal.

\begin{quote}\begin{description}
\item[{Returns}] \leavevmode
Un diccionario con toda las preferencias del usuario recolectadas para cada uno de los Frames disponibles.
\item[{Return type}] \leavevmode
Dictionary
\end{description}\end{quote}

\end{fulllineitems}
%******* Termina función *******

%******* Empieza función *******
\begin{fulllineitems}

\pysiglinewithargsret{\sphinxbfcode{init\_procedure}}{\emph{event}}{}~

\begin{notice}{note}{Note:}
Este método es privado.
\end{notice}

Inicia el procedimiento para ejecutar un M.O.E.A.\break
Los pasos que se realizan son:

\begin{itemize}
\item Recolecta las preferencias ingresadas por el usuario en los Frames que conforman la Ventana Principal.
\item Se sanitizan dichos datos con ayuda del Controller.
\item En caso de no haber problemas con la sanitización, se ejecuta el proceso alojándolo en un hilo \textbf{(Thread)} para que permita seguir teniendo acceso a la Ventana Principal; por el contrario si hubo alguna falla regresa un mensaje de error.
\end{itemize}

Gracias a este método el proyecto entero tiene la característica 
de ser Multi-Hilo \textbf{(ó Multi-Threading)}, es decir, se pueden 
ejecutar varios procedimientos de manera independiente.

\begin{quote}\begin{description}
\item[{Parameters}] \leavevmode\begin{itemize}
\item\textbf{\texttt{event}} (\emph{\texttt{String}}) -- El evento del elemento gráfico que activa esta función.
\end{itemize}
\end{description}\end{quote}

\end{fulllineitems}
%******* Termina función *******

%******* Empieza función *******
\begin{fulllineitems}

\pysiglinewithargsret{\sphinxbfcode{initialize\_frames}}{}{}~

\begin{notice}{note}{Note:}
Este método es privado.
\end{notice}

Método que inicializa los Frames que se 
colocarán en la Ventana Principal como opciones.

\end{fulllineitems}
%******* Termina función *******

%******* Empieza función *******
\begin{fulllineitems}

\pysiglinewithargsret{\sphinxbfcode{load\_images}}{}{}~

\begin{notice}{note}{Note:}
Este método es privado.
\end{notice}

Carga las imágenes que se encuentran en el directorio 
View/Images para que puedan ser usadas por los Frames.\break
Es importante recalcar que el método sólo carga imágenee .gif 
ya que son la extensión más estable para que se muestren 
las imágenes en la interfaz gráfica.

\begin{quote}\begin{description}
\item[{Returns}] \leavevmode
Un diccionario con todas las imágenes cargadas.
\item[{Return type}] \leavevmode
Dictionary
\end{description}\end{quote}

\end{fulllineitems}
%******* Termina función *******

%******* Empieza función *******
\begin{fulllineitems}

\pysiglinewithargsret{\sphinxbfcode{obtain\_results}}{\emph{execution\_task\_count}, \emph{generations\_queue}, \emph{gathered\_information}, \emph{sanitized\_information}}{}~

\begin{notice}{note}{Note:}
Este método es privado.
\end{notice}

Ejecuta un M.O.E.A..\break
Esta es la función que se coloca en un hilo para ser llevada a 
cabo de manera independiente con la finalidad de dejar libre la 
Ventana Principal y de manera secundaria ejecutar varios 
procedimientos simultáneamente.

\begin{quote}\begin{description}
\item[{Parameters}] \leavevmode\begin{itemize}
\item \textbf{\texttt{execution\_task\_count}} (\emph{\texttt{Integer}}) -- El número de proceso actual.
\item \textbf{\texttt{generations\_queue}} (\emph{\texttt{Instance}}) -- Una referencia a una cola \textbf{(Queue)} donde los procesos escribirán su avance
en cuanto a las generaciones transcurridas.
\item \textbf{\texttt{gathered\_information}} (\emph{\texttt{Dictionary}}) -- La información que el usuario ingresó al momento de iniciar el proceso actual.
\item \textbf{\texttt{sanitized\_information}} (\emph{\texttt{Dictionary}}) -- La información anterior sanitizada.
\end{itemize}
\end{description}\end{quote}

\end{fulllineitems}
%******* Termina función *******

%******* Empieza función *******
\begin{fulllineitems}

\pysiglinewithargsret{\sphinxbfcode{restore\_settings}}{\emph{event}}{}~

\begin{notice}{note}{Note:}
Este método es privado.
\end{notice}

Limpia y deja por defecto los valores estándar del Frame
mostrado actualmente en la Ventana Principal.\break
El método no aplica para regresar a M.O.P.'s \textbf{(Multi Objective Problems)}
cargados anteriormente.

\begin{quote}\begin{description}
\item[{Parameters}] \leavevmode\begin{itemize}
\item \textbf{\texttt{event}} (\emph{\texttt{String}}) -- El evento del elemento gráfico que acciona esta función.
\end{itemize}
\end{description}\end{quote}

\end{fulllineitems}
%******* Termina función *******

%******* Empieza función *******
\begin{fulllineitems}

\pysiglinewithargsret{\sphinxbfcode{update\_frame}}{\emph{event}}{}~

\begin{notice}{note}{Note:}
Este método es privado.
\end{notice}

Muestra en la Ventana Principal el Frame actual.

\begin{quote}\begin{description}
\item[{Parameters}] \leavevmode\begin{itemize}
\item \textbf{\texttt{event}} (\emph{\texttt{String}}) -- El evento del elemento gráfico que 
activa la función.
\end{itemize}
\end{description}\end{quote}

\end{fulllineitems}
%******* Termina función *******

%******* Empieza función *******
\begin{fulllineitems}

\pysiglinewithargsret{\sphinxbfcode{load\_mop\_example}}{\emph{elements}}{}
Carga el M.O.P \textbf{(Multi Objective Problem)} seleccionado
a los Frames correspondientes \textbf{(Objective Functions y Decision Variables)}.

\begin{quote}\begin{description}
\item[{Parameters}] \leavevmode\begin{itemize}
\item \textbf{\texttt{elements}} (\emph{\texttt{Array}}) -- Un arreglo que contiene dos elementos, el primero son las funciones objetivo precargadas mientras que el segundo son las variables de decisión también precargadas. Ambas provienen del menú secundario \textbf{(véase View/Additional/}\break\textbf{MenuInternalOption/InternalOptionTab/}\break\textbf{MOPExampleFrame.py, Controller/XML/}\break\textbf{MOPExamples.xml)}.
\end{itemize}
\end{description}\end{quote}

\end{fulllineitems}
%******* Termina función *******

%******* Empieza función *******
\begin{fulllineitems}

\pysiglinewithargsret{\sphinxbfcode{resource\_path}}{\emph{relative\_path}}{}~
\vspace{-0.1cm}
Esta función se utiliza para poder crear ejecutables
apropiadamente.\break
A grandes rasgos el ejecutable se empaqueta en un directorio
llamado \_MEIPASS, entonces aquí se implementa la búsqueda
de dicho archivo devolviendo un path \textbf{(ruta)}.

\begin{quote}\begin{description}
\item[{Returns}] \leavevmode
La ruta del directorio \_MEIPASS.
\item[{Return type}] \leavevmode
String
\end{description}\end{quote}

\end{fulllineitems}
%******* Termina función *******

%******* Empieza función *******
\begin{fulllineitems}

\pysiglinewithargsret{\sphinxbfcode{run}}{}{}
Lanza la Ventana Principal.

\end{fulllineitems}
%******* Termina función *******

\end{fulllineitems}
%******* Termina clase *******

%******* Empieza módulo *******
\subsection{Main (módulo)}
%Se coloca el vínculo interno procedente de esta misma sección (a_3_2).
\label{sec:a_3_2}
Contiene todos los elementos gráficos para que el usuario
pueda configurar los atributos que intervienen en la ejecución 
de un M.O.E.A.\medskip\break
A continuación se colocan todos los elementos que constituen el 
módulo en cuestión:

%******* Empieza módulo *******
\subsubsection{Home (módulo)}
%Se coloca el vínculo interno procedente de esta misma sección (a_3_2_1).
\label{sec:a_3_2_1}
Contiene toda la información posible para poder describir tanto 
los elementos que conforman el programa como su correcto uso.\medskip\break
Los elementos que constituyen a este módulo son:

%******* Empieza clase *******
\paragraph{HomeFrame (clase)}
%Se coloca el vínculo interno procedente de esta misma sección (a_3_2_1_1).
\label{sec:a_3_2_1_1}

%******* Empieza descripción *******
\begin{fulllineitems}

\begin{DUlineblock}{0em}
\item[] Unifica dos elementos: Canvas e IntroductionFrame.\break
La razón de haber hecho esto es que, cuando se añaden demasiados 
elementos al IntroductionFrame, se tiene que agregar una barra de 
desplazamiento para poder acceder a los que se encuentran hasta abajo.\break
Dentro del ambiente de Tkinter, el elemento más sencillo para lograr 
esto es un Canvas, por ello se anida el IntroductionFrame al Canvas.
\end{DUlineblock}

\begin{quote}\begin{description}
\item[{Parameters}] \leavevmode\begin{itemize}
\item \textbf{\texttt{parent}} (\emph{\texttt{Tkinter.Frame}}) -- Frame padre al que pertenece.
\item \textbf{\texttt{features}} (\emph{\texttt{Dictionary}}) -- Conjunto de técnicas con sus respectivos parámetros para que se puedan cargar automáticamente en este Frame \textbf{(véase
Controller/XMLParser.py)}.
\end{itemize}

\item[{Returns}] \leavevmode
Tkinter.Frame
\item[{Return type}] \leavevmode
Instance
\end{description}\end{quote}

%******* Termina descripción *******

%******* Empieza función *******
\begin{fulllineitems}

\pysiglinewithargsret{\sphinxbfcode{update\_scrollbar}}{\emph{event=None}}{}~

\begin{notice}{note}{Note:}
Este método es privado.
\end{notice}

Actualiza la barra de desplazamiento de acuerdo al número de 
elementos existentes en el Frame, esto para poder hacer un 
recorrido apropiado de la barra.

\begin{quote}\begin{description}
\item[{Parameters}] \leavevmode\begin{itemize}
\item \textbf{\texttt{event}} (\emph{\texttt{String}}) -- Elemento que ejecutó esta función.
\end{itemize}
\end{description}\end{quote}

\end{fulllineitems}
%******* Termina función *******

%******* Empieza función *******
\begin{fulllineitems}

\pysiglinewithargsret{\sphinxbfcode{move\_to\_section}}{\emph{y\_coordinate}}{}
Mueve la barra de desplazamiento (y por ende el contenido)
con base en la coordenada (en Y) que se le pase como parámetro.

\begin{quote}\begin{description}
\item[{Parameters}] \leavevmode\begin{itemize}
\item \textbf{\texttt{y\_coordinate}} -- Coordenada que se necesita para hace el desplazamiento. Oscila entre 0 y 1.
\end{itemize}
\end{description}\end{quote}

\end{fulllineitems}
%******* Termina función *******

%******* Empieza función *******
\begin{fulllineitems}

\pysiglinewithargsret{\sphinxbfcode{restore\_settings}}{}{}
Restaura la configuración del Frame a la que tenía por
defecto.

\end{fulllineitems}
%******* Termina función *******

\end{fulllineitems}
%******* Termina clase *******

La clase actual se apoya del elemento mostrado a continuación:

%******* Empieza clase *******
\subparagraph{IntroductionFrame (clase)}
%Se coloca el vínculo interno procedente de esta misma sección (a_3_2_1_1_1).
\label{sec:a_3_2_1_1_1}
%******* Empieza descripción *******
\begin{fulllineitems}

\begin{DUlineblock}{0em}
\item[] Contiene información básica y concisa sobre el producto 
de software, la cual es organizada y mostrada de acuerdo al número 
de secciones existentes en éste.\break
De manera secundaria proporciona la infraestructura para poder 
darle al usuario un desplazamiento más rápido entre dichas 
secciones.
\end{DUlineblock}

\begin{quote}\begin{description}
\item[{Parameters}] \leavevmode\begin{itemize}
\item \textbf{\texttt{parent}} (\emph{\texttt{Tkinter.Frame}}) -- Frame padre al que pertenece.
\item \textbf{\texttt{canvas\_function}} (\emph{\texttt{Instance}}) -- Una función alusiva al funcionamiento del Canvas.
\item \textbf{\texttt{features}} (\emph{\texttt{Dictionary}}) -- Conjunto de técnicas con sus respectivos parámetros para que se puedan cargar automáticamente en este Frame.
\end{itemize}

\item[{Returns}] \leavevmode
Tkinter.Frame
\item[{Return type}] \leavevmode
Instance
\end{description}\end{quote}

%******* Termina descripción *******

%******* Empieza función *******
\begin{fulllineitems}

\pysiglinewithargsret{\sphinxbfcode{go\_to\_selected\_section}}{\emph{event}}{}~

\begin{notice}{note}{Note:}
Este método es privado.
\end{notice}

Con base en la liga elegida por el usuario, realiza el
desplazamiento hacia la sección correspondiente.

\begin{quote}\begin{description}
\item[{Parameters}] \leavevmode\begin{itemize}
\item \textbf{\texttt{event}} (\emph{\texttt{String}}) -- Elemento que ejecutó esta función.
\end{itemize}
\end{description}\end{quote}

\end{fulllineitems}
%******* Termina función *******

\end{fulllineitems}
%******* Termina clase *******
%******* Termina módulo *******

%******* Empieza módulo *******
\subsubsection{DecisionVariable (módulo)}
%Se coloca el vínculo interno procedente de esta misma sección (a_3_2_2).
\label{sec:a_3_2_2}

Proporciona los elementos gráficos para que el usuario pueda
insertar, modificar y eliminar variables de decisión con sus 
respectivos rangos.\medskip\break
Los elementos que constituyen al módulo son:

%******* Empieza clase *******
\paragraph{DecisionVariableFrame (clase)}
%Se coloca el vínculo interno procedente de esta misma sección (a_3_2_2_1).
\label{sec:a_3_2_2_1}
%******* Empieza descripción *******
\begin{fulllineitems}

\begin{DUlineblock}{0em}
\item[] Realiza la fusión de Canvas y VariableFrame, debido a que, 
cuando se agregan numerosas variables al VariableFrame, se debe 
insertar una barra de desplazamiento para poder acceder a aquéllos 
que se encuentren hasta abajo.\break
Dentro del ambiente de Tkinter, el elemento más sencillo para lograr 
este efecto es un Canvas, por ello se anida el VariableFrame al Canvas.
\end{DUlineblock}

\begin{quote}\begin{description}
\item[{Parameters}] \leavevmode\begin{itemize}
\item \textbf{\texttt{parent}} (\emph{\texttt{Tkinter.Frame}}) -- Frame padre al que pertenece.
\item \textbf{\texttt{features}} (\emph{\texttt{Dictionary}}) -- Conjunto de técnicas con sus respectivos parámetros para que se puedan cargar automáticamente en este Frame \textbf{(véase Controller/XMLParser.py)}.
\end{itemize}

\item[{Returns}] \leavevmode
Tkinter.Frame
\item[{Return type}] \leavevmode
Instance
\end{description}\end{quote}

%******* Termina descripción *******

%******* Empieza función *******
\begin{fulllineitems}

\pysiglinewithargsret{\sphinxbfcode{activate\_scroll}}{\emph{event}}{}~

\begin{notice}{note}{Note:}
Este método es privado.
\end{notice}

Actualiza la barra de desplazamiento y con base en esta acción
la activa o desactiva.

\begin{quote}\begin{description}
\item[{Parameters}] \leavevmode\begin{itemize}
\item \textbf{\texttt{event}} (\emph{\texttt{String}}) -- Elemento que ejecutó esta función.
\end{itemize}
\end{description}\end{quote}

\end{fulllineitems}
%******* Termina función *******

%******* Empieza función *******
\begin{fulllineitems}

\pysiglinewithargsret{\sphinxbfcode{update\_scrollbar}}{\emph{event=None}}{}~

\begin{notice}{note}{Note:}
Este método es privado.
\end{notice}

Actualiza la barra de desplazamiento de acuerdo al número de elementos
existentes en el Frame, esto para poder hacer un recorrido apropiado de 
la barra.

\begin{quote}\begin{description}
\item[{Parameters}] \leavevmode\begin{itemize}
\item \textbf{\texttt{event}} (\emph{\texttt{String}}) -- Elemento que ejecutó esta función.
\end{itemize}
\end{description}\end{quote}

\end{fulllineitems}
%******* Termina función *******

%******* Empieza función *******
\begin{fulllineitems}

\pysiglinewithargsret{\sphinxbfcode{get\_information}}{}{}
Regresa la información recabada en el Frame.

\begin{quote}\begin{description}
\item[{Returns}] \leavevmode
\item[{Return type}] \leavevmode
Dictionary
\end{description}\end{quote}

\end{fulllineitems}
%******* Termina función *******

%******* Empieza función *******
\begin{fulllineitems}

\pysiglinewithargsret{\sphinxbfcode{insert\_mop\_example}}{\emph{variables}}{}~
Inserta un M.O.P \textbf{(Multi Objective Problem ó Problema Multi Objetivo)}.\break
En este caso significa que se insertarán las variables con sus respectivos 
rangos en el Frame para poder hacer pruebas rápidas en el programa, habiendo 
antes limpiado por completo el contenido del Frame.\break
\textbf{(véase Controller/XML/MOPExample.xml)}\break
\textbf{(véase View/Additional/MenuInternalOption/InternalOptionFrame.py)}.\break

\begin{quote}\begin{description}
\item[{Parameters}] \leavevmode\begin{itemize}
\item \textbf{\texttt{functions}} (\emph{\texttt{List}}) -- Lista de variables para ser insertadas en el Frame.
\end{itemize}
\end{description}\end{quote}

\end{fulllineitems}
%******* Termina función *******

%******* Empieza función *******
\begin{fulllineitems}

\pysiglinewithargsret{\sphinxbfcode{restore\_settings}}{}{}
Restaura el contenido del Frame, en este caso significa que 
se eliminará todo lo que esté en éste y se dejará una casilla 
vacía libre.

\end{fulllineitems}
%******* Termina función *******

\end{fulllineitems}
%******* Termina clase *******

La clase actual se basa en el siguiente elemento:

%******* Empieza clase *******
\subparagraph{VariableFrame (clase)}
%Se coloca el vínculo interno procedente de esta misma sección (a_3_2_2_1_1).
\label{sec:a_3_2_2_1_1}
%******* Empieza descripción *******
\begin{fulllineitems}

\begin{DUlineblock}{0em}
\item[] Proporciona bases gráficas para que el usuario pueda insertar
variables de decisión, así como información relativa a éstas.\break
En términos generales, el usuario insertará casillas para ingresar variables
de decisión, indicando también el valor mínimo y máximo que podrán tener.\medskip\break
Es importante comentar que todas las variables de decisión deben contener
rangos finitos, es decir, no se contemplan valores infinitos, aunque algunos
M.O.P.'s \textbf{(Multi Objective Problems ó Problemas Multi Objetivo)} 
manejan este tipo de rangos.         
\end{DUlineblock}

\begin{quote}\begin{description}
\item[{Parameters}] \leavevmode\begin{itemize}
\item \textbf{\texttt{parent}} (\emph{\texttt{Tkinter.Frame}}) -- Frame padre al que pertenece.
\item \textbf{\texttt{features}} (\emph{\texttt{Dictionary}}) -- Conjunto de técnicas con sus respectivos parámetros para que se puedan cargar automáticamente en este Frame \textbf{(véase Controller/XMLParser.py)}.
\end{itemize}

\item[{Returns}] \leavevmode
Tkinter.Frame
\item[{Return type}] \leavevmode
Instance
\end{description}\end{quote}

%******* Termina descripción *******

%******* Empieza función *******
\begin{fulllineitems}

\pysiglinewithargsret{\sphinxbfcode{add\_variable}}{\emph{event}}{}~
\begin{notice}{note}{Note:}
Este método es privado.
\end{notice}

Agrega una casilla al Frame.\break
Esta función se usa si fue ejecutada por un 
evento.

\begin{quote}\begin{description}
\item[{Parameters}] \leavevmode\begin{itemize}
\item \textbf{\texttt{event}} (\emph{\texttt{String}}) -- Identificador del elemento gráfico que activó la función.
\end{itemize}
\end{description}\end{quote}

\end{fulllineitems}
%******* Termina función *******

%******* Empieza función *******
\begin{fulllineitems}

\pysiglinewithargsret{\sphinxbfcode{delete\_single\_variable}}{\emph{event}}{}~

\begin{notice}{note}{Note:}
Este método es privado.
\end{notice}

Elimina una casilla y todos los elementos gráficos que 
la acompañan.\break
También elimina todo rastro que se encuentre en las 
estructuras lógicas.

\begin{quote}\begin{description}
\item[{Parameters}] \leavevmode\begin{itemize}
\item \textbf{\texttt{event}} (\emph{\texttt{String}}) -- Identificador del elemento gráfico que activó la función.
\end{itemize}
\end{description}\end{quote}

\end{fulllineitems}
%******* Termina función *******

%******* Empieza función *******
\begin{fulllineitems}

\pysiglinewithargsret{\sphinxbfcode{grid\_widgets}}{}{}~

\begin{notice}{note}{Note:}
Este método es privado.
\end{notice}

Coloca elementos en el Frame.

\end{fulllineitems}
%******* Termina función *******

%******* Empieza función *******
\begin{fulllineitems}

\pysiglinewithargsret{\sphinxbfcode{get\_current\_elements}}{}{}
Regresa el número actual de casillas en el Frame.

\begin{quote}\begin{description}
\item[{Returns}] \leavevmode
Cantidad de elementos en la estructura rows, donde se guardan las casillas (Entries).
\item[{Return type}] \leavevmode
Integer
\end{description}\end{quote}

\end{fulllineitems}
%******* Termina función *******

%******* Empieza función *******
\begin{fulllineitems}

\pysiglinewithargsret{\sphinxbfcode{get\_information}}{}{}
Toma la información del Frame y regresa las variables con 
sus rangos que el usuario ingresó.

\begin{quote}\begin{description}
\item[{Returns}] \leavevmode
Un diccionario que contiene una lista con las variables (y rangos) escritas.
\item[{Return type}] \leavevmode
Dictionary
\end{description}\end{quote}

\end{fulllineitems}
%******* Termina función *******

%******* Empieza función *******
\begin{fulllineitems}

\pysiglinewithargsret{\sphinxbfcode{insert\_mop\_example}}{\emph{variables}}{}~
\vspace{-0.3cm}

Inserta un M.O.P (Multi Objective Problem) que no es más 
que un conjunto de variables con sus rangos para que se 
pueda hacer más rápidamente una prueba.\break
Previo a ésto se limpia el Frame para insertar únicamente 
el M.O.P.\break
\textbf{(véase Controller/XML/MOPExample.xml)}\break
\textbf{(véase View/Additional/MenuInternalOption/InternalOptionFrame.py)}.

\begin{quote}\begin{description}
\item[{Parameters}] \leavevmode\begin{itemize}
\item \textbf{\texttt{functions}} (\emph{\texttt{List}}) -- Conjunto de variables para insertar en el Frame.
\end{itemize}
\end{description}\end{quote}

\end{fulllineitems}
%******* Termina función *******

%******* Empieza función *******
\begin{fulllineitems}

\pysiglinewithargsret{\sphinxbfcode{insert\_variable}}{\emph{variable=None}}{}~
\vspace{-0.3cm}

Coloca en el Frame una colección de elementos:\break
{[}casilla para insertar variable ,casilla de rango minimo, casilla de rango máximo, botón para eliminar{]}\break
Si el parámetro function es \textbf{None}, se añade 
la casilla vacía, de lo contrario se agrega ésta con 
la variable y sus rangos.

\begin{quote}\begin{description}
\item[{Parameters}] \leavevmode\begin{itemize}
\item \textbf{\texttt{function}} (\emph{\texttt{String}}) -- Una terna (nombre de la variable, rango máximo, rango mínimo) para ser insertada en las casillas correspondientes.
\end{itemize}
\end{description}\end{quote}

\end{fulllineitems}
%******* Termina función *******

%******* Empieza función *******
\begin{fulllineitems}

\pysiglinewithargsret{\sphinxbfcode{restore\_settings}}{}{}
Restaura el contenido del Frame a sus valores por defecto.\break
Esto significa que borrará cualquier contenido que se 
encuentre en existencia y dejará una casilla vacía.

\end{fulllineitems}
%******* Termina función *******

\end{fulllineitems}
%******* Termina clase *******
%******* Termina módulo *******

%******* Empieza módulo *******
\subsubsection{ObjectiveFunction (módulo)}
%Se coloca el vínculo interno procedente de esta misma sección (a_3_2_3).
\label{sec:a_3_2_3}
Proporciona los elementos gráficos para que el usuario
pueda insertar, modificar y eliminar funciones objetivo.\medskip\break
Sus elementos que lo conforman son:

%******* Empieza clase *******
\paragraph{ObjectiveFunctionFrame (clase)}
%Se coloca el vínculo interno procedente de esta misma sección (a_3_2_3_1).
\label{sec:a_3_2_3_1}
%******* Empieza descripción *******
\begin{fulllineitems}

\begin{DUlineblock}{0em}
\item[] Unifica dos elementos: Canvas y FunctionFrame.\break
La razón de haber hecho esto es que, cuando se agregan muchas 
funciones al FunctionFrame, se tiene que agregar una barra de 
desplazamiento para poder acceder a los que se encuentran hasta 
abajo.\break
Dentro del ambiente de Tkinter, el elemento más sencillo para 
lograr esto es un Canvas, por ello se anida el FunctionFrame al 
Canvas.
\end{DUlineblock}

\begin{quote}\begin{description}
\item[{Parameters}] \leavevmode\begin{itemize}
\item \textbf{\texttt{parent}} (\emph{\texttt{Tkinter.Frame}}) -- Frame padre al que pertenece.
\item \textbf{\texttt{features}} (\emph{\texttt{Dictionary}}) -- Conjunto de técnicas con sus respectivos parámetros para que se puedan cargar automáticamente en este Frame \textbf{(véase Controller/XMLParser.py)}.
\end{itemize}

\item[{Returns}] \leavevmode
Tkinter.Frame
\item[{Return type}] \leavevmode
Instance
\end{description}\end{quote}

%******* Termina descripción *******

%******* Empieza función *******
\begin{fulllineitems}

\pysiglinewithargsret{\sphinxbfcode{activate\_scroll}}{\emph{event}}{}~

\begin{notice}{note}{Note:}
Este método es privado.
\end{notice}

Actualiza la barra de desplazamiento y con base en esta acción
la activa o desactiva.

\begin{quote}\begin{description}
\item[{Parameters}] \leavevmode\begin{itemize}
\item \textbf{\texttt{event}} (\emph{\texttt{String}}) -- Elemento que ejecutó esta función.
\end{itemize}
\end{description}\end{quote}

\end{fulllineitems}
%******* Termina función *******

%******* Empieza función *******
\begin{fulllineitems}

\pysiglinewithargsret{\sphinxbfcode{update\_scrollbar}}{\emph{event=None}}{}~

\begin{notice}{note}{Note:}
Este método es privado.
\end{notice}

Actualiza la barra de desplazamiento de acuerdo al número 
de elementos existentes en el Frame, esto para poder hacer 
un recorrido apropiado de la barra.

\begin{quote}\begin{description}
\item[{Parameters}] \leavevmode\begin{itemize}
\item \textbf{\texttt{event}} (\emph{\texttt{String}}) -- Elemento que ejecutó esta función.
\end{itemize}
\end{description}\end{quote}

\end{fulllineitems}
%******* Termina función *******

%******* Empieza función *******
\begin{fulllineitems}

\pysiglinewithargsret{\sphinxbfcode{get\_information}}{}{}
Regresa la información recabada en el Frame.

\begin{quote}\begin{description}
\item[{Returns}] \leavevmode
Un diccionario que contiene una lista con las funciones escritas.
\item[{Return type}] \leavevmode
Dictionary
\end{description}\end{quote}

\end{fulllineitems}
%******* Termina función *******

%******* Empieza función *******
\begin{fulllineitems}

\pysiglinewithargsret{\sphinxbfcode{insert\_mop\_example}}{\emph{functions}}{}~
\vspace{-0.3cm}

Inserta un M.O.P (Multi Objective Problem).\break
En este caso significa que se insertarán funciones
para poder hacer pruebas rápidas en el programa.\break
\textbf{(véase Controller/XML/MOPExample.xml)}\break
\textbf{(véase View/Additional/MenuInternalOption/InternalOptionFrame.py)}.

\begin{quote}\begin{description}
\item[{Parameters}] \leavevmode\begin{itemize}
\item \textbf{\texttt{functions}} (\emph{\texttt{List}}) -- Lista de funciones para ser insertadas en el Frame.
\end{itemize}
\end{description}\end{quote}

\end{fulllineitems}
%******* Termina función *******

%******* Empieza función *******
\begin{fulllineitems}

\pysiglinewithargsret{\sphinxbfcode{restore\_settings}}{}{}
Restaura el contenido del Frame, en este caso significa 
que se eliminará todo lo que esté en éste y se dejará 
una casilla vacía libre.

\end{fulllineitems}
%******* Termina función *******

\end{fulllineitems}
%******* Termina clase *******

La clase actual toma como fundamento lo siguiente:

%******* Empieza clase *******
\subparagraph{FunctionFrame (clase)}
%Se coloca el vínculo interno procedente de esta misma sección (a_3_2_3_1_1).
\label{sec:a_3_2_3_1_1}

%******* Empieza descripción *******
\begin{fulllineitems}

\begin{DUlineblock}{0em}
\item[] Esta clase proporciona una base gráfica para que 
el usuario pueda agregar tantas functiones objetivo como 
desee.\break
A grandes rasgos el usuario podrá agregar casillas donde 
se colocarán las funciones objetivo, esto utilizando un botón.\break
De igual manera, las casillas pueden ser eliminadas
usando un ícono que estará cerca de cada una de éstas.\break
Importante es mencionar que las funciones deben estar 
escritas en sintaxis de Python.
\end{DUlineblock}

\begin{quote}\begin{description}
\item[{Parameters}] \leavevmode\begin{itemize}
\item \textbf{\texttt{parent}} (\emph{\texttt{Tkinter.Frame}}) -- Frame padre al que pertenece.
\item \textbf{\texttt{features}} (\emph{\texttt{Dictionary}}) -- Conjunto de técnicas con sus respectivos parámetros para que se puedan cargar automáticamente en este Frame \textbf{(véase Controller/XMLParser.py)}.
\end{itemize}

\item[{Returns}] \leavevmode
Tkinter.Frame
\item[{Return type}] \leavevmode
Instance
\end{description}\end{quote}

%******* Termina descripción *******

%******* Empieza función *******
\begin{fulllineitems}

\pysiglinewithargsret{\sphinxbfcode{add\_function}}{\emph{event}}{}~

\begin{notice}{note}{Note:}
Este método es privado.
\end{notice}

Agrega una casilla al Frame.\break
Esta función se usa si fue ejecutada 
por un evento.

\begin{quote}\begin{description}
\item[{Parameters}] \leavevmode\begin{itemize}
\item \textbf{\texttt{event}} (\emph{\texttt{String}}) -- Identificador del elemento gráfico que activó la función.
\end{itemize}
\end{description}\end{quote}

\end{fulllineitems}
%******* Termina función *******

%******* Empieza función *******
\begin{fulllineitems}

\pysiglinewithargsret{\sphinxbfcode{delete\_single\_function}}{\emph{event}}{}~

\begin{notice}{note}{Note:}
Este método es privado.
\end{notice}

Elimina una casilla y todos los elementos 
gráficos que la acompañan.\break
También elimina todo rastro que se encuentre 
en las estructuras lógicas.

\begin{quote}\begin{description}
\item[{Parameters}] \leavevmode\begin{itemize}
\item \textbf{\texttt{event}} (\emph{\texttt{String}}) -- Identificador del elemento gráfico que activó la función.
\end{itemize}
\end{description}\end{quote}

\end{fulllineitems}
%******* Termina función *******

%******* Empieza función *******
\begin{fulllineitems}

\pysiglinewithargsret{\sphinxbfcode{grid\_widgets}}{}{}~

\begin{notice}{note}{Note:}
Este método es privado.
\end{notice}

Coloca elementos en el Frame.

\end{fulllineitems}
%******* Termina función *******

%******* Empieza función *******
\begin{fulllineitems}

\pysiglinewithargsret{\sphinxbfcode{get\_current\_elements}}{}{}
Regresa el número actual de casillas 
en el Frame.

\begin{quote}\begin{description}
\item[{Returns}] \leavevmode
Cantidad de elementos en la estructura rows, donde se guardan las casillas (Entries).

\item[{Return type}] \leavevmode
Integer
\end{description}\end{quote}

\end{fulllineitems}
%******* Termina función *******

%******* Empieza función *******
\begin{fulllineitems}

\pysiglinewithargsret{\sphinxbfcode{get\_information}}{}{}
Toma la información del Frame y regresa las funciones 
objectivo que el usuario insertó.

\begin{quote}\begin{description}
\item[{Returns}] \leavevmode
Un diccionario que contiene una lista con las funciones escritas.
\item[{Return type}] \leavevmode
Dictionary
\end{description}\end{quote}

\end{fulllineitems}
%******* Termina función *******

%******* Empieza función *******
\begin{fulllineitems}

\pysiglinewithargsret{\sphinxbfcode{insert\_function}}{\emph{function=None}}{}~
\vspace{-0.3cm}

Coloca en el Frame una colección de elementos:\break
{[}casilla para insertar funcion, opción de maximizar, opción de minimizar, botón para eliminar{]}\break
Si el parámetro function es \textbf{None}, se agrega la casilla 
vacía, de lo contrario se añade ésta con la función.

\begin{quote}\begin{description}
\item[{Parameters}] \leavevmode\begin{itemize}
\item \textbf{\texttt{function}} (\emph{\texttt{String}}) -- Una función para ser insertada en el primer elemento de la colección.
\end{itemize}
\end{description}\end{quote}

\end{fulllineitems}
%******* Termina función *******

%******* Empieza función *******
\begin{fulllineitems}

\pysiglinewithargsret{\sphinxbfcode{insert\_mop\_example}}{\emph{functions}}{}~
\vspace{-0.3cm}

Inserta un M.O.P (Multi Objective Problem) que no es más 
que un conjunto de funciones para que se pueda hacer más 
rápidamente una prueba.\break
Previo a ésto se limpia el Frame para insertar únicamente 
el M.O.P.\break
\textbf{(véase Controller/XML/MOPExample.xml)}\break
\textbf{(véase View/Additional/MenuInternalOption/InternalOptionFrame.py)}.

\begin{quote}\begin{description}
\item[{Parameters}] \leavevmode\begin{itemize}
\item \textbf{\texttt{functions}} (\emph{\texttt{List}}) -- Conjunto de funciones para insertar en el Frame.
\end{itemize}
\end{description}\end{quote}

\end{fulllineitems}
%******* Termina función *******

%******* Empieza función *******
\begin{fulllineitems}

\pysiglinewithargsret{\sphinxbfcode{restore\_settings}}{}{}
Restaura el contenido del Frame a sus valores por 
defecto.\break
Esto significa que borrará cualquier contenido que se 
encuentre en existencia y dejará una casilla vacía.

\end{fulllineitems}
%******* Termina función *******

\end{fulllineitems}
%******* Termina clase *******
%******* Termina módulo *******

%******* Empieza módulo *******
\subsubsection{Population (módulo)}
%Se coloca el vínculo interno procedente de esta misma sección (a_3_2_4).
\label{sec:a_3_2_4}
Proporciona las estructuras gráficas para que el usuario 
pueda configurar atributos de la Población.\break
Los elementos que conforman al módulo son los siguientes:

%******* Empieza módulo *******
\paragraph{PopulationFrame (clase)}
%Se coloca el vínculo interno procedente de esta misma sección (a_3_2_4_1).
\label{sec:a_3_2_4_1}

%******* Empieza descripción *******
\begin{fulllineitems}

\begin{DUlineblock}{0em}
\item[] Unifica y mantiene un control sobre las clases 
PopulaceFrame y FitnessFrame, esto con el fin de poder 
colocar los elementos apropiadamente y agilizar el 
intercambio de información con el usuario.
\end{DUlineblock}

\begin{quote}\begin{description}
\item[{Parameters}] \leavevmode\begin{itemize}
\item \textbf{\texttt{parent}} (\emph{\texttt{Tkinter.Frame}}) -- Frame padre al que pertenece.
\item \textbf{\texttt{features}} (\emph{\texttt{Dictionary}}) -- Conjunto de técnicas con sus respectivos parámetros para que se puedan cargar automáticamente en este Frame \textbf{(véase Controller/XMLParser.py)}.
\end{itemize}

\item[{Returns}] \leavevmode
Tkinter.Frame
\item[{Return type}] \leavevmode
Instance
\end{description}\end{quote}

%******* Termina descripción *******

%******* Empieza función *******
\begin{fulllineitems}

\pysiglinewithargsret{\sphinxbfcode{get\_information}}{}{}
Toma la información propiciada en cada Frame y después
la unifica para regresar un sólo conjunto de información.

\begin{quote}\begin{description}
\item[{Returns}] \leavevmode
Un diccionario con la información de PopulaceFrame y FitnessFrame.
\item[{Return type}] \leavevmode
Dictionary
\end{description}\end{quote}

\end{fulllineitems}
%******* Termina función *******

%******* Empieza función *******
\begin{fulllineitems}

\pysiglinewithargsret{\sphinxbfcode{restore\_settings}}{}{}
Restaura los valores por defecto en ambos Frames.

\end{fulllineitems}
%******* Termina función *******

\end{fulllineitems}
%******* Termina clase *******

%******* Empieza clase *******
\paragraph{TemplatePopulationFrame (clase)}
%Se coloca el vínculo interno procedente de esta misma sección (a_3_2_4_2).
\label{sec:a_3_2_4_2}

%******* Empieza descripción *******
\begin{fulllineitems}

\begin{DUlineblock}{0em}
\item[] Esta clase proporciona la infraestructura gráfica para que el 
usuario pueda  elegir técnicas y configurar atributos 
concernientes al Fitness de una Población y a la Población 
en general.\break
A grandes rasgos se trata de una plantilla que deberán implementar 
las clases FitnessFrame y PopulaceFrame.\break
La clase permite la selección de cada posible técnica disponible y 
automáticamente se muestran los parámetros necesarios \textbf{(si los hay)} 
para cada una de éstas.
\end{DUlineblock}

\begin{quote}\begin{description}
\item[{Parameters}] \leavevmode\begin{itemize}
\item \textbf{\texttt{parent}} (\emph{\texttt{Tkinter.Frame}}) -- Frame padre al que pertenece.
\item \textbf{\texttt{name}} (\emph{\texttt{String}}) -- Identificador \textbf{(único)} que tendrá el Frame.
\item \textbf{\texttt{features}} (\emph{\texttt{Dictionary}}) -- Conjunto de técnicas con sus respectivos parámetros para que se puedan cargar automáticamente en este frame \textbf{(véase Controller/XMLParser.py)}.
\end{itemize}

\item[{Returns}] \leavevmode
Tkinter.Frame
\item[{Return type}] \leavevmode
Instance
\end{description}\end{quote}

%******* Termina descripción *******

%******* Empieza función *******
\begin{fulllineitems}

\pysiglinewithargsret{\sphinxbfcode{create\_dynamic\_widgets}}{}{}~

\begin{notice}{note}{Note:}
Este método es privado.
\end{notice}

Inicializa los elementos dinámicos del Frame, esto es, de 
acuerdo al tipo que lleva cada parámetro se creará un widget 
diferente.

\end{fulllineitems}
%******* Termina función *******

%******* Empieza función *******
\begin{fulllineitems}

\pysiglinewithargsret{\sphinxbfcode{update\_widgets}}{\emph{event=None}}{}~

\begin{notice}{note}{Note:}
Este método es privado.
\end{notice}

Realiza solamente la actualización y colocación de elementos 
dinámicos en el Frame.\break
Si el parámetro event es distinto de \textbf{None}, significa 
que se lanzó un evento que provocará que se actualicen los 
parámetros de acuerdo con la técnica seleccionada.

\begin{quote}\begin{description}
\item[{Parameters}] \leavevmode\begin{itemize}
\item \textbf{\texttt{event}} (\emph{\texttt{String}}) -- Contiene el valor del elemento que ejecutó esta función.
\end{itemize}
\end{description}\end{quote}

\end{fulllineitems}
%******* Termina función *******

%******* Empieza función *******
\begin{fulllineitems}

\pysiglinewithargsret{\sphinxbfcode{get\_information}}{}{}
Recolecta la información que ha seleccionado e introducido 
el usuario, también la organiza para que se pueda utilizar 
apropiadamente.

\begin{quote}\begin{description}
\item[{Returns}] \leavevmode

Un diccionario que contiene:

\begin{itemize}
\item \textbf{Clase},
\item \textbf{Técnica},
\item \textbf{Parametros}
\end{itemize}

\item[{Return type}] \leavevmode
Dictionary
\end{description}\end{quote}

\end{fulllineitems}
%******* Termina función *******

%******* Empieza función *******
\begin{fulllineitems}

\pysiglinewithargsret{\sphinxbfcode{grid\_widgets}}{}{}
Permite la colocación adecuada de elementos estáticos y 
dinámicos, considerando además el espacio o características 
necesarias de redimensionamiento para éstos últimos.

\end{fulllineitems}
%******* Termina función *******

%******* Empieza función *******
\begin{fulllineitems}

\pysiglinewithargsret{\sphinxbfcode{restore\_settings}}{}{}
Asigna los valores por defecto tanto de las técnicas como 
de sus respectivos parámetros, también limpia aquéllos en 
donde se hayan insertado valores.

\end{fulllineitems}
%******* Termina función *******

\end{fulllineitems}
%******* Termina clase *******

Los siguientes elementos implementan la plantilla actual:

%******* Empieza clase *******
\subparagraph{PopulaceFrame (clase)}
%Se coloca el vínculo interno procedente de esta misma sección (a_3_2_4_2_1).
\label{sec:a_3_2_4_2_1}
%******* Empieza descripción *******
\begin{fulllineitems}

\begin{DUlineblock}{0em}
\item[] Esta clase proporciona la infraestructura gráfica para que 
el usuario pueda elegir métodos y características concernientes a 
la conformación de la Población.\break
También hereda atributos de la clase TemplatePopulationFrame con el 
fin de establecer una forma más rápida y ordenada de colocar componentes 
y recolectar la información de éstos.
\end{DUlineblock}

\begin{quote}\begin{description}
\item[{Parameters}] \leavevmode\begin{itemize}
\item \textbf{\texttt{parent}} (\emph{\texttt{Tkinter.Frame}}) -- Frame padre al que pertenece.
\item \textbf{\texttt{name}} (\emph{\texttt{String}}) -- Identificador \textbf{(único)} que tendrá el Frame.
\item \textbf{\texttt{features}} (\emph{\texttt{Dictionary}}) -- Conjunto de técnicas con sus respectivos parámetros para que se puedan cargar automáticamente en este Frame \textbf{(véase Controller/XMLParser.py)}.
\end{itemize}

\item[{Returns}] \leavevmode
Tkinter.Frame
\item[{Return type}] \leavevmode
Instance
\end{description}\end{quote}

%******* Termina descripción *******

%******* Empieza función *******
\begin{fulllineitems}

\pysiglinewithargsret{\sphinxbfcode{get\_information}}{}{}
Recolecta la información genérica \textbf{(usando el método de la clase Padre)}, 
y también se le añade aquélla recolectada exclusivamente 
en esta clase.

\begin{quote}\begin{description}
\item[{Returns}] \leavevmode

Un diccionario que contiene:

\begin{itemize}
\item \textbf{Métodos genéricos,}
\item \textbf{Número de Generaciones,}
\item \textbf{Tamaño de la Población,}
\item \textbf{Número de Decimales.}
\end{itemize}

\item[{Return type}] \leavevmode
Dictionary
\end{description}\end{quote}

\end{fulllineitems}
%******* Termina función *******

%******* Empieza función *******
\begin{fulllineitems}

\pysiglinewithargsret{\sphinxbfcode{restore\_settings}}{}{}~
\vspace{-0.3cm}

Por un lado, restaura el contenido de los elementos pertenecientes 
sólo a esta clase, y por el otro, activa el método de la clase 
Padre que realiza una restauración de los elementos genéricos.

\end{fulllineitems}
%******* Termina función *******

\end{fulllineitems}
%******* Termina función *******

%******* Empieza clase *******
\subparagraph{FitnessFrame (clase)}
%Se coloca el vínculo interno procedente de esta misma sección (a_3_2_4_2_2).
\label{sec:a_3_2_4_2_2}
%******* Empieza descripción *******
\begin{fulllineitems}

\begin{DUlineblock}{0em}
\item[] Esta clase proporciona la infraestructura gráfica para que 
el usuario pueda elegir métodos concernientes a la asignación del 
Fitness para la Población.\break
Además hereda atributos de la clase TemplatePopulationFrame para 
facilitar la colocacion y extracción de información pertinente 
para el usuario.
\end{DUlineblock}

\begin{quote}\begin{description}
\item[{Parameters}] \leavevmode\begin{itemize}
\item \textbf{\texttt{parent}} (\emph{\texttt{Tkinter.Frame}}) -- Frame padre al que pertenece.
\item \textbf{\texttt{name}} (\emph{\texttt{String}}) -- Identificador \textbf{(único)} que tendrá el Frame.
\item \textbf{\texttt{features}} (\emph{\texttt{Dictionary}}) -- Conjunto de técnicas con sus respectivos parámetros para que se puedan cargar automáticamente en este Frame \textbf{(véase Controller/XMLParser.py)}.
\end{itemize}

\item[{Returns}] \leavevmode
Tkinter.Frame
\item[{Return type}] \leavevmode
Instance

\end{description}\end{quote}

%******* Termina descripción *******

%******* Empieza función *******
\begin{fulllineitems}

\pysiglinewithargsret{\sphinxbfcode{get\_information}}{}{}
Llama al método de la clase Padre, el cual recopila toda la 
información elegida por el usuario y la regresa en forma de 
diccionario.

\begin{quote}\begin{description}
\item[{Returns}] \leavevmode
Diccionario con información de los métodos genéricos.
\item[{Return type}] \leavevmode
Dictionary
\end{description}\end{quote}

\end{fulllineitems}
%******* Termina función *******

%******* Empieza función *******
\begin{fulllineitems}

\pysiglinewithargsret{\sphinxbfcode{restore\_settings}}{}{}
Llamar al método de la clase Padre, el cual restaura los 
valores por defecto de los elementos dinámicos y estáticos 
del Frame.

\end{fulllineitems}
%******* Termina función *******

\end{fulllineitems}
%******* Termina clase *******
%******* Termina módulo *******

%******* Empieza módulo *******
\subsubsection{GeneticOperator (módulo)}
%Se coloca el vínculo interno procedente de esta misma sección (a_3_2_5).
\label{sec:a_3_2_5}
Proporciona los elementos gráficos para que el usuario pueda
realizar operaciones relacionadas con la Selección, Cruza 
y Mutación de Individuos de una Población.\medskip\break
Los elementos que lo conforman son:

%******* Empieza clase *******
\paragraph{GeneticOperatorFrame (clase)}
%Se coloca el vínculo interno procedente de esta misma sección (a_3_2_5_1).
\label{sec:a_3_2_5_1}
%******* Empieza descripción *******
\begin{fulllineitems}

\begin{DUlineblock}{0em}
\item[] Reúne y controla las clases SelectionFrame, CrossoverFrame y  
MutationFrame con la finalidad de colocar los elementos gráficos apropiadamente y 
agilizar el intercambio de información con el usuario.
\end{DUlineblock}

\begin{quote}\begin{description}
\item[{Parameters}] \leavevmode\begin{itemize}
\item \textbf{\texttt{parent}} (\emph{\texttt{Tkinter.Frame}}) -- Frame padre al que pertenece.
\item \textbf{\texttt{features}} (\emph{\texttt{Dictionary}}) -- Conjunto de técnicas con sus respectivos parámetros para que se puedan cargar automáticamente en este Frame \textbf{(véase Controller/XMLParser.py)}.
\end{itemize}

\item[{Returns}] \leavevmode
Tkinter.Frame
\item[{Return type}] \leavevmode
Instance
\end{description}\end{quote}

%******* Termina descripción *******

%******* Empieza función *******
\begin{fulllineitems}

\pysiglinewithargsret{\sphinxbfcode{get\_information}}{}{}
Toma la información propiciada en cada Frame y 
después la unifica para regresar un sólo conjunto 
de información.

\begin{quote}\begin{description}
\item[{Returns}] \leavevmode
Un diccionario con la información de SelectionFrame, CrossoverFrame y MutationFrame.
\item[{Return type}] \leavevmode
Dictionary
\end{description}\end{quote}

\end{fulllineitems}
%******* Termina función *******

%******* Empieza función *******
\begin{fulllineitems}

\pysiglinewithargsret{\sphinxbfcode{restore\_settings}}{}{}
Realiza la restauración de información y contenido 
en cada uno de los Frames.

\end{fulllineitems}
%******* Termina función *******

\end{fulllineitems}
%******* Termina clase *******

%******* Empieza clase *******
\paragraph{TemplateGeneticOperatorFrame (clase)}
%Se coloca el vínculo interno procedente de esta misma sección (a_3_2_5_2).
\label{sec:a_3_2_5_2}
%******* Empieza descripción *******
\begin{fulllineitems}

\begin{DUlineblock}{0em}
\item[] Proporciona la infraestructura gráfica para que el usuario 
pueda elegir técnicas y configurar atributos concernientes a la 
Selección, Cruza y Mutación de Individuos de una Población.\break
A grandes rasgos se trata de una plantilla que deberán implementar 
las clases SelectionFrame, CrossoverFrame y MutationFrame.\break
La clase permite la selección de cada posible técnica disponible 
y automáticamente se muestran los parámetros necesarios \textbf{(si los hay)} 
para cada una de éstas.
\end{DUlineblock}

\begin{quote}\begin{description}
\item[{Parameters}] \leavevmode\begin{itemize}
\item \textbf{\texttt{parent}} (\emph{\texttt{Tkinter.Frame}}) -- Frame padre al que pertenece.
\item \textbf{\texttt{name}} (\emph{\texttt{String}}) -- Identificador \textbf{(único)} que tendrá el Frame.
\item \textbf{\texttt{features}} (\emph{\texttt{Dictionary}}) -- Conjunto de técnicas con sus respectivos parámetros para que se puedan cargar automáticamente en este Frame \textbf{(véase Controller/XMLParser.py)}.
\item \textbf{\texttt{sort\_techniques}} (\emph{\texttt{Boolean}}) -- Indica si las técnicas disponibles se ordenan alfabéticamente
o no.
\end{itemize}

\item[{Returns}] \leavevmode
Tkinter.Frame
\item[{Return type}] \leavevmode
Instance
\end{description}\end{quote}

%******* Termina descripción *******

%******* Empieza función *******
\begin{fulllineitems}

\pysiglinewithargsret{\sphinxbfcode{dynamic\_widgets}}{}{}~

\begin{notice}{note}{Note:}
Este método es privado.
\end{notice}

Inicializa los elementos dinámicos del Frame, esto es, de acuerdo 
al tipo que lleva cada parámetro se creará un widget diferente.

\end{fulllineitems}
%******* Termina función *******

%******* Empieza función *******
\begin{fulllineitems}

\pysiglinewithargsret{\sphinxbfcode{update\_widgets}}{\emph{event=None}}{}~

\begin{notice}{note}{Note:}
Este método es privado.
\end{notice}

Realiza solamente la actualización y colocación de elementos dinámicos 
en el Frame.\break
Si el parámetro event es distinto de \textbf{None}, significa que se 
lanzó un evento que provocará que se actualicen los parámetros de 
acuerdo con la técnica seleccionada.

\begin{quote}\begin{description}
\item[{Parameters}] \leavevmode\begin{itemize}
\item \textbf{\texttt{event}} (\emph{\texttt{String}}) -- Contiene el valor del elemento que ejecutó esta función.
\end{itemize}
\end{description}\end{quote}

\end{fulllineitems}
%******* Termina función *******

%******* Empieza función *******
\begin{fulllineitems}

\pysiglinewithargsret{\sphinxbfcode{get\_information}}{}{}
Recolecta la información que ha seleccionado e introducido 
el usuario, también la organiza para que se pueda utilizar 
apropiadamente.

\begin{quote}\begin{description}
\item[{Returns}] \leavevmode

Un diccionario que contiene:

\begin{itemize}
\item \textbf{Clase},
\item \textbf{Técnica},
\item \textbf{Parámetros}
\end{itemize}

\item[{Return type}] \leavevmode
Dictionary
\end{description}\end{quote}

\end{fulllineitems}
%******* Termina función *******

%******* Empieza función *******
\begin{fulllineitems}

\pysiglinewithargsret{\sphinxbfcode{grid\_widgets}}{}{}
Permite la colocación adecuada de elementos estáticos y 
dinámicos, considerando además el espacio o características 
necesarias de redimensionamiento para éstos últimos.

\end{fulllineitems}
%******* Termina función *******

%******* Empieza función *******
\begin{fulllineitems}

\pysiglinewithargsret{\sphinxbfcode{restore\_settings}}{}{}
Asigna los valores por defecto tanto de las técnicas como 
de sus respectivos parámetros, también limpia aquéllos en 
donde se hayan insertado valores.

\end{fulllineitems}
%******* Termina función *******

\end{fulllineitems}
%******* Termina clase *******

Las clases que implementan esta plantilla son las siguientes:

%******* Empieza clase *******
\subparagraph{SelectionFrame (clase)}
%Se coloca el vínculo interno procedente de esta misma sección (a_3_2_5_2_1).
\label{sec:a_3_2_5_2_1}
%******* Empieza descripción *******
\begin{fulllineitems}

\begin{DUlineblock}{0em}
\item[] Esta clase proporciona la infraestructura gráfica para que 
el usuario pueda elegir métodos y características relacionadas con 
la selección de Individuos.\break
También hereda atributos de la clase TemplateGeneticOperatorFrame 
para facilitar la carga de elementos en el Frame y su correspondiente 
recolección de información.
\end{DUlineblock}

\begin{quote}\begin{description}
\item[{Parameters}] \leavevmode\begin{itemize}
\item \textbf{\texttt{parent}} (\emph{\texttt{Tkinter.Frame}}) -- Frame padre al que pertenece.
\item \textbf{\texttt{name}} (\emph{\texttt{String}}) -- Identificador \textbf{(único)} que tendrá el Frame.
\item \textbf{\texttt{features}} (\emph{\texttt{Dictionary}}) -- Conjunto de técnicas con sus respectivos parámetros para que se puedan cargar automáticamente en este Frame \textbf{(véase Controller/XMLParser.py)}.
\end{itemize}

\item[{Returns}] \leavevmode
Tkinter.Frame
\item[{Return type}] \leavevmode
Instance
\end{description}\end{quote}

%******* Termina descripción *******

%******* Empieza función *******
\begin{fulllineitems}

\pysiglinewithargsret{\sphinxbfcode{get\_information}}{}{}
Recolecta la información relativa a esta clase haciendo 
uso del método de la clase Padre.

\begin{quote}\begin{description}
\item[{Returns}] \leavevmode
Diccionario con información de los métodos genéricos.
\item[{Return type}] \leavevmode
Dictionary
\end{description}\end{quote}

\end{fulllineitems}
%******* Termina función *******

%******* Empieza función *******
\begin{fulllineitems}

\pysiglinewithargsret{\sphinxbfcode{restore\_settings}}{}{}
Ejecuta el método de la clase Padre, el cual restaura los 
valores por defecto de los elementos dinámicos y estáticos 
del Frame.

\end{fulllineitems}
%******* Terminas función *******

\end{fulllineitems}
%******* Termina clase *******

%******* Empieza clase *******
\subparagraph{CrossoverFrame (clase)}
%Se coloca el vínculo interno procedente de esta misma sección (a_3_2_5_2_2).
\label{sec:a_3_2_5_2_2}
%******* Empieza descripción *******
\begin{fulllineitems}

\begin{DUlineblock}{0em}
\item[] Esta clase proporciona la infraestructura gráfica para que 
el usuario pueda elegir técnicas y características concernientes a 
la Cruza entre Individuos.\break
También hereda atributos de la clase TemplateGeneticOperatorFrame 
para facilitar la carga de elementos en el Frame y su correspondiente 
recolección de información.
\end{DUlineblock}

\begin{quote}\begin{description}
\item[{Parameters}] \leavevmode\begin{itemize}
\item \textbf{\texttt{parent}} (\emph{\texttt{Tkinter.Frame}}) -- Frame padre al que pertenece.
\item \textbf{\texttt{name}} (\emph{\texttt{String}}) -- Identificador \textbf{(único)} que tendrá el Frame.
\item \textbf{\texttt{features}} (\emph{\texttt{Dictionary}}) -- Conjunto de técnicas con sus respectivos parámetros para que se puedan cargar automáticamente en este Frame \textbf{(véase Controller/XMLParser.py)}.
\end{itemize}

\item[{Returns}] \leavevmode
Tkinter.Frame
\item[{Return type}] \leavevmode
Instance
\end{description}\end{quote}

%******* Termina descripción *******

%******* Empieza función *******
\begin{fulllineitems}

\pysiglinewithargsret{\sphinxbfcode{get\_information}}{}{}
Recolecta la información genérica \textbf{(usando el método de la clase Padre)}, 
y también se le añade aquélla recolectada exclusivamente 
en esta clase.

\begin{quote}\begin{description}
\item[{Returns}] \leavevmode

Un diccionario que contiene:

\begin{itemize}
\item \textbf{Métodos genéricos,}
\item \textbf{Probabilidad de cruza.}
\end{itemize}

\item[{Return type}] \leavevmode
Dictionary
\end{description}\end{quote}

\end{fulllineitems}
%******* Termina función *******

%******* Empieza función *******
\begin{fulllineitems}

\pysiglinewithargsret{\sphinxbfcode{restore\_settings}}{}{}
Ejecuta el método de la clase Padre, el cual restaura los 
valores por defecto de los elementos dinámicos y estáticos 
del Frame.

\end{fulllineitems}
%******* Termina función *******

\end{fulllineitems}
%******* Termina clase *******

%******* Empieza clase *******
\subparagraph{MutationFrame (clase)}
%Se coloca el vínculo interno procedente de esta misma sección (a_3_2_5_2_3).
\label{sec:a_3_2_5_2_3}
%******* Empieza descripción *******
\begin{fulllineitems}

\begin{DUlineblock}{0em}
\item[] Esta clase proporciona la infraestructura gráfica para que el 
usuario pueda elegir técnicas y características relativas a la Mutación 
de Individuos.\break
También hereda atributos de la clase TemplateGeneticOperatorFrame 
para facilitar la carga automática de elementos en el Frame y su 
consecuente recolección de información.
\end{DUlineblock}

\begin{quote}\begin{description}
\item[{Parameters}] \leavevmode\begin{itemize}
\item \textbf{\texttt{parent}} (\emph{\texttt{Tkinter.Frame}}) -- Frame padre al que pertenece.
\item \textbf{\texttt{name}} (\emph{\texttt{String}}) -- Identificador \textbf{(único)} que tendrá el Frame.
\item \textbf{\texttt{features}} (\emph{\texttt{Dictionary}}) -- Conjunto de técnicas con sus respectivos parámetros para que se puedan cargar automáticamente en este Frame \textbf{(véase Controller/XMLParser.py)}.
\end{itemize}

\item[{Returns}] \leavevmode
Tkinter.Frame
\item[{Return type}] \leavevmode
Instance
\end{description}\end{quote}

%******* Termina descripción *******

%******* Empieza función *******
\begin{fulllineitems}

\pysiglinewithargsret{\sphinxbfcode{get\_information}}{}{}
Recolecta la información genérica \textbf{(usando el método de la clase Padre)}, 
y también se le añade aquélla recolectada exclusivamente 
en esta clase.

\begin{quote}\begin{description}
\item[{Returns}] \leavevmode

Un diccionario que contiene:

\begin{itemize}
\item \textbf{Métodos genéricos,}
\item \textbf{Probabilidad de mutación.}
\end{itemize}

\item[{Return type}] \leavevmode
Dictionary
\end{description}\end{quote}

\end{fulllineitems}
%******* Termina función *******

%******* Empieza función *******
\begin{fulllineitems}

\pysiglinewithargsret{\sphinxbfcode{restore\_settings}}{}{}
Ejecuta el método de la clase Padre, el cual restaura 
los valores por defecto de los elementos dinámicos y 
estáticos del Frame.

\end{fulllineitems}
%******* Termina función *******

\end{fulllineitems}
%******* Termina clase *******
%******* Termina módulo *******

%******* Empieza módulo *******
\subsubsection{MOEA (módulo)}
%Se coloca el vínculo interno procedente de esta misma sección (a_3_2_6).
\label{sec:a_3_2_6}
Proporciona los elementos gráficos para que el usuario 
realice configuraciones concernientes a los M.O.E.A.s \textbf{(Multi-Objective Evolutionary Algorithms
ó Algoritmos Evolutivos Multiobjetivo)} y sus atributos 
relacionados.\break
Sus elementos son los siguientes:

%******* Empieza clase *******
\paragraph{MOEAFrame (clase)}
%Se coloca el vínculo interno procedente de esta misma sección (a_3_2_6_1).
\label{sec:a_3_2_6_1}
%******* Empieza descripción *******
\begin{fulllineitems}

\begin{DUlineblock}{0em}
\item[] Unifica los Frames AlgorithmFrame y SharingFunctionFrame, 
la razón de ésto es para facilitar el acomodo de componentes 
de manera individual, para así garantizar un acceso asequible 
a la información.
\end{DUlineblock}

\begin{quote}\begin{description}
\item[{Parameters}] \leavevmode\begin{itemize}
\item \textbf{\texttt{parent}} (\emph{\texttt{Tkinter.Frame}}) -- Frame padre al que pertenece.
\item \textbf{\texttt{features}} (\emph{\texttt{Dictionary}}) -- Conjunto de técnicas con sus respectivos parámetros para que se puedan cargar automáticamente en este Frame \textbf{(véase Controller/XMLParser.py)}.
\end{itemize}

\item[{Returns}] \leavevmode
Tkinter.Frame
\item[{Return type}] \leavevmode
Instance
\end{description}\end{quote}

%******* Termina descripción *******

%******* Empieza función *******
\begin{fulllineitems}

\pysiglinewithargsret{\sphinxbfcode{get\_information}}{}{}
Toma la información solicitada en cada Frame y después
la unifica para regresar un sólo conjunto de información.

\begin{quote}\begin{description}
\item[{Returns}] \leavevmode
Un diccionario con la información de AlgorithmFrame y SharingFunctionFrame.
\item[{Return type}] \leavevmode
Dictionary
\end{description}\end{quote}
\end{fulllineitems}
%******* Termina función *******

%******* Empieza función *******
\begin{fulllineitems}

\pysiglinewithargsret{\sphinxbfcode{restore\_settings}}{}{}
Restaura los valores por defecto en cada Frame.

\end{fulllineitems}
%******* Termina función *******

\end{fulllineitems}
%******* Termina clase *******

La clase actual se apoya de los siguientes elementos:

%******* Empieza clase *******
\subparagraph{AlgorithmFrame (clase)}
%Se coloca el vínculo interno procedente de esta misma sección (a_3_2_6_1_1).
\label{sec:a_3_2_6_1_1}
%******* Empieza descripción *******
\begin{fulllineitems}

\begin{DUlineblock}{0em}
\item[] Esta clase proporciona una base gráfica para que el 
usuario pueda seleccionar técnicas con sus parámetros 
correspondientes \textbf{(si es que tienen)} referentes a los 
M.O.E.A.'s \textbf{(Multi-Objective Evolutionary Algorithms ó Algoritmos Evolutivos Multiobjetivo)}.
\end{DUlineblock}

\begin{quote}\begin{description}
\item[{Parameters}] \leavevmode\begin{itemize}
\item \textbf{\texttt{parent}} (\emph{\texttt{Tkinter.Frame}}) -- Frame padre al que pertenece.
\item \textbf{\texttt{name}} (\emph{\texttt{String}}) -- Identificador \textbf{(único)} que tendrá el Frame.
\item \textbf{\texttt{features}} (\emph{\texttt{Dictionary}}) -- Conjunto de técnicas con sus respectivos parámetros para que se puedan cargar automáticamente en este Frame \textbf{(véase Controller/XMLParser.py)}.
\end{itemize}

\item[{Returns}] \leavevmode
Tkinter.Frame
\item[{Return type}] \leavevmode
Instance
\end{description}\end{quote}

%******* Termina descripción *******

%******* Empieza función *******
\begin{fulllineitems}

\pysiglinewithargsret{\sphinxbfcode{create\_dynamic\_widgets}}{}{}~

\begin{notice}{note}{Note:}
Este método es privado.
\end{notice}

Inicializa los elementos dinámicos del Frame, esto es, de acuerdo 
al tipo que lleva cada parámetro se creará un widget diferente.

\end{fulllineitems}
%******* Termina función *******

%******* Empieza función *******
\begin{fulllineitems}

\pysiglinewithargsret{\sphinxbfcode{grid\_widgets}}{}{}~

\begin{notice}{note}{Note:}
Este método es privado.
\end{notice}

Coloca elementos en el Frame, tanto estáticos como dinámicos.

\end{fulllineitems}
%******* Termina función *******

%******* Empieza función *******
\begin{fulllineitems}

\pysiglinewithargsret{\sphinxbfcode{update\_widgets}}{\emph{event=None}}{}~

\begin{notice}{note}{Note:}
Este método es privado.
\end{notice}

Realiza solamente la actualización y colocación de elementos dinámicos 
en el Frame.\break
Si el parámetro event es distinto de \textbf{None}, significa que se 
lanzó un evento que provocará que se actualicen los parámetros de 
acuerdo con la técnica seleccionada.

\begin{quote}\begin{description}
\item[{Parameters}] \leavevmode\begin{itemize}
\item \textbf{\texttt{event}} (\emph{\texttt{String}}) -- Contiene el valor del elemento que ejecutó esta función.
\end{itemize}
\end{description}\end{quote}

\end{fulllineitems}
%******* Termina función *******

%******* Empieza función *******
\begin{fulllineitems}

\pysiglinewithargsret{\sphinxbfcode{get\_information}}{}{}
Recolecta la información que ha seleccionado e introducido 
el usuario, también la organiza para que se pueda utilizar 
apropiadamente.

\begin{quote}\begin{description}
\item[{Returns}] \leavevmode

Un diccionario que contiene:

\begin{itemize}
\item \textbf{Clase},
\item \textbf{Técnica},
\item \textbf{Parámetros.}
\end{itemize}

\item[{Return type}] \leavevmode
Dictionary
\end{description}\end{quote}

\end{fulllineitems}
%******* Termina función *******

%******* Empieza función *******
\begin{fulllineitems}

\pysiglinewithargsret{\sphinxbfcode{restore\_settings}}{}{}
Asigna los valores por defecto tanto de las técnicas como 
de sus respectivos parámetros, también limpia aquéllos en 
donde se hayan insertado valores.

\end{fulllineitems}
%******* Termina función *******

\end{fulllineitems}
%******* Termina clase *******

%******* Empieza clase *******
\subparagraph{SharingFunctionFrame (clase)}
%Se coloca el vínculo interno procedente de esta misma sección (a_3_2_6_1_2).
\label{sec:a_3_2_6_1_2}
%******* Empieza descripción *******
\begin{fulllineitems}

\begin{DUlineblock}{0em}
\item[] Esta clase proporciona una base gráfica para que el 
usuario pueda seleccionar métodos con sus respectivos parámetros 
\textbf{(si es que tienen)} referentes a Sharing Function.\break         
Una técnica de Sharing Function sirve para aplicar una selección 
más intensiva de Individuos en caso de haber un ``empate'' entre 
éstos.
\end{DUlineblock}

\begin{quote}\begin{description}
\item[{Parameters}] \leavevmode\begin{itemize}
\item \textbf{\texttt{parent}} (\emph{\texttt{Tkinter.Frame}}) -- Frame padre al que pertenece.
\item \textbf{\texttt{name}} (\emph{\texttt{String}}) -- Identificador \textbf{(único)} que tendrá el Frame.
\item \textbf{\texttt{features}} (\emph{\texttt{Dictionary}}) -- Conjunto de técnicas con sus respectivos parámetros para que se puedan cargar automáticamente en este Frame \textbf{(véase Controller/XMLParser.py)}.
\end{itemize}

\item[{Returns}] \leavevmode
Tkinter.Frame
\item[{Return type}] \leavevmode
Instance
\end{description}\end{quote}

%******* Termina descripción *******
\index{\_SharingFunctionFrame\_\_create\_dynamic\_widgets() (SharingFunctionFrame method)}

%******* Empieza función *******
\begin{fulllineitems}

\pysiglinewithargsret{\sphinxbfcode{create\_dynamic\_widgets}}{}{}~

\begin{notice}{note}{Note:}
Este método es privado.
\end{notice}

Inicializa los elementos dinámicos del Frame, esto es, de 
acuerdo al tipo que lleva cada parámetro se creará un widget 
diferente.

\end{fulllineitems}
%******* Termina función *******

%******* Empieza función *******
\begin{fulllineitems}

\pysiglinewithargsret{\sphinxbfcode{grid\_widgets}}{}{}~

\begin{notice}{note}{Note:}
Este método es privado.
\end{notice}

Coloca elementos en el Frame, tanto estáticos como 
dinámicos.

\end{fulllineitems}
%******* Termina función *******

%******* Empieza función *******
\begin{fulllineitems}

\pysiglinewithargsret{\sphinxbfcode{update\_widgets}}{\emph{event=None}}{}~

\begin{notice}{note}{Note:}
Este método es privado.
\end{notice}

Realiza solamente la actualización y colocación de elementos 
dinámicos en el Frame.\break
Si el parámetro event es distinto de \textbf{None}, significa 
que se lanzó un evento que provocará que se actualicen los 
parámetros de acuerdo con la técnica seleccionada.

\begin{quote}\begin{description}
\item[{Parameters}] \leavevmode\begin{itemize}
\item \textbf{\texttt{event}} (\emph{\texttt{String}}) -- Contiene el valor del elemento que ejecutó esta función.
\end{itemize}
\end{description}\end{quote}

\end{fulllineitems}
%******* Termina función *******

%******* Empieza función *******
\begin{fulllineitems}

\pysiglinewithargsret{\sphinxbfcode{get\_information}}{}{}
Recolecta la información que ha seleccionado e introducido 
el usuario, también la organiza para que se pueda utilizar 
apropiadamente.

\begin{quote}\begin{description}
\item[{Returns}] \leavevmode

Un diccionario que contiene:

\begin{itemize}
\item \textbf{Clase},
\item \textbf{Técnica},
\item \textbf{Parámetros.}
\end{itemize}

\item[{Return type}] \leavevmode
Dictionary
\end{description}\end{quote}

\end{fulllineitems}
%******* Termina función *******

%******* Empieza función *******
\begin{fulllineitems}

\pysiglinewithargsret{\sphinxbfcode{restore\_settings}}{}{}
Asigna los valores por defecto tanto de las técnicas como de sus 
respectivos parámetros, también limpia aquéllos en donde se hayan 
insertado valores.

\end{fulllineitems}
%******* Termina función *******

\end{fulllineitems}
%******* Termina clase *******
%******* Termina módulo *******
%******* Termina módulo *******

%******* Empieza módulo *******
\subsection{Additional (módulo)}
%Se coloca el vínculo interno procedente de esta misma sección (a_3_3).
\label{sec:a_3_3}
Proporciona elementos gráficos que, aunque no tienen cabilda 
en la Ventana Principal, sí contienen herramientas auxiliares 
de importancia.\medskip\break
El módulo consta de los siguientes elementos:

%******* Empieza clase *******
\subsubsection{GenerationSignalToplevel (clase)}
%Se coloca el vínculo interno procedente de esta misma sección (a_3_3_1).
\label{sec:a_3_3_1}
%******* Empieza descripción *******
\begin{fulllineitems}

\begin{DUlineblock}{0em}
\item[] Se trata de un Toplevel \textbf{(ventana independiente)} que 
muestra el progreso de las generaciones al momento de ejecutar un Task.\break
Esta ventana aunque es creada y mostrada en los procesos de la capa View, 
será en \textbf{Model/MOEA} en donde se utilice y actualice, ya que la idea 
es crear una ``señal'' que indique al usuario el progreso del MOEA en 
ejecución para que se dé una idea del desempeño del algoritmo.
\end{DUlineblock}

\begin{quote}\begin{description}
\item[{Parameters}] \leavevmode\begin{itemize}
\item \textbf{\texttt{parent}} (\emph{\texttt{Tkinter.Frame}}) -- Frame padre al que pertenece.
\item \textbf{\texttt{path\_image\_logo}} (\emph{\texttt{String}}) -- La ruta al logotipo que se usa en esta ventana independiente.
\item \textbf{\texttt{execution\_task\_number}} (\emph{\texttt{Integer}}) -- Número que indica el actual Task en ejecución \textbf{(véase View/Additional/}\break\textbf{ResultsGrapher/ResultsGrapherToplevel.py)}.
\end{itemize}

\item[{Returns}] \leavevmode
Tkinter.Toplevel
\item[{Return type}] \leavevmode
Instance
\end{description}\end{quote}

%******* Termina descripción *******

%******* Empieza función *******
\begin{fulllineitems}

\pysiglinewithargsret{\sphinxbfcode{center}}{}{}~

\begin{notice}{note}{Note:}
Este método es privado.
\end{notice}

Centra la ventana independiente con respecto de la 
Ventana Principal.\break
En otras palabras, la ventana independiente será 
colocada en el centro de la Ventana Principal.

\end{fulllineitems}
%******* Termina  función *******

%******* Empieza función *******
\begin{fulllineitems}

\pysiglinewithargsret{\sphinxbfcode{do\_nothing}}{}{}~

\begin{notice}{note}{Note:}
Este método es privado.
\end{notice}

Simplemente es una función ``dummy'' que no realiza 
nada.\break
Es utilizada como sustituto de la función del ícono 
``Cerrar'' y así evitar que el usario intencionadamente 
intente ocluir la ventana del número de generaciones.

\end{fulllineitems}
%******* Termina función *******

%******* Empieza función *******
\begin{fulllineitems}

\pysiglinewithargsret{\sphinxbfcode{close}}{}{}
Oculta y elimina toda referencia gráfica y lógica de la 
ventana independiente, indicando así que el número de 
generaciones ha alcanzado su límite.

\end{fulllineitems}
%******* Termina función *******

%******* Empieza función *******
\begin{fulllineitems}

\pysiglinewithargsret{\sphinxbfcode{hide}}{}{}
Oculta la ventana independiente de la pantalla pero no 
la elimina de los registros gráficos.

\end{fulllineitems}
%******* Termina función *******

%******* Empieza función *******
\begin{fulllineitems}

\pysiglinewithargsret{\sphinxbfcode{show}}{}{}
Reactiva la ventana independiente, realizando 
además durante esta ejecución un par de consignas 
más para dar una experiencia de usuario suficiente 
y concisa.

\end{fulllineitems}
%******* Termina función *******

%******* Empieza función *******
\begin{fulllineitems}

\pysiglinewithargsret{\sphinxbfcode{update\_current\_generation}}{\emph{current\_generation}}{}~
\vspace{-0.3cm}

Actualiza la generación actual en la ventana independiente.\break
Típicamente esta función será usada en todos los algoritmos de la 
capa Model/MOEA, pues es allí donde se designará el progreso del 
algoritmo que a su vez se verá reflejado en la capa de View.

\begin{quote}\begin{description}
\item[{Parameters}] \leavevmode\begin{itemize}
\item \textbf{\texttt{current\_generation}} (\emph{\texttt{Integer}}) -- La generación que está corriendo actualmente en el MOEA.s
\end{itemize}

\item[{Returns}] \leavevmode
1 si se  ha alcanzado la generación límite, 0 en otro caso.
\item[{Return type}] \leavevmode
Integer
\end{description}\end{quote}

\end{fulllineitems}
%******* Termina función *******

%******* Empieza función *******
\begin{fulllineitems}

\pysiglinewithargsret{\sphinxbfcode{update\_number\_of\_generations}}{\emph{number\_of\_generations}}{}
Actualiza el número total de generaciones.\break
Generalmente esta función será llamada desde Controller/Controller.py 
ya que ahí es donde se decide si las configuraciones iniciales son 
adecuadas para poder ejecutar el algoritmo.

\begin{quote}\begin{description}
\item[{Parameters}] \leavevmode\begin{itemize}
\item \textbf{\texttt{number\_of\_generations}} (\emph{\texttt{Integer.}}) -- El número de generaciones total que tendrá el MOEA.
\end{itemize}
\end{description}\end{quote}

\end{fulllineitems}
%******* Termina función *******

\end{fulllineitems}
%******* Termina clase  *******
%******* Termina módulo *******

%******* Empieza módulo *******
\subsection{MenuInternalOption (módulo)}
%Se coloca el vínculo interno procedente de esta misma sección (a_3_3_2).
\label{sec:a_3_3_2}
Contiene elementos gráficos que permiten acceder a configuraciones 
internas del programa y también a M.O.P.s \textbf{(Multi-Objective Problems)} 
previamente cargados para hacer uso fácil de ellos.

%******* Empieza clase *******
\subsubsection{MenuInternalOption (clase)}
%Se coloca el vínculo interno procedente de esta misma sección (a_3_3_2_1).
\label{sec:a_3_3_2_1}
%******* Empieza descripción *******
\begin{fulllineitems}

\begin{DUlineblock}{0em}
\item[] Se crea el Menú de Opciones Internas o Menú Secundario.\break
Básicamente se trata de una serie de características que, aunque no
forman parte esencial del programa, sí ofrecen alternativas que 
pueden facilitar la experiencia de usuario.\break
Este menú será atado al Frame Principal y desde allí el usuario podrá
tener acceso a las opciones que aquí se describen.
\end{DUlineblock}

\begin{quote}\begin{description}
\item[{Parameters}] \leavevmode\begin{itemize}
\item \textbf{\texttt{parent}} (\emph{\texttt{Tkinter.Frame}}) -- El Frame Padre al que pertenece esta implementación.
\item \textbf{\texttt{path\_image\_logo}} (\emph{\texttt{String}}) -- La ruta al logotipo que se usa en esta ventana independiente.
\item \textbf{\texttt{features}} (\emph{\texttt{Dictionary}}) -- Un diccionario con las características que deberá tener cada una de las opciones listadas.
\end{itemize}

\item[{Returns}] \leavevmode
Tkinter.Menu
\item[{Return type}] \leavevmode
Instance
\end{description}\end{quote}

%******* Termina descripción *******

%******* Empieza función *******
\begin{fulllineitems}

\pysiglinewithargsret{\sphinxbfcode{launch\_about\_toplevel}}{}{}~

\begin{notice}{note}{Note:}
Este método es privado.
\end{notice}

Abre la ventana independiente \textbf{(Toplevel)} About.
También verifica que se abra una y sólo una instancia de 
dicha ventana.

\end{fulllineitems}
%******* Termina función *******

%******* Empieza función *******
\begin{fulllineitems}

\pysiglinewithargsret{\sphinxbfcode{launch\_internal\_option\_toplevel}}{}{}~

\begin{notice}{note}{Note:}
Este método es privado.
\end{notice}

Abre la ventana independiente \textbf{(Toplevel)} Internal 
Options \textbf{(o simplemente Options)}.\break
También verifica que se abra una y sólo una instancia de 
dicha ventana.

\end{fulllineitems}
%******* Empieza función *******

%******* Empieza función *******
\begin{fulllineitems}

\pysiglinewithargsret{\sphinxbfcode{about\_toplevel\_custom\_close}}{}{}
Indica que la única instancia que debe crearse
para la opción About está disponible.

\end{fulllineitems}
%******* Empieza función *******

%******* Empieza función *******
\begin{fulllineitems}

\pysiglinewithargsret{\sphinxbfcode{internal\_option\_toplevel\_custom\_close}}{}{}
Indica que la única instancia que debe crearse
para la opción Options está disponible.

\end{fulllineitems}
%******* Termina función *******

\end{fulllineitems}
%******* Termina clase *******

El módulo consta de las siguientes características:

%******* Empieza clase *******
\paragraph{InternalOptionToplevel (clase)}
%Se coloca el vínculo interno procedente de esta misma sección (a_3_3_2_2).
\label{sec:a_3_3_2_2}
%******* Empieza descripción *******
\begin{fulllineitems}

\begin{DUlineblock}{0em}
\item[] Contiene un Menú pequeño con pestañas que indican las
características internas del sistema a las que puede tener acceso el 
usuario.\break
En su mayoría se trata de características que muestran los métodos,
técnicas y sistemas auxiliares que garantizan un manejo más armonioso 
del programa y si así lo desea el usuario, modificarlos para ajustar
su desempeño.
\end{DUlineblock}

\begin{quote}\begin{description}
\item[{Parameters}] \leavevmode\begin{itemize}
\item \textbf{\texttt{parent}} (\emph{\texttt{Tkinter.Menu}}) -- El elemento Padre al que pertenece la actual ventana independiente \textbf{(Toplevel)}.
\item \textbf{\texttt{path\_image\_logo}} (\emph{\texttt{String}}) -- La ruta al logotipo que se usa en esta ventana independiente.
\item \textbf{\texttt{features}} (\emph{\texttt{Dictionary}}) -- Un diccionario que contiene las características necesarias que serán mostradas en esta ventana independiente.
\item \textbf{\texttt{custom\_function}} (\emph{\texttt{Instance}}) -- Una variable que contiene una función, la cual redefinirá más apropiadamente el comportamiento de la actual Ventana Principal con respecto de su Frame Padre.
\end{itemize}

\item[{Returns}] \leavevmode
La ventana independiente que contiene la información
\item[{Return type}] \leavevmode
Tkinter.Toplevel
\end{description}\end{quote}

%******* Termina descripción *******

%******* Empieza función *******
\begin{fulllineitems}

\pysiglinewithargsret{\sphinxbfcode{center}}{}{}~
\begin{notice}{note}{Note:}
Este método es privado.
\end{notice}

Centra la ventana independiente con respecto de la 
Ventana Principal.\break
En otras palabras, la ventana independiente será colocada 
en el centro de la Ventana Principal.

\end{fulllineitems}
%******* Termina función *******

%******* Empieza función *******
\begin{fulllineitems}

\pysiglinewithargsret{\sphinxbfcode{close}}{}{}~

\begin{notice}{note}{Note:}
Este método es privado.
\end{notice}

Cierra y elimina todo rastro de esta 
ventana independiente.

\end{fulllineitems}
%******* Termina función *******

\end{fulllineitems}
%******* Termina clase *******

%******* Empieza módulo *******
\paragraph{InternalOptionTab (módulo)}
%Se coloca el vínculo interno procedente de esta misma sección (a_3_3_2_3).
\label{sec:a_3_3_2_3}
\begin{fulllineitems}
\begin{DUlineblock}{0em}
\item[] Contiene las partes gráficas que conformarán cada una de las pestañas
concernientes al Toplevel \textbf{(ventana independiente)} de opciones internas
\textbf{(InternalOptionToplevel)}.\break
Consta de los siguientes elementos:
\end{DUlineblock}
\end{fulllineitems}

%******* Empieza clase *******
\subparagraph{MOPExampleFrame (clase)}
%Se coloca el vínculo interno procedente de esta misma sección (a_3_3_2_3_1).
\label{sec:a_3_3_2_3_1}
%******* Empieza descripción *******
\begin{fulllineitems}

\begin{DUlineblock}{0em}
\item[] Unifica dos elementos: Canvas y MOPFrame.\break
La razón de esto es que, en promedio la información mostrada 
por MOPFrame rebasará el tamaño de la ventana de la información 
final \textbf{(véase View/Additional/}\break\textbf{ResultsGrapher/ResultsGrapherTopLevel.py)}, 
es entonces que se deben agregar barras de desplazamiento para poder acceder 
al contenido que quedaría oculto.\break
Uno de los elementos en Tkinter más sencillos que cumplen con este 
cometido es un Canvas. Luego entonces esa es la razón de tal fusión.
\end{DUlineblock}

\begin{quote}\begin{description}
\item[{Parameters}] \leavevmode\begin{itemize}
\item \textbf{\texttt{parent}} (\emph{\texttt{Tkinter.Toplevel}}) -- El elemento Padre al que pertenece el actual
Frame.
\item \textbf{\texttt{features}} (\emph{\texttt{Dictionary}}) -- Un diccionario que contiene las características necesarias que serán mostradas en este Frame.
\end{itemize}

\item[{Returns}] \leavevmode
El Frame que contiene la información señalada.
\item[{Return type}] \leavevmode
Tkinter.Frame
\end{description}\end{quote}

%******* Termina descripción *******

%******* Empieza función *******
\begin{fulllineitems}

\pysiglinewithargsret{\sphinxbfcode{update\_scrollbar}}{\emph{event}}{}~

\begin{notice}{note}{Note:}
Este método es privado.
\end{notice}

Actualiza la barra de desplazamiento de acuerdo al 
número de elementos existentes en el Frame, esto 
para poder hacer un recorrido apropiado de la barra.

\begin{quote}\begin{description}
\item[{Parameters}] \leavevmode\begin{itemize}
\item\textbf{\texttt{event}} (\emph{\texttt{String}}) -- Elemento que ejecutó esta función.
\end{itemize}
\end{description}\end{quote}

\end{fulllineitems}
%******* Termina función *******

\end{fulllineitems}
%******* Termina clase *******

La clase actual tiene como base el siguiente elemento:

%******* Empieza clase *******
\subparagraph{MOPFrame (clase)}
%Se coloca el vínculo interno procedente de esta misma sección (a_3_3_2_3_1_1).
\label{sec:a_3_3_2_3_1_1}
%******* Empieza descripción *******
\begin{fulllineitems}

\begin{DUlineblock}{0em}
\item[] Muestra la información relativa a los M.O.P.'s y
provee de métodos que facilitan la carga de éstos en la
Ventana Principal.\break
Un M.O.P. \textbf{(Multi Objective Problem)} es un conjunto 
de funciones y variables bien definidas que ya han sido previamente 
estudiadas, así como su comportamiento en conjunto; la idea es 
proporcionarle al usuario un ambiente de carga fácil de datos 
para que pueda probar los ejemplos ya tratados por muchos 
autores en los libros que se citarán en el trabajo escrito.
\end{DUlineblock}

\begin{quote}\begin{description}
\item[{Parameters}] \leavevmode\begin{itemize}
\item \textbf{\texttt{parent}} (\emph{\texttt{Tkinter.Frame}}) -- El elemento Padre al que pertenece el actual Frame.
\item \textbf{\texttt{grandparent}} (\emph{\texttt{Tkinter.Toplevel}}) -- El elemento Padre del Padre al que pertenece el actual Frame.
\item \textbf{\texttt{features}} (\emph{\texttt{Dictionary}}) -- Un diccionario que contiene las características necesarias que serán mostradas en este Frame.
\end{itemize}

\item[{Returns}] \leavevmode
El Frame que contiene la información señalada.
\item[{Return type}] \leavevmode
Tkinter.Frame
\end{description}\end{quote}

%******* Termina descripción *******

%******* Empieza función *******
\begin{fulllineitems}

\pysiglinewithargsret{\sphinxbfcode{get\_mop\_example}}{\emph{event}}{}~

\begin{notice}{note}{Note:}
Este método es privado.
\end{notice}

Con base en la selección de M.O.P.
hecha por el usuario, se carga éste
en la Ventana Principal.

\begin{quote}\begin{description}
\item[{Parameters}] \leavevmode\begin{itemize}
\item\textbf{\texttt{event}} (\emph{\texttt{String}}) -- El evento del elemento gráfico que
activa esta función.
\end{itemize}
\end{description}\end{quote}

\end{fulllineitems}
%******* Termina función *******

%******* Empieza función *******
\begin{fulllineitems}

\pysiglinewithargsret{\sphinxbfcode{update\_current\_mop}}{\emph{event=None}}{}~

\begin{notice}{note}{Note:}
Este método es privado.
\end{notice}

Despliega la información relacionada con el
M.O.P. seleccionado.

\begin{quote}\begin{description}
\item[{Parameters}] \leavevmode\begin{itemize}
\item \textbf{\texttt{event}} (\emph{\texttt{String}}) -- El evento del elemento gráfico que
activa esta función.
\end{itemize}
\end{description}\end{quote}

\end{fulllineitems}
%******* Termina función *******

\end{fulllineitems}
%******* Termina clase *******

%******* Empieza clase *******
\subparagraph{FeatureFrame (clase)}
%Se coloca el vínculo interno procedente de esta misma sección (a_3_3_2_3_2).
\label{sec:a_3_3_2_3_2}
%******* Empieza descripción *******
\begin{fulllineitems}

\begin{DUlineblock}{0em}
\item[] Unifica dos elementos: Canvas y CharacteristicFrame.\break
La razón de esto es que, en promedio la información mostrada 
por CharacteristicFrame rebasará el tamaño de la ventana de la 
información final \textbf{(véase View/}\break\textbf{Additional/ResultsGrapher/ResultsGrapherTopLevel.py)}, es 
entonces que se deben agregar barras de desplazamiento para poder 
acceder al contenido que quedaría oculto.\break
Uno de los elementos en Tkinter más sencillos que cumplen con 
este cometido es un Canvas. Luego entonces esa es la razón de 
tal fusión.
\end{DUlineblock}

\begin{quote}\begin{description}
\item[{Parameters}] \leavevmode\begin{itemize}
\item \textbf{\texttt{parent}} (\emph{\texttt{Tkinter.Toplevel}}) -- El elemento Padre al que pertenece el actual Frame.
\item \textbf{\texttt{features}} (\emph{\texttt{Dictionary}}) -- Un diccionario que contiene las características necesarias que serán mostradas en este Frame.
\end{itemize}

\item[{Returns}] \leavevmode
El Frame que contiene la información señalada.
\item[{Return type}] \leavevmode
Tkinter.Frame
\end{description}\end{quote}

%******* Termina descripción *******

%******* Empieza función *******
\begin{fulllineitems}

\pysiglinewithargsret{\sphinxbfcode{update\_scrollbar}}{\emph{event}}{}~

\begin{notice}{note}{Note:}
Este método es privado.
\end{notice}

Actualiza la barra de desplazamiento de acuerdo al 
número de elementos existentes en el Frame, esto 
para poder hacer un recorrido apropiado de la barra.
\begin{quote}\begin{description}
\item[{Parameters}] \leavevmode\begin{itemize}
\item \textbf{\texttt{event}} (\emph{\texttt{String}}) -- Elemento que ejecutó esta función.
\end{itemize}
\end{description}\end{quote}

\end{fulllineitems}
%******* Termina función *******

\end{fulllineitems}
%******* Termina clase *******

La clase actual se apoya del siguiente elemento:

%******* Empieza clase *******
\subparagraph{CharacteristicFrame (clase)}
%Se coloca el vínculo interno procedente de esta misma sección (a_3_3_2_3_2_1).
\label{sec:a_3_3_2_3_2_1}
%******* Empieza descripción *******
\begin{fulllineitems}


\begin{DUlineblock}{0em}
\item[] Despliega información concerniente a todas las técnicas 
\textbf{(con sus respectivos parámetros)} disponibles para el usuario.\break
Se agrupan éstas en las mismas categorías que presenta el programa,
más en concreto, las secciones que conforman a la Ventana Principal 
\textbf{(véase View/}\break\textbf{MainWindow.py)}.\break
También señala someramente las instrucciones necesarias para que el
programa pueda reconocer cualquier técnica que desarrolle el usuario.
\end{DUlineblock}

\begin{quote}\begin{description}
\item[{Parameters}] \leavevmode\begin{itemize}
\item \textbf{\texttt{parent}} (\emph{\texttt{Tkinter.Toplevel}}) -- El elemento Padre al que pertenece el actual Frame.
\item \textbf{\texttt{features}} (\emph{\texttt{Dictionary}}) -- Un diccionario que contiene las características necesarias que serán mostradas en este Frame.
\end{itemize}

\item[{Returns}] \leavevmode
El Frame que contiene la información señalada.
\item[{Return type}] \leavevmode
Tkinter.Frame
\end{description}\end{quote}

%******* Termina descripción *******

\end{fulllineitems}
%******* Termina clase *******

%******* Empieza clase *******
\subparagraph{PythonExpressionFrame (clase)}
%Se coloca el vínculo interno procedente de esta misma sección (a_3_3_2_3_3).
\label{sec:a_3_3_2_3_3}
%******* Empieza descripción *******
\begin{fulllineitems}

\begin{DUlineblock}{0em}
\item[] Realiza la fusión de Canvas y ExpressionFrame, debido a 
que, cuando se agregan numerosas variables al ExpressionFrame, 
se debe insertar una barra de desplazamiento para poder acceder 
a aquéllos que se encuentren hasta abajo.\break
Dentro del ambiente de Tkinter, el elemento más sencillo para 
lograr este efecto es un Canvas, por ello se anida el ExpressionFrame 
al Canvas.
\end{DUlineblock}

\begin{quote}\begin{description}
\item[{Parameters}] \leavevmode\begin{itemize}
\item \textbf{\texttt{parent}} (\emph{\texttt{Tkinter.Frame}}) -- Frame padre al que pertenece.
\item \textbf{\texttt{features}} (\emph{\texttt{Dictionary}}) -- Conjunto de técnicas con sus respectivos parámetros para que se puedan cargar automáticamente en este Frame \textbf{(véase Controller/XML/}\break\textbf{PythonExpressions.xml)}.
\end{itemize}

\item[{Returns}] \leavevmode
Tkinter.Frame
\item[{Return type}] \leavevmode
Instance
\end{description}\end{quote}

%******* Termina descripción *******

%******* Empieza función *******
\begin{fulllineitems}

\pysiglinewithargsret{\sphinxbfcode{activate\_scroll}}{\emph{event=None}}{}~

\begin{notice}{note}{Note:}
Este método es privado.
\end{notice}

Actualiza la barra de desplazamiento y con base en esta 
acción la activa o desactiva.

\begin{quote}\begin{description}
\item[{Parameters}] \leavevmode\begin{itemize}
\item \textbf{\texttt{event}} (\emph{\texttt{String}}) -- Elemento que ejecutó esta función.
\end{itemize}
\end{description}\end{quote}

\end{fulllineitems}
%******* Termina función *******

%******* Empieza función *******
\begin{fulllineitems}

\pysiglinewithargsret{\sphinxbfcode{update\_scrollbar}}{\emph{event=None}}{}~

\begin{notice}{note}{Note:}
Este método es privado.
\end{notice}

Actualiza la barra de desplazamiento de acuerdo al 
número de elementos existentes en el Frame, esto 
para poder hacer un recorrido apropiado de la barra.

\begin{quote}\begin{description}
\item[{Parameters}] \leavevmode\begin{itemize}
\item \textbf{\texttt{event}} (\emph{\texttt{String}}) -- Elemento que ejecutó esta función.
\end{itemize}
\end{description}\end{quote}

\end{fulllineitems}
%******* Termina función *******

\end{fulllineitems}
%******* Termina clase *******

La clase actual toma como referencia el siguiente elemento:

%******* Empieza clase *******
\subparagraph{ExpressionFrame (clase)}
%Se coloca el vínculo interno procedente de esta misma sección (a_3_3_2_3_3_1).
\label{sec:a_3_3_2_3_3_1}
%******* Empieza descripción *******
\begin{fulllineitems}

\begin{DUlineblock}{0em}
\item[] Ofrece opciones simples para mostrar y añadir expresiones
de Python.\break 
Lo anterior ocurre ya que al momento de crear y evaluar funciones 
objetivo hay algunas palabras reservadas que no pueden ser usadas 
en Python directamente si no se hace un renombramiento apropiado.\break
Dicha información se encuentra en \textbf{Controller/XML/PythonExpressions.xml}
\end{DUlineblock}

\begin{quote}\begin{description}
\item[{Parameters}] \leavevmode\begin{itemize}
\item \textbf{\texttt{parent}} (\emph{\texttt{Tkinter.Toplevel}}) -- El elemento Padre al que pertenece el actual Frame.
\item \textbf{\texttt{features}} (\emph{\texttt{Dictionary}}) -- Un diccionario que contiene las características necesarias que serán mostradas en este Frame.
\end{itemize}

\item[{Returns}] \leavevmode
El Frame que contiene la información señalada.
\item[{Return type}] \leavevmode
Tkinter.Frame
\end{description}\end{quote}

%******* Termina descripción *******

%******* Empieza función *******
\begin{fulllineitems}

\pysiglinewithargsret{\sphinxbfcode{add\_expression}}{\emph{event}}{}~

\begin{notice}{note}{Note:}
Este método es privado.
\end{notice}

Inserta una casilla que conforma una expresión
dentro del Frame.

\begin{quote}\begin{description}
\item[{Parameters}] \leavevmode\begin{itemize}
\item \textbf{\texttt{event}} (\emph{\texttt{String}}) -- Identificador del elemento gráfico que activó la función.
\end{itemize}
\end{description}\end{quote}

\end{fulllineitems}
%******* Termina función *******

%******* Empieza función *******
\begin{fulllineitems}

\pysiglinewithargsret{\sphinxbfcode{delete\_single\_expression}}{\emph{event}}{}~

\begin{notice}{note}{Note:}
Este método es privado.
\end{notice}

Elimina una expresión y todos los elementos 
gráficos que la acompañan.\break
También elimina todo rastro que se encuentre 
en las estructuras lógicas.

\begin{quote}\begin{description}
\item[{Parameters}] \leavevmode\begin{itemize}
\item \textbf{\texttt{event}} (\emph{\texttt{String}}) -- Identificador del elemento gráfico que activó la función.
\end{itemize}
\end{description}\end{quote}

\end{fulllineitems}
%******* Termina función *******

%******* Empieza función *******
\begin{fulllineitems}

\pysiglinewithargsret{\sphinxbfcode{get\_information}}{}{}~

\begin{notice}{note}{Note:}
Este método es privado.
\end{notice}

Toma la información del Frame \textbf{(en específico de las casillas)} 
y regresa las expresiones con sus respectivos equivalentes en Python.

\begin{quote}\begin{description}
\item[{Returns}] \leavevmode
Una lista que contiene arreglos de dos elementos donde el primero es la expresión normal mientras que el segundo es la expresión equivalente en Python.
\item[{Return type}] \leavevmode
List
\end{description}\end{quote}

\end{fulllineitems}
%******* Termina función *******

%******* Empieza función *******
\begin{fulllineitems}

\pysiglinewithargsret{\sphinxbfcode{insert\_expression}}{\emph{expression=None}}{}~

\begin{notice}{note}{Note:}
Este método es privado.
\end{notice}

Coloca en el Frame una colección de elementos:\break
{[}etiqueta para expresión normal, expresión normal, etiqueta para expresión de Pyrhon, expresión de Python, botón para eliminar{]}\break
Si el parámetro expression es \textbf{None}, se añade 
la casilla vacía, de lo contrario se agrega ésta 
con la información pertinente.\break

\begin{quote}\begin{description}
\item[{Parameters}] \leavevmode\begin{itemize}
\item \textbf{\texttt{expression}} (\emph{\texttt{Array}}) -- Un arreglo con dos elementos, el primero contiene la expresión normal mientras que el segundo maneja la información de la expresión equivalente en Python.
\end{itemize}
\end{description}\end{quote}

\end{fulllineitems}
%******* Termina función *******

%******* Empieza función *******
\begin{fulllineitems}

\pysiglinewithargsret{\sphinxbfcode{load\_expressions}}{}{}~

\begin{notice}{note}{Note:}
Este método es privado.
\end{notice}

Carga las expresiones a manera de contenido gráfico
en el Frame.\break
Dichas expresiones son tomadas del archivo \textbf{Controller/XML/}\break\textbf{PythonExpressions.xml.}

\end{fulllineitems}
%******* Termina función *******

%******* Empieza función *******
\begin{fulllineitems}

\pysiglinewithargsret{\sphinxbfcode{save\_changes}}{\emph{event}}{}~

\begin{notice}{note}{Note:}
Este método es privado.
\end{notice}

Toma la información existente en las casillas y procede a 
sobreescribir el archivo \textbf{Controller/XML/PythonExpressions.xml} 
con la información recién recabada.

\begin{quote}\begin{description}
\item[{Parameters}] \leavevmode\begin{itemize}
\item \textbf{\texttt{event}} (\emph{\texttt{String}}) -- Identificador del elemento gráfico que activó la función.
\end{itemize}
\end{description}\end{quote}

\end{fulllineitems}
%******* Termina función *******

%******* Empieza función *******
\begin{fulllineitems}

\pysiglinewithargsret{\sphinxbfcode{get\_current\_elements}}{}{}
Regresa el número actual de casillas en el Frame.

\begin{quote}\begin{description}
\item[{Returns}] \leavevmode
Cantidad de elementos en la estructura rows, donde se guardan las casillas \textbf{(Entry's)}.
\item[{Return type}] \leavevmode
Integer
\end{description}\end{quote}

\end{fulllineitems}
%******* Termina función *******

\end{fulllineitems}
%******* Termina clase *******

%******* Empieza clase *******
\paragraph{AboutToplevel (clase)}
%Se coloca el vínculo interno procedente de esta misma sección (a_3_3_2_4).
\label{sec:a_3_3_2_4}
%******* Empieza descripción *******
\begin{fulllineitems}

\begin{DUlineblock}{0em}
\item[] Esta ventana independiente \textbf{(Toplevel)} 
proporciona información básica del programa así como 
de sus desarrolladores.
\end{DUlineblock}

\begin{quote}\begin{description}
\item[{Parameters}] \leavevmode\begin{itemize}
\item \textbf{\texttt{parent}} (\emph{\texttt{Tkinter.Menu}}) -- El elemento Padre al que pertenece la actual ventana independiente \textbf{(Toplevel)}.
\item \textbf{\texttt{path\_image\_logo}} (\emph{\texttt{String}}) -- La ruta al logotipo que se usa en esta ventana independiente.
\item \textbf{\texttt{custom\_function}} (\emph{\texttt{Instance}}) -- Una variable que contiene una función, la cual redefinirá más apropiadamente el comportamiento de la actual Ventana Principal con respecto de su Frame Padre.
\end{itemize}

\item[{Returns}] \leavevmode
Tkinter.Toplevel
\item[{Return type}] \leavevmode
Instance
\end{description}\end{quote}

%******* Termina descripción *******

%******* Empieza función *******
\begin{fulllineitems}

\pysiglinewithargsret{\sphinxbfcode{center}}{}{}~

\begin{notice}{note}{Note:}
Este método es privado.
\end{notice}

Centra la ventana independiente con respecto de 
la Ventana Principal.\break
En otras palabras, la ventana independiente será 
colocada en el centro de la Ventana Principal.
\end{fulllineitems}
%******* Termina función *******

%******* Empieza función *******
\begin{fulllineitems}

\pysiglinewithargsret{\sphinxbfcode{close}}{\emph{custom\_function}}{}~

\begin{notice}{note}{Note:}
Este método es privado.
\end{notice}

Cierra y elimina todo rastro de esta ventana independiente.

\begin{quote}\begin{description}
\item[{Parameters}] \leavevmode\begin{itemize}
\item \textbf{\texttt{custom\_function}} (\emph{\texttt{Instance}}) -- Una variable que contiene una función que ha de 
ejecutarse dentro de este método.
\end{itemize}
\end{description}\end{quote}

\end{fulllineitems}
%******* Termina función *******

\end{fulllineitems}
%******* Termina clase *******
%******* Termina módulo *******

%******* Empieza módulo *******
\subsubsection{ResultsGrapher (módulo)}
%Se coloca el vínculo interno procedente de esta misma sección (a_3_3_3).
\label{sec:a_3_3_3}
Proporciona los elementos gráficos para poder presentar
las gráficas de los resultados que ha arrojado la ejecución
de algún M.O.E.A.\medskip\break
Consta de los siguientes elementos:

%******* Empieza clase *******
\paragraph{ResultsGrapherToplevel (clase)}
%Se coloca el vínculo interno procedente de esta misma sección (a_3_3_3_1).
\label{sec:a_3_3_3_1}
%******* Empieza descripción *******
\begin{fulllineitems}

\begin{DUlineblock}{0em}
\item[] Esta clase lanza una ventana independiente que muestra los 
resultados arrojados por una configuración previa del usuario.\break
Primero que nada es menester mencionar que una ventana independiente 
es un Toplevel en Tkinter, la cual es casi ajena a la Ventana Principal 
\textbf{(véase View/Main/}\break\textbf{MainWindow.py)}, pero si ésta última es cerrada, 
se eliminarán también las ventanas independientes creadas.\medskip\break
Cada ventana independiente mostrará el número de Task, es decir, el 
orden en el que fue procesada la información con respecto de otros Tasks.\break
Entiéndase por Task a una ejecución de algún algoritmo MOEA bajo un 
cierto conjunto de configuraciones iniciales.\break
Así, los Tasks serán mostrados en una ventana independiente. La numeración 
de los Tasks irá siempre en orden progresivo, lo que significa que el número 
será reinicializado sólamente volviendo a ejecutar el programa principal.\break
De esta manera es posible tener varias ventanas independientes abiertas y 
en cuestiones más generales, es posible ejecutar varios Tasks simultáneamente, 
ya que el programa es multi-threading en ese sentido.\medskip\break
Finalmente, la información será mostrada en dos pestañas: en una 
\textbf{(SummaryFrame)} se otorga un resumen de todas las funciones objetivo, 
variables de decisión, MOEA usado y configuraciones adicionales en el Task.\break
En la otra \textbf{(GraphFrame)} se colocan todas las gráficas pertinentes 
producto de la ejecución del MOEA con las funciones objectivo, variables de 
decisión y configuraciones ingresadas \textbf{(véase Model/Community/Community.py)} \textbf{(véase View/}\break\textbf{Additional/ResultsGrapher/GraphFrame.py)}.\break
Si por cualquier circunstancia llega a haber una falla interna durante la 
ejecución del proceso, ninguna de las dos pestañas será mostrada y en su 
lugar aparecerá una de error \textbf{(ErrorFrame)}, especificando además 
el tipo de error y en qué parte de Model \textbf{(ó Modelo)} ocurrió.
\end{DUlineblock}

\begin{quote}\begin{description}
\item[{Parameters}] \leavevmode\begin{itemize}
\item \textbf{\texttt{parent}} (\emph{\texttt{Tkinter.Frame}}) -- Frame padre al que pertenece.
\item \textbf{\texttt{path\_image\_logo}} (\emph{\texttt{String}}) -- La ruta al logotipo que se usa en esta ventana independiente.
\item \textbf{\texttt{execution\_task\_count}} (\emph{\texttt{Integer}}) -- Número que indica el actual Task en ejecución.
\item \textbf{\texttt{main\_features}} (\emph{\texttt{Dictionary}}) -- Diccionario que contiene, entre otras cosas, los nombres de los parámetros asociados a cada técnica.
\item \textbf{\texttt{gathered\_information}} (\emph{\texttt{Dictionary}}) -- Diccionario que contiene todas las configuraciones recabadas ingresadas por el usuario \textbf{(véase View/Main/MainWindow.py)}.
\item \textbf{\texttt{final\_results}} (\emph{\texttt{Dictionary}}) -- Diccionario que contiene la información procesada lista para graficar\textbf{(véase View/Additional/ResultsGrapher/GraphFrame.py)}.
\end{itemize}

\item[{Returns}] \leavevmode
Tkinter.Toplevel
\item[{Return type}] \leavevmode
Instance
\end{description}\end{quote}

%******* Termina descripción *******

%******* Empieza función *******
\begin{fulllineitems}

\pysiglinewithargsret{\sphinxbfcode{center}}{}{}~

\begin{notice}{note}{Note:}
Este método es privado.
\end{notice}

Centra la ventana independiente con respecto de la 
Ventana Principal.\break
En otras palabras, la ventana independiente será 
colocada en el centro de Ventana Principal.

\end{fulllineitems}
%******* Termina función *******

%******* Empieza función *******
\begin{fulllineitems}

\pysiglinewithargsret{\sphinxbfcode{create\_renamed\_settings}}{}{}~

\begin{notice}{note}{Note:}
Este método es privado.
\end{notice}

Tal como su nombre lo dice, renombra las funciones objetivo y 
variables de decisión para posteriormente almacenarlas en una 
estructura por cada tipo.\break
Renombrar una función o variable de decisión es hacer un mapeo 
que consista en:\break
Elemento\_renombrado -\textgreater{} elemento original.\break
Para el caso de la función objetivo, el renombramiento se da 
anteponiendo la letra \textbf{F} seguido de la posición en la 
que fue insertada originalmente por el usuario.\break
El caso es análogo para la variable de decisión, sólo que la 
letra es \textbf{V}.\break
La idea de renombrar las funciones y variables surge como 
alternativa al momento de graficar los datos \textbf{(véase View/Additional/ResultsGrapher/}\break\textbf{GraphFrame.py)}, 
ya que el usuario puede ingresar funciones muy largas o variables 
con identificadores muy complejos y esto en la parte gráfica se 
vería muy amontonado; por ello fue preferible mostrar la parte 
renombrada en la sección de GraphFrame y colocar la muestra original 
en el SummaryFrame.         

\end{fulllineitems}
%******* Termina función *******

\end{fulllineitems}
%******* Termina clase *******

La clase actual consta de los siguientes elementos:

%******* Empieza clase *******
\subparagraph{GraphFrame (clase)}
%Se coloca el vínculo interno procedente de esta misma sección (a_3_3_3_1_1).
\label{sec:a_3_3_3_1_1}
%******* Empieza descripción *******
\begin{fulllineitems}

\begin{DUlineblock}{0em}
\item[] Proporciona un Frame que contiene gráficas 
alimentadas por los resultados obtenidos al ejecutar 
algún MOEA, el cual ha sido refinado por las configuraciones
recabadas de la Ventana Principal \textbf{(véase Model/MOEA)}.
\end{DUlineblock}

\begin{quote}\begin{description}
\item[{Parameters}] \leavevmode\begin{itemize}
\item \textbf{\texttt{parent}} (\emph{\texttt{Tkinter.Frame}}) -- Frame padre al que pertenece.
\item \textbf{\texttt{execution\_task\_count}} (\emph{\texttt{Integer}}) -- Número que indica el actual Task en ejecución.
\item \textbf{\texttt{objective\_functions}} (\emph{\texttt{List}}) -- Lista que contiene las funciones objetivo renombradas.
\item \textbf{\texttt{decision\_variables}} (\emph{\texttt{List}}) -- Lista que contiene las variables de decisión renombradas.
\item \textbf{\texttt{final\_results}} (\emph{\texttt{Dictionary}}) -- Diccionario que contiene la información para graficar. Se divide en dos categorías principales: Frente de Pareto y Mejor Individuo por Generación.
\end{itemize}

\item[{Returns}] \leavevmode
Tkinter.Frame
\item[{Return type}] \leavevmode
Instance
\end{description}\end{quote}

%******* Termina descripción *******

%******* Empieza función *******
\begin{fulllineitems}

\pysiglinewithargsret{\sphinxbfcode{change\_canvas\_category}}{}{}~

\begin{notice}{note}{Note:}
Este método es privado.
\end{notice}

Realiza el cambio de Canvas de la categoría de funciones 
objetivo a la de variables y decision y viceversa, tomando 
en cuenta factores como por ejemplo si alguna de las dos 
categorías tiene un OptionMenu asociado \textbf{(para entonces colocarlo apropiadamente)} 
e identificando siempre el último Canvas seleccionado de la 
categoría anterior para que cuando sea oportuno se vuelva 
a colocar.

\end{fulllineitems}
%******* Termina función *******

%******* Empieza función *******
\begin{fulllineitems}

\pysiglinewithargsret{\sphinxbfcode{change\_inner\_canvas}}{\emph{event}}{}~

\begin{notice}{note}{Note:}
Este método es privado.
\end{notice}

Realiza el cambio de Canvas dentro de una misma categoría, 
esto en caso en que los datos hayan arrojado más de una gráfica.\break
El cambio se hace con ayuda de su OptionMenu asociado.

\begin{quote}\begin{description}
\item[{Parameters}] \leavevmode\begin{itemize}
\item \textbf{\texttt{event}} (\emph{\texttt{String}}) -- Elemento que ejecutó esta función.
\end{itemize}
\end{description}\end{quote}

\end{fulllineitems}
%******* Termina función *******

%******* Empieza función *******
\begin{fulllineitems}

\pysiglinewithargsret{\sphinxbfcode{create\_2d\_canvas}}{\emph{x\_label}, \emph{x\_index}, \emph{y\_label}, \emph{y\_index}, \emph{collection\_points}}{}~

\begin{notice}{note}{Note:}
Este método es privado.
\end{notice}

Crea una gráfica en 2 dimensiones que es envuelta en 
un Canvas.

\begin{quote}\begin{description}
\item[{Parameters}] \leavevmode\begin{itemize}
\item \textbf{\texttt{x\_label}} (\emph{\texttt{String}}) -- Nombre para el eje X de la gráfica.
\item \textbf{\texttt{x\_index}} (\emph{\texttt{Integer}}) -- Posición dentro de collection\_points para los datos del eje X.
\item \textbf{\texttt{y\_label}} (\emph{\texttt{String}}) -- Nombre para el eje Y de la gráfica.
\item \textbf{\texttt{y\_index}} (\emph{\texttt{Integer}}) -- Posición dentro de collection\_points para los datos del eje Y.
\item \textbf{\texttt{collection\_points}} (\emph{\texttt{Dictionary}}) -- Diccionario que contiene los puntos a graficar.
\end{itemize}

\item[{Returns}] \leavevmode
Canvas
\item[{Return type}] \leavevmode
matplotlib.backends.backend\_tkagg.FigureCanvasTkAgg
\end{description}\end{quote}

\end{fulllineitems}
%******* Termina función *******

%******* Empieza función *******
\begin{fulllineitems}

\pysiglinewithargsret{\sphinxbfcode{create\_3d\_canvas}}{\emph{x\_label}, \emph{x\_index}, \emph{y\_label}, \emph{y\_index}, \emph{z\_label}, \emph{z\_index}, \emph{collection\_points}}{}~

\begin{notice}{note}{Note:}
Este método es privado.
\end{notice}

Crea una gráfica en 3 dimensiones que es envuelta en 
un Canvas.

\begin{quote}\begin{description}
\item[{Parameters}] \leavevmode\begin{itemize}
\item \textbf{\texttt{x\_label}} (\emph{\texttt{String}}) -- Nombre para el eje X de la gráfica.
\item \textbf{\texttt{x\_index}} (\emph{\texttt{Integer}}) -- Posición dentro de collection\_points para los datos del eje X.
\item \textbf{\texttt{y\_label}} (\emph{\texttt{String}}) -- Nombre para el eje Y de la gráfica.
\item \textbf{\texttt{y\_index}} (\emph{\texttt{Integer}}) -- Posición dentro de collection\_points para los datos del eje Y.
\item \textbf{\texttt{z\_label}} (\emph{\texttt{String}}) -- Nombre para el eje Z de la gráfica.
\item \textbf{\texttt{z\_index}} (\emph{\texttt{Integer}}) -- Posición dentro de collection\_points para los datos del eje Z.
\item \textbf{\texttt{collection\_points}} (\emph{\texttt{Dictionary}}) -- Diccionario que contiene los puntos a graficar.
\end{itemize}

\item[{Returns}] \leavevmode
Canvas
\item[{Return type}] \leavevmode
matplotlib.backends.backend\_tkagg.FigureCanvasTkAgg
\end{description}\end{quote}

\end{fulllineitems}
%******* Termina función *******

%******* Empieza función *******
\begin{fulllineitems}

\pysiglinewithargsret{\sphinxbfcode{create\_decision\_variables\_canvas}}{\emph{decision\_variables}, \emph{collection\_points}}{}~

\begin{notice}{note}{Note:}
Este método es privado
\end{notice}

Crea los Canvas para las variables de decisión.

\begin{quote}\begin{description}
\item[{Parameters}] \leavevmode\begin{itemize}
\item \textbf{\texttt{decision\_variables}} (\emph{\texttt{List}}) -- Lista que contiene las variables de decisión renombradas.
\item \textbf{\texttt{collection\_points}} (\emph{\texttt{Dictionary}}) -- Diccionario que contiene los valores de las funciones objetivo de todos los Individuos en la Población final.
\end{itemize}
\end{description}\end{quote}

\end{fulllineitems}
%******* Termina función *******

%******* Empieza función *******
\begin{fulllineitems}

\pysiglinewithargsret{\sphinxbfcode{create\_objective\_functions\_canvas}}{\emph{objective\_functions}, \emph{collection\_points}}{}~

\begin{notice}{note}{Note:}
Este método es privado.
\end{notice}

Crea los Canvas para las funciones objetivo.

\begin{quote}\begin{description}
\item[{Parameters}] \leavevmode\begin{itemize}
\item \textbf{\texttt{objective\_functions}} (\emph{\texttt{List}}) -- Lista que contiene las funciones objetivo renombradas.
\item \textbf{\texttt{collection\_points}} (\emph{\texttt{Dictionary}}) -- Diccionario que contiene los valores de las funciones objetivo de todos los Individuos en la Población final.
\end{itemize}
\end{description}\end{quote}

\end{fulllineitems}
%******* Termina función *******

\end{fulllineitems}
%******* Termina clase *******

La clase actual toma como referencia el siguiente elemento:

%******* Empieza clase *******
\subparagraph{CustomNavigationToolbar2TkAgg (clase)}
%Se coloca el vínculo interno procedente de esta misma sección (a_3_3_3_1_1_1).
\label{sec:a_3_3_3_1_1_1}
%******* Empieza descripción *******
\begin{fulllineitems}

\begin{DUlineblock}{0em}
\item[] Proporciona una Barra de Navegación \textbf{(ó NavigationToolbar)} 
que se anexa a cada una de las gráficas con el fin de facilitar la exploración 
y almacenamiento de los datos obtenidos.\break
Por defecto la barra de navegación original se encuentra obsoleta a las 
necesidades inherentes a este proyecto, por ello es que se crea una barra 
personalizada que responde a requerimientos tales como la obtención apropiada 
de imágenes relativas a las gráficas así como su correcto funcionamiento sin 
importar el sistema operativo empleado.
\end{DUlineblock}

\begin{quote}\begin{description}
\item[{Parameters}] \leavevmode\begin{itemize}
\item \textbf{\texttt{canvas}} (\emph{\texttt{matplotlib.backends.backend\_tkagg.\break FigureCanvasTkAgg}}) -- La estructura que contiene tanto a la gráfica como a la Barra de Navegación.
\item \textbf{\texttt{window}} (\emph{\texttt{Tkinter.Frame}}) -- El Frame que contiene a canvas.
\item \textbf{\texttt{parent\_frame}} (\emph{\texttt{Tkinter.Frame}}) -- El Frame que contiene a window, en este caso ResultsGrapherToplevel.py.
\item \textbf{\texttt{execution\_task\_count}} (\emph{\texttt{Integer}}) -- Un identificador que precisa el número de tarea \textbf{(Task)} en ejecución \textbf{(véase View/Additional/ResultsGrapher/}\break\textbf{ResultsGrapherToplevel.py)}.
\item \textbf{\texttt{image\_text}} (\emph{\texttt{String}}) -- El nombre que tendrán por defecto las imágenes resultantes al guardarse en el equipo de cómputo.
\end{itemize}

\item[{Returns}] \leavevmode
matplotlib.backends.backend\_tkagg.NavigationToolbar2TkAgg
\item[{Rype}] \leavevmode
Instance
\end{description}\end{quote}

%******* Termina descripción *******

%******* Empieza función *******
\begin{fulllineitems}

\pysiglinewithargsret{\sphinxbfcode{save\_figure}}{\emph{*args}}{}~

\begin{notice}{note}{Note:}
Este método sobreescribe al original.
\end{notice}

Arroja una ventana emergente modificada para guardar archivos, 
en este caso las gráficas.\break
Las modificaciones con respecto de la función original consisten 
en agregar un título para tener conocimiento de las imágenes del 
Task que se van a guardar.\break
Además se modifica el comportamiento de la ventana para adherirlo 
a la ventana del Task y no a la Ventana Principal.

\begin{quote}\begin{description}
\item[{Parameters}] \leavevmode\begin{itemize}
\item \textbf{\texttt{args}} (\emph{\texttt{Tuple}}) -- Un listado con parámetros que aunque no se ocupan en el método se coloca porque así lo estructuraron los desarrolladores originales de la biblioteca.
\end{itemize}
\end{description}\end{quote}

\end{fulllineitems}
%******* Termina función *******

\end{fulllineitems}
%******* Termina clase *******

%******* Empieza clase *******
\subparagraph{SummaryFrame (clase)}
%Se coloca el vínculo interno procedente de esta misma sección (a_3_3_3_1_2).
\label{sec:a_3_3_3_1_2}
%******* Empieza descripción *******
\begin{fulllineitems}

\begin{DUlineblock}{0em}
\item[] Unifica dos elementos: Canvas y ContentFrame.\break
La razón de esto es que, en promedio la información 
mostrada por ContentFrame rebasará el tamaño de la 
ventana de la información final \textbf{(véase View/Additional/}\break\textbf{ResultsGrapher/ResultsGrapherTopLevel.py)}, 
es entonces que se deben agregar barras de desplazamiento 
para poder acceder al contenido que quedaría oculto.\break
Uno de los elementos en Tkinter más sencillos que 
cumplen con este cometido es un Canvas. Luego entonces 
esa es la razón de tal fusión.
\end{DUlineblock}

\begin{quote}\begin{description}
\item[{Parameters}] \leavevmode\begin{itemize}
\item \textbf{\texttt{parent}} (\emph{\texttt{Tkinter.Frame}}) -- Frame padre al que pertenece.
\item \textbf{\texttt{renamed\_objective\_functions}} (\emph{\texttt{Dictionary}}) -- Diccionario de funciones objetivo renombradas \textbf{(véase View/Additional/ResultsGrapher/ResultsGrapherToplevel.py)}.
\item \textbf{\texttt{renamed\_decision\_variables}} (\emph{\texttt{Dictionary}}) -- Diccionario de variables de decisión renombradas  
\textbf{(véase View/Additional/ResultsGrapher/ResultsGrapherToplevel.py)}.
\item \textbf{\texttt{main\_features}} (\emph{\texttt{Dictionary}}) -- Diccionario que contiene, entre otras cosas, los nombres de los
parámetros asociados a cada técnica.
\item \textbf{\texttt{gathered\_information}} (\emph{\texttt{Dictionary}}) -- Diccionario que contiene todas las configuraciones 
recabadas ingresadas por el usuario \textbf{(véase View/Main/MainWindow.py)}.
\end{itemize}

\item[{Returns}] \leavevmode
Tkinter.Frame
\item[{Return type}] \leavevmode
Instance
\end{description}\end{quote}

%******* Termina descripción *******

%******* Empieza función *******
\begin{fulllineitems}

\pysiglinewithargsret{\sphinxbfcode{update\_scrollbar}}{\emph{event}}{}~

\begin{notice}{note}{Note:}
Este método es privado.
\end{notice}

Actualiza la barra de desplazamiento de acuerdo al 
número de elementos existentes en el Frame, esto 
para poder hacer un recorrido apropiado de la barra.

\begin{quote}\begin{description}
\item[{Parameters}] \leavevmode\begin{itemize}
\item \textbf{\texttt{event}} (\emph{\texttt{String}}) -- Elemento que ejecutó esta función.
\end{itemize}
\end{description}\end{quote}

\end{fulllineitems}
%******* Termina función *******

\end{fulllineitems}
%******* Termina clase *******

La clase actual toma como base el siguiente elemento:

%******* Empieza clase *******
\subparagraph{ContentFrame (clase)}
%Se coloca el vínculo interno procedente de esta misma sección (a_3_3_3_1_2_1).
\label{sec:a_3_3_3_1_2_1}
%******* Empieza descripción *******
\begin{fulllineitems}

\begin{DUlineblock}{0em}
\item[] Recaba el contenido de todas las funciones 
objetivo, variables de decisión y demás parámetros 
que el usuario ingresó para poder ejecutar un Task 
determinado.\break
Es entonces que plasma toda esta información en un 
Frame para que el usuario pueda cotejar los datos 
ingresados con los resultados obtenidos \textbf{(véase View/Additional/}\break\textbf{ResultsGrapher/GraphFrame.py)}.
\end{DUlineblock}

\begin{quote}\begin{description}
\item[{Parameters}] \leavevmode\begin{itemize}
\item \textbf{\texttt{parent}} (\emph{\texttt{Tkinter.Frame}}) -- Frame padre al que pertenece.
\item \textbf{\texttt{renamed\_objective\_functions}} (\emph{\texttt{Dictionary}}) -- Diccionario de funciones objetivo renombradas \textbf{(véase View/Additional/ResultsGrapher/ResultsGrapherToplevel.py)}.
\item \textbf{\texttt{renamed\_decision\_variables}} (\emph{\texttt{Dictionary}}) -- Diccionario de variables de decisión renombradas \textbf{(véase View/Additional/ResultsGrapher/ResultsGrapherToplevel.py)}.
\item \textbf{\texttt{main\_features}} (\emph{\texttt{Dictionary}}) -- Diccionario que contiene, entre otras cosas, los nombres de los parámetros asociados a cada técnica.
\item \textbf{\texttt{gathered\_information}} (\emph{\texttt{Dictionary}}) -- Diccionario que contiene todas las configuraciones  recabadas ingresadas por el usuario \textbf{(véase View/Main/MainWindow.py)}.
\end{itemize}

\item[{Returns}] \leavevmode
Tkinter.Frame
\item[{Return type}] \leavevmode
Instance
\end{description}\end{quote}

\end{fulllineitems}
%******* Termina clase *******

%******* Empieza clase *******
\subparagraph{ErrorFrame (clase)}
%Se coloca el vínculo interno procedente de esta misma sección (a_3_3_3_1_3).
\label{sec:a_3_3_3_1_3}
%******* Empieza descripción *******
\begin{fulllineitems}

\begin{DUlineblock}{0em}
\item[] Este Frame surge si durante el proceso 
interno en el Modelo \textbf{(véase Model/MOEA)} se 
suscita algún error del cual el método no se pueda 
recuperar.\break
Entonces aquí se desplegará toda la información 
relativa a la falla, asímismo funciona como medida 
de contingencia para darle una salida al programa 
y evitar que se quede atorado.
\end{DUlineblock}

\begin{quote}\begin{description}
\item[{Parameters}] \leavevmode\begin{itemize}
\item \textbf{\texttt{parent}} (\emph{\texttt{Tkinter.Frame}}) -- Frame padre al que pertenece.
\item \textbf{\texttt{final\_results}} (\emph{\texttt{Dictionary}}) -- Diccionario que contiene en este caso las características alusivas a la falla \textbf{(véase Model/MOEA)}.
\end{itemize}

\item[{Returns}] \leavevmode
Tkinter.Frame
\item[{Return type}] \leavevmode
Instance
\end{description}\end{quote}

\end{fulllineitems}
%******* Termina clase *******
%******* Termina módulo *******
%******* Termina módulo *******

%******* Termina documento *******
\end{document}

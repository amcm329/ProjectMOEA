%Autor: Aarón Martín Castillo Medina.
%Asesora: Dra. Katya Rodríguez Vázquez.
%Contacto: katya.rodriguez@iimas.unam.mx; amcm329@hotmail.com

%En este archivo se conjuntan las partes que conforman el manual 
%técnico para el programa M.O.E.A. Software, además se introducen 
%algunas directivas para personalizar el comportamiento del trabajo 
%para determinadas secciones.


%Aquí se indica que la tesis consta de un artículo a dos páginas con
%letra de tamaño 12.
%Se decidió emplear un artículo en vez de un reporte porque en el 
%primero se tiene la posibilidad de adjuntar título, autor y páginas
%junto a la parte escrita sin dejar espacios, mientras que en la 
%segunda opción se hubieran tenido que modificar varias directivas.
\documentclass[12pt,twoside]{article}

%Este paquete permite la carga adecuada de subarchivos .tex, respetando
%(con ayuda de la opción subpreambles=true) los paquetes cargados en cada
%subarchivo.
\usepackage[subpreambles=true]{standalone}

%Se lleva a cabo la carga de paquetes específicamente seleccionados
%para este archivo.
\usepackage{packages_used_general}

%Para ajustar los colores de los hipervínculos.
\hypersetup{
            colorlinks = true,
            linkcolor = black,
            filecolor = magenta,      
            urlcolor = cyan
           }
 
%Aquí se indica la ruta de las imágenes que se utilizarán en la tesis.
\graphicspath{ 
              {images/}
             }

%El número de subsecciones máximo en el texto es de 4 para iniciar, 
%por ello es que con esta directiva se incrementa dicha cantidad.
%No confundir con el número máximo de niveles de anidamiento en
%un listado.
\setcounter{secnumdepth}{6}

%A continuacipón se contemplan las características que tendrá
%el pie de las páginas.
%Con esta línea se coloca un margen en la parte inferior para
%identificar al pie de página.
\renewcommand{\footrulewidth}{0.4pt}% default es 0pt

%En la parte izquierda del pie de página (sin importar si es par 
%o impar) se pondrá la palabra "UNAM".
\lfoot{UNAM}

%En la parte derecha del pie de página (sin importar si es par 
%o impar) se pone la palabra "Facultad de Ciencias".
\rfoot{Facultad de Ciencias}

%Enla parte centro del pie de página (sin importar si es par 
%o impar) se coloca un espacio vacío.
\cfoot{}

%Se modifica el formato de todos los capítulos de la tesis.
\titleformat{\chapter}[display]{\normalfont\huge\bfseries}{\chaptertitlename\ \thechapter}{11pt}{\Huge}

%Las siguientes líneas corresponden a la información que se 
%mostrará en el encabezado de las páginas.
%A la izquierda de cada página par se mostrará el texto 
%"No. Página - Capítulo No. X"
%Para la bibliografía y el apéndice en el encabezado izquiero de páginas pares
%sólo se mostrarán las palabras "Bibliografía" y "Apéndice" respectivamente.        
\fancyhead[LE]{\textbf{\thepage}\ \ \ \ \ Manual Técnico}

%A la derecha de cada página par no se mostrará nada.
\fancyhead[RE]{}

%A la izquierda de cada página impar no se mostrará nada.
\fancyhead[LO]{}

%A la derecha de cada página impar se muestra la información 
%"Nombre del capítulo - No. Página"
\fancyhead[RO]{Manual Técnico \ \ \ \ \ \textbf{\thepage}}

%Con ayuda del paquete "titling" se pueden insertar elementos
%previos al título, en este caso las imágenes correspondientes a
%los escudos de la Universidad Nacional Autónoma de México y la 
%Facultad de Ciencias respectivamente.
\pretitle{%
\begin{center}

%Para indicar que el título tendrá un tamaño LARGE.
\LARGE
%Para dejar un párrafo antes de las imágenes.
\par
%Se inserta la imagen del escudo de la U.N.A.M.
\raisebox{-.5\height}{\includegraphics[width=3.6cm]{escudo_unam.jpg}}%
%Esta sentencia indica que la siguiente imagen se coloque
%en la parte horizontal contraria a la imagen anterior.
\hfill
%A continuación se agrega la imagen relativa al escudo de la
%Facultad de Ciencias.
\raisebox{-.5\height}{\includegraphics[width=3.6cm]{escudo_ciencias.jpg}}%
%Para dejar un párrafo después de las imágenes.
\par
}

%La parte del postítulo sirve para cerrar adecuadamente la 
%información precisada en el pretítulo.
\posttitle{\end{center}}

%Para eliminar espacio vertical entre la primera sección en blanco
%y las imágenes de los escudos.
\droptitle = -25mm

%Se coloca el título del documento.
%El código vspace se utiliza para añadir espacio adicional entre
%las imágenes de los escudos y el título.
\title{\vspace{+0.62cm}\textbf{Manual Técnico de M.O.E.A. Software}}

%Se plasman autores con información correspondiente a sus correos
%de contacto.
\author{Dra. Katya Rodríguez Vázquez \\ \href{mailto:katya.rodriguez@iimas.unam.mx}{\textcolor{blue}{katya.rodriguez@iimas.unam.mx}} 
   \and Aarón Martín Castillo Medina \\ \href{mailto:amcm329@hotmail.com}{\textcolor{blue}{amcm329@hotmail.com}}}

%Se agrega la fecha del documento, aunque ésta está vacía es imperativo
%colocarla para que no existan fallos en la compilación.
\date{}	

%Con esta instrucción se activan el formato personalizado para encabezado
%y pie de página descrito anteriormente.
\pagestyle{fancy}      

%Comienza el documento.
\begin{document}

%Toda la información creada anteriormente con respecto del título, autores
%y fecha se plasma en esta sección mediante la directiva \maketitle.
\maketitle

%Aquí se indica que se va a remover espacio vertical entre los autores
%y el inicio del contenido.
\vspace{-0.5cm}
El presente manual contiene la información técnica relacionada con el 
producto de software denominado \textbf{M.O.E.A. Software}, la cual consiste 
en un compendio de las funcionalidades creadas.\break
Este documento pretende ser una extensión especializada exclusivamente 
en implementaciones algorítmicas del mencionado producto con la finalidad 
de familiarizar al usuario con el contenido subyacente en el código fuente 
para que pueda interaccionar con éste a su merced, pudiendo reproducirlo o 
modificarlo según sea el caso.\medskip\break
Es importante mencionar que al ser el producto de software mismo una fusión 
entre la teoría comprendida en el trabajo escrito y sus correspondientes 
implementaciones técnicas, el manual por ende también incluye referencias 
que manifiestan la teoría detrás de cada una de las caracteríticas creadas.\break
Para esto se ofrece al usuario una explicación precisa y concisa de los términos
técnicos que se usarán a lo largo del manual.\medskip\break
Comenzando por un breve señalamiento de la arquitectura de diseño, ésta se refiere 
al esquema o plantilla que se suele usar para adecuar la organización del código 
y así agruparlo en módulos con características similares con la finalidad de 
optimizar tiempos de detección y corrección de errores, amén de la facilidad de 
lectura del código gracias a la estructura del mismo.\break
Entonces se hace notar el uso de la así denominada MVC \textbf{(Model-View-Controller 
ó }\break\textbf{Modelo-Vista-Controlador)}.\break
Siguiendo este tipo de organización, se colocan las funcionalidades en tres categorías 
principales, que son:

\begin{itemize}
\item \textbf{Model (ó Modelo)}, se almacenan todos los elementos que realizan 
el proceso analítico, en este caso todo lo relacionado con la ejecución de M.O.E.A.’s 
y la recolección de resultados.
\item \textbf{View (ó Vista)}, se coloca todo aquéllo asociado a la interfaz gráfica 
del programa y en el caso del proyecto, la graficación apropiada de resultados.
\item \textbf{Controller (ó Controlador)}, se guarda toda la parafernalia relativa 
al control de las comunicaciones entre la Vista y el Modelo.\medskip\break
\end{itemize}

El proceso usual de interacción entre dichas categorías es el siguiente:

\begin{enumerate}
\item El usuario inserta las configuraciones pertinentes en la Vista, las cuales 
permitirán obtener resultados detallados del M.O.E.A. que se fuera a ejecutar.
\item El Controlador obtiene las configuraciones previamente insertadas por el 
usuario; durante esta etapa se realiza una verificación y saneamiento de dichas 
configuraciones. Si el proceso fue exitoso se procede ir al paso \textbf{(3)}, 
en cualquier otro caso se retrocede al paso \textbf{(1)} con una notificación de error.
\item El Modelo se encarga de ejecutar el algoritmo precisado por el usuario en 
\textbf{(1)}, para ello se le proporcionan todas las configuraciones adjuntas. 
Una vez concluido el proceso el Modelo le regresa los resultados al Controlador.
\item El Controlador toma los resultados y a su vez los transfiere a la Vista, 
la cual se encarga de mostrar al usuario los datos finales de manera gráfica.\medskip\break
\end{enumerate}

Ahora, haciendo énfasis en el contenido del programa, \textbf{M.O.E.A. Software} 
consta de 60 archivos útiles con un promedio de 200 líneas de código por cada 
uno aproximadamente sin considerar las líneas de comentarios.\break
Se resalta la denominación \textit{útiles} porque el usuario encontrará archivos 
adicionales cuyo único fin es el de proporcionar meta-información del programa 
o habilitar detalles técnicos muy específicos que escapan al enfoque de 
desarrollo.\medskip\break
Dirigiendo la atención hacia el interior del código fuente el usuario notará 
que todos los archivos están debidamente comentados, además se incluyen 
explicaciones técnicas breves para que éste pueda familiarizarse no sólo con el 
código sino con conceptos o elementos de programación utilizados en el 
producto de software.\medskip\break
Con respecto del formato de enunciamiento de comentarios se distingue una 
sintaxis ligeramente diferente al lenguaje natural. Esto se debe a que se ha usado 
la correspondiente a \break\textbf{Sphinx \href{http://www.sphinx-doc.org/en/stable/}{\textcolor{blue}{(http://www.sphinx-doc.org/en/stable/)}}}, 
el cual es un programa que automatiza la creación visualmente apropiada de 
documentación para productos de software.\break
Aún teniendo esto en cuenta no se debe tener dificultad en la interpretación de 
la información basada en los comentarios y en el caso de encontrarse con alguna 
incomodidad el sitio de Sphinx ofrece referencias básicas de traducción entre 
su sintaxis y el lenguaje cotidiano.\medskip\break
En lo concerniente a los tipos de archivos creados, se muestran las siguientes 
clasificaciones \textbf{(de menor a mayor jerarquía)}:

\begin{itemize}
\item \textbf{Script}, el archivo básico de implementación, por lo general en 
éste se describe infraestructura simple como métodos o variables; su creación 
se atribuye a concebir el uso de técnicas disponibles para el usuario de una 
manera más eficiente y menos burocrática.\break
Por lo general, este tipo de archivos se encuentran en el Modelo ya que su uso 
alude a entornos más dinámicos que requieran de una rápida respuesta.
\item \textbf{Clase}, similar a los Scripts con la salvedad de que las Clases 
contienen una estructura más robusta pero también más estática. Su uso se 
limita en este proyecto a la generación de la interfaz gráfica \textbf{(en la Vista)} 
y en el Controlador ya que, por regla general éstos deben estar predispuestos 
a las necesidades del usuario, ante lo cual se tiene el menester de destinar 
mayores recursos.
\item \textbf{Módulo}, corresponde principalmente a una agrupación de Clases 
y/o Scripts con características en común.
\item \textbf{Sección}, mantiene relación directa con las categorías mencionadas 
en la arquitectura MVC. Contiene a todos los elementos anteriores.\medskip\break
\end{itemize}

Conviene mencionar que en algunas partes del manual se establecen sugerencias 
de referencias anteponiendo la palabra \textit{véase} como identificador; en este 
caso se aclara que se trata de rutas relacionadas con el código fuente por lo que 
el usuario debe localizarlas en las secciones pertinentes \textbf{(usando el archivo 
Begin.py como punto de referencia)} para poder acceder al contenido y entonces 
asociar el código entre distintos archivos.\medskip\break
A continuación de muestra el diagrama del contenido del producto de software, 
donde se ven desplegadas tanto la jerarquía de los elementos como las 
relaciones entre éstos:

%Latex por defecto maneja niveles de anidamiento de listas de 5, para un número
%mayor se utilizan paquetes especiales como enumitem, sin embargo este paquete
%causa conflicto con las secciones y en particular con el paquete sphinx.sty 
%usado en dichas secciones.
%La solución para ello fue alternar el uso de itemize y enumerate para listas
%anidadas ya que de esta manera el "contador" se reinicia y por ende nunca se
%pasa del límite establecido.
%No se debe confuncir el número máximo de niveles de anidamiento con el número
%máximo de subsecciones permitidas en el documento.
\begin{itemize}
\item[$\blacksquare$] \hyperref[sec:a_1]{\textbf{\textcolor{blue}{\large Begin (script)}}} 
\item[$\blacksquare$] \hyperref[sec:a_2]{\textbf{\textcolor{blue}{\large Model (sección)}}} 
      \begin{enumerate}[$\square$]
      \item \hyperref[sec:a_2_1]{\textcolor{blue}{ChromosomalRepresentation (módulo)}} 
            \begin{itemize}
            \item[$\blacktriangleright$] \hyperref[sec:a_2_1_1]{\textcolor{blue}{BinaryRepresentation (script)}} 
            \item[$\blacktriangleright$] \hyperref[sec:a_2_1_2]{\textcolor{blue}{FloatPointRepresentation (script)}}
            \end{itemize}
      \item \hyperref[sec:a_2_2]{\textcolor{blue}{Community (clase)}}
            \begin{itemize}
            \item[$\blacktriangleright$] \hyperref[sec:a_2_2_1]{\textcolor{blue}{Population (clase)}}
                  \begin{enumerate}[$\triangleright$]
                  \item \hyperref[sec:a_2_2_1_1]{\textcolor{blue}{Individual (clase)}}
                  \end{enumerate}
            \end{itemize}
      \item \hyperref[sec:a_2_3]{\textcolor{blue}{Fitness (módulo)}}
            \begin{itemize}
            \item[$\blacktriangleright$] \hyperref[sec:a_2_3_1]{\textcolor{blue}{LinearRankingFitness (script)}}
            \item[$\blacktriangleright$] \hyperref[sec:a_2_3_2]{\textcolor{blue}{NonLinearRankingFitness (script)}}
            \item[$\blacktriangleright$] \hyperref[sec:a_2_3_3]{\textcolor{blue}{ProportionalFitness (script)}}
            \end{itemize}
      \item \hyperref[sec:a_2_4]{\textcolor{blue}{Operator (módulo)}}
            \begin{itemize}
            \item[$\blacktriangleright$] \hyperref[sec:a_2_4_1]{\textcolor{blue}{Selection (módulo)}}
                  \begin{enumerate}[$\triangleright$]
                  \item \hyperref[sec:a_2_4_1_1]{\textcolor{blue}{Roulette (script)}}
                  \item \hyperref[sec:a_2_4_1_2]{\textcolor{blue}{ProbabilisticTournament (script)}}
                  \item \hyperref[sec:a_2_4_1_3]{\textcolor{blue}{StochasticUniversalSampling (script)}}
                  \end{enumerate}
            \item[$\blacktriangleright$] \hyperref[sec:a_2_4_2]{\textcolor{blue}{Crossover (módulo)}}
                  \begin{enumerate}[$\triangleright$]
                  \item \hyperref[sec:a_2_4_2_1]{\textcolor{blue}{NPointsCrossover (script)}}
                  \item \hyperref[sec:a_2_4_2_2]{\textcolor{blue}{UniformCrossover (script)}}
                  \end{enumerate}
            \item[$\blacktriangleright$] \hyperref[sec:a_2_4_3]{\textcolor{blue}{Mutation (módulo)}}
                  \begin{enumerate}[$\triangleright$]
                  \item \hyperref[sec:a_2_4_3_1]{\textcolor{blue}{BinaryMutation (script)}}
                  \item \hyperref[sec:a_2_4_3_2]{\textcolor{blue}{FloatPointMutation (script)}}
                  \end{enumerate}
            \end{itemize}
      \item \hyperref[sec:a_2_5]{\textcolor{blue}{SharingFunction (módulo)}}
            \begin{itemize}
            \item[$\blacktriangleright$] \hyperref[sec:a_2_5_1]{\textcolor{blue}{GenotypicSimilarity (módulo)}}
                  \begin{enumerate}[$\triangleright$]
                  \item \hyperref[sec:a_2_5_1_1]{\textcolor{blue}{HammingDistance (script)}}
                  \end{enumerate}
            \item[$\blacktriangleright$] \hyperref[sec:a_2_5_2]{\textcolor{blue}{PhenotypicSimilarity (módulo)}}
                  \begin{enumerate}[$\triangleright$]
                  \item \hyperref[sec:a_2_5_2_1]{\textcolor{blue}{EuclideanDistance (script)}}
                  \end{enumerate}
            \end{itemize}
      \item \hyperref[sec:a_2_6]{\textcolor{blue}{MOEA (módulo)}} 
            \begin{itemize}
            \item[$\blacktriangleright$] \hyperref[sec:a_2_6_1]{\textcolor{blue}{VEGA (script)}}
            \item[$\blacktriangleright$] \hyperref[sec:a_2_6_2]{\textcolor{blue}{SPEAII (script)}}
            \item[$\blacktriangleright$] \hyperref[sec:a_2_6_3]{\textcolor{blue}{MOGA (script)}} 
            \item[$\blacktriangleright$] \hyperref[sec:a_2_6_4]{\textcolor{blue}{NSGAII (script)}}
            \end{itemize}
      \end{enumerate}
\item[$\blacksquare$] \hyperref[sec:a_3]{\textbf{\textcolor{blue}{\large View (sección)}}}     
      \begin{enumerate}[$\square$]
      \item \hyperref[sec:a_3_1]{\textcolor{blue}{MainWindow (clase)}}  
      \item \hyperref[sec:a_3_2]{\textcolor{blue}{Main (módulo)}}
            \begin{itemize} 
            \item[$\blacktriangleright$] \hyperref[sec:a_3_2_1]{\textcolor{blue}{Home (módulo)}}
                  \begin{enumerate}[$\triangleright$]
                  \item \hyperref[sec:a_3_2_1_1]{\textcolor{blue}{HomeFrame (clase)}}
                        \begin{itemize}
                        \item[$\bullet$] \hyperref[sec:a_3_2_1_1_1]{\textcolor{blue}{IntroductionFrame (clase)}}
                        \end{itemize}
                  \end{enumerate}
            \item[$\blacktriangleright$] \hyperref[sec:a_3_2_2]{\textcolor{blue}{DecisionVariable (módulo)}}
                  \begin{enumerate}[$\triangleright$]
                  \item \hyperref[sec:a_3_2_2_1]{\textcolor{blue}{DecisionVariableFrame (clase)}}
                        \begin{itemize}
                        \item[$\bullet$] \hyperref[sec:a_3_2_2_1_1]{\textcolor{blue}{VariableFrame (clase)}}
                        \end{itemize}
                  \end{enumerate}
            \item[$\blacktriangleright$] \hyperref[sec:a_3_2_3]{\textcolor{blue}{ObjectiveFunction (módulo)}}
                  \begin{enumerate}[$\triangleright$]
                  \item \hyperref[sec:a_3_2_3_1]{\textcolor{blue}{ObjectiveFunctionFrame (clase)}}
                        \begin{itemize}
                        \item[$\bullet$] \hyperref[sec:a_3_2_3_1_1]{\textcolor{blue}{FunctionFrame (clase)}}
                        \end{itemize}
                  \end{enumerate}
            \item[$\blacktriangleright$] \hyperref[sec:a_3_2_4]{\textcolor{blue}{Population (módulo)}}
                  \begin{enumerate}[$\triangleright$]
                  \item \hyperref[sec:a_3_2_4_1]{\textcolor{blue}{PopulationFrame (clase)}}
                  \item \hyperref[sec:a_3_2_4_2]{\textcolor{blue}{TemplatePopulationFrame (clase)}}
                        \begin{itemize}
                        \item[$\bullet$] \hyperref[sec:a_3_2_4_2_1]{\textcolor{blue}{PopulaceFrame (clase)}}
                        \item[$\bullet$] \hyperref[sec:a_3_2_4_2_2]{\textcolor{blue}{FitnessFrame (clase)}}
                        \end{itemize}
                  \end{enumerate}
            \item[$\blacktriangleright$] \hyperref[sec:a_3_2_5]{\textcolor{blue}{GeneticOperator (módulo)}}
                  \begin{enumerate}[$\triangleright$]
                  \item \hyperref[sec:a_3_2_5_1]{\textcolor{blue}{GeneticOperatorFrame (clase)}}
                  \item \hyperref[sec:a_3_2_5_2]{\textcolor{blue}{TemplateGeneticOperatorFrame (clase)}}
                        \begin{itemize}
                        \item[$\bullet$] \hyperref[sec:a_3_2_5_2_1]{\textcolor{blue}{SelectionFrame (clase)}}
                        \item[$\bullet$] \hyperref[sec:a_3_2_5_2_2]{\textcolor{blue}{CrossoverFrame (clase)}}
                        \item[$\bullet$] \hyperref[sec:a_3_2_5_2_3]{\textcolor{blue}{MutationFrame (clase)}}
                        \end{itemize}
                  \end{enumerate}
            \item[$\blacktriangleright$] \hyperref[sec:a_3_2_6]{\textcolor{blue}{MOEA (módulo)}}
                  \begin{enumerate}[$\triangleright$]
                  \item \hyperref[sec:a_3_2_6_1]{\textcolor{blue}{MOEAFrame (clase)}}
                        \begin{itemize}
                        \item[$\bullet$] \hyperref[sec:a_3_2_6_1_1]{\textcolor{blue}{AlgorithmFrame (clase)}}
                        \item[$\bullet$] \hyperref[sec:a_3_2_6_1_2]{\textcolor{blue}{SharingFunctionFrame (clase)}}
                        \end{itemize}
                  \end{enumerate}
            \end{itemize} 
      \item \hyperref[sec:a_3_3]{\textcolor{blue}{Additional (módulo)}}
            \begin{itemize}
            \item[$\blacktriangleright$] \hyperref[sec:a_3_3_1]{\textcolor{blue}{GenerationSignalToplevel (clase)}}
            \item[$\blacktriangleright$] \hyperref[sec:a_3_3_2]{\textcolor{blue}{MenuInternalOption (módulo)}}
                  \begin{enumerate}[$\triangleright$]
                  \item \hyperref[sec:a_3_3_2_1]{\textcolor{blue}{MenuInternalOption (clase)}}
                  \item \hyperref[sec:a_3_3_2_2]{\textcolor{blue}{InternalOptionToplevel (clase)}}
                  \item \hyperref[sec:a_3_3_2_3]{\textcolor{blue}{InternalOptionTab (módulo)}}
                        \begin{itemize}
                        \item[$\bullet$] \hyperref[sec:a_3_3_2_3_1]{\textcolor{blue}{MOPExampleFrame (clase)}}
                              \begin{enumerate}[$\circ$]
                              \item \hyperref[sec:a_3_3_2_3_1_1]{\textcolor{blue}{MOPFrame (clase)}}
                              \end{enumerate}
                        \item[$\bullet$] \hyperref[sec:a_3_3_2_3_2]{\textcolor{blue}{FeatureFrame (clase)}}
                              \begin{enumerate}[$\circ$]
                              \item \hyperref[sec:a_3_3_2_3_2_1]{\textcolor{blue}{CharacteristicFrame (clase)}}
                              \end{enumerate}
                        \item[$\bullet$] \hyperref[sec:a_3_3_2_3_3]{\textcolor{blue}{PythonExpressionFrame (clase)}}
                              \begin{enumerate}[$\circ$]
                              \item \hyperref[sec:a_3_3_2_3_3_1]{\textcolor{blue}{ExpressionFrame (clase)}}
                              \end{enumerate}
                        \end{itemize}
                  \item \hyperref[sec:a_3_3_2_4]{\textcolor{blue}{AboutToplevel (clase)}}
                  \end{enumerate}
            \item[$\blacktriangleright$] \hyperref[sec:a_3_3_3]{\textcolor{blue}{ResultsGrapher (módulo)}}
                  \begin{enumerate}[$\triangleright$]
                  \item \hyperref[sec:a_3_3_3_1]{\textcolor{blue}{ResultsGrapherToplevel (clase)}}
                        \begin{itemize}
                        \item[$\bullet$] \hyperref[sec:a_3_3_3_1_1]{\textcolor{blue}{GraphFrame (clase)}}
                              \begin{enumerate}[$\circ$]
                              \item \hyperref[sec:a_3_3_3_1_1_1]{\textcolor{blue}{CustomNavigationToolbar2TkAgg (clase)}}
                              \end{enumerate}
                        \item[$\bullet$] \hyperref[sec:a_3_3_3_1_2]{\textcolor{blue}{SummaryFrame (clase)}}
                              \begin{enumerate}[$\circ$]
                              \item \hyperref[sec:a_3_3_3_1_2_1]{\textcolor{blue}{ContentFrame (clase)}}
                              \end{enumerate}
                        \item[$\bullet$] \hyperref[sec:a_3_3_3_1_3]{\textcolor{blue}{ErrorFrame (clase)}}
                        \end{itemize}
                  \end{enumerate}
            \end{itemize}
      \end{enumerate} 
\item[$\blacksquare$] \hyperref[sec:a_4]{\textbf{\textcolor{blue}{\large Controller (sección)}}}     
      \begin{enumerate}[$\square$] 
      \item \hyperref[sec:a_4_1]{\textcolor{blue}{Controller (clase)}}
      \item \hyperref[sec:a_4_2]{\textcolor{blue}{XMLParser (clase)}}
      \item \hyperref[sec:a_4_3]{\textcolor{blue}{Verifier (clase)}}
      \end{enumerate}
\end{itemize}
%Termina el listado.

\section{Begin (script)}
%Se coloca el vínculo interno procedente de esta misma sección (a_1).
\label{sec:a_1}
Este archivo funge como un launcher \textbf{(disparador)} el cual simplemente 
crea y muestra la ventana principal \textbf{(véase la sección View)}.

%A continuación se incluyen los subarchivos relacionados con la
%documentación, divididos en Modelo (ó Model), Vista (ó View) y
%Controlador (ó Controller), respectivamente.
%Autor: Aarón Martín Castillo Medina.
%Asesora: Dra. Katya Rodríguez Vázquez
%Contacto: katya.rodriguez@iimas.unam.mx; amcm329@hotmail.com

%Este archivo contiene información relacionada con la capa Modelo
%(ó Model), la cual representa tanto física como lógicamente a uno de 
%los componentes que conforman el producto de software y por tanto al 
%Manual Técnico.


%Se indica que el documento es de tipo reporte bajo el paquete standalone.
\documentclass[class=report, crop=false]{standalone}

%Se cargan los paquetes relacionados con los subapéndices (elementos que
%conforman el Apéndice en su totalidad).
\usepackage{packages_used_section}

%Comienza el documento.
\begin{document}

\section{Model (sección)}
%Se coloca el vínculo interno procedente de esta misma sección (a_2).
\label{sec:a_2}
La sección Model \textbf{(ó Modelo)} contiene toda la base lógica del 
programa, más en específico, todas las características para poder ejecutar 
MOEA's apropiadamente alimentados con los datos obtenidos por la sección 
View \textbf{(ó Vista)} usando la sección Controller \textbf{(ó Controlador)} 
como intermediaria.\break
Una vez que se obtenga algún resultado, éste será transmitido a la sección View 
a través del Controller.\medskip\break
A continuación se observan las subcategorías que conforman a la sección 
en cuestión:

%******* Empieza módulo *******
\subsection{ChromosomalRepresentation (módulo)}
%Se coloca el vínculo interno procedente de esta misma sección (a_2_1).
\label{sec:a_2_1}
Ofrece elementos para elaborar una codificación adecuada.\break
Entiéndase por codificación a la forma de determinar el 
cromosoma y sus propiedades; cabe mencionar que el cromosoma será portado 
por los Individuals \textbf{(ó Individuos)}.\medskip\break
Es importante mencionar que cualquier codificación implementada debe ser 
sustentada en los métodos correspondientes al Crossover \textbf{(ó Cruza)} y 
Mutation \textbf{(ó Mutación)}, ésto porque dichas operaciones funcionan con 
cromosomas.\medskip\break
De esta manera, la idea es que el usuario pueda crear sus propias codificaciones,
por lo que, además de agregar la descripción de la codificación a 
Controller/XML/Features.xml \textbf{(véase el archivo mencionado en la sección de código)}, 
deberá implementar por lo menos las siguientes funciones:

%******* Empieza función *******
\begin{fulllineitems}
\pysiglinewithargsret{\sphinxbfcode{calculate\_length\_subchromosomes}}{\emph{vector\_variables}, \emph{number\_of\_decimals}, \emph{representation\_parameters}}{}~
\vspace{-0.1cm}

Por cada variable de decisión se crea una porción del cromosoma, 
entonces en esta función se calcula el tamaño de cada porción 
\textbf{(ó subcromosoma)}, ya que al final las operaciones de cruza 
y mutación se realizarán sobre el súper crosomoma, el cual es la 
concatenacion de todos los subcromosomas.\break
Por eso es importante identificar el tamaño de cada subcromosoma, 
así como sus límites dentro del súper cromosoma.

\begin{quote}\begin{description}
\item[{Parameters}] \leavevmode\begin{itemize}
\item \textbf{\texttt{vector\_variables}} (\emph{\texttt{List}}) -- El vector de variables de decisión, donde cada variable trae consigo sus límites inferior
y superior.
\item \textbf{\texttt{number\_of\_decimals}} (\emph{\texttt{Integer}}) -- El número de decimales que deberá traer cada variable de decisión.
\item \textbf{\texttt{representation\_parameters}} (\emph{\texttt{Dictionary}}) -- Un diccionario que contiene todas las opciones adicionales para cada tipo de codificación.
\end{itemize}

\item[{Returns}] \leavevmode
Una lista que contiene el tamaño del cromosoma por cada variable de 
decisión. Dado que el orden de las variables de decisión es inmutable, 
se preserva el mismo y por ello la lista contiene sólo los tamaños.
\item[{Return type}] \leavevmode
List
\end{description}\end{quote}

\end{fulllineitems}
%******* Termina función *******

%******* Empieza función *******
\begin{fulllineitems}
\pysiglinewithargsret{\sphinxbfcode{create\_chromosome}}{\emph{length\_subchromosomes}, \emph{vector\_variables}, \emph{number\_of\_decimals}, \emph{representation\_parameters}}{}~
\vspace{-0.1cm}

Función que crea el cromosoma. Se usa la como apoyo el método \break
\textbf{calculate\_length\_subchromosomes} descrito con anterioridad.

\begin{quote}\begin{description}
\item[{Parameters}] \leavevmode\begin{itemize}
\item \textbf{\texttt{length\_subchromosomes}} (\emph{\texttt{List}}) -- La lista que contiene los tamaños de cada variable de decisión.
\item \textbf{\texttt{vector\_variables}} (\emph{\texttt{List}}) -- La lista que contiene las variables de decisión, así como sus rangos.
\item \textbf{\texttt{number\_of\_decimals}} (\emph{\texttt{Integer}}) -- El número de decimales que traerá cada variable de decisión.
\item \textbf{\texttt{representation\_parameters}} (\emph{\texttt{Dictionary}}) -- Un diccionario que contiene todas las opciones adicionales para cada tipo de codificación.
\end{itemize}

\item[{Returns}] \leavevmode
El cromosoma devuelto en forma de lista.
\item[{Return type}] \leavevmode
List
\end{description}\end{quote}

\end{fulllineitems}
%******* Termina función *******

%******* Empieza función *******
\begin{fulllineitems}
\pysiglinewithargsret{\sphinxbfcode{evaluate\_subchromosomes}}{\emph{complete\_chromosome}, \emph{length\_subchromosomes}, \emph{vector\_variables,number\_of\_decimals}, \emph{representation\_parameters}}{}~
\vspace{-0.1cm}

Tomando en cuenta que el cromosoma ya ha sido creado usando 
los tamaños de los subcromosomas, en esta función se procede 
a evaluar el súper cromosoma partiéndolo en los subcromosomas 
pertinentes y evaluando individualmente cada uno de éstos.

\begin{quote}\begin{description}
\item[{Parameters}] \leavevmode\begin{itemize}
\item \textbf{\texttt{complete\_chromosome}} (\emph{\texttt{List}}) -- El súper cromosoma a ser evaluado.
\item \textbf{\texttt{length\_subchromosomes}} (\emph{\texttt{List}}) -- La lista que contiene los tamaños de cada variable de decisión.
\item \textbf{\texttt{vector\_variables}} (\emph{\texttt{List}}) -- La lista que contiene las variables de decisión, así como sus rangos.
\item \textbf{\texttt{number\_of\_decimals}} (\emph{\texttt{Integer}}) -- El número de decimales que traerá cada variable de decisión.
\item \textbf{\texttt{representation\_parameters}} (\emph{\texttt{Dictionary}}) -- Un diccionario que contiene todas las opciones adicionales para cada tipo de codificación.
\end{itemize}

\item[{Returns}] \leavevmode
Un diccionario que contiene como llave la variable de decisión 
y como valor la evaluación del subcromosoma correspondiente.
\item[{Return type}] \leavevmode
Dictionary
\end{description}\end{quote}

\end{fulllineitems}

Sólo concierne añadir un detalle adicional; se asume por defecto que las
funciones antes mencionadas se encuentran implementadas en cada uno de los
elementos de este módulo, por ello es que primordialmente se mostrarán 
aquéllas que no se contemplen en el esquema original, es decir, funciones 
auxiliares.\break
En el caso muy específico en el que alguna de las funciones obligatorias 
contenga información importante también se adjuntarán en el documento.\medskip\break  
A continuación se develan los elementos que constituyen a 
este módulo:
%******* Termina función *******

%******* Empieza script *******
\subsubsection{BinaryRepresentation (script)}
%Se coloca el vínculo interno procedente de esta misma sección (a_2_1_1).
\label{sec:a_2_1_1}
Contiene todas las funcionalidades requeridas para que se pueda hacer 
uso de una codificación de tipo Binary \textbf{(ó Binaria)}; ésto 
significa que los alelos que conforman al cromosoma serán exclusivamente 
0 ó 1.

%******* Empieza función *******
\begin{fulllineitems}

\pysiglinewithargsret{\sphinxbfcode{binary\_to\_decimal}}{\emph{chromosome}}{}
Método que convierte un número binario a decimal.\break
Este es un ejemplo de método que se puede agregar
adicionalmente siempre y cuando se implementen las 
funciones que se han mencionado ya.

\begin{quote}\begin{description}
\item[{Parameters}] \leavevmode\begin{itemize}
\item \textbf{\texttt{chromosome}} (\emph{\texttt{List}}) -- El cromosoma sobre el cual se hará
la evaluación.
\end{itemize}

\item[{Returns}] \leavevmode
La representación en decimal del número binario.
\item[{Return type}] \leavevmode
Integer
\end{description}\end{quote}

\end{fulllineitems}
%******* Termina función *******

%******* Empieza función *******
\begin{fulllineitems}

\pysiglinewithargsret{\sphinxbfcode{evaluate\_subchromosomes}}{\emph{complete\_chromosome}, \emph{length\_subchromosomes}, \emph{vector\_variables}, \emph{number\_of\_decimals}, \emph{representation\_parameters}}{}
Realiza una evaluación de los subcromosomas para la codificación 
binaria \textbf{(ó Binary)}.\break
En términos generales se toma cada porción del subcrosomoma 
\textbf{(tomando en cuenta que previamente se calcularon sus longitudes)} 
y así se convierte a decimal, considerando la expansión numérica.\break
Posteriormente para obtener el número final se hace lo siguiente:

\begin{center}\(Conversi\acute{o}n(subcromosoma) = A + DN(subcromosoma) \cdot \frac{B - A}{2^M - 1}\)
\end{center}

Donde:

\begin{itemize}
\item \textbf{A} es el límite inferior que toma la variable de 
decisión.
\item \textbf{B} es el límite superior que toma la variable de 
decisión.
\item \textbf{M} es la longitud del subcromosoma asociado a la 
variable de decisión.
\item \textbf{DN (Decimal number)} es el número en decimal del 
subcromosoma asociado a la variable de decisión.
\end{itemize}

\end{fulllineitems}
%******* Termina función *******
%******* Termina script *******

%******* Empieza script *******
\subsubsection{FloatPointRepresentation (script)}
%Se coloca el vínculo interno procedente de esta misma sección (a_2_1_2).
\label{sec:a_2_1_2}
Este script contiene todas las funcionalidades requeridas para 
que se pueda hacer uso de una codificación de tipo Float Point 
\textbf{(ó Punto Flotante)}; ésto significa que los alelos que 
conforman al cromosoma serán números de punto flotante.\break
Un número de punto flotante es aquél que tiene una parte entera 
y una decimal; cabe mencionar que si el número en cuestión no 
tiene expansión decimal, se le considera un número de representación 
Integer \textbf{(ó Entera)}; ésto porque en algunas fuentes se 
manejan la representación de Punto Flotante y Entera por separado.
%******* Termina script *******
%******* Termina módulo *******

%******* Empieza clase *******
\subsection{Community (clase)}
%Se coloca el vínculo interno procedente de esta misma sección (a_2_2).
\label{sec:a_2_2}
%******* Empieza descripción *******
\begin{fulllineitems}

\begin{DUlineblock}{0em}
\item[] Proporciona toda la infraestructura lógica para poder construir 
poblaciones y operar con éstas, además de transacciones relacionadas 
con sus elementos de manera individual.\break
Se le llama Community porque aludiendo a su significado una Community 
\textbf{(ó Comunidad)} consta de al menos una Population \textbf{(o Población)}. 
De esta manera se deduce que en algún momento habrán métodos que 
involucren a más de una Población.
\end{DUlineblock}

\begin{quote}\begin{description}
\item[{Parameters}] \leavevmode\begin{itemize}
\item \textbf{\texttt{vector\_functions}} (\emph{\texttt{List}}) -- Lista que contiene las funciones objetivo previamente 
saneadas por Controller/Controller.py.
\item \textbf{\texttt{vector\_variables}} (\emph{\texttt{List}}) -- Lista que contiene las variables de decisión previamente 
saneadas por Controller/Controller.py.
\item \textbf{\texttt{available\_expressions}} (\emph{\texttt{Dictionary}}) -- Diccionario que contiene algunas funciones escritas como azúcar sintáctica
para que puedan ser utilizadas más fácilmente por el usuario y evaluadas
más ŕapidamente en el programa \textbf{(véase Controller/XML/PythonExpressions.xml)}.
\item \textbf{\texttt{number\_of\_decimals}} (\emph{\texttt{Integer}}) -- El número de decimales que tendrán las soluciones; con este número se determina
en gran medida el tamaño del cromosoma.
\item \textbf{\texttt{representation\_instance}} (\emph{\texttt{Instance}}) -- Instancia de la técnica de representación que eligió el usuario
\textbf{(véase Controller/Verifier.py)}.
\item \textbf{\texttt{representation\_parameters}} (\emph{\texttt{Dictionary}}) -- Diccionario que contiene todos los parámetros adicionales a la técnica
de representación considerada por el usuario.
\item \textbf{\texttt{fitness\_instance}} (\emph{\texttt{Instance}}) -- Instancia de la técnica de Fitness que eligió el usuario
\textbf{(véase Controller/Verifier.py)}.
\item \textbf{\texttt{fitness\_parameters}} (\emph{\texttt{Dictionary}}) -- Diccionario que contiene todos los parámetros adicionales a la técnica
de Fitness seleccionada por el usuario.
\item \textbf{\texttt{sharing\_function\_instance}} (\emph{\texttt{Instance}}) -- Instancia de la técnica de Sharing Function seleccionada por el usuario
\textbf{(véase Controller/Verifier.py)}.
\item \textbf{\texttt{sharing\_function\_parameters}} (\emph{\texttt{Dictionary}}) -- Diccionario que contiene todos los parámetros adicionales a la técnica
de Fitness seleccionada por el usuario.
\item \textbf{\texttt{selection\_instance}} (\emph{\texttt{Instance}}) -- Instancia de la técnica de selección \textbf{(Selection)} elegida por el usuario
\textbf{(véase Controller/Verifier.py)}.
\item \textbf{\texttt{selection\_parameters}} (\emph{\texttt{Dictionary}}) -- Diccionario que contiene todos los parámetros adicionales a la técnica
de selección \textbf{(Selection)} usada por el usuario.
\item \textbf{\texttt{crossover\_instance}} (\emph{\texttt{Instance}}) -- Instancia de la técnica de cruza \textbf{(Crossover)} tomada por el usuario
\textbf{(véase Controller/Verifier.py)}.
\item \textbf{\texttt{crossover\_parameters}} (\emph{\texttt{Dictionary}}) -- Diccionario que contiene todos los parámetros adicionales a la técnica
de cruza \textbf{(Crossover)} manejada por el usuario.
\item \textbf{\texttt{mutation\_instance}} (\emph{\texttt{Instance}}) -- Instancia de la técnica de mutación \textbf{(Mutation)} tomada por el usuario
\textbf{(véase Controller/Verifier.py)}.
\item \textbf{\texttt{mutation\_parameters}} (\emph{\texttt{Dictionary}}) -- Diccionario que contiene todos los parámetros adicionales a la técnica
de mutación \textbf{(Mutation)} seleccionada por el usuario.
\end{itemize}
\item[{Returns}] \leavevmode
Model.Community.Community
\item[{Return type}] \leavevmode
Instance
\end{description}\end{quote}

%******* Termina descripción *******

%******* Empieza función *******
\begin{fulllineitems}

\pysiglinewithargsret{\sphinxbfcode{compare\_dominance}}{\emph{current}, \emph{challenger}, \emph{allowed\_functions}}{}

\begin{notice}{note}{Note:}
Este método es privado.
\end{notice}

Permite realizar la comparación de las funciones objetivo 
de los Individuos current y challenger tomadas una a una 
para indicar así quién es el dominado y quién es el que domina. 
Cabe mencionar que más apropiadamente se le conoce como dominancia 
fuerte de Pareto.

\begin{quote}\begin{description}
\item[{Parameters}] \leavevmode\begin{itemize}
\item \textbf{\texttt{current}} (\emph{\texttt{Instance}}) -- El Individuo inicial para comprobar dominancia.
\item \textbf{\texttt{challenger}} (\emph{\texttt{Instance}}) -- El Individuo que reta al inicial para comprobar dominancia.
\item \textbf{\texttt{allowed\_functions}} (\emph{\texttt{List}}) -- Lista que indica cuáles son las funciones objetivo que deben 
compararse.
\end{itemize}

\item[{Returns}] \leavevmode
True si current domina a challenger, False en otro caso.
\item[{Return type}] \leavevmode
Boolean
\end{description}\end{quote}

\end{fulllineitems}
%******* Termina función *******

%******* Empieza función *******
\begin{fulllineitems}

\pysiglinewithargsret{\sphinxbfcode{get\_best\_individual\_results}}{\emph{population}}{}

\begin{notice}{note}{Note:}
Este método es privado.
\end{notice}

Obtiene los valores de las variables de decisión y de las 
funciones objetivo por cada Individuo.

\begin{quote}\begin{description}
\item[{Parameters}] \leavevmode\begin{itemize}
\item \textbf{\texttt{population}} (\emph{\texttt{List}}) -- Una lista que contiene los mejores Individuos por generación.
\end{itemize}
\item[{Returns}] \leavevmode
Una lista que contiene por un lado la tupla (generacion, funciones)
y por otro la tupla (generación, variables). Esto por cada generación.
\item[{Return type}] \leavevmode
List
\end{description}\end{quote}

\end{fulllineitems}
%******* Termina función *******

%******* Empieza función *******
\begin{fulllineitems}

\pysiglinewithargsret{\sphinxbfcode{get\_pareto\_results}}{\emph{population}}{}

\begin{notice}{note}{Note:}
Este método es privado.
\end{notice}

Obtiene el frente de Pareto, el complemento del frente de Pareto y 
el óptimo de Pareto.\break
Para una mejor orientación léase la parte escrita del proyecto.

\begin{quote}\begin{description}
\item[{Parameters}] \leavevmode\begin{itemize}
\item \textbf{\texttt{population}} (\emph{\texttt{Instance}}) -- La Población sobre la cual se obtendrán estos elementos.
\end{itemize}
\item[{Returns}] \leavevmode
Una lista que contiene el frente de Pareto, su complemento y el óptimo de Pareto.
\item[{Return type}] \leavevmode
List
\end{description}\end{quote}

\end{fulllineitems}
%******* Termina función *******

%******* Empieza función *******
\begin{fulllineitems}

\pysiglinewithargsret{\sphinxbfcode{using\_sharing\_function}}{\emph{individual\_i}, \emph{individual\_j}, \emph{alpha\_share}, \emph{sigma\_share}}{}

\begin{notice}{note}{Note:}
Este método es privado.
\end{notice}

Devuelve un valor que ayuda al cálculo del Sharing Function.\break
A grandes rasgos el sharing function sirve para hacer una 
selección más precisa de los mejores Individuos cuando se da el 
caso de que tienen el mismo número de Individuos dominados.

\begin{quote}\begin{description}
\item[{Parameters}] \leavevmode\begin{itemize}
\item \textbf{\texttt{individual\_i}} (\emph{\texttt{Instance}}) -- Individuo sobre el que se hará la operación.
\item \textbf{\texttt{individual\_j}} (\emph{\texttt{Instance}}) -- Individuo sobre el que se hará la operación.
\item \textbf{\texttt{alpha\_share}} (\emph{\texttt{Float}}) -- El valor necesario para poder calcular la distancia entre Individuos.
\item \textbf{\texttt{sigma\_share}} (\emph{\texttt{Float}}) -- El valor necesario para poder calcular la distancia entre Individuos.
\end{itemize}
\item[{Returns}] \leavevmode
El resultado que contribuirá al sharing function.
\item[{Return type}] \leavevmode
Float
\end{description}\end{quote}

\end{fulllineitems}
%******* Termina función *******

%******* Empieza función *******
\begin{fulllineitems}

\pysiglinewithargsret{\sphinxbfcode{assign\_fonseca\_and\_flemming\_pareto\_rank}}{\emph{population}, \emph{allowed\_functions='All'}}{}
Asigna una puntuación \textbf{(ó rank)} a cada uno de los 
Individuos de una Población con base en su dominancia de Pareto.\break
A grandes rasgos, el algoritmo asigna un rank que consiste en:

\begin{center}\(rank(Individuo) = n\acute{u}mero\_soluciones\_que\_dominan(Individuo) + 1\)
\end{center}

Esta técnica es usada principalmente por M.O.G.A.

\begin{quote}\begin{description}
\item[{Parameters}] \leavevmode\begin{itemize}
\item \textbf{\texttt{population}} (\emph{\texttt{Instance}}) -- La Población sobre la que se hará la operación.
\item \textbf{\texttt{allowed\_functions}} (\emph{\texttt{List}}) -- Lista que contiene las posiciones de las funciones que son admisibles 
para hacer comparaciones. Por defecto tiene el valor ``All''.
\end{itemize}
\end{description}\end{quote}

\end{fulllineitems}
%******* Termina función *******

%******* Empieza función *******
\begin{fulllineitems}

\pysiglinewithargsret{\sphinxbfcode{assign\_goldberg\_pareto\_rank}}{\emph{population}, \emph{additional\_info=False}, \emph{allowed\_functions='All'}}{}
Asigna una puntuación \textbf{(ó rank)} a cada uno de los Individuos 
de una Población con base en su dominancia de Pareto.\break
En términos generales, el algoritmo trabaja con niveles, es decir, 
primero toma los Individuos no dominados y les asigna un valor 0, 
luego los elimina del conjunto y nuevamente aplica la operación sobre 
los no dominados del nuevo conjunto, a los que les asigna el valor 1, 
y así sucesivamente hasta no quedar Individuos.\break
Esta técnica es usada principalmente por N.S.G.A. II.

\begin{quote}\begin{description}
\item[{Parameters}] \leavevmode\begin{itemize}
\item \textbf{\texttt{population}} (\emph{\texttt{Instance}}) -- La Población sobre la que se hará la operación.
\item \textbf{\texttt{additional\_info}} (\emph{\texttt{Boolean}}) -- Un valor que le indica a la función que debe regresar información 
adicional.
\item \textbf{\texttt{allowed\_functions}} (\emph{\texttt{List}}) -- Lista que contiene las posiciones de las funciones que son admisibles 
para hacer comparaciones. Por defecto tiene el valor ``All''.
\end{itemize}

\item[{Returns}] \leavevmode
Si additional\_info es True: un arreglo con dos elementos: en el 
primero se almacena una lista con los niveles de dominancia disponibles 
mientras que el segundo consta de una estructura que contiene todos 
los posibles niveles y asociados a éstos, los cromosomas de los 
Individuos que los conforman.\break
Si additional\_info es False: el método es void \textbf{(no regresa nada)}.
\item[{Return type}] \leavevmode
List
\end{description}\end{quote}

\end{fulllineitems}
%******* Termina función *******

%******* Empieza función *******
\begin{fulllineitems}

\pysiglinewithargsret{\sphinxbfcode{assign\_population\_fitness}}{\emph{population}}{}
Aplica la asignación de Fitness para una Población dada 
usando como base el Ranking de cada Individuo \textbf{(véase Model/Fitness)}.

\begin{quote}\begin{description}
\item[{Parameters}] \leavevmode\begin{itemize}
\item \textbf{\texttt{population}} (\emph{\texttt{Instance}}) -- La Población sobre la que se hará la operación.
\end{itemize}
\end{description}\end{quote}

\end{fulllineitems}
%******* Termina función *******

%******* Empieza función *******
\begin{fulllineitems}

\pysiglinewithargsret{\sphinxbfcode{assign\_zitzler\_and\_thiele\_pareto\_rank}}{\emph{population}, \emph{allowed\_functions='All'}}{}
Asigna una puntuación (rank) a cada uno de los Individuos 
de una Población con base en su dominancia de Pareto.\break
A manera de esbozo, el algoritmo asigna un rank que consiste 
en una razón:

\begin{center}\(rank(Individuo) = \frac{n\acute{u}mero\_soluciones\_dominadas(Individuo)}{tama\tilde{n}o\_poblaci\acute{o}n} + 1\)
\end{center}

Esta técnica es usada principalmente por S.P.E.A. II

\begin{quote}\begin{description}
\item[{Parameters}] \leavevmode\begin{itemize}
\item \textbf{\texttt{population}} (\emph{\texttt{Instance}}) -- La Población sobre la que se hará la operación.
\item \textbf{\texttt{allowed\_functions}} (\emph{\texttt{List}}) -- Lista que contiene las posiciones de las funciones que son admisibles 
para hacer comparaciones. Por defecto tiene el valor ``All''.
\end{itemize}
\end{description}\end{quote}

\end{fulllineitems}
%******* Termina función *******

%******* Empieza función *******
\begin{fulllineitems}

\pysiglinewithargsret{\sphinxbfcode{calculate\_population\_niche\_count}}{\emph{population}}{}
Calcula el valor conocido como niche count que no es 
mas que la suma de los sharing function de todos los 
Individuos j con el Individuo i, con i != j.

\begin{quote}\begin{description}
\item[{Parameters}] \leavevmode\begin{itemize}
\item \textbf{\texttt{population}} (\emph{\texttt{Instance}}) -- Conjunto sobre el que se hará la operación.
\end{itemize}
\end{description}\end{quote}

\end{fulllineitems}
%******* Termina función *******

%******* Empieza función *******
\begin{fulllineitems}

\pysiglinewithargsret{\sphinxbfcode{calculate\_population\_pareto\_dominance}}{\emph{population}, \emph{allowed\_functions}}{}
Realiza la comparación de dominancia entre todos los elementos 
de la Población con base en la evaluación de sus funciones 
objetivo.

\begin{quote}\begin{description}
\item[{Parameters}] \leavevmode\begin{itemize}
\item \textbf{\texttt{population}} (\emph{\texttt{Instance}}) -- La Población sobre la que se hará la operación.
\item \textbf{\texttt{allowed\_functions}} (\emph{\texttt{List}}) -- Lista que indica las funciones objetivo permitidas para hacer la 
comparación.
\end{itemize}
\end{description}\end{quote}

\end{fulllineitems}
%******* Termina función *******

%******* Empieza función *******
\begin{fulllineitems}

\pysiglinewithargsret{\sphinxbfcode{calculate\_population\_shared\_fitness}}{\emph{population}}{}
Calcula el Shared Fitness \textbf{(ó Fitness Compartido)} 
de cada uno de los Individuos de la Población.

\begin{quote}\begin{description}
\item[{Parameters}] \leavevmode\begin{itemize}
\item \textbf{\texttt{population}} (\emph{\texttt{Instance}}) -- Conjunto sobre el que se hará la operación.
\end{itemize}
\end{description}\end{quote}

\end{fulllineitems}
%******* Termina función *******

%******* Empieza función *******
\begin{fulllineitems}

\pysiglinewithargsret{\sphinxbfcode{create\_population}}{\emph{set\_chromosomes}}{}~
\vspace{-0.1cm}
Crea una Población usando un conjunto de 
cromosomas como base.

\begin{quote}\begin{description}
\item[{Parameters}] \leavevmode\begin{itemize}
\item \textbf{\texttt{set\_chromosomes}} (\emph{\texttt{List}}) -- Conjunto de cromosomas.
\end{itemize}
\item[{Returns}] \leavevmode
Model.Community.Population
\item[{Return type}] \leavevmode
Instance
\end{description}\end{quote}

\end{fulllineitems}
%******* Termina función *******

%******* Empieza función *******
\begin{fulllineitems}

\pysiglinewithargsret{\sphinxbfcode{evaluate\_population\_functions}}{\emph{population}}{}
Evalúa cada uno de los subcromosomas de los Individuos de la 
Población \textbf{(Population)}.\break
De manera adicional obtiene el listado de los valores extremos 
tanto de variables de decisión como de funciones objetivo para 
el cálculo del sigma share \textbf{(véase el método using\_sharing\_function)}. 

\begin{quote}\begin{description}
\item[{Parameters}] \leavevmode\begin{itemize}
\item \textbf{\texttt{population}} (\emph{\texttt{Instance}}) -- La Población sobre la que se hará la operación.
\end{itemize}
\end{description}\end{quote}

\end{fulllineitems}
%******* Termina función *******

%******* Empieza función *******
\begin{fulllineitems}

\pysiglinewithargsret{\sphinxbfcode{execute\_crossover\_and\_mutation}}{\emph{selected\_parents\_chromosomes}}{}
Realiza la cruza y mutación de los Individuos. Para el caso 
de la cruza ésta se lleva a cabo siempre entre dos Individuos, 
mientras que la mutación es unaria.

\begin{quote}\begin{description}
\item[{Parameters}] \leavevmode\begin{itemize}
\item \textbf{\texttt{selected\_parents\_chromosomes}} (\emph{\texttt{List}}) -- El conjunto de cromosomas sobre los cuales se aplicarán dichos operadores genéticos.
\end{itemize}
\item[{Returns}] \leavevmode
Una instancia del tipo Model.Community.Population.
\item[{Return type}] \leavevmode
Instance
\end{description}\end{quote}

\end{fulllineitems}
%******* Termina función *******

%******* Empieza función *******
\begin{fulllineitems}

\pysiglinewithargsret{\sphinxbfcode{execute\_selection}}{\emph{parents}}{}
Realiza la ejecución de la técnica de selección por medio 
de una instancia que se creó previamente \textbf{(véase Controller/Verifier.py)}.

\begin{quote}\begin{description}
\item[{Parameters}] \leavevmode\begin{itemize}
\item \textbf{\texttt{parents}} (\emph{\texttt{Instance}}) -- El conjunto de Individuos sobre el cual se aplicará la técnica
\end{itemize}
\item[{Returns}] \leavevmode
Una lista con los cromosomas de aquellos Individuos seleccionados.
\item[{Return type}] \leavevmode
List
\end{description}\end{quote}

\end{fulllineitems}
%******* Termina función *******

%******* Empieza función *******
\begin{fulllineitems}

\pysiglinewithargsret{\sphinxbfcode{get\_best\_individual}}{\emph{population}}{}
Obtiene el mejor Individuo dentro de una Población. 
Para estos fines el mejor Individuo es aquél que tenga 
mejor dominancia.

\begin{quote}\begin{description}
\item[{Parameters}] \leavevmode\begin{itemize}
\item \textbf{\texttt{population}} (\emph{\texttt{Instance}}) -- La Población sobre la cual se hará la búsqueda.
\end{itemize}
\item[{Returns}] \leavevmode
El Individuo que cumple con la característica de la mayor dominancia.
\item[{Return type}] \leavevmode
Instance
\end{description}\end{quote}

\end{fulllineitems}
%******* Termina función *******

%******* Empieza función *******
\begin{fulllineitems}

\pysiglinewithargsret{\sphinxbfcode{get\_results}}{\emph{best\_individual\_along\_generations}, \emph{external\_set\_population}}{}
Recolecta la información y la almacena en una estructura 
que contiene dos categorías principales: funciones objetivo 
y variables de decisión. Por cada una existen las subcategorías 
Pareto y mejor Individuo, en referencia al óptimo o frente de 
Pareto \textbf{(según corresponda)} y a los valores del mejor 
Individuo por generación \textbf{(véase View/Additional/ResultsGrapher/GraphFrame.py)}.

\begin{quote}\begin{description}
\item[{Parameters}] \leavevmode\begin{itemize}
\item \textbf{\texttt{best\_individual\_along\_generations}} (\emph{\texttt{List}}) -- Una lista que contiene los mejores Individuos por generación.
\item \textbf{\texttt{external\_set\_population}} (\emph{\texttt{Instance}}) -- La Población sobre la cual se efectuarán las operaciones.
\end{itemize}
\item[{Returns}] \leavevmode
Un diccionario con los elementos mostrados en la descripción.
\item[{Return type}] \leavevmode
Dictionary
\end{description}\end{quote}

\end{fulllineitems}
%******* Termina función *******

%******* Empieza función *******
\begin{fulllineitems}
\pysiglinewithargsret{\sphinxbfcode{init\_population}}{\emph{population\_size}}{}~
\vspace{-0.1cm}
Crea una Población de manera aleatoria.

\begin{quote}\begin{description}
\item[{Parameters}] \leavevmode\begin{itemize}
\item \textbf{\texttt{population\_size}} (\emph{\texttt{Integer}}) -- El tamaño de la Población.
\end{itemize}
\item[{Returns}] \leavevmode
Model.Community.Community.Population
\item[{Return type}] \leavevmode
Instance
\end{description}\end{quote}

\end{fulllineitems}
\end{fulllineitems}

La clase en cuestión se apoya de los siguientes 
elementos:
%******* Termina función *******
%******* Termina clase *******

%******* Empieza clase *******
\subsubsection{Population (clase)}
%Se coloca el vínculo interno procedente de esta misma sección (a_2_2_1).
\label{sec:a_2_2_1}

%******* Empieza descripción *******
\begin{fulllineitems}

\begin{DUlineblock}{0em}
\item[] Consiste en un conjunto de instancias de la clase Individual, 
proporcionando además métodos y atributos que se manifiestan 
tanto en grupo como de manera individual.
\end{DUlineblock}

\begin{quote}\begin{description}
\item[{Parameters}] \leavevmode\begin{itemize}
\item \textbf{\texttt{population\_size}} (\emph{\texttt{Integer}}) -- El tamaño de la Población.
\item \textbf{\texttt{vector\_functions}} (\emph{\texttt{List}}) -- Lista con las funciones objetivo.
\item \textbf{\texttt{vector\_variables}} (\emph{\texttt{List}}) -- Lista con las variables de decisión y sus rangos.
\item \textbf{\texttt{available\_expressions}} (\emph{\texttt{Dictionary}}) -- Diccionario que contiene algunas funciones escritas como azúcar sintáctica
para que puedan ser utilizadas más fácilmente por el usuario y evaluadas más ŕapidamente en el programa \textbf{(véase Controller/XML/PythonExpressions.xml)}.
\item \textbf{\texttt{number\_of\_decimals}} (\emph{\texttt{Integer}}) -- Número de decimales que tendrá cada solución \textbf{(tanto en variables de decisión como funciones objetivo)}.
\end{itemize}
\item[{Returns}] \leavevmode
Model.Community.Population
\item[{Return type}] \leavevmode
Instance
\end{description}\end{quote}

%******* Termina descripción *******

%******* Empieza función *******
\begin{fulllineitems}

\pysiglinewithargsret{\sphinxbfcode{add\_individual}}{\emph{position}, \emph{complete\_chromosome}}{}
Añade un Individuo a la Población.

\begin{quote}\begin{description}
\item[{Parameters}] \leavevmode\begin{itemize}
\item \textbf{\texttt{position}} (\emph{\texttt{Integer}}) -- La posición dentro del arreglo de Individuos 
donde se colocará el nuevo elemento.
\item \textbf{\texttt{complete\_chromosome}} (\emph{\texttt{Array}}) -- El cromosoma del Individuo.
\end{itemize}
\end{description}\end{quote}

\end{fulllineitems}
%******* Termina función *******

%******* Empieza función *******
\begin{fulllineitems}

\pysiglinewithargsret{\sphinxbfcode{calculate\_population\_properties}}{}{}
Calcula atributos individuales con base en los 
valores de toda la Población.

\end{fulllineitems}
%******* Termina función *******

%******* Empieza función *******
\begin{fulllineitems}

\pysiglinewithargsret{\sphinxbfcode{get\_decision\_variables\_extreme\_values}}{}{}
Regresa el listado de los valores máximo y mínimo de las 
variables de decisión para el cálculo de sigma share.

\begin{quote}\begin{description}
\item[{Returns}] \leavevmode
Una colección con los valores máximo y mínimo para las
variables de decisión.
\item[{Return type}] \leavevmode
Dictionary
\end{description}\end{quote}

\end{fulllineitems}
%******* Termina función *******

%******* Empieza función *******
\begin{fulllineitems}

\pysiglinewithargsret{\sphinxbfcode{get\_individuals}}{}{}
Regresa los Individuos de la Población.

\begin{quote}\begin{description}
\item[{Returns}] \leavevmode
Estructura que contiene a los Individuos de la Población.
\item[{Return type}] \leavevmode
Array
\end{description}\end{quote}

\end{fulllineitems}
%******* Empieza función *******

%******* Termina función *******
\begin{fulllineitems}

\pysiglinewithargsret{\sphinxbfcode{get\_length\_vector\_functions}}{}{}
Regresa el número de elementos del vector de funciones 
objetivo.
\begin{quote}\begin{description}
\item[{Returns}] \leavevmode
Número de funciones objetivo.
\item[{Return type}] \leavevmode
Integer
\end{description}\end{quote}

\end{fulllineitems}
%******* Termina función *******

%******* Empieza función *******
\begin{fulllineitems}

\pysiglinewithargsret{\sphinxbfcode{get\_objective\_functions\_extreme\_values}}{}{}
Regresa el listado de los valores máximo y mínimo de las 
funciones objetivo para el cálculo de sigma share.

\begin{quote}\begin{description}
\item[{Returns}] \leavevmode
El listado con los valores máximo y mínimo para las
funciones objetivo.
\item[{Return type}] \leavevmode
List
\end{description}\end{quote}

\end{fulllineitems}
%******* Termina función *******

%******* Empieza función *******
\begin{fulllineitems}

\pysiglinewithargsret{\sphinxbfcode{get\_size}}{}{}
Otorga el tamaño de la Población.

\begin{quote}\begin{description}
\item[{Returns}] \leavevmode
El tamaño de la Población.
\item[{Return type}] \leavevmode
Integer
\end{description}\end{quote}

\end{fulllineitems}
%******* Termina función *******

%******* Empieza función *******
\begin{fulllineitems}

\pysiglinewithargsret{\sphinxbfcode{get\_total\_expected\_value}}{}{}
Regresa el valor esperado de la Población.

\begin{quote}\begin{description}
\item[{Returns}] \leavevmode
El valor esperado.
\item[{Return type}] \leavevmode
Float
\end{description}\end{quote}

\end{fulllineitems}
%******* Termina función *******

%******* Empieza función *******
\begin{fulllineitems}

\pysiglinewithargsret{\sphinxbfcode{get\_total\_fitness}}{}{}
Captura el Fitness total de la Población.

\begin{quote}\begin{description}
\item[{Returns}] \leavevmode
El valor del Fitness poblacional.
\item[{Return type}] \leavevmode
Float
\end{description}\end{quote}

\end{fulllineitems}
%******* Termina función *******

%******* Empieza función *******
\begin{fulllineitems}

\pysiglinewithargsret{\sphinxbfcode{get\_vector\_variables}}{}{}
Regresa el vector de variables de decisión.

\begin{quote}\begin{description}
\item[{Returns}] \leavevmode
Conjunto que contiene las variables de decisión con sus rangos.
\item[{Return type}] \leavevmode
List
\end{description}\end{quote}

\end{fulllineitems}
%******* Termina función *******

%******* Empieza función *******
\begin{fulllineitems}

\pysiglinewithargsret{\sphinxbfcode{print\_info}}{}{}
Imprime en texto las características de los Individuos
de la Población, tanto grupales como individuales 
\textbf{(en consola)}.

\end{fulllineitems}
%******* Termina función *******

%******* Empieza función *******
\begin{fulllineitems}

\pysiglinewithargsret{\sphinxbfcode{set\_decision\_variables\_extreme\_values}}{\emph{decision\_variables\_extreme\_values}}{}
Actualiza el listado de valores máximo y mínimo de las
variables de decisión para el cálculo de sigma share.

\begin{quote}\begin{description}
\item[{Parameters}] \leavevmode\begin{itemize}
\item \textbf{\texttt{decision\_variables\_extreme\_values}} (\emph{\texttt{Dictionary}}) -- Un conjunto con los valores máximo y mínimo
de cada una de las variables de decisión.
\end{itemize}
\end{description}\end{quote}

\end{fulllineitems}
%******* Termina función *******

%******* Empieza función *******
\begin{fulllineitems}

\pysiglinewithargsret{\sphinxbfcode{set\_objective\_functions\_extreme\_values}}{\emph{objective\_functions\_extreme\_values}}{}
Actualiza el listado de valores máximo y mínimo de las
funciones objetivo para el cálculo de sigma share.

\begin{quote}\begin{description}
\item[{Parameters}] \leavevmode\begin{itemize}
\item \textbf{\texttt{objective\_functions\_extreme\_values}} (\emph{\texttt{List}}) -- Una lista con los valores máximo y mínimo
de cada una de las funciones objetivo.
\end{itemize}
\end{description}\end{quote}

\end{fulllineitems}
%******* Termina función *******

%******* Empieza función *******
\begin{fulllineitems}

\pysiglinewithargsret{\sphinxbfcode{set\_total\_fitness}}{\emph{value}}{}
Actualiza el Fitness total de la Población.

\begin{quote}\begin{description}
\item[{Parameters}] \leavevmode\begin{itemize}
\item \textbf{\texttt{value}} (\emph{\texttt{Float}}) -- El valor a actualizar.
\end{itemize}
\end{description}\end{quote}

\end{fulllineitems}
%******* Termina función *******

%******* Empieza función *******
\begin{fulllineitems}

\pysiglinewithargsret{\sphinxbfcode{shuffle\_individuals}}{}{}
Desordena los elementos de la Población.

\end{fulllineitems}
%******* Termina función *******

%******* Empieza función *******
\begin{fulllineitems}

\pysiglinewithargsret{\sphinxbfcode{sort\_individuals}}{\emph{method}, \emph{is\_descendent}}{}
Ordena los Individuos de acuerdo a algún 
criterio dado.

\begin{quote}\begin{description}
\item[{Parameters}] \leavevmode\begin{itemize}
\item \textbf{\texttt{method}} (\emph{\texttt{String}}) -- El método o atributo sobre el cual se hará la comparación.
\item \textbf{\texttt{is\_descendent}} (\emph{\texttt{Boolean}}) -- Indica si el ordenamiento es ascendente o descendente.
\end{itemize}
\end{description}\end{quote}

\end{fulllineitems}

\end{fulllineitems}

La clase actual tiene como base el siguiente 
elemento:
%******* Termina función *******
%******* Termina clase *******

%******* Empieza clase *******
\paragraph{Individual (clase)}
%Se coloca el vínculo interno procedente de esta misma sección (a_2_2_1_1).
\label{sec:a_2_2_1_1}

%******* Empieza descripción *******
\begin{fulllineitems}

\begin{DUlineblock}{0em}
\item[] La base de toda operación lógica.\break
Consiste en una abstracción de un elemento simple en función
de un ecosistema.\break
Si bien la parte esencial es el cromosoma, en esta implementación 
se añaden algunos elementos extra con la finalidad de facilitar 
ciertas operaciones.
\end{DUlineblock}

\begin{quote}\begin{description}
\item[{Parameters}] \leavevmode\begin{itemize}
\item \textbf{\texttt{complete\_chromosome}} (\emph{\texttt{Array}}) -- El cromosoma que conformará al Individuo.
\item \textbf{\texttt{vector\_functions}} (\emph{\texttt{List}}) -- Lista que contiene a las funciones objetivo.
\item \textbf{\texttt{available\_expressions}} (\emph{\texttt{Dictionary}}) -- Diccionario que contiene algunas funciones escritas como azúcar sintáctica
para que puedan ser utilizadas más fácilmente por el usuario y evaluadas más ŕapidamente en el programa \textbf{(véase Controller/XML/PythonExpressions.xml)}.
\item \textbf{\texttt{number\_of\_decimals}} (\emph{\texttt{Integer}}) -- El número de decimales que deberá tener cada solución, influye en el
comportamiento del cromosoma.
\end{itemize}
\item[{Returns}] \leavevmode
Model.Community.Population.Individual
\item[{Return type}] \leavevmode
Instance
\end{description}\end{quote}

%******* Termina descripción *******

%******* Empieza función *******
\begin{fulllineitems}

\pysiglinewithargsret{\sphinxbfcode{evaluate\_single\_function}}{\emph{function}, \emph{expressions}}{}

\begin{notice}{note}{Note:}
Este método es privado.
\end{notice}

Evalúa una función objetivo.

\begin{quote}\begin{description}
\item[{Parameters}] \leavevmode\begin{itemize}
\item \textbf{\texttt{function}} (\emph{\texttt{String}}) -- La función que será evaluada.
\item \textbf{\texttt{expressions}} (\emph{\texttt{Dictionary}}) -- El diccionario que ayuda a evaluar la función.
Expressions = variables + constantes + funciones built-in.
\end{itemize}
\item[{Returns}] \leavevmode
La función evaluada.
\item[{Return type}] \leavevmode
Float
\end{description}\end{quote}

\end{fulllineitems}
%******* Termina función *******

%******* Empieza función *******
\begin{fulllineitems}

\pysiglinewithargsret{\sphinxbfcode{evaluate\_functions}}{\emph{decision\_variables}}{}
Evalúa todas las funciones objetivo.

\begin{quote}\begin{description}
\item[{Parameters}] \leavevmode\begin{itemize}
\item \textbf{\texttt{decision\_variables}} (\emph{\texttt{List}}) -- El vector de variables de decisión.
\end{itemize}
\end{description}\end{quote}

\end{fulllineitems}
%******* Termina función *******

%******* Empieza función *******
\begin{fulllineitems}

\pysiglinewithargsret{\sphinxbfcode{get\_complete\_chromosome}}{}{}
Regresa el cromosoma del Individuo.

\begin{quote}\begin{description}
\item[{Returns}] \leavevmode
El cromosoma.
\item[{Return type}] \leavevmode
Array
\end{description}\end{quote}

\end{fulllineitems}
%******* Termina función *******

%******* Empieza función *******
\begin{fulllineitems}

\pysiglinewithargsret{\sphinxbfcode{get\_decision\_variables}}{}{}~
\vspace{-0.2cm}

Da el vector de variables de decisión.

\begin{quote}\begin{description}
\item[{Returns}] \leavevmode
El vector de variables de decisión.
\item[{Return type}] \leavevmode
List
\end{description}\end{quote}

\end{fulllineitems}
%******* Termina función *******

%******* Empieza función *******
\begin{fulllineitems}

\pysiglinewithargsret{\sphinxbfcode{get\_evaluated\_functions}}{}{}
Regresa las funciones objetivo evaluadas.

\begin{quote}\begin{description}
\item[{Returns}] \leavevmode
Las funciones objetivo evaluadas.
\item[{Return type}] \leavevmode
List
\end{description}\end{quote}

\end{fulllineitems}
%******* Termina función *******

%******* Empieza función *******
\begin{fulllineitems}

\pysiglinewithargsret{\sphinxbfcode{get\_expected\_value}}{}{}~
\vspace{-0.2cm}

Se obtiene el valor esperado del Individuo.\break
Por definición, el valor esperado es el número de posibles
hijos que puede tener un Individuo. Mientras más apto, más 
hijos.

\begin{quote}\begin{description}
\item[{Returns}] \leavevmode
El valor esperado.
\item[{Return type}] \leavevmode
Float
\end{description}\end{quote}

\end{fulllineitems}
%******* Termina función *******

%******* Empieza función *******
\begin{fulllineitems}

\pysiglinewithargsret{\sphinxbfcode{get\_fitness}}{}{}
Regresa el Fitness del Individuo.
\begin{quote}\begin{description}
\item[{Returns}] \leavevmode
El Fitness.
\item[{Return type}] \leavevmode
Float
\end{description}\end{quote}

\end{fulllineitems}
%******* Termina función *******

%******* Empieza función *******
\begin{fulllineitems}

\pysiglinewithargsret{\sphinxbfcode{get\_niche\_count}}{}{}
Regresa el valor niche para el Individuo.

\begin{quote}\begin{description}
\item[{Returns}] \leavevmode
El tamaño niche.
\item[{Return type}] \leavevmode
Float
\end{description}\end{quote}

\end{fulllineitems}
%******* Termina función *******

%******* Empieza función *******
\begin{fulllineitems}

\pysiglinewithargsret{\sphinxbfcode{get\_pareto\_dominated}}{}{}
Regresa el número de soluciones que dominan al 
Individuo actual.

\begin{quote}\begin{description}
\item[{Returns}] \leavevmode
El número de soluciones que dominan a la actual.
\item[{Return type}] \leavevmode
Integer
\end{description}\end{quote}

\end{fulllineitems}
%******* Termina función *******

%******* Empieza función *******
\begin{fulllineitems}

\pysiglinewithargsret{\sphinxbfcode{get\_pareto\_dominates}}{}{}
Regresa el número de soluciones que son dominadas por 
el actual Individuo.

\begin{quote}\begin{description}
\item[{Returns}] \leavevmode
El número de soluciones dominadas.
\item[{Return type}] \leavevmode
Integer
\end{description}\end{quote}

\end{fulllineitems}
%******* Termina función *******

%******* Empieza función *******
\begin{fulllineitems}

\pysiglinewithargsret{\sphinxbfcode{get\_rank}}{}{}
Regresa la puntuación \textbf{(rank)} que se le 
designó al Individuo \textbf{(véase Model/Community/Community.py)}.

\begin{quote}\begin{description}
\item[{Returns}] \leavevmode
El rango.
\item[{Return type}] \leavevmode
Float
\end{description}\end{quote}

\end{fulllineitems}
%******* Termina función *******

%******* Empieza función *******
\begin{fulllineitems}

\pysiglinewithargsret{\sphinxbfcode{get\_vector\_functions}}{}{}
Obtiene el vector de funciones objetivo.

\begin{quote}\begin{description}
\item[{Returns}] \leavevmode
El vector de funciones objetivo.
\item[{Return type}] \leavevmode
List
\end{description}\end{quote}

\end{fulllineitems}
%******* Termina función *******

%******* Empieza función *******
\begin{fulllineitems}

\pysiglinewithargsret{\sphinxbfcode{print\_info}}{}{}
Imprime las características básicas del 
Individuo \textbf{(en consola)}.

\end{fulllineitems}
%******* Termina función *******

%******* Empieza función *******
\begin{fulllineitems}

\pysiglinewithargsret{\sphinxbfcode{set\_expected\_value}}{\emph{value}}{}
Actualiza el valor esperado del Individuo.

\begin{quote}\begin{description}
\item[{Parameters}] \leavevmode\begin{itemize}
\item \textbf{\texttt{value}} (\emph{\texttt{Float}}) -- El valor a actualizar.
\end{itemize}
\end{description}\end{quote}

\end{fulllineitems}
%******* Termina función *******

%******* Empieza función *******
\begin{fulllineitems}

\pysiglinewithargsret{\sphinxbfcode{set\_fitness}}{\emph{value}}{}
Actualiza el valor del Fitness.
\begin{quote}\begin{description}

\item[{Parameters}] \leavevmode\begin{itemize}
\item \textbf{\texttt{value}} (\emph{\texttt{Float}}) -- El valor a actualizar.
\end{itemize}
\end{description}\end{quote}

\end{fulllineitems}
%******* Termina función *******

%******* Empieza función *******
\begin{fulllineitems}

\pysiglinewithargsret{\sphinxbfcode{set\_niche\_count}}{\emph{value}}{}
Actualiza el valor niche.

\begin{quote}\begin{description}
\item[{Parameters}] \leavevmode\begin{itemize}
\item \textbf{\texttt{value}} (\emph{\texttt{Float}}) -- El valor a actualizar.
\end{itemize}
\end{description}\end{quote}

\end{fulllineitems}
%******* Termina función *******

%******* Empieza función *******
\begin{fulllineitems}

\pysiglinewithargsret{\sphinxbfcode{set\_pareto\_dominated}}{\emph{value}}{}
Actualiza el número de soluciones que dominan a 
la solución actual.

\begin{quote}\begin{description}
\item[{Parameters}] \leavevmode\begin{itemize}
\item \textbf{\texttt{value}} (\emph{\texttt{Integer}}) -- El valor a actualizar.
\end{itemize}
\end{description}\end{quote}

\end{fulllineitems}
%******* Termina función *******

%******* Empieza función *******
\begin{fulllineitems}

\pysiglinewithargsret{\sphinxbfcode{set\_pareto\_dominates}}{\emph{value}}{}
Actualiza el número de soluciones dominadas por el
Individuo actual.

\begin{quote}\begin{description}
\item[{Parameters}] \leavevmode\begin{itemize}
\item \textbf{\texttt{value}} (\emph{\texttt{Integer}}) -- El valor a actualizar.
\end{itemize}
\end{description}\end{quote}

\end{fulllineitems}
%******* Termina función *******

%******* Empieza función *******
\begin{fulllineitems}

\pysiglinewithargsret{\sphinxbfcode{set\_rank}}{\emph{rank}}{}
Actualiza el rango del Individuo.

\begin{quote}\begin{description}
\item[{Parameters}] \leavevmode\begin{itemize}
\item \textbf{\texttt{rank}} (\emph{\texttt{Float}}) -- El valor a actualizar.
\end{itemize}
\end{description}\end{quote}

\end{fulllineitems}

\end{fulllineitems}
%******* Termina función *******
%******* Termina clase *******

%******* Empieza módulo *******
\subsection{Fitness (módulo)}
%Se coloca el vínculo interno procedente de esta misma sección (a_2_3).
\label{sec:a_2_3}
Este módulo provee técnicas que calculan el Fitness 
\textbf{(ó Aptitud)} de los Individuals \textbf{(ó Individuos)} 
de una Population \textbf{(ó Población)}.\medskip\break
Se entiende por Fitness a un número que indica la calidad del
Individuo \textbf{(en particular de sus variables de decisión)} 
frente a las funciones objetivo al momento de ser evaluadas, esto 
es, a mayor Fitness, mayor es la probalidad de que las variables 
de decisión del Individuo sean la solución óptima para las funciones 
objetivo.\medskip\break
La asignación del Fitness depende en gran medida del ranking que 
se les haya otorgado a los Individuos previamente 
\textbf{(véase Model/Community/Community.py)}.\break
Indirectamente, esto nos indica que un Individuo con un Fitness alto
tiene más probabilidades de ser elegido en los métodos de Selection 
\textbf{(ó Selección)} y propagar su carga genética; así en las 
funciones de dicha sección \textbf{(Model/Operator/Selection)} el 
criterio para escoger a un Individuo está basado comúnmente en su 
Fitness.\medskip\break
Al final la meta es que el usuario cree sus propias versiones de 
asignación de Fitness, para lo cual es imperativo que, además de 
agregar la descripción de la codificación a Controller/XML/Features.xml 
\textbf{(véase el archivo mencionado en la sección de código)}, se 
implemente la siguiente función:

%******* Empieza función *******
\begin{fulllineitems}

\pysiglinewithargsret{\sphinxbfcode{assign\_fitness}}{\emph{population}, \emph{fitness\_parameters}}{}
Realiza la asignación de Fitness de los Individuos.\break
Dentro de esta se suelen usar métodos de la clase Population 
\textbf{(véase Model/}\break\textbf{Community/Population/Population.py)} y de 
la clase Individual \textbf{(véase Model/}\break\textbf{Community/Population/Individual/Individual.py)}, 
por lo que es muy recomendable que el usuario verifique las 
funciones disponibles. Algunas de las que se ocupan más 
frecuentemente son:

\begin{itemize}
\item \textbf{get\_rank (Individual)}
\item \textbf{set\_fitness (Individual).}
\item \textbf{set\_total\_fitness (Population)}
\item \textbf{calculate\_population\_properties (Population)}
\end{itemize}

\begin{quote}\begin{description}
\item[{Parameters}] \leavevmode\begin{itemize}
\item \textbf{\texttt{population}} (\emph{\texttt{Instance}}) -- La Población sobre la cual se hará el cálculo de Fitness por cada Individuo.
\item \textbf{\texttt{fitness\_parameters}} (\emph{\texttt{Dictionary}}) -- Un diccionario que puede contener opciones adicionales para el cálculo
de Fitness.
\end{itemize}
\end{description}\end{quote}

\end{fulllineitems}

Sólo concierne añadir un detalle adicional; se asume por defecto que las
funciones antes mencionadas se encuentran implementadas en cada uno de los
elementos de este módulo, por ello es que primordialmente se mostrarán 
aquéllas que no se contemplen en el esquema original, es decir, funciones 
auxiliares.\break
En el caso muy específico en el que alguna de las funciones obligatorias 
contenga información importante también se adjuntarán en el documento.\medskip\break  
Ahora se muestran los elementos que conforman el 
módulo actual:
%******* Termina función *******

%******* Empieza script *******
\subsubsection{LinearRankingFitness (script)}
%Se coloca el vínculo interno procedente de esta misma sección (a_2_3_1).
\label{sec:a_2_3_1}
Se implementa la asignación de Fitness conocida como 
Linear Ranking \textbf{(ó Ranking Lineal)}. Es denominada 
así porque el Fitness se asigna con una función lineal que 
tiene como fundamento la posición que ocupa el Individuo dentro 
de la Población.\break
El procedimiento es: tomando en cuenta el ranking asignado a 
los Individuals \textbf{(ó Individuos)} por la clase Community 
\textbf{(véase Model/Community/Community.py)} se ordenan de 
acuerdo a este valor y entonces el Fitness se basa en la posición 
que cada uno de los Individuos ocupa. Más en específico, el Fitness 
está proporcionado por la siguiente fórmula:

\begin{center}\(Fitness(Individuo) = 2 - SP + \frac{2 \cdot (SP - 1) \cdot posici\acute{o}n(Individuo)}{tama\tilde{n}o\_poblaci\acute{o}n - 1}\)
\end{center}

Donde:

\begin{itemize}
\item \textbf{SP (Selective Pressure ó Presión Selectiva)} es un valor que oscila entre 1 y 2.
\item \textbf{Posición(Individuo)} es la que ocupa el Individuo de acuerdo al rank.
\end{itemize}

Haciendo un análisis somero en la fórmula, se puede apreciar que los
Individuos con mejor Fitness serán aquéllos que se encuentren en las 
últimas posiciones, sin embargo los rankings que se manejan en este 
proyecto son inversamente proporcionales a la calidad de los Individuos 
\textbf{(véase Model/Community/Community.py)}; por ello es importante 
ordenar a los Individuos de manera descendente para que la operación 
tenga sentido. La función encargada de esto se llama sort\_individuals 
y está en \textbf{Model/Community/}\break\textbf{Population/Population.py}.
%******* Termina script *******

%******* Empieza script *******
\subsubsection{NonLinearRankingFitness (script)}
%Se coloca el vínculo interno procedente de esta misma sección (a_2_3_2).
\label{sec:a_2_3_2}
Se implementa la asignación de Fitness conocida como Non-Linear Ranking
\textbf{(ó Ranking No Lineal)} que, a diferencia de los demás métodos, 
la aplica usando como base la posición del Individual \textbf{(ó Individuo)} 
en la Population \textbf{(ó Población)} como resultado de las operaciones 
de ranking \textbf{(véase Model/Community/Community.py)}.\break
Posteriormente el Fitness se constituye tomando la posición del 
Individuo y una función polinomial \textbf{(la cual es una función no lineal, de ahí el nombre)}.
La fórmula es la siguiente:

\begin{center}\(Fitness(Individuo) = \frac{TP \cdot X^{posici\acute{o}n(Individuo)}}{\sum_{i=1}^{TP}X^{i - 1}}\)
\end{center}

Donde:

\begin{itemize}
\item \textbf{TP} es el tamaño de la Población.
\item \textbf{Posición(Individuo)} es la que ocupa éste de acuerdo al ranking previo.
\item \textbf{X} es la solución al polinomio: \((SP - TP) \cdot X^{TP - 1} + SP \cdot X^{TP - 2} + ... + SP \cdot X + SP = 0\)
\item \textbf{SP (Selective Pressure ó Presión Selectiva)} varía entre 1 y 2.
\end{itemize}

Haciendo un análisis somero en la fórmula, se puede apreciar que los
Individuos con mejor Fitness serán aquéllos que se encuentren en las 
últimas posiciones, sin embargo los rankings que se manejan en este 
proyecto son inversamente proporcionales a la calidad de los Individuos 
\textbf{(véase Model/Community/Community.py)}; por ello es importante 
ordenar a los Individuos de manera descendente para que la operación 
tenga sentido. La función encargada de esto se llama sort\_individuals 
y está en \textbf{Model/Community/}\break\textbf{Population/Population.py}.

%******* Empieza función *******
\begin{fulllineitems}

\pysiglinewithargsret{\sphinxbfcode{calculate\_root}}{\emph{polynome}, \emph{x\_0}, \emph{epsilon}}{}~\begin{description}
\item[] Calcula la solución de un polinomio usando el método 
Newton-Raphson.\break
A grandes rasgos el funcionamiento es el siguiente:
\end{description}

Tomando como base el punto \(x_0\) se obtiene \(x_1\) así:

\begin{center}\(x_1 = x_0 - \frac{f(x_0)}{f'(x_0)}\)
\end{center}

Una vez obtenido \(x_1\) se calcula \(x_2\) de la misma manera:

\begin{center}\(x_2 = x_1 - \frac{f(x_1)}{f'(x_1)}\)
\end{center}

El proceso se repite para `n' iteraciones hasta que 
el valor alcance la precisión de epsilon ó el polinomio ya 
no tenga más derivadas. Concretando lo anterior:

\begin{center}\(x_{n+1} = x_n - \frac{f(x_n)}{f'(x_n)}\)
\end{center}

Cuando \(x_{n+1}\) se acerque a epsilon ó cuando el
polinomio no sea más derivable el método se detendrá.

\begin{quote}\begin{description}
\item[{Parameters}] \leavevmode\begin{itemize}
\item \textbf{\texttt{polynome}} (\emph{\texttt{List}}) -- El polinomio en el que se buscará la solución.
\item \textbf{\texttt{x\_0}} (\emph{\texttt{Float}}) -- el punto sobre el que se hará la evaluación del polinomio.
\item \textbf{\texttt{epsilon}} (\emph{\texttt{Float}}) -- La precisión decimal que se necesita para poder devolver
el resultado.
\end{itemize}
\item[{Returns}] \leavevmode
La solución al polinomio.
\item[{Return type}] \leavevmode
Float
\end{description}\end{quote}

\end{fulllineitems}
%******* Termina función *******

%******* Empieza función *******
\begin{fulllineitems}

\pysiglinewithargsret{\sphinxbfcode{derivate}}{\emph{polynome}}{}
Método que calcula la derivada de un polinomio, 
modificando directamente éste sin regresar nada.

\begin{quote}\begin{description}
\item[{Parameters}] \leavevmode\begin{itemize}
\item \textbf{\texttt{polynome}} (\emph{\texttt{List}}) -- El polinomio inicial.
\end{itemize}
\end{description}\end{quote}

\end{fulllineitems}
%******* Termina función *******

%******* Empieza función *******
\begin{fulllineitems}

\pysiglinewithargsret{\sphinxbfcode{evaluate\_polynome}}{\emph{polynome}, \emph{x}}{}
Evalúa un polinomio en un cierto valor.

\begin{quote}\begin{description}
\item[{Parameters}] \leavevmode\begin{itemize}
\item \textbf{\texttt{polynome}} (\emph{\texttt{List}}) -- El polinomio a evaluar.
\item \textbf{\texttt{x}} (\emph{\texttt{Float}}) -- El valor sobre el que se evaluará el polinomio.
\end{itemize}
\item[{Returns}] \leavevmode
La evaluación del polinomio.
\item[{Return type}] \leavevmode
Float
\end{description}\end{quote}

\end{fulllineitems}
%******* Termina función *******
%******* Termina script *******

%******* Empieza script *******
\subsubsection{ProportionalFitness (script)}
%Se coloca el vínculo interno procedente de esta misma sección (a_2_3_3).
\label{sec:a_2_3_3}
Se desarrolla la asignación de Fitness conocida como 
Proportional \textbf{(ó Proporcional)}.\break
La función \textbf{(ó fórmula)} utilizada es la siguiente:

\begin{center}\(Fitness(Individuo) = \frac{F_0(Individuo)}{\sum_{i=1}^{tama\tilde{n}o\_poblaci\acute{o}n}F_0(Individuo_i)}\)
\end{center}

Donde:

\begin{itemize}
\item \(F_0\) es conocido como el valor de la función objetivo del Individuo. Nótese que \(F_0\) debe ser proporcional al Fitness del Individuo.
\end{itemize}

De acuerdo a la información provista anteriormente, la asignación 
es llamada así porque, como dice el nombre, el Fitness de un Individuo 
corresponde a la parte proporcional de la cantidad total de \(F_0\) de 
la Population \textbf{(ó Población)}.\break
De esta manera es posible ajustar los valores para que no existan 
Fitness dispares. Con respecto de \(F_0\) es importante considerar 
que, dado que se está manejando un sistema multi objetivo puede haber 
más de un valor en existencia,  por ello se necesita una cantidad que 
conjunte estas evaluaciones el cual es el rank, sin embargo el rank 
es inversamente proporcional a la calidad de un Individuo.\break
Entonces se debe hacer una modificación para garantizar que exista un valor
proporcional al Fitness del Individuo, por lo cual \(F_0\) se reescribe así:

\begin{center}\(F_0(Individuo) = tama\tilde{n}o\_poblaci\acute{o}n - rank(Individuo)\)
\end{center}

Reescribiendo la fórmula inicial se tiene lo siguiente:

\begin{center}\(Fitness(Individuo) = \frac{tama\tilde{n}o\_poblaci\acute{o}n - rank(Individuo)}{\sum_{i=1}^{tama\tilde{n}o\_poblaci\acute{o}n}[tama\tilde{n}o\_poblaci\acute{o}n - rank(Individuo_i)]}\)
\end{center}

Con esta actualización ya es posible obtener un Fitness acorde al 
rank del Individuo sin alterar la esencia de la técnica.

%******* Termina script *******
%******* Termina módulo *******

%******* Empieza módulo *******
\subsection{Operator (módulo)}
%Se coloca el vínculo interno procedente de esta misma sección (a_2_4).
\label{sec:a_2_4}
En éste se encuentran implementadas todas aquellas funcionalidades 
que intervengan en el proceso de la creación de una nueva Population 
\textbf{(ó Población)} hija.\medskip\break
La finalidad de ésto es propagar y realizar combinaciones de la 
carga genética de los Individuals \textbf{(ó Individuos)} más aptos 
mediante el cromosoma \textbf{(véase Model/}\break\textbf{ChromosomalRepresentation)} 
para obtener soluciones con una mejor calidad que sus predecesoras.\medskip\break
Para este punto es importante mencionar que la calidad de un Individuo 
es directamente proporcional a su Fitness \textbf{(véase Model/Fitness)}.\medskip\break
En términos generales, la manera de construir una Población hija es 
la siguiente:

\begin{itemize}
\item De la Población actual y tomando como base el Fitness de cada Individuo se seleccionan aquéllos que serán los elegidos para reproducirse. Nótese que un Individuo puede ser tomado en cuenta más de una vez si se da el caso.
\item Con base en los elegidos, se toman sus respectivos cromosomas y se realiza la operación de Crossover \textbf{(ó Cruza)}. Ésta es una simulación de una reproducción de tipo sexual donde se toman dos padres para ``procrear'' dos hijos. \break Las características de los hijos dependerán de las técnicas usadas \textbf{(véase Model/}\break\textbf{Operator/Crossover)}.
\item Se toman los hijos y uno a uno se les aplica la operación de mutación.
\end{itemize}

Al final Población hija constará de los hijos ``mutados''.\medskip\break
A continuación se muestran las siguientes subcategorías correspondientes 
a los pasos descritos anteriormente, cada una con sus respectivas técnicas 
desarrolladas:

%******* Empieza módulo *******
\subsubsection{Selection (módulo)}
%Se coloca el vínculo interno procedente de esta misma sección (a_2_4_1).
\label{sec:a_2_4_1}
En esta sección se encuentran implementadas todas las técnicas 
relacionadas con la selección de Individuos.\break
Como se ha mencionado antes, durante dicha operación la 
importancia de la elección radica en el Fitness de cada Individuo, 
además un Individuo puede ser seleccionado más de una vez si la 
causa lo amerita.\break
Así, se elegirán tantos Individuos como elementos haya en la Población.\medskip\break
El objetivo radica en mantener el equilibrio entre una ``selección justa'' y
la oportunidad de permitir a los Individuos con una calidad media o baja 
la propagación de su carga genética.\medskip\break
Al final se busca que el usuario desarrolle sus propias técnicas de 
selección, por lo cual, además de añadir el método en el listado 
localizado en \textbf{Controller/XML/Features.xml}, deberá implementar 
la siguiente función:

%******* Empieza función *******
\begin{fulllineitems}

\pysiglinewithargsret{\sphinxbfcode{execute\_selection\_technique}}{\emph{population}, \emph{selection\_parameters}}{}~
\vspace{-0.1cm}

Lleva a cabo la selección de Individuos de una Población. 
Es importante recalcar que, la función que más se ocupa es:

\begin{center}\textbf{get\_fitness (Model/Community/Population/Individual.py)}
\end{center}

Aunque existen otras que pueden tener relevancia para el usuario 
\textbf{(véase Model/}\break\textbf{Community/Population.py)}.\medskip\break
Como medida adicional, para los eventos de Crossover \textbf{(ó Cruza)} 
y Mutation \textbf{(ó Mutación)} se recomienda ampliamente que este 
método regrese únicamente los cromosomas asociados a los Individuos, 
ya que ésto facilita sobremaneralas operaciones mencionadas.

\begin{quote}\begin{description}
\item[{Parameters}] \leavevmode\begin{itemize}
\item \textbf{\texttt{population}} (\emph{\texttt{Instance}}) -- La Población sobre la cual se se seleccionarán los Individuos.
\item \textbf{\texttt{selection\_parameters}} (\emph{\texttt{Dictionary}}) -- Un diccionario que puede contener opciones adicionales para la
selección de Individuos.
\end{itemize}
\item[{Returns}] \leavevmode
Una lista que contiene los cromosomas de los Individuos seleccionados.
\item[{Return type}] \leavevmode
List
\end{description}\end{quote}

\end{fulllineitems}

Es conveniente mencionar que se asume por defecto que las funciones antes 
mencionadas se encuentran implementadas en cada uno de los elementos de este 
módulo, por ello es que primordialmente se mostrarán aquéllas que no se 
contemplen en el esquema original, es decir, funciones auxiliares.\break
En el caso muy específico en el que alguna de las funciones obligatorias 
contenga información importante también se adjuntarán en el documento.\medskip\break  
A continuación se vislumbran los elementos característicos 
de este módulo:
%******* Termina función *******

%******* Empieza script *******
\paragraph{Roulette (script)}
%Se coloca el vínculo interno procedente de esta misma sección (a_2_4_1_1).
\label{sec:a_2_4_1_1}

\begin{fulllineitems}

\begin{DUlineblock}{0em}
\item[] Se implementa el método de selección conocido como 
Roulette \textbf{(ó Ruleta)}.\break
También es llamado Proportional Selection \textbf{(ó Selección Proporcional)}.\break
En la función se distinguen dos etapas principales: construir 
la ruleta y ``ponerla a girar'' para que se elija al elemento.\medskip\break
Para la primera etapa se toma como base el Valor Esperado 
\textbf{(ó Expected Value)} de cada Individuo 
\textbf{(véase Model/Community/Population/Individual.py)}.\medskip\break
El Valor Esperado para fines de este proyecto es el número de 
``hijos'' que un Individuo puede ofrecer. Éste se calcula de la 
siguiente forma:

\begin{center}\(Valor\_Esperado(Individuo) = \frac{tama\tilde{n}o\_poblaci\acute{o}n \cdot Fitness(Individuo)}{\sum_{i=1}^{tama\tilde{n}o\_poblaci\acute{o}n}Fitness(Individuo_i)}\)
\end{center}

Al final aquéllos con Valores Esperados altos tendrán lugar 
a mayores espacios en la ruleta y por ende su probabilidad 
de elección aumenta.\medskip\break
Para recorrer la ruleta en realidad se toma un valor aleatorio 
entre 0 y la suma de los Valores Esperados. Entonces se van 
sumando los Valores Esperados de los Individuos hasta que se 
exceda el valor aleatorio mencionado antes. Aquel elemento cuyo 
Valor Esperado haya excedido la suma se considera el elegido y 
es seleccionado para la etapa de cruza.\medskip\break
Para la selección de Individuos se efectúa la segunda operación 
tantas veces como el tamaño de la Población.\break
Cabe mencionar que el Valor Esperado ya se calcula de manera 
automática en este proyecto \textbf{(véase Model/Community/Population/Population.py)}.
\end{DUlineblock}

\end{fulllineitems}
%******* Termina script *******

%******* Empieza script *******
\paragraph{ProbabilisticTournament (script)}
%Se coloca el vínculo interno procedente de esta misma sección (a_2_4_1_2).
\label{sec:a_2_4_1_2}

\begin{fulllineitems}

\begin{DUlineblock}{0em}
\item[] Se desarrolla la técnica conocida como Torneo 
Probabilístico \textbf{(ó Probabilistic Tournament)}.\break
Tal como lo sugiere el nombre, la selección será 
llevada a cabo en forma de competencia directa 
entre los Individuos.\break
Tradicionalmente se comparan sus Fitness y de esta 
manera el Individuo ganador es aquél con la cantidad 
mayor de Fitness, pero dado que se maneja un esquema 
probabilístico la decisión no depende totalmente del 
factor antes mencionado.\medskip\break
De esta manera se pueden recapitular los siguientes pasos:

\begin{itemize}
\item Tomar k \((2 \leqslant k \leqslant tama\tilde{n}o\_poblaci\acute{o}n)\) Individuos de la Población.
\item Realizar el torneo de manera secuencial entre los elementos seleccionados anteriormente, esto es, tomar el elemento A y enfrentarlo con B, al resultado de la batalla anterior enfrentarlo con C y así sucesivamente.
\end{itemize}

Para ello por cada encuentro se crea un número aleatorio 
entre 0 y 1, si el número es menor a 0.5 se toma al elemento 
con menor Fitness, de lo contrario se elige al de mayor 
Fitness.\break
La operación se lleva a cabo hasta que se tenga un ganador 
de los k Individuos.

Los dos pasos anteriores se repiten hasta que se hayan obtenido 
tantos Individuos como el tamaño de la Población.
\end{DUlineblock}

\end{fulllineitems}
%******* Termina script *******

%******* Empieza script *******
\paragraph{StochasticUniversalSampling (script)}
%Se coloca el vínculo interno procedente de esta misma sección (a_2_4_1_3).
\label{sec:a_2_4_1_3}

\begin{fulllineitems}

\begin{DUlineblock}{0em}
\item[] Se determina la técnica conocida como Stochastic 
Universal Sampling \textbf{(ó Muestreo Estocástico Universal)}.\break
Primero que nada es menester mencionar que es necesario 
el uso del Expected Value \textbf{(ó Valor Esperado)} 
de cada Individuo.\medskip\break
Para fines concernientes a este proyecto, se trata 
del número de ``hijos'' que un Individuo puede 
ofrecer. Éste se calcula de la siguiente forma:

\begin{center}\(Valor\_Esperado(Individuo) = \frac{tama\tilde{n}o\_poblaci\acute{o}n \cdot Fitness(Individuo)}{\sum_{i=1}^{tama\tilde{n}o\_poblaci\acute{o}n}Fitness(Individuo_i)}\)
\end{center}

Con base a lo anterior, el método consiste en lo siguiente:

\begin{itemize}
\item Se selecciona un valor aleatorio entre 0 y 1, a éste se le llamará Pointer \textbf{(ó Puntero)}
\item De manera secuencial se seleccionarán tantos Individuos como el tamaño de la Población, los cuales deben estar igualmente espaciados en su Valor Esperado tomando como referencia el valor de Pointer.
\end{itemize}

Es importante aclarar el segundo punto, así que se abordará 
desde una perspectiva computacional:

\begin{itemize}
\item Se deben tener variables adicionales que indiquen la acumulación tanto del Pointer \textbf{(CP, Cumulative Pointers)} como de los Valores Esperados \textbf{(CEV, Cumulative Expected Value)} así como al Individuo actual que está siendo seleccionado \textbf{(I)}.
\item Para averiguar si un Individuo está igualmente espaciado en su Valor Esperado con respecto de los demás basándose en Pointer, basta con corroborar que:

\begin{center}\(CP + Pointer > CEV + EV\)
\end{center}

\item Si la condición descrita es verdadera los valores EV e I deben actualizarse \textbf{(I se ajusta al siguiente Individuo)} ya que esto indica que se buscará al siguiente Individuo espaciado equitativamente con el valor Pointer. No se hace nada si la condición es falsa.
\item Independientemente del valor de la condición anterior, CP y CEV deben actualizarse durante todo el ciclo.
\end{itemize}

Cabe mencionar que si la lista de Individuos se agota, 
se puede volver a iterar desde el inicio teniendo cautela en 
conservar CEV y CP.
\end{DUlineblock}

\end{fulllineitems}
%******* Termina script *******
%******* Termina módulo *******

%******* Empieza módulo *******
\subsubsection{Crossover (módulo)}
%Se coloca el vínculo interno procedente de esta misma sección (a_2_4_2).
\label{sec:a_2_4_2}
Aquí se desarrollan las técnicas de Crossover 
\textbf{(ó Cruza)}.\medskip\break
Prosiguiendo con el ciclo de creación de una nueva Población, 
es en este apartado donde se lleva a cabo la concepción de 
nuevos Individuos.\break
Debido a esto se busca crear ``hijos'' más aptos que respondan 
mejor ante la problemática fundamentada, es decir, concebir 
soluciones que se adapten mejor a los criterios establecidos
por el usuario desde un inicio basándose en las soluciones 
predecesoras.\medskip\break
Es menester mencionar que esta función es meramente binaria, 
lo cual significa que siempre deben haber dos padres, además 
se debe hacer hincapié en que la Cruza se ejecuta a nivel cromosómico 
\textbf{(véase Model/ChromosomalRepresentation)},por lo que se 
debe tener mesura con el tratamiento de los métodos, dicho de otra 
manera, cada Representación Cromosómica debe ir acompañada de al 
menos una función de Cruza.\medskip\break
Como dato para posteriores referencias, un gen hace referencia 
a una casilla del cromosoma mientras que un alelo es el valor 
que puede existir en un gen.\break
Entonces se persigue que el usuario construya sus propias funciones 
de Cruza, para lo cual, además de añadir el método en el listado 
localizado en \textbf{Controller/XML/Features.xml}, deberá implementar 
la siguiente función:

%******* Empieza función *******
\begin{fulllineitems}

\pysiglinewithargsret{\sphinxbfcode{execute\_crossover\_technique}}{\emph{chromosome\_a}, \emph{chromosome\_b}, \emph{crossover\_parameters}}{}~
\vspace{-0.1cm}

Lleva a cabo la cruza de dos Individuos a nivel cromosómico.\break
Además esta función debe retornar siempre dos hijos los cuales 
serán la cruza de A con B y la cruza de B con A, esto nos indica que, 
con el objetivo de incrementar la calidad de los Individuos sin perder 
la carga genética ganada o introducir elementos riesgosos, la cruza 
consiste en generar un nuevo Individuo y su recíproco; así se garantiza 
una adecuada y controlada descendencia.\medskip\break
Finalmente, esta función debe contar con la probabilidad de Cruza, la 
cual indica si se debe o no hacer la operación cromosómica; en caso 
de ser la respuesta negativa los hijos resultan en copias idénticas 
de los padres.

\begin{quote}\begin{description}
\item[{Parameters}] \leavevmode\begin{itemize}
\item \textbf{\texttt{chromosome\_a}} (\emph{\texttt{List}}) -- El cromosoma del Individuo A.
\item \textbf{\texttt{chromosome\_b}} (\emph{\texttt{List}}) -- El cromosoma del Individuo B.
\item \textbf{\texttt{crossover\_parameters}} (\emph{\texttt{Dictionary}}) -- Un diccionario que puede contener opciones adicionales para la
cruza de Individuos.
\end{itemize}
\item[{Returns}] \leavevmode
Un arreglo con dos cromososomas, el primero es la cruza de A con B, mientras que el segundo
es la cruza de B con A.
\item[{Return type}] \leavevmode
Array
\end{description}\end{quote}

\end{fulllineitems}

Es menester considerar que se asume por defecto que las funciones antes 
mencionadas se encuentran implementadas en cada uno de los elementos de 
este módulo, por ello es que primordialmente se mostrarán aquéllas que 
no se contemplen en el esquema original, es decir, funciones auxiliares.\break
En el caso muy específico en el que alguna de las funciones obligatorias 
contenga información importante también se adjuntarán en el documento.\medskip\break  
Se colocan los elementos alusivos a este módulo:
%******* Termina función *******

%******* Empieza script *******
\paragraph{NPointsCrossover (script)}
%Se coloca el vínculo interno procedente de esta misma sección (a_2_4_2_1).
\label{sec:a_2_4_2_1}

\begin{fulllineitems}

\begin{DUlineblock}{0em}
\item[] Se implementa el método que lleva por nombre N-Points 
Crossover \textbf{(ó Cruza en N-Puntos)}.\break
Para comenzar, esta técnica está elaborada para usarse 
tanto por Representación Cromosómica \textbf{(véase Model/ChromosomalRepresentation)} 
de tipo FloatPoint \textbf{(ó de Punto Flotante)} como Binary \textbf{(ó Binaria)}.\medskip\break
Su funcionamiento consiste en construir a los descendientes 
usando sub-bloques de cromosomas de cada uno de los padres, 
determinados éstos por una cierta cantidad de puntos de corte, 
de ahí el nombre.\break
Aterrizando lo anterior de una manera concisa se tiene lo siguiente:

\begin{itemize}
\item Consideremos a los cromosomas de los padres Padre I: \(I_1I_2...I_n\) y Padre J: \(J_1J_2...J_n\)
\item Posteriormente se determinan aleatoriamente los puntos de corte, cabe mencionar que si los cromosomas son de tamaño n, pueden existir máximo n - 1 puntos. Supongamos que se crean k puntos \((1 \leqslant k \leqslant n - 1)\) y por lo tanto cada cromosoma queda separado en k + 1 bloques.\break
De esta manera obtenemos:
      \begin{itemize}
      \item Padre I en bloques \textbf{(BI)}: \(BI_1BI_2...BI_{k + 1}\);
      \item Padre J en bloques \textbf{(BJ)}: \(BJ_1BJ_2...BJ_{k + 1}\).
      \end{itemize}
\item Finalmente cada hijo constará de la alternancia de bloques de manera secuencial comenzando por el bloque inicial de un padre determinado, dicho de otra forma, los hijos estarán constituidos de la siguiente manera:
      \begin{itemize}
      \item Para el hijo \(H_1\): \(BI_1BJ_2...BI_{k + 1}\)
      \item Para el hijo \(H_2\): \(BJ_1BI_2...BJ_{k + 1}\)
      \end{itemize}
\end{itemize}

Sólo queda mencionar que hasta el cierre de este proyecto 
no existe una manera transparente desde el View \textbf{(ó Vista)} 
de conocer, dada una representación Binaria y un conjunto de variables 
de decisión y funciones objetivo, el número máximo de puntos de 
corte permitidos para este procedimiento, sin embargo, una manera 
de mitigar esta situación fue contemplar algún posible caso de error 
en esta sección y mandar un mensaje de error a la Vista por si 
llegase a suceder algún desperfecto durante el proceso.
\end{DUlineblock}

\end{fulllineitems}
%******* Termina script *******

%******* Empieza script *******
\paragraph{UniformCrossover (script)}
%Se coloca el vínculo interno procedente de esta misma sección (a_2_4_2_2).
\label{sec:a_2_4_2_2}

\begin{fulllineitems}

\begin{DUlineblock}{0em}
\item[] Se lleva a cabo la implementación de la técnica conocida 
como Uniform Crossover \textbf{(ó Cruza Uniforme)}.\medskip\break
Primero que nada esta operación está fabricada para usarse 
tanto con la Representación Cromosómica \textbf{(véase Model/ChromosomalRepresentation)}
de tipo FloatPoint \textbf{(ó Punto Flotante)} como Binary 
\textbf{(ó Binaria)}.\break
La característica de este procedimiento es crear nuevos Individuos 
intercambiando secuencialmente los genes de sus padres; visto 
de una manera más estructurada consiste en lo siguiente:

\begin{itemize}
\item Tenemos a los cromosomas de los padres Padre A: \(A_1A_2...A_n\) y Padre B: \(B_1B_2...B_n\)
\item Ahora, cada hijo será construido con genes de uno y sólo uno de los padres a menos que se indique lo contrario; este movimiento será posible con una variable denominada Pmask \textbf{(Pm)} que toma valores de 0 a 1 y una probabilidad de Pmask \textbf{(Pp)} que también toma valores de 0 a 1. Entonces lo anterior se puede declarar así:
\item Para el hijo \((H_1)\) que tomará sus genes del padre A \textbf{(PA)}:
      \begin{itemize}
      \item Si \(Pm \leqslant Pp\ entonces\ H_1(i) = A_i,\ en\ otro\ caso\ H_1(i) = B_i; 1 \leqslant i \leqslant n\)
      \end{itemize}
\item Para el hijo \((H_2)\) que tomará sus genes del padre B \textbf{(PB)}:
      \begin{itemize}
      \item Si \(Pm \leqslant Pp\ entonces\ H_2(i) = B_i,\ en\ otro\ caso\ H_1(i) = A_i; 1 \leqslant i \leqslant n\)
      \end{itemize}
\end{itemize}

\end{DUlineblock}

\end{fulllineitems}
%******* Termina script *******
%******* Termina módulo *******

%******* Empieza módulo *******
\subsubsection{Mutation (módulo)}
%Se coloca el vínculo interno procedente de esta misma sección (a_2_4_3).
\label{sec:a_2_4_3}
En esta parte se encuentran detalladas las técnicas 
relacionadas con Mutation \textbf{(ó Mutación)}.\medskip\break
Retomando el proceso de creación de una nueva Población, 
es aquí donde una vez obtenidos los hijos, se modifican 
pequeñas porciones \textbf{(genes)} de sus cromosomas 
de manera individual.\break
Con ésto se persigue principalmente que estas ínfimas 
alteraciones permitan incrementar la exploración del 
material genético y por ende otorgar Individuos aún más 
aptos sin caer en el peligro de perder características 
valiosas en la Población.\medskip\break
Considerando lo anterior, lo primero que hay que tomar en 
cuenta es que la operación de Mutación es unaria, esto 
significa que sólo se puede mutar el cromosoma de un 
Individuo a la vez.\break
También y reiterando la información pasada, la Mutación 
es una operación que se lleva a cabo a nivel cromosómico 
\textbf{(véase Model/ChromosomalRepresentation)}, por lo 
que se debe tener mesura con el tratamiento de los métodos, 
dicho de otra manera, cada Representación Cromosómica debe 
ir acompañada de al menos una función de Mutación.\medskip\break
Como dato para posteriores referencias, un gen hace referencia 
a una casilla del cromosoma, mientras que un alelo es el 
valor que puede existir en un gen.\break
Así, se invita a que el usuario construya sus propias 
versiones de Mutación, por lo cual, además de añadir el 
método en el listado localizado en \textbf{Controller/XML/Features.xml}, 
deberá implementar la siguiente función:

%******* Empieza función *******
\begin{fulllineitems}

\pysiglinewithargsret{\sphinxbfcode{execute\_mutation\_technique}}{\emph{chromosome}, \emph{mutation\_parameters}}{}~
\vspace{-0.1cm}

Lleva a cabo mutación del Individuo a nivel cromosómico.\break
A grandes rasgos, modifica los alelos de los genes tomando 
en cuenta la gama de valores a los que se pueden transformar 
\textbf{(por ejemplo, una mutación de representación Binaria puede transformarse sólo en valores 0 ó 1)}.\medskip\break
El método debe retornar siempre el cromosoma mutado.\break
Finalmente, esta función debe contar con la probabilidad de Mutación,
la cual indica si se debe o no hacer la operación cromosómica 
por cada gen; en caso de ser la respuesta negativa el Individuo 
no experimenta modificación alguna en el gen y se pasa al 
siguiente y así sucesivamente.

\begin{quote}\begin{description}
\item[{Parameters}] \leavevmode\begin{itemize}
\item \textbf{\texttt{chromosome}} (\emph{\texttt{List}}) -- El cromosoma para ser mutado.
\item \textbf{\texttt{mutation\_parameters}} (\emph{\texttt{Dictionary}}) -- Un diccionario que puede contener opciones adicionales para la
mutación del cromosoma del Individuo.
\end{itemize}
\item[{Returns}] \leavevmode
El cromosoma modificado.
\item[{Return type}] \leavevmode
List
\end{description}\end{quote}

\end{fulllineitems}

Se debe tomar en cuenta que se asume por defecto que las funciones antes 
mencionadas se encuentran implementadas en cada uno de los elementos de 
este módulo, por ello es que primordialmente se mostrarán aquéllas que 
no se contemplen en el esquema original, es decir, funciones auxiliares.\break
En el caso muy específico en el que alguna de las funciones obligatorias 
contenga información importante también se adjuntarán en el documento.\medskip\break  
A continuación se muestran los elementos concernientes a este módulo:
%******* Termina función *******

%******* Empieza script *******
\paragraph{BinaryMutation (script)}
%Se coloca el vínculo interno procedente de esta misma sección (a_2_4_3_1).
\label{sec:a_2_4_3_1}

\begin{fulllineitems}

\begin{DUlineblock}{0em}
\item[] Se implementa el método conocido como Binary Mutation 
\textbf{(ó Mutación Binaria)}.\break
El procedimiento es el siguiente:

\begin{itemize}
\item Se trata cada gen individualmente y se modifica de acuerdo a una probabilidad de Mutación asignada, si ésta es suficiente se procede a hacer el cambio, en otro caso se deja el alelo asociado al gen intacto.
\item Retomando el caso en que se puede modificar el alelo del gen se verifica su valor actual y ya que se maneja una representación Binaria su transformación es muy simple: si se encuentra un 0, el alelo toma el valor 1 y viceversa.
\end{itemize}

\end{DUlineblock}

\end{fulllineitems}
%******* Termina script *******

%******* Empieza script *******
\paragraph{FloatPointMutation (script)}
%Se coloca el vínculo interno procedente de esta misma sección (a_2_4_3_2).
\label{sec:a_2_4_3_2}

\begin{fulllineitems}

\begin{DUlineblock}{0em}
\item[] Se concreta el método conocido como Float Point 
Mutation \textbf{(ó Mutación de Punto Flotante)}.\break
El procedimiento es el siguiente:

\begin{itemize}
\item Se trata cada gen individualmente y se modifica de acuerdo a una probabilidad de Mutación asignada, si ésta es suficiente se procede a hacer el cambio, en otro caso se deja el alelo asociado al gen intacto.
\item Retomando el caso en que se puede modificar el alelo del gen se verifica los límites de la variable de decisión que está ligada a éste, así como la precisión decimal. Entonces se crea el nuevo número con la precisión decimal requerida y se sustituye por el anterior.
\end{itemize}

\end{DUlineblock}

\end{fulllineitems}
%******* Termina script *******
%******* Termina módulo *******

%******* Empieza módulo *******
\subsection{SharingFunction (módulo)}
%Se coloca el vínculo interno procedente de esta misma sección (a_2_5).
\label{sec:a_2_5}
En esta sección se almacenan las técnicas relativas 
al Sharing Function \textbf{(ó Función de Compartición)}.\medskip\break
El objetivo de estas técnicas se delega a un rol secundario pero 
aún así muy importante y consiste en realizar un filtrado más 
minucioso de los mejores Individuos y así tomar a los candidatos 
elegidos para dejar descendencia.\break
La operación es útil en casos en el que la calidad de los 
Individuos es muy similar y entonces se desea seleccionar a 
los que son superiores, sin embargo, es menester mencionar que, 
en exceso, dicha selección parsimoniosa puede dar lugar a un efecto 
negativo del Selective Pressure \textbf{(ó Presión Selectiva, véase Model/MOEA)}.\break
Esto provoca que, lejos de dar una Población de elementos 
óptimos, los Indviduos se queden estancados puesto que al tener 
todos cargas genéticas muy similares, existe una pobre exploración 
genética en sus cromosomas y entonces no se llegará a una optimización 
de funciones objetivo adecuada.\medskip\break
Es por ello que no todos los MOEAS \textbf{(véase Model/MOEA)} 
lo utilizan, sin embargo se decidió implementar esta sección ya que 
extrapolando las circunstancias, en cualquier momento se puede hacer 
uso de técnicas de esta índole.\medskip\break
Haciendo énfasis en la parte matemática, el Sharing Function funciona así:\medskip\break
Cada Individual \textbf{(ó Individuo)} tendrá asociado un Shared Fitness 
\textbf{(ó Fitness Compartido)} que fungirá como el Fitness original asignado 
a cada Individo y el cual será obtenido de la siguiente manera:

\begin{center}\(SharedFitness(Individuo) = \frac{Fitness(Individuo)}{NicheCount(Individuo)}\)
\end{center}

Para fines de implementación el Shared Fitness será colocado en 
la misma variable utilizada para almacenar el Fitness original, 
esto por cada Individuo.\medskip\break
El Niche Count es un valor que indica qué tan cercano en calidad 
se encuentra un Individuo con respecto de los demás. La forma de 
calcularlo es la siguiente:

\begin{center}\(NicheCount(Individuo) = \sum_{j=1}^{tama\tilde{n}o\_poblaci\acute{o}n}SF(D(Individuo,Individuo_j))\)
\end{center}

Donde \(D(Individuo_i,Individuo_j)\) es la distancia que existe 
entre el Individuo i y el Individuo j; mientras que el SF es el 
Sharing Function.\break
Entonces el SF se define como:

\begin{center}\(SF(D(Individuo_i,Individuo_j)) = \left\{ \begin{array}{lcc}
              1 - (\frac{D(Individuo_i,Individuo_j)}{\sigma_{share}})^{\alpha},\ \ si\ \ D < \sigma_{share}. \\
              \\ 0,\ \ en\ cualquier\ otro\ caso. \\
         \end{array}
\right.\)
\end{center}

Donde \(\alpha\) es una variable que casi siempre se asigna a 
1 \textbf{(aunque en este proyecto se le da la libertad al usuario de seleccionar valores distintos)} y 
\(\sigma_{share}\) marca el límite en el cual dos Individuos se 
consideran cercanos en calidad, es decir, viven en el mismo Niche.\medskip\break
Llegados a este punto, si bien la parte que se utilizará finalmente 
es el Shared Fitness, sólo las técnicas concernientes a 
\(D(Individuo_i,Individuo_j)\) serán las que se implementen en esta 
sección, pues lo demás siempre se mantendrá estático.\break
Siendo más específicos con base en lo anterior, existen dos 
tipos de funciones de Distancia:

\begin{itemize}
\item De Similaridad Genotípica \textbf{(ó Genotypic Similarity)}.
\item De Similaridad Fenotípica \textbf{(ó Phenotypic Similarity)}.
\end{itemize}

La primera indica en pocas palabras que la comparación se 
hará usando únicamente características relacionadas con el 
cromosoma, mientras que la segunda implicará la comparación 
de características externas como las funciones objetivo evaluadas 
con las variables de decisión de cada Individuo ó las variables de 
decisión por sí solas.\medskip\break
Eventualmente se desea que el usuario implemente sus propias funciones, 
por ello es que, además de añadir el método en el listado localizado 
en \textbf{Controller/XML/Features.xml}, deberá implementar las siguientes 
funciones:

%******* Empieza función *******
\begin{fulllineitems}

\pysiglinewithargsret{\sphinxbfcode{calculate\_sigma\_share}}{\emph{population}, \emph{sharing\_function\_parameters}}{}~
\vspace{-0.1cm}

Realiza el cálculo del factor \(\sigma_{share}\) sobre el cual se 
hará el cuestionamiento de Individuos cercanos en calidad.\break
Es importante mencionar que la función debe regresar un escalar 
que representa el límite máximo para el cual dos Individuos se 
consideran en el mismo Niche.

\begin{quote}\begin{description}
\item[{Parameters}] \leavevmode\begin{itemize}
\item \textbf{\texttt{population}} (\emph{\texttt{Instance}}) -- La Población sobre la cual se hará el cálculo correspondiente.
\item \textbf{\texttt{sharing\_function\_parameters}} (\emph{\texttt{Dictionary}}) -- Un diccionario que puede contener opciones adicionales para
el cálculo de la distancia entre Individuos.
\end{itemize}
\item[{Returns}] \leavevmode
Un valor escalar que representa el límite de cercanía para cualesquiera dos Individuos
de una Población.
\item[{Return type}] \leavevmode
Float
\end{description}\end{quote}

\end{fulllineitems}
%******* Termina función *******

%******* Empieza función *******
\begin{fulllineitems}

\pysiglinewithargsret{\sphinxbfcode{calculate\_distance}}{\emph{individual\_i}, \emph{individual\_j}, \emph{sharing\_function\_parameters}}{}~
\vspace{-0.1cm}

Calcula la distancia de calidad que existe entre dos Individuos 
cualesquiera.\break
Dada la simpleza del método, se puede usar independientemente 
de las categorías antes especificadas.\break
Es importante resaltar que la función debe regresar un escalar 
que aluda a la distancia entre los Individuos.

\begin{quote}\begin{description}
\item[{Parameters}] \leavevmode\begin{itemize}
\item \textbf{\texttt{individual\_i}} (\emph{\texttt{Instance}}) -- El Individuo para calcular distancia.
\item \textbf{\texttt{individual\_j}} (\emph{\texttt{Instance}}) -- El Individuo para calcular distancia.
\item \textbf{\texttt{sharing\_function\_parameters}} (\emph{\texttt{Dictionary}}) -- Un diccionario que puede contener opciones adicionales para
el cálculo de la distancia entre Individuos.
\end{itemize}
\item[{Returns}] \leavevmode
Un valor escalar que indica la distancia entre los Individuos.
\item[{Return type}] \leavevmode
Float
\end{description}\end{quote}

\end{fulllineitems}

A continuación se muestran las subcategorías 
correspondientes:
%******* Termina función *******

%******* Empieza módulo *******
\subsubsection{GenotypicSimilarity (módulo)}
%Se coloca el vínculo interno procedente de esta misma sección (a_2_5_1).
\label{sec:a_2_5_1}
La similaridad Genotípica \textbf{(ó Genotypic Similarity)}, 
es una subcategoría que calcula las distancias entre dos 
Individuos cualesquiera usando para ello características 
Genotípicas de éstos, lo cual quiere decir que se emplearán 
rasgos meramente internos endémicos de los Individuos.\medskip\break
Para fines del proyecto típicamente se utiliza el cromosoma 
y/o sus características asociadas, no obstante siendo sensatos 
con el término, el cromosoma no es la única herramienta que 
se puede usar sino cualquier rasgo interno.\medskip\break
Es necesario considerar que, por defecto las funciones antes 
mencionadas se encuentran implementadas en cada uno de los elementos de 
este módulo, por ello es que primordialmente se mostrarán aquéllas que 
no se contemplen en el esquema original, es decir, funciones auxiliares.\break
En el caso muy específico en el que alguna de las funciones obligatorias 
contenga información importante también se adjuntarán en el documento.\break  
Ahora se muestran los elementos implementados para esta subcategoría:

%******* Empieza script *******
\paragraph{HammingDistance (script)}
%Se coloca el vínculo interno procedente de esta misma sección (a_2_5_1_1).
\label{sec:a_2_5_1_1}

\begin{fulllineitems}

\begin{DUlineblock}{0em}
\item[] La Distancia de Hamming \textbf{(ó Hamming Distance)} es una 
implementación perteneciente a la subcategoría Genotypic 
Similarity \textbf{(ó Similaridad Genotípica)}.
Esta consiste en comparar los alelos entre los cromosomas 
de los Individuos y devolver un valor numérico que indica en 
cuántos alelos los cromosomas de los Individuos resultaron 
tener valores diferentes.\break
Como consecuencia lógica, la magnitud de la Distancia de 
Hamming es inversamente proporcional a la calidad de los 
Individuos.\break
Es ampliamente usada para la Representación Cromosómica 
\textbf{(véase Model/}\break\textbf{ChromosomalRepresentation)} de tipo 
Binario \textbf{(ó Binary)}, aunque su uso no se limita 
sólo a esta codificación.\medskip\break
Con respecto del cálculo del \(\sigma_{share}\), éste se hace 
tomando en cuenta el número máximo permitido de genes 
diferentes entre dos cromosomas cualesquiera.\break
Dicha cantidad es deducida solicitándole al usuario únicamente 
el porcentaje máximo permitido, con base en éste se determina 
entonces el número en concreto.
\end{DUlineblock}

\end{fulllineitems}
%******* Termina script *******
%******* Termina módulo *******

%******* Empieza módulo *******
\subsubsection{PhenotypicSimilarity (módulo)}
%Se coloca el vínculo interno procedente de esta misma sección (a_2_5_2).
\label{sec:a_2_5_2}
La Similaridad Fenotípica \textbf{(ó Phenotypic Similarity)} 
es una subcategoría que calcula las distancias entre cualesquiera 
dos Individuos usando características concernientes al Fenotipo, 
es decir, rasgos exteriores de los Individuos.\medskip\break
Para fines relativos al proyecto, dichos atributos tradicionalmente 
no son otra cosa que las funciones objetivo evaluadas de cada 
Individuo, usando para ello las variables de decisión que cada uno 
lleva consigo.\break
Aún considerando lo anterior, siendo más generales, cualquier 
característica externa que se relacione con el Individuo puede 
ser utilizada.\medskip\break
Se hace mención en el hecho de que por defecto las funciones antes 
mencionadas se encuentran implementadas en cada uno de los elementos de 
este módulo, por ello es que primordialmente se mostrarán aquéllas que 
no se contemplen en el esquema original, es decir, funciones auxiliares.\break
En el caso muy específico en el que alguna de las funciones obligatorias 
contenga información importante también se adjuntarán en el documento.\medskip\break  
El presente módulo consta de los siguientes scripts:

%******* Empieza script *******
\paragraph{EuclideanDistance (script)}
%Se coloca el vínculo interno procedente de esta misma sección (a_2_5_2_1).
\label{sec:a_2_5_2_1}

\begin{fulllineitems}

\begin{DUlineblock}{0em}
\item[] La Distancia Euclidiana \textbf{(ó Euclidean Distance)} es una 
implementación de cálculo de distancia entre dos Individuos 
que pertenece a la subcategoría Phenotypic Similarity \textbf{(ó Similaridad Fenotípica)}.\break
Esta versión está dirigida para las Funciones Objetivo 
\textbf{(ó Objective Functions)} que poseen cada uno de los 
Individuos \textbf{(ó Individuals)} de una Población \textbf{(ó Population)}.\medskip\break
Primero que nada para obtener el cálculo de \(\sigma_{share}\) 
la operación está regida por la siguiente fórmula:

\begin{center}\(\sigma_{share} = \frac{\sum_{j=1}^{n\acute{u}m\_funciones\_objetivo}(max(F_j) - min(F_j))}{tama\tilde{n}o\_poblaci\acute{o}n - 1}\)
\end{center}

Lo anterior significa que se van a obtener los valores máximo y 
mínimo de cada función objetivo, se restan entre sí y al resultado 
anterior se le divide entre el tamaño de la Población menos uno; 
esto por cada generación.\break
La forma de hacer el cálculo de la distancia es la siguiente:\break
Supongamos que tenemos los vectores \(U = (u_1,u_2,...,u_n)\) y \(V = (v_1,v_2,...,v_n)\). 
Entonces la Distancia Euclidiana se define como:

\begin{center}\(d_E(U,V) = \sqrt{(v_1 - u_1)^2 + (v_2 - u_2)^2 + ... + (v_n - u_n)^2}\)
\end{center}

Para los fines que nos conciernen, los vectores \(U\ y\ V\) serán las 
evaluaciones en las funciones objetivos de cada Individuo participante.\break
Finalmente es menester mencionar que, aunque tradicionalmente esta 
técnica se usa para Representaciones Cromosómicas \textbf{(véase Model/ChromosomalRepresentation)} 
de tipo FloatPoint \textbf{(ó Punto Flotante)}, en sentido estricto no 
se encuentra limitada sólo a este tipo de codificación.
\end{DUlineblock}

\end{fulllineitems}
%******* Termina script *******
%******* Termina módulo *******
%******* Termina módulo *******

%******* Empieza módulo *******
\subsection{MOEA (módulo)}
%Se coloca el vínculo interno procedente de esta misma sección (a_2_6).
\label{sec:a_2_6}
En esta parte se encuentran desarrolladas todas las 
técnicas concernientes al uso de M.O.E.A.'s \textbf{(Multi-Objective Evolutionary Algorithms
ó Algoritmos Evolutivos Multiobjetivo)}.\medskip\break
Un M.O.E.A. es la convergencia y culminación de todas las 
técnicas que se han implementado en la sección Model 
\textbf{(ó Modelo)} con la finalidad de ofrecer una 
solución óptima ante un problema multiobjetivo mediante el 
uso de Algoritmos Evolutivos.\medskip\break
Primero, solucionar un problema multiobjetivo aterrizado en 
un lenguaje matemático consiste en lo siguiente:\break
Tenemos un vector de funciones objetivo:

\begin{center}\(F(\vec{x}) = [f_1(\vec{x}),f_2(\vec{x}),...,f_n(\vec{x})]^T;\ con\ n \geqslant 1.\)
\end{center}

Donde:

\begin{center}\(\vec{x} = [x_1,x_2,...,x_k]^T;\ k \geqslant 1.\)
\end{center}

Representa al vector de variables de decisión que ``alimenta'' a 
cada función objetivo.\break
La meta es encontrar un vector especial 
de variables de decisión, denominado:

\begin{center}\(\vec{x*} = [x_1*,x_2*,...,x_k*]^T;\ k \geqslant 1.\)
\end{center}

Tal que:

\begin{center}\(f_i(\vec{x*}) \leqslant f_i(\vec{x});\ 1 \leqslant i \leqslant n;\ \forall f \in F\).
\end{center}

Dicho de otra forma, se debe encontrar el vector de variables 
de decisión que minimize todas y cada una de las funciones 
objetivo en existencia.\break
Adicionalmente, todo vector de variables de decisión debe estar 
sujeto a las restricciones:

\begin{center}\(h_i(\vec{x}) = 0;\ 1 \leqslant i \leqslant p\ \ (restricciones\ de\ igualdad).\)
\end{center}
\begin{center}\(g_i(\vec{x}) \leqslant 0;\ 1 \leqslant i \leqslant m\ \ (restricciones\ de\ desigualdad).\)
\end{center}

Las cuales para fines de este proyecto son aquéllas a las que 
se encuentran afianzadas las variables de decisión 
\textbf{(véase View/Main/DecisionVariable/VariableFrame.py)}.\medskip\break
Una definición adicional que sin lugar a dudas se verá utilizada 
es la de \emph{dominancia} entre vectores de variables de decisión,
para ello tomemos dos vectores \(U = (u_1,u_2,...,u_k)\) y \(V = (v_1,v_2,...,v_k)\), 
se dice que \textbf{U domina a V ó V es dominada por U} si:

\begin{center}\(\forall i \in \{1,...,k\}\ \ u_i \leqslant v_i \land \exists i \in \{1,...,k\}; \ \ u_i < v_i\).
\end{center}

Lo anterior significa que \(U\) debe ser mejor que \(V\) en 
cada uno de sus componentes para garantizar la dominancia.\break
La simbología que se suele usar para identificar este hecho es \(u \succ v\).\medskip\break 
Algo importante a mencionar es que en las definiciones se trata 
únicamente la minimización de funciones objetivo porque, en caso 
de querer la maximización, simplemente se realiza la sustitución:

\begin{center}\(f'_i(\vec{x}) = -f_i(\vec{x});\ 1 \leqslant i \leqslant n,\ para\ alguna\ f \in F.\)
\end{center}

Es decir, minimizando la función negativa se obtiene el máximo.\break 
El proyecto ya contempla este tipo de casos \textbf{(véase View/Main/ObjectiveFunction/}\break\textbf{FunctionFrame)}.\break
Como dato adicional, es menester añadir que, en un escenario 
típico muchas de las funciones objetivo entrarán en conflicto, 
esto quiere decir que en algunas se buscará el mínimo mientras 
que en otras, el máximo.\medskip\break
Con base en lo anterior, el funcionamiento de un M.O.E.A. 
\textbf{(resolver un problema de optimización multiobjetivo usando algoritmos genéticos)} 
generalmente se lleva a cabo de la siguiente manera:\break

\begin{enumerate} 
\item Usando una Representación Cromosómica \textbf{(véase Model/ChromosomalRepresentation)}, crear la Población Padre y evaluar cada uno de los Individuos respecto a las funciones objetivo.
\item Asignar un Ranking a los Individuos de la Población Padre \textbf{(véase Model/}\break\textbf{Community/Community.py)}.
\item Con base en el Ranking, asignar el Fitness a cada uno de los Individuos \textbf{(véase Model/Fitness)}.
\item Tomando en cuenta el Fitness, aplicar las operaciones de Selección, Cruza y Mutación con la finalidad de crear una Población Hija \textbf{(véase Model/GeneticOperator)}. Todos los métodos empleados en este punto deben funcionar acorde a la Representación Cromosómica del punto 1.
\item \textbf{(Opcional)} Utilizar el Fitness Compartido para aplicar una elección más minuciosa de los mejores Individuos en la Población Hija \textbf{(véase Model/SharingFunction)}.
\item Designar a la Población Hija como la nueva Población Padre.
\item Repetir los pasos 2 a 6 hasta haber alcanzado un número límite de generaciones \textbf{(iteraciones)}.
\end{enumerate}

A grandes rasgos la diferencia entre un M.O.E.A. y otro es la Presión 
Selectiva \textbf{(ó Selective Pressure)} que se aplica durante 
el procedimiento, para fines de este proyecto se trata de la 
tolerancia para seleccionar a los Individuos de calidad media o 
baja frente a los mejores.\break 
Una baja Presión Selectiva permite elegir Individuos no tan 
aptos; el caso es análogo para una alta Presión Selectiva.\medskip\break
Es por eso que se han tomado los M.O.E.A.'s más representativos, 
pues se desea ilustrar la consistencia y eficacia de dichos métodos 
en general a través de variadas circunstancias.\break
Tomando en cuenta lo anterior, la finalidad es que el usuario desarrolle
sus propios M.O.E.A.'s, por ello es que, además de añadir el método 
en el listado localizado en \textbf{Controller/}\break\textbf{XML/Features.xml}, deberá 
implementar la siguiente función:

%******* Empieza función *******
\begin{fulllineitems}

\pysiglinewithargsret{\sphinxbfcode{execute\_moea}}{\emph{execution\_task\_count}, \emph{generations\_queue}, \emph{generations}, \emph{population\_size}, \emph{vector\_functions}, \emph{vector\_variables}, \emph{available\_expressions}, \emph{number\_of\_decimals}, \emph{community\_instance}, \emph{algorithm\_parameters}, \emph{representation\_instance}, \emph{representation\_parameters}, \emph{fitness\_instance}, \emph{fitness\_parameters}, \emph{sharing\_function\_instance}, \emph{sharing\_function\_parameters}, \emph{selection\_instance}, \emph{selection\_parameters}, \emph{crossover\_instance}, \emph{crossover\_parameters}, \emph{mutation\_instance},\emph{mutation\_parameters}}{}~
\vspace{-0.1cm}

Devuelve la solución óptima para un conjunto de funciones objetivo 
\textbf{vector\_functions} ligadas a un conjunto de restricciones 
\textbf{vector\_variables} tomando como fundamento el uso de 
algoritmos genéticos.\break
El método se apoya de las características subyacentes; en lo 
concerniente a la devolución de resultados se recomienda ver el 
método \textbf{get\_results} localizado en \textbf{Model/Community/Community.py}.

\begin{quote}\begin{description}
\item[{Parameters}] \leavevmode\begin{itemize}
\item \textbf{\texttt{execute\_task\_count}} (\emph{\texttt{Integer}}) -- El identificador que se utiliza para orquestar el orden en que el método será ejecutado con respecto de los demás \textbf{(véase View/Additional/}\break\textbf{ResultsGrapher/ResultsGrapherTopLevel.py)}.
\item \textbf{\texttt{generations\_queue}} (\emph{\texttt{Instance}}) -- Una estructura auxiliar \textbf{(Queue o Cola)} que es necesaria para indicar a la interfaz gráfica el progreso del método \textbf{(véase Controller/Controller.py, View/MainWindow.py, View/Additional/}\break\textbf{ResultsGrapher/ResultsGrapherTopLevel.py)} .
\item \textbf{\texttt{generations}} (\emph{\texttt{Integer}}) -- El número de generaciones \textbf{(iteraciones)} que se emplearán para la ejecución del método.
\item \textbf{\texttt{population\_size}} (\emph{\texttt{Integer}}) -- El tamaño de la Población \textbf{(número de Individuos)}.
\item \textbf{\texttt{vector\_functions}} (\emph{\texttt{List}}) -- El vector con las funciones objetivo insertadas por el usuario.
\item \textbf{\texttt{vector\_variables}} (\emph{\texttt{List}}) -- El vector con las variables de decisión ingresadas por el usuario.
\item \textbf{\texttt{available\_expressions}} (\emph{\texttt{Dictionary}}) -- Un diccionario con expresiones creadas para que la evaluación de funciones objetivo sea mucho más sencilla \textbf{(véase Controller/Verifier.py, Controller/XML/PythonExpressions.xml, View/Additional/MenuInternalOption/InternalOptionTab/}\break\textbf{PythonExpressionFrame.py)}.
\item \textbf{\texttt{number\_of\_decimals}} (\emph{\texttt{Integer}}) -- La precisión decimal \textbf{(número de decimales)} que tendrán las soluciones inherentes a los Individuos.
\item \textbf{\texttt{community\_instance}} (\emph{\texttt{Instance}}) -- Una instancia de la clase Community \textbf{(véase Controller/Verifier.py, Model/Community/Community.py)}.
\item \textbf{\texttt{algorithm\_parameters}} (\emph{\texttt{Instance}}) -- Un diccionario para añadir opciones adicionales para los M.O.E.A.'s.
\item \textbf{\texttt{representation\_instance}} (\emph{\texttt{Instance}}) -- Una instancia de la técnica de Representación Cromosómica \textbf{(ó Chromosomal Representation)} usada por el usuario \textbf{(véase Controller/Verifier.py, Model/}\break\textbf{ChromosomalRepresentation)}.
\item \textbf{\texttt{representation\_parameters}} (\emph{\texttt{Dictionary}}) -- Un diccionario con opciones adicionales a la técnica de Representación Cromosómica usada.
\item \textbf{\texttt{fitness\_instance}} (\emph{\texttt{Instance}}) -- Una instancia de la técnica de Fitness seleccionada por el usuario \textbf{(véase Controller/Verifier.py, Model/Fitness)}.
\item \textbf{\texttt{fitness\_parameters}} (\emph{\texttt{Dictionary}}) -- Un diccionario con parámetros adicionales para la técnica de Fitness utilizada.
\item \textbf{\texttt{sharing\_function\_instance}} (\emph{\texttt{Instance}}) -- Una instancia de la técnica de Sharing Function \textbf{(ó Función de Compartición)} usada por el usuario \textbf{(véase Controller/Verifier.py, Model/SharingFunction)}.
\item \textbf{\texttt{sharing\_function\_parameters}} (\emph{\texttt{Dictionary}}) -- Un diccionario con opciones adicionales para la técnica de Sharing Function seleccionada.
\item \textbf{\texttt{selection\_instance}} (\emph{\texttt{Instance}}) -- Una instancia de la técnica de Selection \textbf{(ó Selección)} seleccionada por el usuario \textbf{(véase Controller/Verifier.py, Model/Operator/}\break\textbf{Selection)}.
\item \textbf{\texttt{selection\_parameters}} (\emph{\texttt{Dictionary}}) -- Un diccionario con opciones adicionales para la técnica de Selection empleada.
\item \textbf{\texttt{crossover\_instance}} (\emph{\texttt{Instance}}) -- Una instancia de la técnica de Crossover \textbf{(ó Cruza)} seleccionada por el usuario \textbf{(véase Controller/Verifier.py, Model/Operator/}\break\textbf{Crossover)}.
\item \textbf{\texttt{crossover\_parameters}} (\emph{\texttt{Dictionary}}) -- Un diccionario con parámetros adicionales para la técnica de Cruza solicitada.
\item \textbf{\texttt{mutation\_instance}} (\emph{\texttt{Instance}}) -- Una instancia de la técnica de Mutation \textbf{(ó Mutación)} empleada por el usuario \textbf{(véase Controller/Verifier.py, Model/Operator/}\break\textbf{Mutation)}.
\item \textbf{\texttt{mutation\_parameters}} -- Un diccionario con parámetros adicionales para la técnica de Mutación usada.
\end{itemize}
\item[{Returns}] \leavevmode
El diccionario que resulta de aplicar el método \textbf{get\_results} que se encuentra en \textbf{Model/Community/Community.py}.
\item[{Return type}] \leavevmode
Dictionary
\end{description}\end{quote}

\end{fulllineitems}

Se debe tomar en consideración que por defecto las funciones antes 
mencionadas se encuentran implementadas en cada uno de los elementos de 
este módulo, por ello es que primordialmente se mostrarán aquéllas que 
no se contemplen en el esquema original, es decir, funciones auxiliares.\break
En el caso muy específico en el que alguna de las funciones obligatorias 
contenga información importante también se adjuntarán en el documento.\medskip\break  
A continuación se muestra la lista de los M.O.E.A.'s 
implementados:
%******* Termina función *******

%******* Empieza script *******
\subsubsection{VEGA (script)}
%Se coloca el vínculo interno procedente de esta misma sección (a_2_6_1).
\label{sec:a_2_6_1}
Se implementa la técnica M.O.E.A conocida como 
V.E.G.A. \textbf{(Vector Evaluated Genetic Algorithm ó Algoritmo Genético de Vectores Evaluados)}.\break
La forma de proceder del algoritmo es la siguiente:

\begin{enumerate} 
\item Se crea la Población Padre (de tamaño \(n\)).
\item Tomando en cuenta las \(k\) funciones objetivo y la Población Padre, se crean \(k\) subpoblaciones de tamaño \(n/k\) cada una, si este número llega a ser irracional se pueden hacer ajustes con respecto de la distribución de los Individuos.
\item Por cada subpoblación, se aplica la técnica de Selección y obtienen los \(n/k\) Individuos, terminado esto se deben unificar todos los seleccionados de nuevo en una súper Población.
\item Con la súper Población del paso 3, se crea a la Población Hija, la cual pasará a convertirse en la la nueva Población Padre.
\item Se repiten los pasos 2 a 4 hasta haber alcanzado el número de generaciones \textbf{(iteraciones)} límite.
\end{enumerate}

Como se puede apreciar es una implementación muy sencilla 
de optimización multiobjetivo, sin embargo el inconveniente 
que tiene es la fácil pérdida de material genético valioso.\break
Lo anterior significa que un Individuo que en una generación 
previa era el mejor para una función objetivo \(i\) al momento 
de ser separado y seleccionado en una subpoblación \(j\) \textbf{(y por ende analizado bajo la función objetivo \(j\))} 
puede ser muy malo en calidad y por tanto no ser seleccionado;
perdiéndose la ganancia genética hasta el momento obtenida para la función 
objetivo $i$, donde $i \ne j$.\\ Por ello es que se puede decir 
que V.E.G.A. genera soluciones promedio que destacan con una 
calidad media para todas las funciones objetivo.\medskip\break
Finalmente hay que comentar que para este algoritmo no se requiere 
aplicar un Ranking específico, no obstante se ha decidido utilizar 
el de Fonseca \& Flemming \textbf{(véase Model/Community/Community.py)} 
pues es el más sencillo de implementar.

%******* Empieza función *******
\begin{fulllineitems}

\pysiglinewithargsret{\sphinxbfcode{create\_subpopulations}}{\emph{comunidad}, \emph{main\_population}}{}
Método que divide a la Población principal en 
subpoblaciones de acuerdo al número de funciones 
objetivo.

\begin{quote}\begin{description}
\item[{Parameters}] \leavevmode\begin{itemize}
\item \textbf{\texttt{comunidad}} (\emph{\texttt{Instance}}) -- Una instancia de Community para poder crear
poblaciones..
\item \textbf{\texttt{main\_population}} (\emph{\texttt{Instance}}) -- La Población que será dividida.
\end{itemize}
\item[{Returns}] \leavevmode
Una lista con las subpoblaciones \textbf{(de tipo Population)}.
\item[{Return type}] \leavevmode
List
\end{description}\end{quote}

\end{fulllineitems}
%******* Termina función *******
%******* Termina script *******

%******* Empieza script *******
\subsubsection{SPEAII (script)}
%Se coloca el vínculo interno procedente de esta misma sección (a_2_6_2).
\label{sec:a_2_6_2}
Se desarrolla la implementación de la técnica M.O.E.A. 
conocida como S.P.E.A. II \textbf{(Strength Pareto Evolutionary Algorithm ó Algoritmo Evolutivo de Fuerza de Pareto)}.\break
El funcionamiento del algoritmo es el siguiente:

\begin{enumerate}
\item Se inicializa una Población llamada \emph{P} y un conjunto inicialmente vacío llamado \emph{E} \textbf{(E albergará Individuos también)}; ambos son de tamaño n.
\item Se asigna el Fitness a los Individuos de \emph{P} y \emph{E} \textbf{(para ello se evalúan las funciones objetivo de los Individuos de ambos conjuntos y se asigna el Ranking Zitzler \& Thiele)}.
\item A continuación se funden \emph{P} y \emph{E} en una súper Población \textbf{(denominado S también señalado en el algoritmo como Mating Pool, de tamaño n)}.Para ello primero se añaden los Individuos \emph{NO DOMINADOS} de \emph{P} en \emph{S} y posteriormente los \emph{NO DOMINADOS} de \emph{E} en \emph{S}.\break 
Aquí se distinguen dos casos:
      \begin{enumerate}
      \item Si llegasen a faltar Individuos se añaden al azar Individuos \emph{DOMINADOS} de \emph{P} en \emph{S} hasta completar la demanda.
      \item Si después de la fusión el número de Individuos supera a n, entonces se hace un truncamiento en \emph{S} hasta ajustar su tamaño a n.
      \end{enumerate}
\item \emph{S} será la nueva \emph{E}, además se crea la Población Hija de la recién creada \emph{E} \textbf{(E-Child)}.
\item E-Child será la nueva P.
\item Se repiten los pasos 2 a 5 hasta que se haya alcanzado el límite de generaciones \textbf{(iteraciones)}.
\end{enumerate}

Finalmente lo que se regresa es \emph{E}, ya que ahí es 
donde se han almacenado los mejores Individuos de todas 
las generaciones.\medskip\break
La característica de este algoritmo es que tiene una 
Presión Selectiva alta ya que se da prioridad a los 
Individuos no dominados \textbf{(de ahí el nombre de Fuerza de Pareto ó los más fuertes con respecto al principio de Pareto)},
y el hecho de mezclar a \emph{E} y \emph{P} en una 
súper Población garantiza la conservación de los mejores 
Individuos sin importar el transcurso de las generaciones \textbf{(a eso se le conoce como Elitismo)}, 
pero también da una tolerancia, aunque mínima, a los Individuos 
de menor calidad como en el punto 3.\break
Además al momento de actualizar \emph{S} a \emph{E} y 
E-Child a \emph{P} se tiene una especie de seguro de vida, 
es decir, si en algún momento la Población E-Child llegara a
tener una calidad baja se tiene el respaldo de \emph{E} 
para una generación posterior para formar \emph{S}.\medskip\break
Se debe tener en cuenta que el algoritmo originalmente no 
contempla ni una súper Población \emph{S} ni E-Child 
sino que en los pasos 3 y 4 se utiliza solamente \emph{E} 
para referirse tanto a E-child como a \emph{S}, sin embargo 
para no confundir al usuario en la funcionalidad del método 
se decidió colocar contenedores extra para poder diferenciar 
más precisamente a los elementos involucrados.\medskip\break
Algo muy importante a mencionar es que en el paso 1 y al momento 
de crear la Población E-Child es necesario evaluar las funciones 
objetivo, asignar un Ranking y posteriormente un Fitness para 
que se puedan aplicar los operadores geneticos \textbf{(véase Model/GeneticOperator)}, 
para este caso el Ranking es estrictamente el de Zitzler \& Thiele; 
la descripción completa de éste se encuentra en 
\textbf{Model/Community/Community.py}.
%******* Termina script *******

%******* Empieza script *******
\subsubsection{MOGA (script)}
%Se coloca el vínculo interno procedente de esta misma sección (a_2_6_3).
\label{sec:a_2_6_3}
Se desarrolla la técnica M.O.E.A. que lleva por 
nombre M.O.G.A. \textbf{(Multi Objective Genetic Algorithm ó Algoritmo Genético Multi Objetivo)}.\break
Su funcionamiento es el siguiente:

\begin{enumerate} 
\item Se crea la Población Padre, se evalúan las funciones objetivo de sus correspondientes Individuos.
\item Se asigna a los Individuos un Ranking \textbf{(Fonseca \& Flemming)} y posteriormente se calcula el Niche Count de la Población Padre.
\item Tomando en cuenta los valores del punto 2 se obtiene el Fitness para cada Individuo y posteriormente su Shared Fitness.
\item Se aplica el operador de selección sobre la Población Padre para determinar los elegidos para dejar descendencia.
\item Se crea la Población Hija, se evalúan las funciones objetivo de sus correspondientes Individuos.
\item Se asigna a los Individuos un Ranking \textbf{(Fonseca \& Flemming)} y posteriormente se calcula el Niche Count de la Población Hija.
\item Tomando en cuenta los valores del punto 6 se obtiene el Fitness para cada Individuo y posteriormente su Shared Fitness.
\item La Población Hija pasará a ser la nueva Población Padre.
\item Se repiten los pasos 4 a 8 hasta que se haya alcanzado el número límite de generaciones \textbf{(iteraciones)}.
\end{enumerate}

Como se puede apreciar, la implementación de este algoritmo es 
muy sencilla, además se rige casi en su totalidad por el 
Shared Fitness \textbf{(ó Fitness Compartido)}, por lo que 
la Presión Selectiva \textbf{(ó Selective Pressure)} incluida 
dependerá en gran medida de la función de Distancia que se utilice, 
así como de la magnitud indicada por el usuario.\medskip\break
Finalmente es menester mencionar que para esta implementación el 
Ranking utilizado debe ser estrictamente el de Fonseca \& Flemming 
\textbf{(véase Model/Community/Community.py)}.
%******* Termina script *******

%******* Empieza script *******
\subsubsection{NSGAII (script)}
%Se coloca el vínculo interno procedente de esta misma sección (a_2_6_4).
\label{sec:a_2_6_4}
En esta parte se lleva a cabo la implementación del M.O.E.A. 
denominado N.S.G.A. II \textbf{(Non-dominated Sorting Genetic Algorithm ó Algoritmo Genético de Ordenamiento No Dominado)}.\break
La forma de proceder del método es la siguiente:

\begin{enumerate}
\item Se crea una Población Padre \textbf{(de tamaño n)}, a la cual se le evalúan las funciones objetivo de sus Individuos, se les asigna un Ranking \textbf{(Goldberg)} y posteriormente se les otorga un Fitness.
\item Con base en la Población Padre se aplica el operador de Selección para elegir a los Individuos que serán aptos para reproducirse.
\item Usando a los elementos del punto 2, se crea una Población Hija \textbf{(de tamaño n)}.
\item Se crea una súper Población \textbf{(denominado S, de tamaño 2n)} que albergará todos los Individuos tanto de la Población Padre como Hija; a \emph{S} se le evalúan las funciones objetivo de sus Individuos, se les asigna un Ranking \textbf{(Goldberg)} y posteriormente se les otorga un Fitness.
\item La súper Población \emph{S} se divide en subcategorías de acuerdo a los niveles de dominancia que existan, es decir, existirá la categoría de dominancia 0, la cual almacena Individuos que tengan una dominancia de 0 Individuos \textbf{(ningún Individuo los domina)}, existirá la categoría de dominancia 1 con el significado análogo y así sucesivamente hasta haber cubierto todos los niveles de dominancia existentes.
\item Se construye la nueva Población Padre, pare ello constará de los Individuos de \emph{S} donde la prioridad será el nivel de dominancia, es decir, primero se añaden los elementos del nivel 0,luego los del nivel 1 y así en lo sucesivo hasta haber adquirido n elementos.
Se debe aclarar que la adquisición de Individuos por nivel debe ser total, esto significa que no se pueden dejar Individuos sueltos para el mismo nivel de dominancia.\break
Supongamos que a un nivel k existen tantos Individuos que su presunta adquisición supera el tamaño n, en este caso se debe hacer lo siguiente:
      \begin{enumerate}
      \item Se crea una Población provisional \textbf{(Prov)} con los Individuos del nivel k, se evalúan las funciones objetivo a cada uno de sus Individuos, se les asigna un Ranking \textbf{(Goldberg)} y posteriormente se les asigna el Fitness.\break
            Con los valores anteriores se calcula el Niche Count \textbf{(véase Model/}\break\textbf{SharingFunction)} de los Individuos; una vez hecho ésto se seleccionan desde Prov los Individuos faltantes con los mayores Niche Count, esto hasta completar el tamaño n de la nueva Población Padre.
      \end{enumerate}
\item Al haber conformado la nueva Población Padre, se evalúan las funciones objetivo de sus Individuos, se les asigna el Ranking correspondiente \textbf{(Goldberg)} y se les atribuye su Fitness.
\item Se repiten los pasos 2 a 7 hasta haber alcanzado el límite de generaciones \textbf{(iteraciones)}.
\end{enumerate}

Como su nombre lo indica, la característica de este algoritmo es 
la clasificación de los Individuos en niveles para su posterior 
selección.\break
Esto al principio propicia una Presión Selectiva moderada por 
la ausencia de elementos con dominancia baja que suele existir 
en las primeras generaciones, sin embargo en iteraciones posteriores 
se agudiza la Presión Selectiva ya que eventualmente la mayoría de 
los Individuos serán alojados en las primeras categorías de dominancia, 
cubriendo casi instantáneamente la demanda de Individuos necesaria en 
el paso 6, por lo que las categorías posteriores serán cada vez 
menos necesarias con el paso de los ciclos.\medskip\break
Por otra parte la fusión de las Poblaciones en \emph{S} garantiza 
que siempre se conserven a los mejores Individuos independientemente 
de la generación transcurrida, a eso se le llama Elitismo.\break
Por cierto que en el algoritmo original no existe un nombre oficial 
para \emph{S} sino más bien se señala como una estructura genérica, 
sin embargo se le ha formalizado con un identificador para guiar 
apropiadamente al usuario en el flujo del algoritmo.\medskip\break
Para finalizar se señala que el uso del ranking de Goldberg 
\textbf{(véase Model/Community/}\break\textbf{Community.py)} es indispensable.
%******* Termina script *******
%******* Termina módulo *******

%Termina el documento.
\end{document}

%Autor: Aarón Martín Castillo Medina.
%Asesora: Dra. Katya Rodríguez Vázquez
%Contacto: katya.rodriguez@iimas.unam.mx; amcm329@hotmail.com

%Este archivo contiene información relacionada con la capa Vista
%(ó View), la cual representa tanto física como lógicamente a uno de 
%los componentes que conforman el producto de software y por tanto al 
%Manual Técnico. 


%Se indica que el documento es de tipo reporte bajo el paquete standalone.
\documentclass[class=report, crop=false]{standalone}

%Se cargan los paquetes relacionados con los subapéndices (elementos que
%conforman el Apéndice en su totalidad).
\usepackage{packages_used_section}

%Comienza el documento.
\begin{document}

\section{View (sección)}
%Se coloca el vínculo interno procedente de esta misma sección (a_3).
\label{sec:a_3}
La capa View \textbf{(ó Vista)} contiene todos los elementos que 
serán alusivos a la interfaz gráfica. De acuerdo al modelo MVC 
\textbf{(Model-View-Controller)}, opera exclusivamente con la capa 
Controller \textbf{(ó Controlador)}.\medskip\break
A continuación se muestran los elementos que conforman esta sección.

%******* Empieza clase *******
\subsection{MainWindow (clase)}
%Se coloca el vínculo interno procedente de esta misma sección (a_3_1).
\label{sec:a_3_1}

%******* Empieza descripción *******
\begin{fulllineitems}

\begin{DUlineblock}{0em}
\item[] Mezcla todas las estructuras gráficas que forman parte de la 
sección View \textbf{(ó vista)}.\break
Se trata de una Ventana que contendrá todas las opciones necesarias 
para que el usuario pueda ejecutar a voluntad M.O.E.A.'s \textbf{(Multi Objective Evolutionary Algorithm
ó Algoritmo Evolutivo Multi Objetivo)}\medskip\break
El flujo que se suele seguir es el siguiente:

\begin{itemize}
\item El usuario ingresa las características que desea que contenga el M.O.E.A. que será ejecutado.
\item Posteriormente el Controller \textbf{(ó Controlador, véase Controller/Controller.py)} verifica la consistencia de los datos anteriores para que no haya conflicto en el lado del Model \textbf{(ó Modelo)}.
\item Si no existe problema alguno se prosigue con el proceso, en otro caso se arroja un mensaje de error.
\item Siguiendo con el flujo normal se ejecutará una instancia del M.O.E.A. solicitado en la capa de Model \textbf{(ó Modelo)}, la cual tendrá una ventana asociada en View \textbf{(ó Vista)} que indicará el progreso del primero.
\item Cuando una instancia termine de ejecutarse, la ventana del progreso desaparece y en su lugar se muestra otra conteniendo los resultados del M.O.E.A. \textbf{(véase View/Additional/ResultsGrapher/ResultsGrapherToplevel.py)}.
\end{itemize}

Es importante mencionar que esta clase y el proyecto en general están 
diseñados para que se puedan crear varias instancias simultáneamente, 
con ello se espera aprovechar al máximo los recursos computacionales 
en los que el proyecto fuera a ejecutarse.
\end{DUlineblock}

\begin{quote}\begin{description}
\item[{Returns}] \leavevmode
Tkinter.Frame
\item[{Return type}] \leavevmode
Instance
\end{description}\end{quote}

%******* Termina descripción *******

%******* Empieza función *******
\begin{fulllineitems}

\pysiglinewithargsret{\sphinxbfcode{change\_frame}}{\emph{current\_frame\_name}}{}~

\begin{notice}{note}{Note:}
Este método es privado.
\end{notice}

Hace el cambio en la Ventana Principal ocultando un 
Frame y mostrando otro.

\begin{quote}\begin{description}
\item[{Parameters}] \leavevmode\begin{itemize}
\item \textbf{\texttt{current\_frame\_name}} (\emph{\texttt{Tkinter.Frame}}) -- El Frame que se va a mostrar en la Ventana Principal.
\end{itemize}
\end{description}\end{quote}

\end{fulllineitems}
%******* Termina función *******

%******* Empieza función *******
\begin{fulllineitems}

\pysiglinewithargsret{\sphinxbfcode{check\_queues}}{}{}~

\begin{notice}{note}{Note:}
Este método es privado.
\end{notice}

Una vez iniciado un proceso que ejecuta un M.O.E.A., este 
método revisa periódicamente las colas \textbf{(Queues)} 
sobre las cuales los procesos escribirán todo tipo de 
información pertinente.

\end{fulllineitems}
%******* Termina función *******

%******* Empieza función *******
\begin{fulllineitems}

\pysiglinewithargsret{\sphinxbfcode{get\_information}}{}{}~

\begin{notice}{note}{Note:}
Este método es privado.
\end{notice}

Método que obtiene los datos ingresados por el usuario
de cada uno de los Frames asociados a esta Ventana Principal.

\begin{quote}\begin{description}
\item[{Returns}] \leavevmode
Un diccionario con toda las preferencias del usuario recolectadas para cada uno de los Frames disponibles.
\item[{Return type}] \leavevmode
Dictionary
\end{description}\end{quote}

\end{fulllineitems}
%******* Termina función *******

%******* Empieza función *******
\begin{fulllineitems}

\pysiglinewithargsret{\sphinxbfcode{init\_procedure}}{\emph{event}}{}~

\begin{notice}{note}{Note:}
Este método es privado.
\end{notice}

Inicia el procedimiento para ejecutar un M.O.E.A.\break
Los pasos que se realizan son:

\begin{itemize}
\item Recolecta las preferencias ingresadas por el usuario en los Frames que conforman la Ventana Principal.
\item Se sanitizan dichos datos con ayuda del Controller.
\item En caso de no haber problemas con la sanitización, se ejecuta el proceso alojándolo en un hilo \textbf{(Thread)} para que permita seguir teniendo acceso a la Ventana Principal; por el contrario si hubo alguna falla regresa un mensaje de error.
\end{itemize}

Gracias a este método el proyecto entero tiene la característica 
de ser Multi-Hilo \textbf{(ó Multi-Threading)}, es decir, se pueden 
ejecutar varios procedimientos de manera independiente.

\begin{quote}\begin{description}
\item[{Parameters}] \leavevmode\begin{itemize}
\item\textbf{\texttt{event}} (\emph{\texttt{String}}) -- El evento del elemento gráfico que activa esta función.
\end{itemize}
\end{description}\end{quote}

\end{fulllineitems}
%******* Termina función *******

%******* Empieza función *******
\begin{fulllineitems}

\pysiglinewithargsret{\sphinxbfcode{initialize\_frames}}{}{}~

\begin{notice}{note}{Note:}
Este método es privado.
\end{notice}

Método que inicializa los Frames que se 
colocarán en la Ventana Principal como opciones.

\end{fulllineitems}
%******* Termina función *******

%******* Empieza función *******
\begin{fulllineitems}

\pysiglinewithargsret{\sphinxbfcode{load\_images}}{}{}~

\begin{notice}{note}{Note:}
Este método es privado.
\end{notice}

Carga las imágenes que se encuentran en el directorio 
View/Images para que puedan ser usadas por los Frames.\break
Es importante recalcar que el método sólo carga imágenee .gif 
ya que son la extensión más estable para que se muestren 
las imágenes en la interfaz gráfica.

\begin{quote}\begin{description}
\item[{Returns}] \leavevmode
Un diccionario con todas las imágenes cargadas.
\item[{Return type}] \leavevmode
Dictionary
\end{description}\end{quote}

\end{fulllineitems}
%******* Termina función *******

%******* Empieza función *******
\begin{fulllineitems}

\pysiglinewithargsret{\sphinxbfcode{obtain\_results}}{\emph{execution\_task\_count}, \emph{generations\_queue}, \emph{gathered\_information}, \emph{sanitized\_information}}{}~

\begin{notice}{note}{Note:}
Este método es privado.
\end{notice}

Ejecuta un M.O.E.A..\break
Esta es la función que se coloca en un hilo para ser llevada a 
cabo de manera independiente con la finalidad de dejar libre la 
Ventana Principal y de manera secundaria ejecutar varios 
procedimientos simultáneamente.

\begin{quote}\begin{description}
\item[{Parameters}] \leavevmode\begin{itemize}
\item \textbf{\texttt{execution\_task\_count}} (\emph{\texttt{Integer}}) -- El número de proceso actual.
\item \textbf{\texttt{generations\_queue}} (\emph{\texttt{Instance}}) -- Una referencia a una cola \textbf{(Queue)} donde los procesos escribirán su avance
en cuanto a las generaciones transcurridas.
\item \textbf{\texttt{gathered\_information}} (\emph{\texttt{Dictionary}}) -- La información que el usuario ingresó al momento de iniciar el proceso actual.
\item \textbf{\texttt{sanitized\_information}} (\emph{\texttt{Dictionary}}) -- La información anterior sanitizada.
\end{itemize}
\end{description}\end{quote}

\end{fulllineitems}
%******* Termina función *******

%******* Empieza función *******
\begin{fulllineitems}

\pysiglinewithargsret{\sphinxbfcode{restore\_settings}}{\emph{event}}{}~

\begin{notice}{note}{Note:}
Este método es privado.
\end{notice}

Limpia y deja por defecto los valores estándar del Frame
mostrado actualmente en la Ventana Principal.\break
El método no aplica para regresar a M.O.P.'s \textbf{(Multi Objective Problems)}
cargados anteriormente.

\begin{quote}\begin{description}
\item[{Parameters}] \leavevmode\begin{itemize}
\item \textbf{\texttt{event}} (\emph{\texttt{String}}) -- El evento del elemento gráfico que acciona esta función.
\end{itemize}
\end{description}\end{quote}

\end{fulllineitems}
%******* Termina función *******

%******* Empieza función *******
\begin{fulllineitems}

\pysiglinewithargsret{\sphinxbfcode{update\_frame}}{\emph{event}}{}~

\begin{notice}{note}{Note:}
Este método es privado.
\end{notice}

Muestra en la Ventana Principal el Frame actual.

\begin{quote}\begin{description}
\item[{Parameters}] \leavevmode\begin{itemize}
\item \textbf{\texttt{event}} (\emph{\texttt{String}}) -- El evento del elemento gráfico que 
activa la función.
\end{itemize}
\end{description}\end{quote}

\end{fulllineitems}
%******* Termina función *******

%******* Empieza función *******
\begin{fulllineitems}

\pysiglinewithargsret{\sphinxbfcode{load\_mop\_example}}{\emph{elements}}{}
Carga el M.O.P \textbf{(Multi Objective Problem)} seleccionado
a los Frames correspondientes \textbf{(Objective Functions y Decision Variables)}.

\begin{quote}\begin{description}
\item[{Parameters}] \leavevmode\begin{itemize}
\item \textbf{\texttt{elements}} (\emph{\texttt{Array}}) -- Un arreglo que contiene dos elementos, el primero son las funciones objetivo precargadas mientras que el segundo son las variables de decisión también precargadas. Ambas provienen del menú secundario \textbf{(véase View/Additional/}\break\textbf{MenuInternalOption/InternalOptionTab/}\break\textbf{MOPExampleFrame.py, Controller/XML/}\break\textbf{MOPExamples.xml)}.
\end{itemize}
\end{description}\end{quote}

\end{fulllineitems}
%******* Termina función *******

%******* Empieza función *******
\begin{fulllineitems}

\pysiglinewithargsret{\sphinxbfcode{resource\_path}}{\emph{relative\_path}}{}~
\vspace{-0.1cm}
Esta función se utiliza para poder crear ejecutables
apropiadamente.\break
A grandes rasgos el ejecutable se empaqueta en un directorio
llamado \_MEIPASS, entonces aquí se implementa la búsqueda
de dicho archivo devolviendo un path \textbf{(ruta)}.

\begin{quote}\begin{description}
\item[{Returns}] \leavevmode
La ruta del directorio \_MEIPASS.
\item[{Return type}] \leavevmode
String
\end{description}\end{quote}

\end{fulllineitems}
%******* Termina función *******

%******* Empieza función *******
\begin{fulllineitems}

\pysiglinewithargsret{\sphinxbfcode{run}}{}{}
Lanza la Ventana Principal.

\end{fulllineitems}
%******* Termina función *******

\end{fulllineitems}
%******* Termina clase *******

%******* Empieza módulo *******
\subsection{Main (módulo)}
%Se coloca el vínculo interno procedente de esta misma sección (a_3_2).
\label{sec:a_3_2}
Contiene todos los elementos gráficos para que el usuario
pueda configurar los atributos que intervienen en la ejecución 
de un M.O.E.A.\medskip\break
A continuación se colocan todos los elementos que constituen el 
módulo en cuestión:

%******* Empieza módulo *******
\subsubsection{Home (módulo)}
%Se coloca el vínculo interno procedente de esta misma sección (a_3_2_1).
\label{sec:a_3_2_1}
Contiene toda la información posible para poder describir tanto 
los elementos que conforman el programa como su correcto uso.\medskip\break
Los elementos que constituyen a este módulo son:

%******* Empieza clase *******
\paragraph{HomeFrame (clase)}
%Se coloca el vínculo interno procedente de esta misma sección (a_3_2_1_1).
\label{sec:a_3_2_1_1}

%******* Empieza descripción *******
\begin{fulllineitems}

\begin{DUlineblock}{0em}
\item[] Unifica dos elementos: Canvas e IntroductionFrame.\break
La razón de haber hecho esto es que, cuando se añaden demasiados 
elementos al IntroductionFrame, se tiene que agregar una barra de 
desplazamiento para poder acceder a los que se encuentran hasta abajo.\break
Dentro del ambiente de Tkinter, el elemento más sencillo para lograr 
esto es un Canvas, por ello se anida el IntroductionFrame al Canvas.
\end{DUlineblock}

\begin{quote}\begin{description}
\item[{Parameters}] \leavevmode\begin{itemize}
\item \textbf{\texttt{parent}} (\emph{\texttt{Tkinter.Frame}}) -- Frame padre al que pertenece.
\item \textbf{\texttt{features}} (\emph{\texttt{Dictionary}}) -- Conjunto de técnicas con sus respectivos parámetros para que se puedan cargar automáticamente en este Frame \textbf{(véase
Controller/XMLParser.py)}.
\end{itemize}

\item[{Returns}] \leavevmode
Tkinter.Frame
\item[{Return type}] \leavevmode
Instance
\end{description}\end{quote}

%******* Termina descripción *******

%******* Empieza función *******
\begin{fulllineitems}

\pysiglinewithargsret{\sphinxbfcode{update\_scrollbar}}{\emph{event=None}}{}~

\begin{notice}{note}{Note:}
Este método es privado.
\end{notice}

Actualiza la barra de desplazamiento de acuerdo al número de 
elementos existentes en el Frame, esto para poder hacer un 
recorrido apropiado de la barra.

\begin{quote}\begin{description}
\item[{Parameters}] \leavevmode\begin{itemize}
\item \textbf{\texttt{event}} (\emph{\texttt{String}}) -- Elemento que ejecutó esta función.
\end{itemize}
\end{description}\end{quote}

\end{fulllineitems}
%******* Termina función *******

%******* Empieza función *******
\begin{fulllineitems}

\pysiglinewithargsret{\sphinxbfcode{move\_to\_section}}{\emph{y\_coordinate}}{}
Mueve la barra de desplazamiento (y por ende el contenido)
con base en la coordenada (en Y) que se le pase como parámetro.

\begin{quote}\begin{description}
\item[{Parameters}] \leavevmode\begin{itemize}
\item \textbf{\texttt{y\_coordinate}} -- Coordenada que se necesita para hace el desplazamiento. Oscila entre 0 y 1.
\end{itemize}
\end{description}\end{quote}

\end{fulllineitems}
%******* Termina función *******

%******* Empieza función *******
\begin{fulllineitems}

\pysiglinewithargsret{\sphinxbfcode{restore\_settings}}{}{}
Restaura la configuración del Frame a la que tenía por
defecto.

\end{fulllineitems}
%******* Termina función *******

\end{fulllineitems}
%******* Termina clase *******

La clase actual se apoya del elemento mostrado a continuación:

%******* Empieza clase *******
\subparagraph{IntroductionFrame (clase)}
%Se coloca el vínculo interno procedente de esta misma sección (a_3_2_1_1_1).
\label{sec:a_3_2_1_1_1}
%******* Empieza descripción *******
\begin{fulllineitems}

\begin{DUlineblock}{0em}
\item[] Contiene información básica y concisa sobre el producto 
de software, la cual es organizada y mostrada de acuerdo al número 
de secciones existentes en éste.\break
De manera secundaria proporciona la infraestructura para poder 
darle al usuario un desplazamiento más rápido entre dichas 
secciones.
\end{DUlineblock}

\begin{quote}\begin{description}
\item[{Parameters}] \leavevmode\begin{itemize}
\item \textbf{\texttt{parent}} (\emph{\texttt{Tkinter.Frame}}) -- Frame padre al que pertenece.
\item \textbf{\texttt{canvas\_function}} (\emph{\texttt{Instance}}) -- Una función alusiva al funcionamiento del Canvas.
\item \textbf{\texttt{features}} (\emph{\texttt{Dictionary}}) -- Conjunto de técnicas con sus respectivos parámetros para que se puedan cargar automáticamente en este Frame.
\end{itemize}

\item[{Returns}] \leavevmode
Tkinter.Frame
\item[{Return type}] \leavevmode
Instance
\end{description}\end{quote}

%******* Termina descripción *******

%******* Empieza función *******
\begin{fulllineitems}

\pysiglinewithargsret{\sphinxbfcode{go\_to\_selected\_section}}{\emph{event}}{}~

\begin{notice}{note}{Note:}
Este método es privado.
\end{notice}

Con base en la liga elegida por el usuario, realiza el
desplazamiento hacia la sección correspondiente.

\begin{quote}\begin{description}
\item[{Parameters}] \leavevmode\begin{itemize}
\item \textbf{\texttt{event}} (\emph{\texttt{String}}) -- Elemento que ejecutó esta función.
\end{itemize}
\end{description}\end{quote}

\end{fulllineitems}
%******* Termina función *******

\end{fulllineitems}
%******* Termina clase *******
%******* Termina módulo *******

%******* Empieza módulo *******
\subsubsection{DecisionVariable (módulo)}
%Se coloca el vínculo interno procedente de esta misma sección (a_3_2_2).
\label{sec:a_3_2_2}

Proporciona los elementos gráficos para que el usuario pueda
insertar, modificar y eliminar variables de decisión con sus 
respectivos rangos.\medskip\break
Los elementos que constituyen al módulo son:

%******* Empieza clase *******
\paragraph{DecisionVariableFrame (clase)}
%Se coloca el vínculo interno procedente de esta misma sección (a_3_2_2_1).
\label{sec:a_3_2_2_1}
%******* Empieza descripción *******
\begin{fulllineitems}

\begin{DUlineblock}{0em}
\item[] Realiza la fusión de Canvas y VariableFrame, debido a que, 
cuando se agregan numerosas variables al VariableFrame, se debe 
insertar una barra de desplazamiento para poder acceder a aquéllos 
que se encuentren hasta abajo.\break
Dentro del ambiente de Tkinter, el elemento más sencillo para lograr 
este efecto es un Canvas, por ello se anida el VariableFrame al Canvas.
\end{DUlineblock}

\begin{quote}\begin{description}
\item[{Parameters}] \leavevmode\begin{itemize}
\item \textbf{\texttt{parent}} (\emph{\texttt{Tkinter.Frame}}) -- Frame padre al que pertenece.
\item \textbf{\texttt{features}} (\emph{\texttt{Dictionary}}) -- Conjunto de técnicas con sus respectivos parámetros para que se puedan cargar automáticamente en este Frame \textbf{(véase Controller/XMLParser.py)}.
\end{itemize}

\item[{Returns}] \leavevmode
Tkinter.Frame
\item[{Return type}] \leavevmode
Instance
\end{description}\end{quote}

%******* Termina descripción *******

%******* Empieza función *******
\begin{fulllineitems}

\pysiglinewithargsret{\sphinxbfcode{activate\_scroll}}{\emph{event}}{}~

\begin{notice}{note}{Note:}
Este método es privado.
\end{notice}

Actualiza la barra de desplazamiento y con base en esta acción
la activa o desactiva.

\begin{quote}\begin{description}
\item[{Parameters}] \leavevmode\begin{itemize}
\item \textbf{\texttt{event}} (\emph{\texttt{String}}) -- Elemento que ejecutó esta función.
\end{itemize}
\end{description}\end{quote}

\end{fulllineitems}
%******* Termina función *******

%******* Empieza función *******
\begin{fulllineitems}

\pysiglinewithargsret{\sphinxbfcode{update\_scrollbar}}{\emph{event=None}}{}~

\begin{notice}{note}{Note:}
Este método es privado.
\end{notice}

Actualiza la barra de desplazamiento de acuerdo al número de elementos
existentes en el Frame, esto para poder hacer un recorrido apropiado de 
la barra.

\begin{quote}\begin{description}
\item[{Parameters}] \leavevmode\begin{itemize}
\item \textbf{\texttt{event}} (\emph{\texttt{String}}) -- Elemento que ejecutó esta función.
\end{itemize}
\end{description}\end{quote}

\end{fulllineitems}
%******* Termina función *******

%******* Empieza función *******
\begin{fulllineitems}

\pysiglinewithargsret{\sphinxbfcode{get\_information}}{}{}
Regresa la información recabada en el Frame.

\begin{quote}\begin{description}
\item[{Returns}] \leavevmode
\item[{Return type}] \leavevmode
Dictionary
\end{description}\end{quote}

\end{fulllineitems}
%******* Termina función *******

%******* Empieza función *******
\begin{fulllineitems}

\pysiglinewithargsret{\sphinxbfcode{insert\_mop\_example}}{\emph{variables}}{}~
Inserta un M.O.P \textbf{(Multi Objective Problem ó Problema Multi Objetivo)}.\break
En este caso significa que se insertarán las variables con sus respectivos 
rangos en el Frame para poder hacer pruebas rápidas en el programa, habiendo 
antes limpiado por completo el contenido del Frame.\break
\textbf{(véase Controller/XML/MOPExample.xml)}\break
\textbf{(véase View/Additional/MenuInternalOption/InternalOptionFrame.py)}.\break

\begin{quote}\begin{description}
\item[{Parameters}] \leavevmode\begin{itemize}
\item \textbf{\texttt{functions}} (\emph{\texttt{List}}) -- Lista de variables para ser insertadas en el Frame.
\end{itemize}
\end{description}\end{quote}

\end{fulllineitems}
%******* Termina función *******

%******* Empieza función *******
\begin{fulllineitems}

\pysiglinewithargsret{\sphinxbfcode{restore\_settings}}{}{}
Restaura el contenido del Frame, en este caso significa que 
se eliminará todo lo que esté en éste y se dejará una casilla 
vacía libre.

\end{fulllineitems}
%******* Termina función *******

\end{fulllineitems}
%******* Termina clase *******

La clase actual se basa en el siguiente elemento:

%******* Empieza clase *******
\subparagraph{VariableFrame (clase)}
%Se coloca el vínculo interno procedente de esta misma sección (a_3_2_2_1_1).
\label{sec:a_3_2_2_1_1}
%******* Empieza descripción *******
\begin{fulllineitems}

\begin{DUlineblock}{0em}
\item[] Proporciona bases gráficas para que el usuario pueda insertar
variables de decisión, así como información relativa a éstas.\break
En términos generales, el usuario insertará casillas para ingresar variables
de decisión, indicando también el valor mínimo y máximo que podrán tener.\medskip\break
Es importante comentar que todas las variables de decisión deben contener
rangos finitos, es decir, no se contemplan valores infinitos, aunque algunos
M.O.P.'s \textbf{(Multi Objective Problems ó Problemas Multi Objetivo)} 
manejan este tipo de rangos.         
\end{DUlineblock}

\begin{quote}\begin{description}
\item[{Parameters}] \leavevmode\begin{itemize}
\item \textbf{\texttt{parent}} (\emph{\texttt{Tkinter.Frame}}) -- Frame padre al que pertenece.
\item \textbf{\texttt{features}} (\emph{\texttt{Dictionary}}) -- Conjunto de técnicas con sus respectivos parámetros para que se puedan cargar automáticamente en este Frame \textbf{(véase Controller/XMLParser.py)}.
\end{itemize}

\item[{Returns}] \leavevmode
Tkinter.Frame
\item[{Return type}] \leavevmode
Instance
\end{description}\end{quote}

%******* Termina descripción *******

%******* Empieza función *******
\begin{fulllineitems}

\pysiglinewithargsret{\sphinxbfcode{add\_variable}}{\emph{event}}{}~
\begin{notice}{note}{Note:}
Este método es privado.
\end{notice}

Agrega una casilla al Frame.\break
Esta función se usa si fue ejecutada por un 
evento.

\begin{quote}\begin{description}
\item[{Parameters}] \leavevmode\begin{itemize}
\item \textbf{\texttt{event}} (\emph{\texttt{String}}) -- Identificador del elemento gráfico que activó la función.
\end{itemize}
\end{description}\end{quote}

\end{fulllineitems}
%******* Termina función *******

%******* Empieza función *******
\begin{fulllineitems}

\pysiglinewithargsret{\sphinxbfcode{delete\_single\_variable}}{\emph{event}}{}~

\begin{notice}{note}{Note:}
Este método es privado.
\end{notice}

Elimina una casilla y todos los elementos gráficos que 
la acompañan.\break
También elimina todo rastro que se encuentre en las 
estructuras lógicas.

\begin{quote}\begin{description}
\item[{Parameters}] \leavevmode\begin{itemize}
\item \textbf{\texttt{event}} (\emph{\texttt{String}}) -- Identificador del elemento gráfico que activó la función.
\end{itemize}
\end{description}\end{quote}

\end{fulllineitems}
%******* Termina función *******

%******* Empieza función *******
\begin{fulllineitems}

\pysiglinewithargsret{\sphinxbfcode{grid\_widgets}}{}{}~

\begin{notice}{note}{Note:}
Este método es privado.
\end{notice}

Coloca elementos en el Frame.

\end{fulllineitems}
%******* Termina función *******

%******* Empieza función *******
\begin{fulllineitems}

\pysiglinewithargsret{\sphinxbfcode{get\_current\_elements}}{}{}
Regresa el número actual de casillas en el Frame.

\begin{quote}\begin{description}
\item[{Returns}] \leavevmode
Cantidad de elementos en la estructura rows, donde se guardan las casillas (Entries).
\item[{Return type}] \leavevmode
Integer
\end{description}\end{quote}

\end{fulllineitems}
%******* Termina función *******

%******* Empieza función *******
\begin{fulllineitems}

\pysiglinewithargsret{\sphinxbfcode{get\_information}}{}{}
Toma la información del Frame y regresa las variables con 
sus rangos que el usuario ingresó.

\begin{quote}\begin{description}
\item[{Returns}] \leavevmode
Un diccionario que contiene una lista con las variables (y rangos) escritas.
\item[{Return type}] \leavevmode
Dictionary
\end{description}\end{quote}

\end{fulllineitems}
%******* Termina función *******

%******* Empieza función *******
\begin{fulllineitems}

\pysiglinewithargsret{\sphinxbfcode{insert\_mop\_example}}{\emph{variables}}{}~
\vspace{-0.3cm}

Inserta un M.O.P (Multi Objective Problem) que no es más 
que un conjunto de variables con sus rangos para que se 
pueda hacer más rápidamente una prueba.\break
Previo a ésto se limpia el Frame para insertar únicamente 
el M.O.P.\break
\textbf{(véase Controller/XML/MOPExample.xml)}\break
\textbf{(véase View/Additional/MenuInternalOption/InternalOptionFrame.py)}.

\begin{quote}\begin{description}
\item[{Parameters}] \leavevmode\begin{itemize}
\item \textbf{\texttt{functions}} (\emph{\texttt{List}}) -- Conjunto de variables para insertar en el Frame.
\end{itemize}
\end{description}\end{quote}

\end{fulllineitems}
%******* Termina función *******

%******* Empieza función *******
\begin{fulllineitems}

\pysiglinewithargsret{\sphinxbfcode{insert\_variable}}{\emph{variable=None}}{}~
\vspace{-0.3cm}

Coloca en el Frame una colección de elementos:\break
{[}casilla para insertar variable ,casilla de rango minimo, casilla de rango máximo, botón para eliminar{]}\break
Si el parámetro function es \textbf{None}, se añade 
la casilla vacía, de lo contrario se agrega ésta con 
la variable y sus rangos.

\begin{quote}\begin{description}
\item[{Parameters}] \leavevmode\begin{itemize}
\item \textbf{\texttt{function}} (\emph{\texttt{String}}) -- Una terna (nombre de la variable, rango máximo, rango mínimo) para ser insertada en las casillas correspondientes.
\end{itemize}
\end{description}\end{quote}

\end{fulllineitems}
%******* Termina función *******

%******* Empieza función *******
\begin{fulllineitems}

\pysiglinewithargsret{\sphinxbfcode{restore\_settings}}{}{}
Restaura el contenido del Frame a sus valores por defecto.\break
Esto significa que borrará cualquier contenido que se 
encuentre en existencia y dejará una casilla vacía.

\end{fulllineitems}
%******* Termina función *******

\end{fulllineitems}
%******* Termina clase *******
%******* Termina módulo *******

%******* Empieza módulo *******
\subsubsection{ObjectiveFunction (módulo)}
%Se coloca el vínculo interno procedente de esta misma sección (a_3_2_3).
\label{sec:a_3_2_3}
Proporciona los elementos gráficos para que el usuario
pueda insertar, modificar y eliminar funciones objetivo.\medskip\break
Sus elementos que lo conforman son:

%******* Empieza clase *******
\paragraph{ObjectiveFunctionFrame (clase)}
%Se coloca el vínculo interno procedente de esta misma sección (a_3_2_3_1).
\label{sec:a_3_2_3_1}
%******* Empieza descripción *******
\begin{fulllineitems}

\begin{DUlineblock}{0em}
\item[] Unifica dos elementos: Canvas y FunctionFrame.\break
La razón de haber hecho esto es que, cuando se agregan muchas 
funciones al FunctionFrame, se tiene que agregar una barra de 
desplazamiento para poder acceder a los que se encuentran hasta 
abajo.\break
Dentro del ambiente de Tkinter, el elemento más sencillo para 
lograr esto es un Canvas, por ello se anida el FunctionFrame al 
Canvas.
\end{DUlineblock}

\begin{quote}\begin{description}
\item[{Parameters}] \leavevmode\begin{itemize}
\item \textbf{\texttt{parent}} (\emph{\texttt{Tkinter.Frame}}) -- Frame padre al que pertenece.
\item \textbf{\texttt{features}} (\emph{\texttt{Dictionary}}) -- Conjunto de técnicas con sus respectivos parámetros para que se puedan cargar automáticamente en este Frame \textbf{(véase Controller/XMLParser.py)}.
\end{itemize}

\item[{Returns}] \leavevmode
Tkinter.Frame
\item[{Return type}] \leavevmode
Instance
\end{description}\end{quote}

%******* Termina descripción *******

%******* Empieza función *******
\begin{fulllineitems}

\pysiglinewithargsret{\sphinxbfcode{activate\_scroll}}{\emph{event}}{}~

\begin{notice}{note}{Note:}
Este método es privado.
\end{notice}

Actualiza la barra de desplazamiento y con base en esta acción
la activa o desactiva.

\begin{quote}\begin{description}
\item[{Parameters}] \leavevmode\begin{itemize}
\item \textbf{\texttt{event}} (\emph{\texttt{String}}) -- Elemento que ejecutó esta función.
\end{itemize}
\end{description}\end{quote}

\end{fulllineitems}
%******* Termina función *******

%******* Empieza función *******
\begin{fulllineitems}

\pysiglinewithargsret{\sphinxbfcode{update\_scrollbar}}{\emph{event=None}}{}~

\begin{notice}{note}{Note:}
Este método es privado.
\end{notice}

Actualiza la barra de desplazamiento de acuerdo al número 
de elementos existentes en el Frame, esto para poder hacer 
un recorrido apropiado de la barra.

\begin{quote}\begin{description}
\item[{Parameters}] \leavevmode\begin{itemize}
\item \textbf{\texttt{event}} (\emph{\texttt{String}}) -- Elemento que ejecutó esta función.
\end{itemize}
\end{description}\end{quote}

\end{fulllineitems}
%******* Termina función *******

%******* Empieza función *******
\begin{fulllineitems}

\pysiglinewithargsret{\sphinxbfcode{get\_information}}{}{}
Regresa la información recabada en el Frame.

\begin{quote}\begin{description}
\item[{Returns}] \leavevmode
Un diccionario que contiene una lista con las funciones escritas.
\item[{Return type}] \leavevmode
Dictionary
\end{description}\end{quote}

\end{fulllineitems}
%******* Termina función *******

%******* Empieza función *******
\begin{fulllineitems}

\pysiglinewithargsret{\sphinxbfcode{insert\_mop\_example}}{\emph{functions}}{}~
\vspace{-0.3cm}

Inserta un M.O.P (Multi Objective Problem).\break
En este caso significa que se insertarán funciones
para poder hacer pruebas rápidas en el programa.\break
\textbf{(véase Controller/XML/MOPExample.xml)}\break
\textbf{(véase View/Additional/MenuInternalOption/InternalOptionFrame.py)}.

\begin{quote}\begin{description}
\item[{Parameters}] \leavevmode\begin{itemize}
\item \textbf{\texttt{functions}} (\emph{\texttt{List}}) -- Lista de funciones para ser insertadas en el Frame.
\end{itemize}
\end{description}\end{quote}

\end{fulllineitems}
%******* Termina función *******

%******* Empieza función *******
\begin{fulllineitems}

\pysiglinewithargsret{\sphinxbfcode{restore\_settings}}{}{}
Restaura el contenido del Frame, en este caso significa 
que se eliminará todo lo que esté en éste y se dejará 
una casilla vacía libre.

\end{fulllineitems}
%******* Termina función *******

\end{fulllineitems}
%******* Termina clase *******

La clase actual toma como fundamento lo siguiente:

%******* Empieza clase *******
\subparagraph{FunctionFrame (clase)}
%Se coloca el vínculo interno procedente de esta misma sección (a_3_2_3_1_1).
\label{sec:a_3_2_3_1_1}

%******* Empieza descripción *******
\begin{fulllineitems}

\begin{DUlineblock}{0em}
\item[] Esta clase proporciona una base gráfica para que 
el usuario pueda agregar tantas functiones objetivo como 
desee.\break
A grandes rasgos el usuario podrá agregar casillas donde 
se colocarán las funciones objetivo, esto utilizando un botón.\break
De igual manera, las casillas pueden ser eliminadas
usando un ícono que estará cerca de cada una de éstas.\break
Importante es mencionar que las funciones deben estar 
escritas en sintaxis de Python.
\end{DUlineblock}

\begin{quote}\begin{description}
\item[{Parameters}] \leavevmode\begin{itemize}
\item \textbf{\texttt{parent}} (\emph{\texttt{Tkinter.Frame}}) -- Frame padre al que pertenece.
\item \textbf{\texttt{features}} (\emph{\texttt{Dictionary}}) -- Conjunto de técnicas con sus respectivos parámetros para que se puedan cargar automáticamente en este Frame \textbf{(véase Controller/XMLParser.py)}.
\end{itemize}

\item[{Returns}] \leavevmode
Tkinter.Frame
\item[{Return type}] \leavevmode
Instance
\end{description}\end{quote}

%******* Termina descripción *******

%******* Empieza función *******
\begin{fulllineitems}

\pysiglinewithargsret{\sphinxbfcode{add\_function}}{\emph{event}}{}~

\begin{notice}{note}{Note:}
Este método es privado.
\end{notice}

Agrega una casilla al Frame.\break
Esta función se usa si fue ejecutada 
por un evento.

\begin{quote}\begin{description}
\item[{Parameters}] \leavevmode\begin{itemize}
\item \textbf{\texttt{event}} (\emph{\texttt{String}}) -- Identificador del elemento gráfico que activó la función.
\end{itemize}
\end{description}\end{quote}

\end{fulllineitems}
%******* Termina función *******

%******* Empieza función *******
\begin{fulllineitems}

\pysiglinewithargsret{\sphinxbfcode{delete\_single\_function}}{\emph{event}}{}~

\begin{notice}{note}{Note:}
Este método es privado.
\end{notice}

Elimina una casilla y todos los elementos 
gráficos que la acompañan.\break
También elimina todo rastro que se encuentre 
en las estructuras lógicas.

\begin{quote}\begin{description}
\item[{Parameters}] \leavevmode\begin{itemize}
\item \textbf{\texttt{event}} (\emph{\texttt{String}}) -- Identificador del elemento gráfico que activó la función.
\end{itemize}
\end{description}\end{quote}

\end{fulllineitems}
%******* Termina función *******

%******* Empieza función *******
\begin{fulllineitems}

\pysiglinewithargsret{\sphinxbfcode{grid\_widgets}}{}{}~

\begin{notice}{note}{Note:}
Este método es privado.
\end{notice}

Coloca elementos en el Frame.

\end{fulllineitems}
%******* Termina función *******

%******* Empieza función *******
\begin{fulllineitems}

\pysiglinewithargsret{\sphinxbfcode{get\_current\_elements}}{}{}
Regresa el número actual de casillas 
en el Frame.

\begin{quote}\begin{description}
\item[{Returns}] \leavevmode
Cantidad de elementos en la estructura rows, donde se guardan las casillas (Entries).

\item[{Return type}] \leavevmode
Integer
\end{description}\end{quote}

\end{fulllineitems}
%******* Termina función *******

%******* Empieza función *******
\begin{fulllineitems}

\pysiglinewithargsret{\sphinxbfcode{get\_information}}{}{}
Toma la información del Frame y regresa las funciones 
objectivo que el usuario insertó.

\begin{quote}\begin{description}
\item[{Returns}] \leavevmode
Un diccionario que contiene una lista con las funciones escritas.
\item[{Return type}] \leavevmode
Dictionary
\end{description}\end{quote}

\end{fulllineitems}
%******* Termina función *******

%******* Empieza función *******
\begin{fulllineitems}

\pysiglinewithargsret{\sphinxbfcode{insert\_function}}{\emph{function=None}}{}~
\vspace{-0.3cm}

Coloca en el Frame una colección de elementos:\break
{[}casilla para insertar funcion, opción de maximizar, opción de minimizar, botón para eliminar{]}\break
Si el parámetro function es \textbf{None}, se agrega la casilla 
vacía, de lo contrario se añade ésta con la función.

\begin{quote}\begin{description}
\item[{Parameters}] \leavevmode\begin{itemize}
\item \textbf{\texttt{function}} (\emph{\texttt{String}}) -- Una función para ser insertada en el primer elemento de la colección.
\end{itemize}
\end{description}\end{quote}

\end{fulllineitems}
%******* Termina función *******

%******* Empieza función *******
\begin{fulllineitems}

\pysiglinewithargsret{\sphinxbfcode{insert\_mop\_example}}{\emph{functions}}{}~
\vspace{-0.3cm}

Inserta un M.O.P (Multi Objective Problem) que no es más 
que un conjunto de funciones para que se pueda hacer más 
rápidamente una prueba.\break
Previo a ésto se limpia el Frame para insertar únicamente 
el M.O.P.\break
\textbf{(véase Controller/XML/MOPExample.xml)}\break
\textbf{(véase View/Additional/MenuInternalOption/InternalOptionFrame.py)}.

\begin{quote}\begin{description}
\item[{Parameters}] \leavevmode\begin{itemize}
\item \textbf{\texttt{functions}} (\emph{\texttt{List}}) -- Conjunto de funciones para insertar en el Frame.
\end{itemize}
\end{description}\end{quote}

\end{fulllineitems}
%******* Termina función *******

%******* Empieza función *******
\begin{fulllineitems}

\pysiglinewithargsret{\sphinxbfcode{restore\_settings}}{}{}
Restaura el contenido del Frame a sus valores por 
defecto.\break
Esto significa que borrará cualquier contenido que se 
encuentre en existencia y dejará una casilla vacía.

\end{fulllineitems}
%******* Termina función *******

\end{fulllineitems}
%******* Termina clase *******
%******* Termina módulo *******

%******* Empieza módulo *******
\subsubsection{Population (módulo)}
%Se coloca el vínculo interno procedente de esta misma sección (a_3_2_4).
\label{sec:a_3_2_4}
Proporciona las estructuras gráficas para que el usuario 
pueda configurar atributos de la Población.\break
Los elementos que conforman al módulo son los siguientes:

%******* Empieza módulo *******
\paragraph{PopulationFrame (clase)}
%Se coloca el vínculo interno procedente de esta misma sección (a_3_2_4_1).
\label{sec:a_3_2_4_1}

%******* Empieza descripción *******
\begin{fulllineitems}

\begin{DUlineblock}{0em}
\item[] Unifica y mantiene un control sobre las clases 
PopulaceFrame y FitnessFrame, esto con el fin de poder 
colocar los elementos apropiadamente y agilizar el 
intercambio de información con el usuario.
\end{DUlineblock}

\begin{quote}\begin{description}
\item[{Parameters}] \leavevmode\begin{itemize}
\item \textbf{\texttt{parent}} (\emph{\texttt{Tkinter.Frame}}) -- Frame padre al que pertenece.
\item \textbf{\texttt{features}} (\emph{\texttt{Dictionary}}) -- Conjunto de técnicas con sus respectivos parámetros para que se puedan cargar automáticamente en este Frame \textbf{(véase Controller/XMLParser.py)}.
\end{itemize}

\item[{Returns}] \leavevmode
Tkinter.Frame
\item[{Return type}] \leavevmode
Instance
\end{description}\end{quote}

%******* Termina descripción *******

%******* Empieza función *******
\begin{fulllineitems}

\pysiglinewithargsret{\sphinxbfcode{get\_information}}{}{}
Toma la información propiciada en cada Frame y después
la unifica para regresar un sólo conjunto de información.

\begin{quote}\begin{description}
\item[{Returns}] \leavevmode
Un diccionario con la información de PopulaceFrame y FitnessFrame.
\item[{Return type}] \leavevmode
Dictionary
\end{description}\end{quote}

\end{fulllineitems}
%******* Termina función *******

%******* Empieza función *******
\begin{fulllineitems}

\pysiglinewithargsret{\sphinxbfcode{restore\_settings}}{}{}
Restaura los valores por defecto en ambos Frames.

\end{fulllineitems}
%******* Termina función *******

\end{fulllineitems}
%******* Termina clase *******

%******* Empieza clase *******
\paragraph{TemplatePopulationFrame (clase)}
%Se coloca el vínculo interno procedente de esta misma sección (a_3_2_4_2).
\label{sec:a_3_2_4_2}

%******* Empieza descripción *******
\begin{fulllineitems}

\begin{DUlineblock}{0em}
\item[] Esta clase proporciona la infraestructura gráfica para que el 
usuario pueda  elegir técnicas y configurar atributos 
concernientes al Fitness de una Población y a la Población 
en general.\break
A grandes rasgos se trata de una plantilla que deberán implementar 
las clases FitnessFrame y PopulaceFrame.\break
La clase permite la selección de cada posible técnica disponible y 
automáticamente se muestran los parámetros necesarios \textbf{(si los hay)} 
para cada una de éstas.
\end{DUlineblock}

\begin{quote}\begin{description}
\item[{Parameters}] \leavevmode\begin{itemize}
\item \textbf{\texttt{parent}} (\emph{\texttt{Tkinter.Frame}}) -- Frame padre al que pertenece.
\item \textbf{\texttt{name}} (\emph{\texttt{String}}) -- Identificador \textbf{(único)} que tendrá el Frame.
\item \textbf{\texttt{features}} (\emph{\texttt{Dictionary}}) -- Conjunto de técnicas con sus respectivos parámetros para que se puedan cargar automáticamente en este frame \textbf{(véase Controller/XMLParser.py)}.
\end{itemize}

\item[{Returns}] \leavevmode
Tkinter.Frame
\item[{Return type}] \leavevmode
Instance
\end{description}\end{quote}

%******* Termina descripción *******

%******* Empieza función *******
\begin{fulllineitems}

\pysiglinewithargsret{\sphinxbfcode{create\_dynamic\_widgets}}{}{}~

\begin{notice}{note}{Note:}
Este método es privado.
\end{notice}

Inicializa los elementos dinámicos del Frame, esto es, de 
acuerdo al tipo que lleva cada parámetro se creará un widget 
diferente.

\end{fulllineitems}
%******* Termina función *******

%******* Empieza función *******
\begin{fulllineitems}

\pysiglinewithargsret{\sphinxbfcode{update\_widgets}}{\emph{event=None}}{}~

\begin{notice}{note}{Note:}
Este método es privado.
\end{notice}

Realiza solamente la actualización y colocación de elementos 
dinámicos en el Frame.\break
Si el parámetro event es distinto de \textbf{None}, significa 
que se lanzó un evento que provocará que se actualicen los 
parámetros de acuerdo con la técnica seleccionada.

\begin{quote}\begin{description}
\item[{Parameters}] \leavevmode\begin{itemize}
\item \textbf{\texttt{event}} (\emph{\texttt{String}}) -- Contiene el valor del elemento que ejecutó esta función.
\end{itemize}
\end{description}\end{quote}

\end{fulllineitems}
%******* Termina función *******

%******* Empieza función *******
\begin{fulllineitems}

\pysiglinewithargsret{\sphinxbfcode{get\_information}}{}{}
Recolecta la información que ha seleccionado e introducido 
el usuario, también la organiza para que se pueda utilizar 
apropiadamente.

\begin{quote}\begin{description}
\item[{Returns}] \leavevmode

Un diccionario que contiene:

\begin{itemize}
\item \textbf{Clase},
\item \textbf{Técnica},
\item \textbf{Parametros}
\end{itemize}

\item[{Return type}] \leavevmode
Dictionary
\end{description}\end{quote}

\end{fulllineitems}
%******* Termina función *******

%******* Empieza función *******
\begin{fulllineitems}

\pysiglinewithargsret{\sphinxbfcode{grid\_widgets}}{}{}
Permite la colocación adecuada de elementos estáticos y 
dinámicos, considerando además el espacio o características 
necesarias de redimensionamiento para éstos últimos.

\end{fulllineitems}
%******* Termina función *******

%******* Empieza función *******
\begin{fulllineitems}

\pysiglinewithargsret{\sphinxbfcode{restore\_settings}}{}{}
Asigna los valores por defecto tanto de las técnicas como 
de sus respectivos parámetros, también limpia aquéllos en 
donde se hayan insertado valores.

\end{fulllineitems}
%******* Termina función *******

\end{fulllineitems}
%******* Termina clase *******

Los siguientes elementos implementan la plantilla actual:

%******* Empieza clase *******
\subparagraph{PopulaceFrame (clase)}
%Se coloca el vínculo interno procedente de esta misma sección (a_3_2_4_2_1).
\label{sec:a_3_2_4_2_1}
%******* Empieza descripción *******
\begin{fulllineitems}

\begin{DUlineblock}{0em}
\item[] Esta clase proporciona la infraestructura gráfica para que 
el usuario pueda elegir métodos y características concernientes a 
la conformación de la Población.\break
También hereda atributos de la clase TemplatePopulationFrame con el 
fin de establecer una forma más rápida y ordenada de colocar componentes 
y recolectar la información de éstos.
\end{DUlineblock}

\begin{quote}\begin{description}
\item[{Parameters}] \leavevmode\begin{itemize}
\item \textbf{\texttt{parent}} (\emph{\texttt{Tkinter.Frame}}) -- Frame padre al que pertenece.
\item \textbf{\texttt{name}} (\emph{\texttt{String}}) -- Identificador \textbf{(único)} que tendrá el Frame.
\item \textbf{\texttt{features}} (\emph{\texttt{Dictionary}}) -- Conjunto de técnicas con sus respectivos parámetros para que se puedan cargar automáticamente en este Frame \textbf{(véase Controller/XMLParser.py)}.
\end{itemize}

\item[{Returns}] \leavevmode
Tkinter.Frame
\item[{Return type}] \leavevmode
Instance
\end{description}\end{quote}

%******* Termina descripción *******

%******* Empieza función *******
\begin{fulllineitems}

\pysiglinewithargsret{\sphinxbfcode{get\_information}}{}{}
Recolecta la información genérica \textbf{(usando el método de la clase Padre)}, 
y también se le añade aquélla recolectada exclusivamente 
en esta clase.

\begin{quote}\begin{description}
\item[{Returns}] \leavevmode

Un diccionario que contiene:

\begin{itemize}
\item \textbf{Métodos genéricos,}
\item \textbf{Número de Generaciones,}
\item \textbf{Tamaño de la Población,}
\item \textbf{Número de Decimales.}
\end{itemize}

\item[{Return type}] \leavevmode
Dictionary
\end{description}\end{quote}

\end{fulllineitems}
%******* Termina función *******

%******* Empieza función *******
\begin{fulllineitems}

\pysiglinewithargsret{\sphinxbfcode{restore\_settings}}{}{}~
\vspace{-0.3cm}

Por un lado, restaura el contenido de los elementos pertenecientes 
sólo a esta clase, y por el otro, activa el método de la clase 
Padre que realiza una restauración de los elementos genéricos.

\end{fulllineitems}
%******* Termina función *******

\end{fulllineitems}
%******* Termina función *******

%******* Empieza clase *******
\subparagraph{FitnessFrame (clase)}
%Se coloca el vínculo interno procedente de esta misma sección (a_3_2_4_2_2).
\label{sec:a_3_2_4_2_2}
%******* Empieza descripción *******
\begin{fulllineitems}

\begin{DUlineblock}{0em}
\item[] Esta clase proporciona la infraestructura gráfica para que 
el usuario pueda elegir métodos concernientes a la asignación del 
Fitness para la Población.\break
Además hereda atributos de la clase TemplatePopulationFrame para 
facilitar la colocacion y extracción de información pertinente 
para el usuario.
\end{DUlineblock}

\begin{quote}\begin{description}
\item[{Parameters}] \leavevmode\begin{itemize}
\item \textbf{\texttt{parent}} (\emph{\texttt{Tkinter.Frame}}) -- Frame padre al que pertenece.
\item \textbf{\texttt{name}} (\emph{\texttt{String}}) -- Identificador \textbf{(único)} que tendrá el Frame.
\item \textbf{\texttt{features}} (\emph{\texttt{Dictionary}}) -- Conjunto de técnicas con sus respectivos parámetros para que se puedan cargar automáticamente en este Frame \textbf{(véase Controller/XMLParser.py)}.
\end{itemize}

\item[{Returns}] \leavevmode
Tkinter.Frame
\item[{Return type}] \leavevmode
Instance

\end{description}\end{quote}

%******* Termina descripción *******

%******* Empieza función *******
\begin{fulllineitems}

\pysiglinewithargsret{\sphinxbfcode{get\_information}}{}{}
Llama al método de la clase Padre, el cual recopila toda la 
información elegida por el usuario y la regresa en forma de 
diccionario.

\begin{quote}\begin{description}
\item[{Returns}] \leavevmode
Diccionario con información de los métodos genéricos.
\item[{Return type}] \leavevmode
Dictionary
\end{description}\end{quote}

\end{fulllineitems}
%******* Termina función *******

%******* Empieza función *******
\begin{fulllineitems}

\pysiglinewithargsret{\sphinxbfcode{restore\_settings}}{}{}
Llamar al método de la clase Padre, el cual restaura los 
valores por defecto de los elementos dinámicos y estáticos 
del Frame.

\end{fulllineitems}
%******* Termina función *******

\end{fulllineitems}
%******* Termina clase *******
%******* Termina módulo *******

%******* Empieza módulo *******
\subsubsection{GeneticOperator (módulo)}
%Se coloca el vínculo interno procedente de esta misma sección (a_3_2_5).
\label{sec:a_3_2_5}
Proporciona los elementos gráficos para que el usuario pueda
realizar operaciones relacionadas con la Selección, Cruza 
y Mutación de Individuos de una Población.\medskip\break
Los elementos que lo conforman son:

%******* Empieza clase *******
\paragraph{GeneticOperatorFrame (clase)}
%Se coloca el vínculo interno procedente de esta misma sección (a_3_2_5_1).
\label{sec:a_3_2_5_1}
%******* Empieza descripción *******
\begin{fulllineitems}

\begin{DUlineblock}{0em}
\item[] Reúne y controla las clases SelectionFrame, CrossoverFrame y  
MutationFrame con la finalidad de colocar los elementos gráficos apropiadamente y 
agilizar el intercambio de información con el usuario.
\end{DUlineblock}

\begin{quote}\begin{description}
\item[{Parameters}] \leavevmode\begin{itemize}
\item \textbf{\texttt{parent}} (\emph{\texttt{Tkinter.Frame}}) -- Frame padre al que pertenece.
\item \textbf{\texttt{features}} (\emph{\texttt{Dictionary}}) -- Conjunto de técnicas con sus respectivos parámetros para que se puedan cargar automáticamente en este Frame \textbf{(véase Controller/XMLParser.py)}.
\end{itemize}

\item[{Returns}] \leavevmode
Tkinter.Frame
\item[{Return type}] \leavevmode
Instance
\end{description}\end{quote}

%******* Termina descripción *******

%******* Empieza función *******
\begin{fulllineitems}

\pysiglinewithargsret{\sphinxbfcode{get\_information}}{}{}
Toma la información propiciada en cada Frame y 
después la unifica para regresar un sólo conjunto 
de información.

\begin{quote}\begin{description}
\item[{Returns}] \leavevmode
Un diccionario con la información de SelectionFrame, CrossoverFrame y MutationFrame.
\item[{Return type}] \leavevmode
Dictionary
\end{description}\end{quote}

\end{fulllineitems}
%******* Termina función *******

%******* Empieza función *******
\begin{fulllineitems}

\pysiglinewithargsret{\sphinxbfcode{restore\_settings}}{}{}
Realiza la restauración de información y contenido 
en cada uno de los Frames.

\end{fulllineitems}
%******* Termina función *******

\end{fulllineitems}
%******* Termina clase *******

%******* Empieza clase *******
\paragraph{TemplateGeneticOperatorFrame (clase)}
%Se coloca el vínculo interno procedente de esta misma sección (a_3_2_5_2).
\label{sec:a_3_2_5_2}
%******* Empieza descripción *******
\begin{fulllineitems}

\begin{DUlineblock}{0em}
\item[] Proporciona la infraestructura gráfica para que el usuario 
pueda elegir técnicas y configurar atributos concernientes a la 
Selección, Cruza y Mutación de Individuos de una Población.\break
A grandes rasgos se trata de una plantilla que deberán implementar 
las clases SelectionFrame, CrossoverFrame y MutationFrame.\break
La clase permite la selección de cada posible técnica disponible 
y automáticamente se muestran los parámetros necesarios \textbf{(si los hay)} 
para cada una de éstas.
\end{DUlineblock}

\begin{quote}\begin{description}
\item[{Parameters}] \leavevmode\begin{itemize}
\item \textbf{\texttt{parent}} (\emph{\texttt{Tkinter.Frame}}) -- Frame padre al que pertenece.
\item \textbf{\texttt{name}} (\emph{\texttt{String}}) -- Identificador \textbf{(único)} que tendrá el Frame.
\item \textbf{\texttt{features}} (\emph{\texttt{Dictionary}}) -- Conjunto de técnicas con sus respectivos parámetros para que se puedan cargar automáticamente en este Frame \textbf{(véase Controller/XMLParser.py)}.
\item \textbf{\texttt{sort\_techniques}} (\emph{\texttt{Boolean}}) -- Indica si las técnicas disponibles se ordenan alfabéticamente
o no.
\end{itemize}

\item[{Returns}] \leavevmode
Tkinter.Frame
\item[{Return type}] \leavevmode
Instance
\end{description}\end{quote}

%******* Termina descripción *******

%******* Empieza función *******
\begin{fulllineitems}

\pysiglinewithargsret{\sphinxbfcode{dynamic\_widgets}}{}{}~

\begin{notice}{note}{Note:}
Este método es privado.
\end{notice}

Inicializa los elementos dinámicos del Frame, esto es, de acuerdo 
al tipo que lleva cada parámetro se creará un widget diferente.

\end{fulllineitems}
%******* Termina función *******

%******* Empieza función *******
\begin{fulllineitems}

\pysiglinewithargsret{\sphinxbfcode{update\_widgets}}{\emph{event=None}}{}~

\begin{notice}{note}{Note:}
Este método es privado.
\end{notice}

Realiza solamente la actualización y colocación de elementos dinámicos 
en el Frame.\break
Si el parámetro event es distinto de \textbf{None}, significa que se 
lanzó un evento que provocará que se actualicen los parámetros de 
acuerdo con la técnica seleccionada.

\begin{quote}\begin{description}
\item[{Parameters}] \leavevmode\begin{itemize}
\item \textbf{\texttt{event}} (\emph{\texttt{String}}) -- Contiene el valor del elemento que ejecutó esta función.
\end{itemize}
\end{description}\end{quote}

\end{fulllineitems}
%******* Termina función *******

%******* Empieza función *******
\begin{fulllineitems}

\pysiglinewithargsret{\sphinxbfcode{get\_information}}{}{}
Recolecta la información que ha seleccionado e introducido 
el usuario, también la organiza para que se pueda utilizar 
apropiadamente.

\begin{quote}\begin{description}
\item[{Returns}] \leavevmode

Un diccionario que contiene:

\begin{itemize}
\item \textbf{Clase},
\item \textbf{Técnica},
\item \textbf{Parámetros}
\end{itemize}

\item[{Return type}] \leavevmode
Dictionary
\end{description}\end{quote}

\end{fulllineitems}
%******* Termina función *******

%******* Empieza función *******
\begin{fulllineitems}

\pysiglinewithargsret{\sphinxbfcode{grid\_widgets}}{}{}
Permite la colocación adecuada de elementos estáticos y 
dinámicos, considerando además el espacio o características 
necesarias de redimensionamiento para éstos últimos.

\end{fulllineitems}
%******* Termina función *******

%******* Empieza función *******
\begin{fulllineitems}

\pysiglinewithargsret{\sphinxbfcode{restore\_settings}}{}{}
Asigna los valores por defecto tanto de las técnicas como 
de sus respectivos parámetros, también limpia aquéllos en 
donde se hayan insertado valores.

\end{fulllineitems}
%******* Termina función *******

\end{fulllineitems}
%******* Termina clase *******

Las clases que implementan esta plantilla son las siguientes:

%******* Empieza clase *******
\subparagraph{SelectionFrame (clase)}
%Se coloca el vínculo interno procedente de esta misma sección (a_3_2_5_2_1).
\label{sec:a_3_2_5_2_1}
%******* Empieza descripción *******
\begin{fulllineitems}

\begin{DUlineblock}{0em}
\item[] Esta clase proporciona la infraestructura gráfica para que 
el usuario pueda elegir métodos y características relacionadas con 
la selección de Individuos.\break
También hereda atributos de la clase TemplateGeneticOperatorFrame 
para facilitar la carga de elementos en el Frame y su correspondiente 
recolección de información.
\end{DUlineblock}

\begin{quote}\begin{description}
\item[{Parameters}] \leavevmode\begin{itemize}
\item \textbf{\texttt{parent}} (\emph{\texttt{Tkinter.Frame}}) -- Frame padre al que pertenece.
\item \textbf{\texttt{name}} (\emph{\texttt{String}}) -- Identificador \textbf{(único)} que tendrá el Frame.
\item \textbf{\texttt{features}} (\emph{\texttt{Dictionary}}) -- Conjunto de técnicas con sus respectivos parámetros para que se puedan cargar automáticamente en este Frame \textbf{(véase Controller/XMLParser.py)}.
\end{itemize}

\item[{Returns}] \leavevmode
Tkinter.Frame
\item[{Return type}] \leavevmode
Instance
\end{description}\end{quote}

%******* Termina descripción *******

%******* Empieza función *******
\begin{fulllineitems}

\pysiglinewithargsret{\sphinxbfcode{get\_information}}{}{}
Recolecta la información relativa a esta clase haciendo 
uso del método de la clase Padre.

\begin{quote}\begin{description}
\item[{Returns}] \leavevmode
Diccionario con información de los métodos genéricos.
\item[{Return type}] \leavevmode
Dictionary
\end{description}\end{quote}

\end{fulllineitems}
%******* Termina función *******

%******* Empieza función *******
\begin{fulllineitems}

\pysiglinewithargsret{\sphinxbfcode{restore\_settings}}{}{}
Ejecuta el método de la clase Padre, el cual restaura los 
valores por defecto de los elementos dinámicos y estáticos 
del Frame.

\end{fulllineitems}
%******* Terminas función *******

\end{fulllineitems}
%******* Termina clase *******

%******* Empieza clase *******
\subparagraph{CrossoverFrame (clase)}
%Se coloca el vínculo interno procedente de esta misma sección (a_3_2_5_2_2).
\label{sec:a_3_2_5_2_2}
%******* Empieza descripción *******
\begin{fulllineitems}

\begin{DUlineblock}{0em}
\item[] Esta clase proporciona la infraestructura gráfica para que 
el usuario pueda elegir técnicas y características concernientes a 
la Cruza entre Individuos.\break
También hereda atributos de la clase TemplateGeneticOperatorFrame 
para facilitar la carga de elementos en el Frame y su correspondiente 
recolección de información.
\end{DUlineblock}

\begin{quote}\begin{description}
\item[{Parameters}] \leavevmode\begin{itemize}
\item \textbf{\texttt{parent}} (\emph{\texttt{Tkinter.Frame}}) -- Frame padre al que pertenece.
\item \textbf{\texttt{name}} (\emph{\texttt{String}}) -- Identificador \textbf{(único)} que tendrá el Frame.
\item \textbf{\texttt{features}} (\emph{\texttt{Dictionary}}) -- Conjunto de técnicas con sus respectivos parámetros para que se puedan cargar automáticamente en este Frame \textbf{(véase Controller/XMLParser.py)}.
\end{itemize}

\item[{Returns}] \leavevmode
Tkinter.Frame
\item[{Return type}] \leavevmode
Instance
\end{description}\end{quote}

%******* Termina descripción *******

%******* Empieza función *******
\begin{fulllineitems}

\pysiglinewithargsret{\sphinxbfcode{get\_information}}{}{}
Recolecta la información genérica \textbf{(usando el método de la clase Padre)}, 
y también se le añade aquélla recolectada exclusivamente 
en esta clase.

\begin{quote}\begin{description}
\item[{Returns}] \leavevmode

Un diccionario que contiene:

\begin{itemize}
\item \textbf{Métodos genéricos,}
\item \textbf{Probabilidad de cruza.}
\end{itemize}

\item[{Return type}] \leavevmode
Dictionary
\end{description}\end{quote}

\end{fulllineitems}
%******* Termina función *******

%******* Empieza función *******
\begin{fulllineitems}

\pysiglinewithargsret{\sphinxbfcode{restore\_settings}}{}{}
Ejecuta el método de la clase Padre, el cual restaura los 
valores por defecto de los elementos dinámicos y estáticos 
del Frame.

\end{fulllineitems}
%******* Termina función *******

\end{fulllineitems}
%******* Termina clase *******

%******* Empieza clase *******
\subparagraph{MutationFrame (clase)}
%Se coloca el vínculo interno procedente de esta misma sección (a_3_2_5_2_3).
\label{sec:a_3_2_5_2_3}
%******* Empieza descripción *******
\begin{fulllineitems}

\begin{DUlineblock}{0em}
\item[] Esta clase proporciona la infraestructura gráfica para que el 
usuario pueda elegir técnicas y características relativas a la Mutación 
de Individuos.\break
También hereda atributos de la clase TemplateGeneticOperatorFrame 
para facilitar la carga automática de elementos en el Frame y su 
consecuente recolección de información.
\end{DUlineblock}

\begin{quote}\begin{description}
\item[{Parameters}] \leavevmode\begin{itemize}
\item \textbf{\texttt{parent}} (\emph{\texttt{Tkinter.Frame}}) -- Frame padre al que pertenece.
\item \textbf{\texttt{name}} (\emph{\texttt{String}}) -- Identificador \textbf{(único)} que tendrá el Frame.
\item \textbf{\texttt{features}} (\emph{\texttt{Dictionary}}) -- Conjunto de técnicas con sus respectivos parámetros para que se puedan cargar automáticamente en este Frame \textbf{(véase Controller/XMLParser.py)}.
\end{itemize}

\item[{Returns}] \leavevmode
Tkinter.Frame
\item[{Return type}] \leavevmode
Instance
\end{description}\end{quote}

%******* Termina descripción *******

%******* Empieza función *******
\begin{fulllineitems}

\pysiglinewithargsret{\sphinxbfcode{get\_information}}{}{}
Recolecta la información genérica \textbf{(usando el método de la clase Padre)}, 
y también se le añade aquélla recolectada exclusivamente 
en esta clase.

\begin{quote}\begin{description}
\item[{Returns}] \leavevmode

Un diccionario que contiene:

\begin{itemize}
\item \textbf{Métodos genéricos,}
\item \textbf{Probabilidad de mutación.}
\end{itemize}

\item[{Return type}] \leavevmode
Dictionary
\end{description}\end{quote}

\end{fulllineitems}
%******* Termina función *******

%******* Empieza función *******
\begin{fulllineitems}

\pysiglinewithargsret{\sphinxbfcode{restore\_settings}}{}{}
Ejecuta el método de la clase Padre, el cual restaura 
los valores por defecto de los elementos dinámicos y 
estáticos del Frame.

\end{fulllineitems}
%******* Termina función *******

\end{fulllineitems}
%******* Termina clase *******
%******* Termina módulo *******

%******* Empieza módulo *******
\subsubsection{MOEA (módulo)}
%Se coloca el vínculo interno procedente de esta misma sección (a_3_2_6).
\label{sec:a_3_2_6}
Proporciona los elementos gráficos para que el usuario 
realice configuraciones concernientes a los M.O.E.A.s \textbf{(Multi-Objective Evolutionary Algorithms
ó Algoritmos Evolutivos Multiobjetivo)} y sus atributos 
relacionados.\break
Sus elementos son los siguientes:

%******* Empieza clase *******
\paragraph{MOEAFrame (clase)}
%Se coloca el vínculo interno procedente de esta misma sección (a_3_2_6_1).
\label{sec:a_3_2_6_1}
%******* Empieza descripción *******
\begin{fulllineitems}

\begin{DUlineblock}{0em}
\item[] Unifica los Frames AlgorithmFrame y SharingFunctionFrame, 
la razón de ésto es para facilitar el acomodo de componentes 
de manera individual, para así garantizar un acceso asequible 
a la información.
\end{DUlineblock}

\begin{quote}\begin{description}
\item[{Parameters}] \leavevmode\begin{itemize}
\item \textbf{\texttt{parent}} (\emph{\texttt{Tkinter.Frame}}) -- Frame padre al que pertenece.
\item \textbf{\texttt{features}} (\emph{\texttt{Dictionary}}) -- Conjunto de técnicas con sus respectivos parámetros para que se puedan cargar automáticamente en este Frame \textbf{(véase Controller/XMLParser.py)}.
\end{itemize}

\item[{Returns}] \leavevmode
Tkinter.Frame
\item[{Return type}] \leavevmode
Instance
\end{description}\end{quote}

%******* Termina descripción *******

%******* Empieza función *******
\begin{fulllineitems}

\pysiglinewithargsret{\sphinxbfcode{get\_information}}{}{}
Toma la información solicitada en cada Frame y después
la unifica para regresar un sólo conjunto de información.

\begin{quote}\begin{description}
\item[{Returns}] \leavevmode
Un diccionario con la información de AlgorithmFrame y SharingFunctionFrame.
\item[{Return type}] \leavevmode
Dictionary
\end{description}\end{quote}
\end{fulllineitems}
%******* Termina función *******

%******* Empieza función *******
\begin{fulllineitems}

\pysiglinewithargsret{\sphinxbfcode{restore\_settings}}{}{}
Restaura los valores por defecto en cada Frame.

\end{fulllineitems}
%******* Termina función *******

\end{fulllineitems}
%******* Termina clase *******

La clase actual se apoya de los siguientes elementos:

%******* Empieza clase *******
\subparagraph{AlgorithmFrame (clase)}
%Se coloca el vínculo interno procedente de esta misma sección (a_3_2_6_1_1).
\label{sec:a_3_2_6_1_1}
%******* Empieza descripción *******
\begin{fulllineitems}

\begin{DUlineblock}{0em}
\item[] Esta clase proporciona una base gráfica para que el 
usuario pueda seleccionar técnicas con sus parámetros 
correspondientes \textbf{(si es que tienen)} referentes a los 
M.O.E.A.'s \textbf{(Multi-Objective Evolutionary Algorithms ó Algoritmos Evolutivos Multiobjetivo)}.
\end{DUlineblock}

\begin{quote}\begin{description}
\item[{Parameters}] \leavevmode\begin{itemize}
\item \textbf{\texttt{parent}} (\emph{\texttt{Tkinter.Frame}}) -- Frame padre al que pertenece.
\item \textbf{\texttt{name}} (\emph{\texttt{String}}) -- Identificador \textbf{(único)} que tendrá el Frame.
\item \textbf{\texttt{features}} (\emph{\texttt{Dictionary}}) -- Conjunto de técnicas con sus respectivos parámetros para que se puedan cargar automáticamente en este Frame \textbf{(véase Controller/XMLParser.py)}.
\end{itemize}

\item[{Returns}] \leavevmode
Tkinter.Frame
\item[{Return type}] \leavevmode
Instance
\end{description}\end{quote}

%******* Termina descripción *******

%******* Empieza función *******
\begin{fulllineitems}

\pysiglinewithargsret{\sphinxbfcode{create\_dynamic\_widgets}}{}{}~

\begin{notice}{note}{Note:}
Este método es privado.
\end{notice}

Inicializa los elementos dinámicos del Frame, esto es, de acuerdo 
al tipo que lleva cada parámetro se creará un widget diferente.

\end{fulllineitems}
%******* Termina función *******

%******* Empieza función *******
\begin{fulllineitems}

\pysiglinewithargsret{\sphinxbfcode{grid\_widgets}}{}{}~

\begin{notice}{note}{Note:}
Este método es privado.
\end{notice}

Coloca elementos en el Frame, tanto estáticos como dinámicos.

\end{fulllineitems}
%******* Termina función *******

%******* Empieza función *******
\begin{fulllineitems}

\pysiglinewithargsret{\sphinxbfcode{update\_widgets}}{\emph{event=None}}{}~

\begin{notice}{note}{Note:}
Este método es privado.
\end{notice}

Realiza solamente la actualización y colocación de elementos dinámicos 
en el Frame.\break
Si el parámetro event es distinto de \textbf{None}, significa que se 
lanzó un evento que provocará que se actualicen los parámetros de 
acuerdo con la técnica seleccionada.

\begin{quote}\begin{description}
\item[{Parameters}] \leavevmode\begin{itemize}
\item \textbf{\texttt{event}} (\emph{\texttt{String}}) -- Contiene el valor del elemento que ejecutó esta función.
\end{itemize}
\end{description}\end{quote}

\end{fulllineitems}
%******* Termina función *******

%******* Empieza función *******
\begin{fulllineitems}

\pysiglinewithargsret{\sphinxbfcode{get\_information}}{}{}
Recolecta la información que ha seleccionado e introducido 
el usuario, también la organiza para que se pueda utilizar 
apropiadamente.

\begin{quote}\begin{description}
\item[{Returns}] \leavevmode

Un diccionario que contiene:

\begin{itemize}
\item \textbf{Clase},
\item \textbf{Técnica},
\item \textbf{Parámetros.}
\end{itemize}

\item[{Return type}] \leavevmode
Dictionary
\end{description}\end{quote}

\end{fulllineitems}
%******* Termina función *******

%******* Empieza función *******
\begin{fulllineitems}

\pysiglinewithargsret{\sphinxbfcode{restore\_settings}}{}{}
Asigna los valores por defecto tanto de las técnicas como 
de sus respectivos parámetros, también limpia aquéllos en 
donde se hayan insertado valores.

\end{fulllineitems}
%******* Termina función *******

\end{fulllineitems}
%******* Termina clase *******

%******* Empieza clase *******
\subparagraph{SharingFunctionFrame (clase)}
%Se coloca el vínculo interno procedente de esta misma sección (a_3_2_6_1_2).
\label{sec:a_3_2_6_1_2}
%******* Empieza descripción *******
\begin{fulllineitems}

\begin{DUlineblock}{0em}
\item[] Esta clase proporciona una base gráfica para que el 
usuario pueda seleccionar métodos con sus respectivos parámetros 
\textbf{(si es que tienen)} referentes a Sharing Function.\break         
Una técnica de Sharing Function sirve para aplicar una selección 
más intensiva de Individuos en caso de haber un ``empate'' entre 
éstos.
\end{DUlineblock}

\begin{quote}\begin{description}
\item[{Parameters}] \leavevmode\begin{itemize}
\item \textbf{\texttt{parent}} (\emph{\texttt{Tkinter.Frame}}) -- Frame padre al que pertenece.
\item \textbf{\texttt{name}} (\emph{\texttt{String}}) -- Identificador \textbf{(único)} que tendrá el Frame.
\item \textbf{\texttt{features}} (\emph{\texttt{Dictionary}}) -- Conjunto de técnicas con sus respectivos parámetros para que se puedan cargar automáticamente en este Frame \textbf{(véase Controller/XMLParser.py)}.
\end{itemize}

\item[{Returns}] \leavevmode
Tkinter.Frame
\item[{Return type}] \leavevmode
Instance
\end{description}\end{quote}

%******* Termina descripción *******
\index{\_SharingFunctionFrame\_\_create\_dynamic\_widgets() (SharingFunctionFrame method)}

%******* Empieza función *******
\begin{fulllineitems}

\pysiglinewithargsret{\sphinxbfcode{create\_dynamic\_widgets}}{}{}~

\begin{notice}{note}{Note:}
Este método es privado.
\end{notice}

Inicializa los elementos dinámicos del Frame, esto es, de 
acuerdo al tipo que lleva cada parámetro se creará un widget 
diferente.

\end{fulllineitems}
%******* Termina función *******

%******* Empieza función *******
\begin{fulllineitems}

\pysiglinewithargsret{\sphinxbfcode{grid\_widgets}}{}{}~

\begin{notice}{note}{Note:}
Este método es privado.
\end{notice}

Coloca elementos en el Frame, tanto estáticos como 
dinámicos.

\end{fulllineitems}
%******* Termina función *******

%******* Empieza función *******
\begin{fulllineitems}

\pysiglinewithargsret{\sphinxbfcode{update\_widgets}}{\emph{event=None}}{}~

\begin{notice}{note}{Note:}
Este método es privado.
\end{notice}

Realiza solamente la actualización y colocación de elementos 
dinámicos en el Frame.\break
Si el parámetro event es distinto de \textbf{None}, significa 
que se lanzó un evento que provocará que se actualicen los 
parámetros de acuerdo con la técnica seleccionada.

\begin{quote}\begin{description}
\item[{Parameters}] \leavevmode\begin{itemize}
\item \textbf{\texttt{event}} (\emph{\texttt{String}}) -- Contiene el valor del elemento que ejecutó esta función.
\end{itemize}
\end{description}\end{quote}

\end{fulllineitems}
%******* Termina función *******

%******* Empieza función *******
\begin{fulllineitems}

\pysiglinewithargsret{\sphinxbfcode{get\_information}}{}{}
Recolecta la información que ha seleccionado e introducido 
el usuario, también la organiza para que se pueda utilizar 
apropiadamente.

\begin{quote}\begin{description}
\item[{Returns}] \leavevmode

Un diccionario que contiene:

\begin{itemize}
\item \textbf{Clase},
\item \textbf{Técnica},
\item \textbf{Parámetros.}
\end{itemize}

\item[{Return type}] \leavevmode
Dictionary
\end{description}\end{quote}

\end{fulllineitems}
%******* Termina función *******

%******* Empieza función *******
\begin{fulllineitems}

\pysiglinewithargsret{\sphinxbfcode{restore\_settings}}{}{}
Asigna los valores por defecto tanto de las técnicas como de sus 
respectivos parámetros, también limpia aquéllos en donde se hayan 
insertado valores.

\end{fulllineitems}
%******* Termina función *******

\end{fulllineitems}
%******* Termina clase *******
%******* Termina módulo *******
%******* Termina módulo *******

%******* Empieza módulo *******
\subsection{Additional (módulo)}
%Se coloca el vínculo interno procedente de esta misma sección (a_3_3).
\label{sec:a_3_3}
Proporciona elementos gráficos que, aunque no tienen cabilda 
en la Ventana Principal, sí contienen herramientas auxiliares 
de importancia.\medskip\break
El módulo consta de los siguientes elementos:

%******* Empieza clase *******
\subsubsection{GenerationSignalToplevel (clase)}
%Se coloca el vínculo interno procedente de esta misma sección (a_3_3_1).
\label{sec:a_3_3_1}
%******* Empieza descripción *******
\begin{fulllineitems}

\begin{DUlineblock}{0em}
\item[] Se trata de un Toplevel \textbf{(ventana independiente)} que 
muestra el progreso de las generaciones al momento de ejecutar un Task.\break
Esta ventana aunque es creada y mostrada en los procesos de la capa View, 
será en \textbf{Model/MOEA} en donde se utilice y actualice, ya que la idea 
es crear una ``señal'' que indique al usuario el progreso del MOEA en 
ejecución para que se dé una idea del desempeño del algoritmo.
\end{DUlineblock}

\begin{quote}\begin{description}
\item[{Parameters}] \leavevmode\begin{itemize}
\item \textbf{\texttt{parent}} (\emph{\texttt{Tkinter.Frame}}) -- Frame padre al que pertenece.
\item \textbf{\texttt{path\_image\_logo}} (\emph{\texttt{String}}) -- La ruta al logotipo que se usa en esta ventana independiente.
\item \textbf{\texttt{execution\_task\_number}} (\emph{\texttt{Integer}}) -- Número que indica el actual Task en ejecución \textbf{(véase View/Additional/}\break\textbf{ResultsGrapher/ResultsGrapherToplevel.py)}.
\end{itemize}

\item[{Returns}] \leavevmode
Tkinter.Toplevel
\item[{Return type}] \leavevmode
Instance
\end{description}\end{quote}

%******* Termina descripción *******

%******* Empieza función *******
\begin{fulllineitems}

\pysiglinewithargsret{\sphinxbfcode{center}}{}{}~

\begin{notice}{note}{Note:}
Este método es privado.
\end{notice}

Centra la ventana independiente con respecto de la 
Ventana Principal.\break
En otras palabras, la ventana independiente será 
colocada en el centro de la Ventana Principal.

\end{fulllineitems}
%******* Termina  función *******

%******* Empieza función *******
\begin{fulllineitems}

\pysiglinewithargsret{\sphinxbfcode{do\_nothing}}{}{}~

\begin{notice}{note}{Note:}
Este método es privado.
\end{notice}

Simplemente es una función ``dummy'' que no realiza 
nada.\break
Es utilizada como sustituto de la función del ícono 
``Cerrar'' y así evitar que el usario intencionadamente 
intente ocluir la ventana del número de generaciones.

\end{fulllineitems}
%******* Termina función *******

%******* Empieza función *******
\begin{fulllineitems}

\pysiglinewithargsret{\sphinxbfcode{close}}{}{}
Oculta y elimina toda referencia gráfica y lógica de la 
ventana independiente, indicando así que el número de 
generaciones ha alcanzado su límite.

\end{fulllineitems}
%******* Termina función *******

%******* Empieza función *******
\begin{fulllineitems}

\pysiglinewithargsret{\sphinxbfcode{hide}}{}{}
Oculta la ventana independiente de la pantalla pero no 
la elimina de los registros gráficos.

\end{fulllineitems}
%******* Termina función *******

%******* Empieza función *******
\begin{fulllineitems}

\pysiglinewithargsret{\sphinxbfcode{show}}{}{}
Reactiva la ventana independiente, realizando 
además durante esta ejecución un par de consignas 
más para dar una experiencia de usuario suficiente 
y concisa.

\end{fulllineitems}
%******* Termina función *******

%******* Empieza función *******
\begin{fulllineitems}

\pysiglinewithargsret{\sphinxbfcode{update\_current\_generation}}{\emph{current\_generation}}{}~
\vspace{-0.3cm}

Actualiza la generación actual en la ventana independiente.\break
Típicamente esta función será usada en todos los algoritmos de la 
capa Model/MOEA, pues es allí donde se designará el progreso del 
algoritmo que a su vez se verá reflejado en la capa de View.

\begin{quote}\begin{description}
\item[{Parameters}] \leavevmode\begin{itemize}
\item \textbf{\texttt{current\_generation}} (\emph{\texttt{Integer}}) -- La generación que está corriendo actualmente en el MOEA.s
\end{itemize}

\item[{Returns}] \leavevmode
1 si se  ha alcanzado la generación límite, 0 en otro caso.
\item[{Return type}] \leavevmode
Integer
\end{description}\end{quote}

\end{fulllineitems}
%******* Termina función *******

%******* Empieza función *******
\begin{fulllineitems}

\pysiglinewithargsret{\sphinxbfcode{update\_number\_of\_generations}}{\emph{number\_of\_generations}}{}
Actualiza el número total de generaciones.\break
Generalmente esta función será llamada desde Controller/Controller.py 
ya que ahí es donde se decide si las configuraciones iniciales son 
adecuadas para poder ejecutar el algoritmo.

\begin{quote}\begin{description}
\item[{Parameters}] \leavevmode\begin{itemize}
\item \textbf{\texttt{number\_of\_generations}} (\emph{\texttt{Integer.}}) -- El número de generaciones total que tendrá el MOEA.
\end{itemize}
\end{description}\end{quote}

\end{fulllineitems}
%******* Termina función *******

\end{fulllineitems}
%******* Termina clase  *******
%******* Termina módulo *******

%******* Empieza módulo *******
\subsection{MenuInternalOption (módulo)}
%Se coloca el vínculo interno procedente de esta misma sección (a_3_3_2).
\label{sec:a_3_3_2}
Contiene elementos gráficos que permiten acceder a configuraciones 
internas del programa y también a M.O.P.s \textbf{(Multi-Objective Problems)} 
previamente cargados para hacer uso fácil de ellos.

%******* Empieza clase *******
\subsubsection{MenuInternalOption (clase)}
%Se coloca el vínculo interno procedente de esta misma sección (a_3_3_2_1).
\label{sec:a_3_3_2_1}
%******* Empieza descripción *******
\begin{fulllineitems}

\begin{DUlineblock}{0em}
\item[] Se crea el Menú de Opciones Internas o Menú Secundario.\break
Básicamente se trata de una serie de características que, aunque no
forman parte esencial del programa, sí ofrecen alternativas que 
pueden facilitar la experiencia de usuario.\break
Este menú será atado al Frame Principal y desde allí el usuario podrá
tener acceso a las opciones que aquí se describen.
\end{DUlineblock}

\begin{quote}\begin{description}
\item[{Parameters}] \leavevmode\begin{itemize}
\item \textbf{\texttt{parent}} (\emph{\texttt{Tkinter.Frame}}) -- El Frame Padre al que pertenece esta implementación.
\item \textbf{\texttt{path\_image\_logo}} (\emph{\texttt{String}}) -- La ruta al logotipo que se usa en esta ventana independiente.
\item \textbf{\texttt{features}} (\emph{\texttt{Dictionary}}) -- Un diccionario con las características que deberá tener cada una de las opciones listadas.
\end{itemize}

\item[{Returns}] \leavevmode
Tkinter.Menu
\item[{Return type}] \leavevmode
Instance
\end{description}\end{quote}

%******* Termina descripción *******

%******* Empieza función *******
\begin{fulllineitems}

\pysiglinewithargsret{\sphinxbfcode{launch\_about\_toplevel}}{}{}~

\begin{notice}{note}{Note:}
Este método es privado.
\end{notice}

Abre la ventana independiente \textbf{(Toplevel)} About.
También verifica que se abra una y sólo una instancia de 
dicha ventana.

\end{fulllineitems}
%******* Termina función *******

%******* Empieza función *******
\begin{fulllineitems}

\pysiglinewithargsret{\sphinxbfcode{launch\_internal\_option\_toplevel}}{}{}~

\begin{notice}{note}{Note:}
Este método es privado.
\end{notice}

Abre la ventana independiente \textbf{(Toplevel)} Internal 
Options \textbf{(o simplemente Options)}.\break
También verifica que se abra una y sólo una instancia de 
dicha ventana.

\end{fulllineitems}
%******* Empieza función *******

%******* Empieza función *******
\begin{fulllineitems}

\pysiglinewithargsret{\sphinxbfcode{about\_toplevel\_custom\_close}}{}{}
Indica que la única instancia que debe crearse
para la opción About está disponible.

\end{fulllineitems}
%******* Empieza función *******

%******* Empieza función *******
\begin{fulllineitems}

\pysiglinewithargsret{\sphinxbfcode{internal\_option\_toplevel\_custom\_close}}{}{}
Indica que la única instancia que debe crearse
para la opción Options está disponible.

\end{fulllineitems}
%******* Termina función *******

\end{fulllineitems}
%******* Termina clase *******

El módulo consta de las siguientes características:

%******* Empieza clase *******
\paragraph{InternalOptionToplevel (clase)}
%Se coloca el vínculo interno procedente de esta misma sección (a_3_3_2_2).
\label{sec:a_3_3_2_2}
%******* Empieza descripción *******
\begin{fulllineitems}

\begin{DUlineblock}{0em}
\item[] Contiene un Menú pequeño con pestañas que indican las
características internas del sistema a las que puede tener acceso el 
usuario.\break
En su mayoría se trata de características que muestran los métodos,
técnicas y sistemas auxiliares que garantizan un manejo más armonioso 
del programa y si así lo desea el usuario, modificarlos para ajustar
su desempeño.
\end{DUlineblock}

\begin{quote}\begin{description}
\item[{Parameters}] \leavevmode\begin{itemize}
\item \textbf{\texttt{parent}} (\emph{\texttt{Tkinter.Menu}}) -- El elemento Padre al que pertenece la actual ventana independiente \textbf{(Toplevel)}.
\item \textbf{\texttt{path\_image\_logo}} (\emph{\texttt{String}}) -- La ruta al logotipo que se usa en esta ventana independiente.
\item \textbf{\texttt{features}} (\emph{\texttt{Dictionary}}) -- Un diccionario que contiene las características necesarias que serán mostradas en esta ventana independiente.
\item \textbf{\texttt{custom\_function}} (\emph{\texttt{Instance}}) -- Una variable que contiene una función, la cual redefinirá más apropiadamente el comportamiento de la actual Ventana Principal con respecto de su Frame Padre.
\end{itemize}

\item[{Returns}] \leavevmode
La ventana independiente que contiene la información
\item[{Return type}] \leavevmode
Tkinter.Toplevel
\end{description}\end{quote}

%******* Termina descripción *******

%******* Empieza función *******
\begin{fulllineitems}

\pysiglinewithargsret{\sphinxbfcode{center}}{}{}~
\begin{notice}{note}{Note:}
Este método es privado.
\end{notice}

Centra la ventana independiente con respecto de la 
Ventana Principal.\break
En otras palabras, la ventana independiente será colocada 
en el centro de la Ventana Principal.

\end{fulllineitems}
%******* Termina función *******

%******* Empieza función *******
\begin{fulllineitems}

\pysiglinewithargsret{\sphinxbfcode{close}}{}{}~

\begin{notice}{note}{Note:}
Este método es privado.
\end{notice}

Cierra y elimina todo rastro de esta 
ventana independiente.

\end{fulllineitems}
%******* Termina función *******

\end{fulllineitems}
%******* Termina clase *******

%******* Empieza módulo *******
\paragraph{InternalOptionTab (módulo)}
%Se coloca el vínculo interno procedente de esta misma sección (a_3_3_2_3).
\label{sec:a_3_3_2_3}
\begin{fulllineitems}
\begin{DUlineblock}{0em}
\item[] Contiene las partes gráficas que conformarán cada una de las pestañas
concernientes al Toplevel \textbf{(ventana independiente)} de opciones internas
\textbf{(InternalOptionToplevel)}.\break
Consta de los siguientes elementos:
\end{DUlineblock}
\end{fulllineitems}

%******* Empieza clase *******
\subparagraph{MOPExampleFrame (clase)}
%Se coloca el vínculo interno procedente de esta misma sección (a_3_3_2_3_1).
\label{sec:a_3_3_2_3_1}
%******* Empieza descripción *******
\begin{fulllineitems}

\begin{DUlineblock}{0em}
\item[] Unifica dos elementos: Canvas y MOPFrame.\break
La razón de esto es que, en promedio la información mostrada 
por MOPFrame rebasará el tamaño de la ventana de la información 
final \textbf{(véase View/Additional/}\break\textbf{ResultsGrapher/ResultsGrapherTopLevel.py)}, 
es entonces que se deben agregar barras de desplazamiento para poder acceder 
al contenido que quedaría oculto.\break
Uno de los elementos en Tkinter más sencillos que cumplen con este 
cometido es un Canvas. Luego entonces esa es la razón de tal fusión.
\end{DUlineblock}

\begin{quote}\begin{description}
\item[{Parameters}] \leavevmode\begin{itemize}
\item \textbf{\texttt{parent}} (\emph{\texttt{Tkinter.Toplevel}}) -- El elemento Padre al que pertenece el actual
Frame.
\item \textbf{\texttt{features}} (\emph{\texttt{Dictionary}}) -- Un diccionario que contiene las características necesarias que serán mostradas en este Frame.
\end{itemize}

\item[{Returns}] \leavevmode
El Frame que contiene la información señalada.
\item[{Return type}] \leavevmode
Tkinter.Frame
\end{description}\end{quote}

%******* Termina descripción *******

%******* Empieza función *******
\begin{fulllineitems}

\pysiglinewithargsret{\sphinxbfcode{update\_scrollbar}}{\emph{event}}{}~

\begin{notice}{note}{Note:}
Este método es privado.
\end{notice}

Actualiza la barra de desplazamiento de acuerdo al 
número de elementos existentes en el Frame, esto 
para poder hacer un recorrido apropiado de la barra.

\begin{quote}\begin{description}
\item[{Parameters}] \leavevmode\begin{itemize}
\item\textbf{\texttt{event}} (\emph{\texttt{String}}) -- Elemento que ejecutó esta función.
\end{itemize}
\end{description}\end{quote}

\end{fulllineitems}
%******* Termina función *******

\end{fulllineitems}
%******* Termina clase *******

La clase actual tiene como base el siguiente elemento:

%******* Empieza clase *******
\subparagraph{MOPFrame (clase)}
%Se coloca el vínculo interno procedente de esta misma sección (a_3_3_2_3_1_1).
\label{sec:a_3_3_2_3_1_1}
%******* Empieza descripción *******
\begin{fulllineitems}

\begin{DUlineblock}{0em}
\item[] Muestra la información relativa a los M.O.P.'s y
provee de métodos que facilitan la carga de éstos en la
Ventana Principal.\break
Un M.O.P. \textbf{(Multi Objective Problem)} es un conjunto 
de funciones y variables bien definidas que ya han sido previamente 
estudiadas, así como su comportamiento en conjunto; la idea es 
proporcionarle al usuario un ambiente de carga fácil de datos 
para que pueda probar los ejemplos ya tratados por muchos 
autores en los libros que se citarán en el trabajo escrito.
\end{DUlineblock}

\begin{quote}\begin{description}
\item[{Parameters}] \leavevmode\begin{itemize}
\item \textbf{\texttt{parent}} (\emph{\texttt{Tkinter.Frame}}) -- El elemento Padre al que pertenece el actual Frame.
\item \textbf{\texttt{grandparent}} (\emph{\texttt{Tkinter.Toplevel}}) -- El elemento Padre del Padre al que pertenece el actual Frame.
\item \textbf{\texttt{features}} (\emph{\texttt{Dictionary}}) -- Un diccionario que contiene las características necesarias que serán mostradas en este Frame.
\end{itemize}

\item[{Returns}] \leavevmode
El Frame que contiene la información señalada.
\item[{Return type}] \leavevmode
Tkinter.Frame
\end{description}\end{quote}

%******* Termina descripción *******

%******* Empieza función *******
\begin{fulllineitems}

\pysiglinewithargsret{\sphinxbfcode{get\_mop\_example}}{\emph{event}}{}~

\begin{notice}{note}{Note:}
Este método es privado.
\end{notice}

Con base en la selección de M.O.P.
hecha por el usuario, se carga éste
en la Ventana Principal.

\begin{quote}\begin{description}
\item[{Parameters}] \leavevmode\begin{itemize}
\item\textbf{\texttt{event}} (\emph{\texttt{String}}) -- El evento del elemento gráfico que
activa esta función.
\end{itemize}
\end{description}\end{quote}

\end{fulllineitems}
%******* Termina función *******

%******* Empieza función *******
\begin{fulllineitems}

\pysiglinewithargsret{\sphinxbfcode{update\_current\_mop}}{\emph{event=None}}{}~

\begin{notice}{note}{Note:}
Este método es privado.
\end{notice}

Despliega la información relacionada con el
M.O.P. seleccionado.

\begin{quote}\begin{description}
\item[{Parameters}] \leavevmode\begin{itemize}
\item \textbf{\texttt{event}} (\emph{\texttt{String}}) -- El evento del elemento gráfico que
activa esta función.
\end{itemize}
\end{description}\end{quote}

\end{fulllineitems}
%******* Termina función *******

\end{fulllineitems}
%******* Termina clase *******

%******* Empieza clase *******
\subparagraph{FeatureFrame (clase)}
%Se coloca el vínculo interno procedente de esta misma sección (a_3_3_2_3_2).
\label{sec:a_3_3_2_3_2}
%******* Empieza descripción *******
\begin{fulllineitems}

\begin{DUlineblock}{0em}
\item[] Unifica dos elementos: Canvas y CharacteristicFrame.\break
La razón de esto es que, en promedio la información mostrada 
por CharacteristicFrame rebasará el tamaño de la ventana de la 
información final \textbf{(véase View/}\break\textbf{Additional/ResultsGrapher/ResultsGrapherTopLevel.py)}, es 
entonces que se deben agregar barras de desplazamiento para poder 
acceder al contenido que quedaría oculto.\break
Uno de los elementos en Tkinter más sencillos que cumplen con 
este cometido es un Canvas. Luego entonces esa es la razón de 
tal fusión.
\end{DUlineblock}

\begin{quote}\begin{description}
\item[{Parameters}] \leavevmode\begin{itemize}
\item \textbf{\texttt{parent}} (\emph{\texttt{Tkinter.Toplevel}}) -- El elemento Padre al que pertenece el actual Frame.
\item \textbf{\texttt{features}} (\emph{\texttt{Dictionary}}) -- Un diccionario que contiene las características necesarias que serán mostradas en este Frame.
\end{itemize}

\item[{Returns}] \leavevmode
El Frame que contiene la información señalada.
\item[{Return type}] \leavevmode
Tkinter.Frame
\end{description}\end{quote}

%******* Termina descripción *******

%******* Empieza función *******
\begin{fulllineitems}

\pysiglinewithargsret{\sphinxbfcode{update\_scrollbar}}{\emph{event}}{}~

\begin{notice}{note}{Note:}
Este método es privado.
\end{notice}

Actualiza la barra de desplazamiento de acuerdo al 
número de elementos existentes en el Frame, esto 
para poder hacer un recorrido apropiado de la barra.
\begin{quote}\begin{description}
\item[{Parameters}] \leavevmode\begin{itemize}
\item \textbf{\texttt{event}} (\emph{\texttt{String}}) -- Elemento que ejecutó esta función.
\end{itemize}
\end{description}\end{quote}

\end{fulllineitems}
%******* Termina función *******

\end{fulllineitems}
%******* Termina clase *******

La clase actual se apoya del siguiente elemento:

%******* Empieza clase *******
\subparagraph{CharacteristicFrame (clase)}
%Se coloca el vínculo interno procedente de esta misma sección (a_3_3_2_3_2_1).
\label{sec:a_3_3_2_3_2_1}
%******* Empieza descripción *******
\begin{fulllineitems}


\begin{DUlineblock}{0em}
\item[] Despliega información concerniente a todas las técnicas 
\textbf{(con sus respectivos parámetros)} disponibles para el usuario.\break
Se agrupan éstas en las mismas categorías que presenta el programa,
más en concreto, las secciones que conforman a la Ventana Principal 
\textbf{(véase View/}\break\textbf{MainWindow.py)}.\break
También señala someramente las instrucciones necesarias para que el
programa pueda reconocer cualquier técnica que desarrolle el usuario.
\end{DUlineblock}

\begin{quote}\begin{description}
\item[{Parameters}] \leavevmode\begin{itemize}
\item \textbf{\texttt{parent}} (\emph{\texttt{Tkinter.Toplevel}}) -- El elemento Padre al que pertenece el actual Frame.
\item \textbf{\texttt{features}} (\emph{\texttt{Dictionary}}) -- Un diccionario que contiene las características necesarias que serán mostradas en este Frame.
\end{itemize}

\item[{Returns}] \leavevmode
El Frame que contiene la información señalada.
\item[{Return type}] \leavevmode
Tkinter.Frame
\end{description}\end{quote}

%******* Termina descripción *******

\end{fulllineitems}
%******* Termina clase *******

%******* Empieza clase *******
\subparagraph{PythonExpressionFrame (clase)}
%Se coloca el vínculo interno procedente de esta misma sección (a_3_3_2_3_3).
\label{sec:a_3_3_2_3_3}
%******* Empieza descripción *******
\begin{fulllineitems}

\begin{DUlineblock}{0em}
\item[] Realiza la fusión de Canvas y ExpressionFrame, debido a 
que, cuando se agregan numerosas variables al ExpressionFrame, 
se debe insertar una barra de desplazamiento para poder acceder 
a aquéllos que se encuentren hasta abajo.\break
Dentro del ambiente de Tkinter, el elemento más sencillo para 
lograr este efecto es un Canvas, por ello se anida el ExpressionFrame 
al Canvas.
\end{DUlineblock}

\begin{quote}\begin{description}
\item[{Parameters}] \leavevmode\begin{itemize}
\item \textbf{\texttt{parent}} (\emph{\texttt{Tkinter.Frame}}) -- Frame padre al que pertenece.
\item \textbf{\texttt{features}} (\emph{\texttt{Dictionary}}) -- Conjunto de técnicas con sus respectivos parámetros para que se puedan cargar automáticamente en este Frame \textbf{(véase Controller/XML/}\break\textbf{PythonExpressions.xml)}.
\end{itemize}

\item[{Returns}] \leavevmode
Tkinter.Frame
\item[{Return type}] \leavevmode
Instance
\end{description}\end{quote}

%******* Termina descripción *******

%******* Empieza función *******
\begin{fulllineitems}

\pysiglinewithargsret{\sphinxbfcode{activate\_scroll}}{\emph{event=None}}{}~

\begin{notice}{note}{Note:}
Este método es privado.
\end{notice}

Actualiza la barra de desplazamiento y con base en esta 
acción la activa o desactiva.

\begin{quote}\begin{description}
\item[{Parameters}] \leavevmode\begin{itemize}
\item \textbf{\texttt{event}} (\emph{\texttt{String}}) -- Elemento que ejecutó esta función.
\end{itemize}
\end{description}\end{quote}

\end{fulllineitems}
%******* Termina función *******

%******* Empieza función *******
\begin{fulllineitems}

\pysiglinewithargsret{\sphinxbfcode{update\_scrollbar}}{\emph{event=None}}{}~

\begin{notice}{note}{Note:}
Este método es privado.
\end{notice}

Actualiza la barra de desplazamiento de acuerdo al 
número de elementos existentes en el Frame, esto 
para poder hacer un recorrido apropiado de la barra.

\begin{quote}\begin{description}
\item[{Parameters}] \leavevmode\begin{itemize}
\item \textbf{\texttt{event}} (\emph{\texttt{String}}) -- Elemento que ejecutó esta función.
\end{itemize}
\end{description}\end{quote}

\end{fulllineitems}
%******* Termina función *******

\end{fulllineitems}
%******* Termina clase *******

La clase actual toma como referencia el siguiente elemento:

%******* Empieza clase *******
\subparagraph{ExpressionFrame (clase)}
%Se coloca el vínculo interno procedente de esta misma sección (a_3_3_2_3_3_1).
\label{sec:a_3_3_2_3_3_1}
%******* Empieza descripción *******
\begin{fulllineitems}

\begin{DUlineblock}{0em}
\item[] Ofrece opciones simples para mostrar y añadir expresiones
de Python.\break 
Lo anterior ocurre ya que al momento de crear y evaluar funciones 
objetivo hay algunas palabras reservadas que no pueden ser usadas 
en Python directamente si no se hace un renombramiento apropiado.\break
Dicha información se encuentra en \textbf{Controller/XML/PythonExpressions.xml}
\end{DUlineblock}

\begin{quote}\begin{description}
\item[{Parameters}] \leavevmode\begin{itemize}
\item \textbf{\texttt{parent}} (\emph{\texttt{Tkinter.Toplevel}}) -- El elemento Padre al que pertenece el actual Frame.
\item \textbf{\texttt{features}} (\emph{\texttt{Dictionary}}) -- Un diccionario que contiene las características necesarias que serán mostradas en este Frame.
\end{itemize}

\item[{Returns}] \leavevmode
El Frame que contiene la información señalada.
\item[{Return type}] \leavevmode
Tkinter.Frame
\end{description}\end{quote}

%******* Termina descripción *******

%******* Empieza función *******
\begin{fulllineitems}

\pysiglinewithargsret{\sphinxbfcode{add\_expression}}{\emph{event}}{}~

\begin{notice}{note}{Note:}
Este método es privado.
\end{notice}

Inserta una casilla que conforma una expresión
dentro del Frame.

\begin{quote}\begin{description}
\item[{Parameters}] \leavevmode\begin{itemize}
\item \textbf{\texttt{event}} (\emph{\texttt{String}}) -- Identificador del elemento gráfico que activó la función.
\end{itemize}
\end{description}\end{quote}

\end{fulllineitems}
%******* Termina función *******

%******* Empieza función *******
\begin{fulllineitems}

\pysiglinewithargsret{\sphinxbfcode{delete\_single\_expression}}{\emph{event}}{}~

\begin{notice}{note}{Note:}
Este método es privado.
\end{notice}

Elimina una expresión y todos los elementos 
gráficos que la acompañan.\break
También elimina todo rastro que se encuentre 
en las estructuras lógicas.

\begin{quote}\begin{description}
\item[{Parameters}] \leavevmode\begin{itemize}
\item \textbf{\texttt{event}} (\emph{\texttt{String}}) -- Identificador del elemento gráfico que activó la función.
\end{itemize}
\end{description}\end{quote}

\end{fulllineitems}
%******* Termina función *******

%******* Empieza función *******
\begin{fulllineitems}

\pysiglinewithargsret{\sphinxbfcode{get\_information}}{}{}~

\begin{notice}{note}{Note:}
Este método es privado.
\end{notice}

Toma la información del Frame \textbf{(en específico de las casillas)} 
y regresa las expresiones con sus respectivos equivalentes en Python.

\begin{quote}\begin{description}
\item[{Returns}] \leavevmode
Una lista que contiene arreglos de dos elementos donde el primero es la expresión normal mientras que el segundo es la expresión equivalente en Python.
\item[{Return type}] \leavevmode
List
\end{description}\end{quote}

\end{fulllineitems}
%******* Termina función *******

%******* Empieza función *******
\begin{fulllineitems}

\pysiglinewithargsret{\sphinxbfcode{insert\_expression}}{\emph{expression=None}}{}~

\begin{notice}{note}{Note:}
Este método es privado.
\end{notice}

Coloca en el Frame una colección de elementos:\break
{[}etiqueta para expresión normal, expresión normal, etiqueta para expresión de Pyrhon, expresión de Python, botón para eliminar{]}\break
Si el parámetro expression es \textbf{None}, se añade 
la casilla vacía, de lo contrario se agrega ésta 
con la información pertinente.\break

\begin{quote}\begin{description}
\item[{Parameters}] \leavevmode\begin{itemize}
\item \textbf{\texttt{expression}} (\emph{\texttt{Array}}) -- Un arreglo con dos elementos, el primero contiene la expresión normal mientras que el segundo maneja la información de la expresión equivalente en Python.
\end{itemize}
\end{description}\end{quote}

\end{fulllineitems}
%******* Termina función *******

%******* Empieza función *******
\begin{fulllineitems}

\pysiglinewithargsret{\sphinxbfcode{load\_expressions}}{}{}~

\begin{notice}{note}{Note:}
Este método es privado.
\end{notice}

Carga las expresiones a manera de contenido gráfico
en el Frame.\break
Dichas expresiones son tomadas del archivo \textbf{Controller/XML/}\break\textbf{PythonExpressions.xml.}

\end{fulllineitems}
%******* Termina función *******

%******* Empieza función *******
\begin{fulllineitems}

\pysiglinewithargsret{\sphinxbfcode{save\_changes}}{\emph{event}}{}~

\begin{notice}{note}{Note:}
Este método es privado.
\end{notice}

Toma la información existente en las casillas y procede a 
sobreescribir el archivo \textbf{Controller/XML/PythonExpressions.xml} 
con la información recién recabada.

\begin{quote}\begin{description}
\item[{Parameters}] \leavevmode\begin{itemize}
\item \textbf{\texttt{event}} (\emph{\texttt{String}}) -- Identificador del elemento gráfico que activó la función.
\end{itemize}
\end{description}\end{quote}

\end{fulllineitems}
%******* Termina función *******

%******* Empieza función *******
\begin{fulllineitems}

\pysiglinewithargsret{\sphinxbfcode{get\_current\_elements}}{}{}
Regresa el número actual de casillas en el Frame.

\begin{quote}\begin{description}
\item[{Returns}] \leavevmode
Cantidad de elementos en la estructura rows, donde se guardan las casillas \textbf{(Entry's)}.
\item[{Return type}] \leavevmode
Integer
\end{description}\end{quote}

\end{fulllineitems}
%******* Termina función *******

\end{fulllineitems}
%******* Termina clase *******

%******* Empieza clase *******
\paragraph{AboutToplevel (clase)}
%Se coloca el vínculo interno procedente de esta misma sección (a_3_3_2_4).
\label{sec:a_3_3_2_4}
%******* Empieza descripción *******
\begin{fulllineitems}

\begin{DUlineblock}{0em}
\item[] Esta ventana independiente \textbf{(Toplevel)} 
proporciona información básica del programa así como 
de sus desarrolladores.
\end{DUlineblock}

\begin{quote}\begin{description}
\item[{Parameters}] \leavevmode\begin{itemize}
\item \textbf{\texttt{parent}} (\emph{\texttt{Tkinter.Menu}}) -- El elemento Padre al que pertenece la actual ventana independiente \textbf{(Toplevel)}.
\item \textbf{\texttt{path\_image\_logo}} (\emph{\texttt{String}}) -- La ruta al logotipo que se usa en esta ventana independiente.
\item \textbf{\texttt{custom\_function}} (\emph{\texttt{Instance}}) -- Una variable que contiene una función, la cual redefinirá más apropiadamente el comportamiento de la actual Ventana Principal con respecto de su Frame Padre.
\end{itemize}

\item[{Returns}] \leavevmode
Tkinter.Toplevel
\item[{Return type}] \leavevmode
Instance
\end{description}\end{quote}

%******* Termina descripción *******

%******* Empieza función *******
\begin{fulllineitems}

\pysiglinewithargsret{\sphinxbfcode{center}}{}{}~

\begin{notice}{note}{Note:}
Este método es privado.
\end{notice}

Centra la ventana independiente con respecto de 
la Ventana Principal.\break
En otras palabras, la ventana independiente será 
colocada en el centro de la Ventana Principal.
\end{fulllineitems}
%******* Termina función *******

%******* Empieza función *******
\begin{fulllineitems}

\pysiglinewithargsret{\sphinxbfcode{close}}{\emph{custom\_function}}{}~

\begin{notice}{note}{Note:}
Este método es privado.
\end{notice}

Cierra y elimina todo rastro de esta ventana independiente.

\begin{quote}\begin{description}
\item[{Parameters}] \leavevmode\begin{itemize}
\item \textbf{\texttt{custom\_function}} (\emph{\texttt{Instance}}) -- Una variable que contiene una función que ha de 
ejecutarse dentro de este método.
\end{itemize}
\end{description}\end{quote}

\end{fulllineitems}
%******* Termina función *******

\end{fulllineitems}
%******* Termina clase *******
%******* Termina módulo *******

%******* Empieza módulo *******
\subsubsection{ResultsGrapher (módulo)}
%Se coloca el vínculo interno procedente de esta misma sección (a_3_3_3).
\label{sec:a_3_3_3}
Proporciona los elementos gráficos para poder presentar
las gráficas de los resultados que ha arrojado la ejecución
de algún M.O.E.A.\medskip\break
Consta de los siguientes elementos:

%******* Empieza clase *******
\paragraph{ResultsGrapherToplevel (clase)}
%Se coloca el vínculo interno procedente de esta misma sección (a_3_3_3_1).
\label{sec:a_3_3_3_1}
%******* Empieza descripción *******
\begin{fulllineitems}

\begin{DUlineblock}{0em}
\item[] Esta clase lanza una ventana independiente que muestra los 
resultados arrojados por una configuración previa del usuario.\break
Primero que nada es menester mencionar que una ventana independiente 
es un Toplevel en Tkinter, la cual es casi ajena a la Ventana Principal 
\textbf{(véase View/Main/}\break\textbf{MainWindow.py)}, pero si ésta última es cerrada, 
se eliminarán también las ventanas independientes creadas.\medskip\break
Cada ventana independiente mostrará el número de Task, es decir, el 
orden en el que fue procesada la información con respecto de otros Tasks.\break
Entiéndase por Task a una ejecución de algún algoritmo MOEA bajo un 
cierto conjunto de configuraciones iniciales.\break
Así, los Tasks serán mostrados en una ventana independiente. La numeración 
de los Tasks irá siempre en orden progresivo, lo que significa que el número 
será reinicializado sólamente volviendo a ejecutar el programa principal.\break
De esta manera es posible tener varias ventanas independientes abiertas y 
en cuestiones más generales, es posible ejecutar varios Tasks simultáneamente, 
ya que el programa es multi-threading en ese sentido.\medskip\break
Finalmente, la información será mostrada en dos pestañas: en una 
\textbf{(SummaryFrame)} se otorga un resumen de todas las funciones objetivo, 
variables de decisión, MOEA usado y configuraciones adicionales en el Task.\break
En la otra \textbf{(GraphFrame)} se colocan todas las gráficas pertinentes 
producto de la ejecución del MOEA con las funciones objectivo, variables de 
decisión y configuraciones ingresadas \textbf{(véase Model/Community/Community.py)} \textbf{(véase View/}\break\textbf{Additional/ResultsGrapher/GraphFrame.py)}.\break
Si por cualquier circunstancia llega a haber una falla interna durante la 
ejecución del proceso, ninguna de las dos pestañas será mostrada y en su 
lugar aparecerá una de error \textbf{(ErrorFrame)}, especificando además 
el tipo de error y en qué parte de Model \textbf{(ó Modelo)} ocurrió.
\end{DUlineblock}

\begin{quote}\begin{description}
\item[{Parameters}] \leavevmode\begin{itemize}
\item \textbf{\texttt{parent}} (\emph{\texttt{Tkinter.Frame}}) -- Frame padre al que pertenece.
\item \textbf{\texttt{path\_image\_logo}} (\emph{\texttt{String}}) -- La ruta al logotipo que se usa en esta ventana independiente.
\item \textbf{\texttt{execution\_task\_count}} (\emph{\texttt{Integer}}) -- Número que indica el actual Task en ejecución.
\item \textbf{\texttt{main\_features}} (\emph{\texttt{Dictionary}}) -- Diccionario que contiene, entre otras cosas, los nombres de los parámetros asociados a cada técnica.
\item \textbf{\texttt{gathered\_information}} (\emph{\texttt{Dictionary}}) -- Diccionario que contiene todas las configuraciones recabadas ingresadas por el usuario \textbf{(véase View/Main/MainWindow.py)}.
\item \textbf{\texttt{final\_results}} (\emph{\texttt{Dictionary}}) -- Diccionario que contiene la información procesada lista para graficar\textbf{(véase View/Additional/ResultsGrapher/GraphFrame.py)}.
\end{itemize}

\item[{Returns}] \leavevmode
Tkinter.Toplevel
\item[{Return type}] \leavevmode
Instance
\end{description}\end{quote}

%******* Termina descripción *******

%******* Empieza función *******
\begin{fulllineitems}

\pysiglinewithargsret{\sphinxbfcode{center}}{}{}~

\begin{notice}{note}{Note:}
Este método es privado.
\end{notice}

Centra la ventana independiente con respecto de la 
Ventana Principal.\break
En otras palabras, la ventana independiente será 
colocada en el centro de Ventana Principal.

\end{fulllineitems}
%******* Termina función *******

%******* Empieza función *******
\begin{fulllineitems}

\pysiglinewithargsret{\sphinxbfcode{create\_renamed\_settings}}{}{}~

\begin{notice}{note}{Note:}
Este método es privado.
\end{notice}

Tal como su nombre lo dice, renombra las funciones objetivo y 
variables de decisión para posteriormente almacenarlas en una 
estructura por cada tipo.\break
Renombrar una función o variable de decisión es hacer un mapeo 
que consista en:\break
Elemento\_renombrado -\textgreater{} elemento original.\break
Para el caso de la función objetivo, el renombramiento se da 
anteponiendo la letra \textbf{F} seguido de la posición en la 
que fue insertada originalmente por el usuario.\break
El caso es análogo para la variable de decisión, sólo que la 
letra es \textbf{V}.\break
La idea de renombrar las funciones y variables surge como 
alternativa al momento de graficar los datos \textbf{(véase View/Additional/ResultsGrapher/}\break\textbf{GraphFrame.py)}, 
ya que el usuario puede ingresar funciones muy largas o variables 
con identificadores muy complejos y esto en la parte gráfica se 
vería muy amontonado; por ello fue preferible mostrar la parte 
renombrada en la sección de GraphFrame y colocar la muestra original 
en el SummaryFrame.         

\end{fulllineitems}
%******* Termina función *******

\end{fulllineitems}
%******* Termina clase *******

La clase actual consta de los siguientes elementos:

%******* Empieza clase *******
\subparagraph{GraphFrame (clase)}
%Se coloca el vínculo interno procedente de esta misma sección (a_3_3_3_1_1).
\label{sec:a_3_3_3_1_1}
%******* Empieza descripción *******
\begin{fulllineitems}

\begin{DUlineblock}{0em}
\item[] Proporciona un Frame que contiene gráficas 
alimentadas por los resultados obtenidos al ejecutar 
algún MOEA, el cual ha sido refinado por las configuraciones
recabadas de la Ventana Principal \textbf{(véase Model/MOEA)}.
\end{DUlineblock}

\begin{quote}\begin{description}
\item[{Parameters}] \leavevmode\begin{itemize}
\item \textbf{\texttt{parent}} (\emph{\texttt{Tkinter.Frame}}) -- Frame padre al que pertenece.
\item \textbf{\texttt{execution\_task\_count}} (\emph{\texttt{Integer}}) -- Número que indica el actual Task en ejecución.
\item \textbf{\texttt{objective\_functions}} (\emph{\texttt{List}}) -- Lista que contiene las funciones objetivo renombradas.
\item \textbf{\texttt{decision\_variables}} (\emph{\texttt{List}}) -- Lista que contiene las variables de decisión renombradas.
\item \textbf{\texttt{final\_results}} (\emph{\texttt{Dictionary}}) -- Diccionario que contiene la información para graficar. Se divide en dos categorías principales: Frente de Pareto y Mejor Individuo por Generación.
\end{itemize}

\item[{Returns}] \leavevmode
Tkinter.Frame
\item[{Return type}] \leavevmode
Instance
\end{description}\end{quote}

%******* Termina descripción *******

%******* Empieza función *******
\begin{fulllineitems}

\pysiglinewithargsret{\sphinxbfcode{change\_canvas\_category}}{}{}~

\begin{notice}{note}{Note:}
Este método es privado.
\end{notice}

Realiza el cambio de Canvas de la categoría de funciones 
objetivo a la de variables y decision y viceversa, tomando 
en cuenta factores como por ejemplo si alguna de las dos 
categorías tiene un OptionMenu asociado \textbf{(para entonces colocarlo apropiadamente)} 
e identificando siempre el último Canvas seleccionado de la 
categoría anterior para que cuando sea oportuno se vuelva 
a colocar.

\end{fulllineitems}
%******* Termina función *******

%******* Empieza función *******
\begin{fulllineitems}

\pysiglinewithargsret{\sphinxbfcode{change\_inner\_canvas}}{\emph{event}}{}~

\begin{notice}{note}{Note:}
Este método es privado.
\end{notice}

Realiza el cambio de Canvas dentro de una misma categoría, 
esto en caso en que los datos hayan arrojado más de una gráfica.\break
El cambio se hace con ayuda de su OptionMenu asociado.

\begin{quote}\begin{description}
\item[{Parameters}] \leavevmode\begin{itemize}
\item \textbf{\texttt{event}} (\emph{\texttt{String}}) -- Elemento que ejecutó esta función.
\end{itemize}
\end{description}\end{quote}

\end{fulllineitems}
%******* Termina función *******

%******* Empieza función *******
\begin{fulllineitems}

\pysiglinewithargsret{\sphinxbfcode{create\_2d\_canvas}}{\emph{x\_label}, \emph{x\_index}, \emph{y\_label}, \emph{y\_index}, \emph{collection\_points}}{}~

\begin{notice}{note}{Note:}
Este método es privado.
\end{notice}

Crea una gráfica en 2 dimensiones que es envuelta en 
un Canvas.

\begin{quote}\begin{description}
\item[{Parameters}] \leavevmode\begin{itemize}
\item \textbf{\texttt{x\_label}} (\emph{\texttt{String}}) -- Nombre para el eje X de la gráfica.
\item \textbf{\texttt{x\_index}} (\emph{\texttt{Integer}}) -- Posición dentro de collection\_points para los datos del eje X.
\item \textbf{\texttt{y\_label}} (\emph{\texttt{String}}) -- Nombre para el eje Y de la gráfica.
\item \textbf{\texttt{y\_index}} (\emph{\texttt{Integer}}) -- Posición dentro de collection\_points para los datos del eje Y.
\item \textbf{\texttt{collection\_points}} (\emph{\texttt{Dictionary}}) -- Diccionario que contiene los puntos a graficar.
\end{itemize}

\item[{Returns}] \leavevmode
Canvas
\item[{Return type}] \leavevmode
matplotlib.backends.backend\_tkagg.FigureCanvasTkAgg
\end{description}\end{quote}

\end{fulllineitems}
%******* Termina función *******

%******* Empieza función *******
\begin{fulllineitems}

\pysiglinewithargsret{\sphinxbfcode{create\_3d\_canvas}}{\emph{x\_label}, \emph{x\_index}, \emph{y\_label}, \emph{y\_index}, \emph{z\_label}, \emph{z\_index}, \emph{collection\_points}}{}~

\begin{notice}{note}{Note:}
Este método es privado.
\end{notice}

Crea una gráfica en 3 dimensiones que es envuelta en 
un Canvas.

\begin{quote}\begin{description}
\item[{Parameters}] \leavevmode\begin{itemize}
\item \textbf{\texttt{x\_label}} (\emph{\texttt{String}}) -- Nombre para el eje X de la gráfica.
\item \textbf{\texttt{x\_index}} (\emph{\texttt{Integer}}) -- Posición dentro de collection\_points para los datos del eje X.
\item \textbf{\texttt{y\_label}} (\emph{\texttt{String}}) -- Nombre para el eje Y de la gráfica.
\item \textbf{\texttt{y\_index}} (\emph{\texttt{Integer}}) -- Posición dentro de collection\_points para los datos del eje Y.
\item \textbf{\texttt{z\_label}} (\emph{\texttt{String}}) -- Nombre para el eje Z de la gráfica.
\item \textbf{\texttt{z\_index}} (\emph{\texttt{Integer}}) -- Posición dentro de collection\_points para los datos del eje Z.
\item \textbf{\texttt{collection\_points}} (\emph{\texttt{Dictionary}}) -- Diccionario que contiene los puntos a graficar.
\end{itemize}

\item[{Returns}] \leavevmode
Canvas
\item[{Return type}] \leavevmode
matplotlib.backends.backend\_tkagg.FigureCanvasTkAgg
\end{description}\end{quote}

\end{fulllineitems}
%******* Termina función *******

%******* Empieza función *******
\begin{fulllineitems}

\pysiglinewithargsret{\sphinxbfcode{create\_decision\_variables\_canvas}}{\emph{decision\_variables}, \emph{collection\_points}}{}~

\begin{notice}{note}{Note:}
Este método es privado
\end{notice}

Crea los Canvas para las variables de decisión.

\begin{quote}\begin{description}
\item[{Parameters}] \leavevmode\begin{itemize}
\item \textbf{\texttt{decision\_variables}} (\emph{\texttt{List}}) -- Lista que contiene las variables de decisión renombradas.
\item \textbf{\texttt{collection\_points}} (\emph{\texttt{Dictionary}}) -- Diccionario que contiene los valores de las funciones objetivo de todos los Individuos en la Población final.
\end{itemize}
\end{description}\end{quote}

\end{fulllineitems}
%******* Termina función *******

%******* Empieza función *******
\begin{fulllineitems}

\pysiglinewithargsret{\sphinxbfcode{create\_objective\_functions\_canvas}}{\emph{objective\_functions}, \emph{collection\_points}}{}~

\begin{notice}{note}{Note:}
Este método es privado.
\end{notice}

Crea los Canvas para las funciones objetivo.

\begin{quote}\begin{description}
\item[{Parameters}] \leavevmode\begin{itemize}
\item \textbf{\texttt{objective\_functions}} (\emph{\texttt{List}}) -- Lista que contiene las funciones objetivo renombradas.
\item \textbf{\texttt{collection\_points}} (\emph{\texttt{Dictionary}}) -- Diccionario que contiene los valores de las funciones objetivo de todos los Individuos en la Población final.
\end{itemize}
\end{description}\end{quote}

\end{fulllineitems}
%******* Termina función *******

\end{fulllineitems}
%******* Termina clase *******

La clase actual toma como referencia el siguiente elemento:

%******* Empieza clase *******
\subparagraph{CustomNavigationToolbar2TkAgg (clase)}
%Se coloca el vínculo interno procedente de esta misma sección (a_3_3_3_1_1_1).
\label{sec:a_3_3_3_1_1_1}
%******* Empieza descripción *******
\begin{fulllineitems}

\begin{DUlineblock}{0em}
\item[] Proporciona una Barra de Navegación \textbf{(ó NavigationToolbar)} 
que se anexa a cada una de las gráficas con el fin de facilitar la exploración 
y almacenamiento de los datos obtenidos.\break
Por defecto la barra de navegación original se encuentra obsoleta a las 
necesidades inherentes a este proyecto, por ello es que se crea una barra 
personalizada que responde a requerimientos tales como la obtención apropiada 
de imágenes relativas a las gráficas así como su correcto funcionamiento sin 
importar el sistema operativo empleado.
\end{DUlineblock}

\begin{quote}\begin{description}
\item[{Parameters}] \leavevmode\begin{itemize}
\item \textbf{\texttt{canvas}} (\emph{\texttt{matplotlib.backends.backend\_tkagg.\break FigureCanvasTkAgg}}) -- La estructura que contiene tanto a la gráfica como a la Barra de Navegación.
\item \textbf{\texttt{window}} (\emph{\texttt{Tkinter.Frame}}) -- El Frame que contiene a canvas.
\item \textbf{\texttt{parent\_frame}} (\emph{\texttt{Tkinter.Frame}}) -- El Frame que contiene a window, en este caso ResultsGrapherToplevel.py.
\item \textbf{\texttt{execution\_task\_count}} (\emph{\texttt{Integer}}) -- Un identificador que precisa el número de tarea \textbf{(Task)} en ejecución \textbf{(véase View/Additional/ResultsGrapher/}\break\textbf{ResultsGrapherToplevel.py)}.
\item \textbf{\texttt{image\_text}} (\emph{\texttt{String}}) -- El nombre que tendrán por defecto las imágenes resultantes al guardarse en el equipo de cómputo.
\end{itemize}

\item[{Returns}] \leavevmode
matplotlib.backends.backend\_tkagg.NavigationToolbar2TkAgg
\item[{Rype}] \leavevmode
Instance
\end{description}\end{quote}

%******* Termina descripción *******

%******* Empieza función *******
\begin{fulllineitems}

\pysiglinewithargsret{\sphinxbfcode{save\_figure}}{\emph{*args}}{}~

\begin{notice}{note}{Note:}
Este método sobreescribe al original.
\end{notice}

Arroja una ventana emergente modificada para guardar archivos, 
en este caso las gráficas.\break
Las modificaciones con respecto de la función original consisten 
en agregar un título para tener conocimiento de las imágenes del 
Task que se van a guardar.\break
Además se modifica el comportamiento de la ventana para adherirlo 
a la ventana del Task y no a la Ventana Principal.

\begin{quote}\begin{description}
\item[{Parameters}] \leavevmode\begin{itemize}
\item \textbf{\texttt{args}} (\emph{\texttt{Tuple}}) -- Un listado con parámetros que aunque no se ocupan en el método se coloca porque así lo estructuraron los desarrolladores originales de la biblioteca.
\end{itemize}
\end{description}\end{quote}

\end{fulllineitems}
%******* Termina función *******

\end{fulllineitems}
%******* Termina clase *******

%******* Empieza clase *******
\subparagraph{SummaryFrame (clase)}
%Se coloca el vínculo interno procedente de esta misma sección (a_3_3_3_1_2).
\label{sec:a_3_3_3_1_2}
%******* Empieza descripción *******
\begin{fulllineitems}

\begin{DUlineblock}{0em}
\item[] Unifica dos elementos: Canvas y ContentFrame.\break
La razón de esto es que, en promedio la información 
mostrada por ContentFrame rebasará el tamaño de la 
ventana de la información final \textbf{(véase View/Additional/}\break\textbf{ResultsGrapher/ResultsGrapherTopLevel.py)}, 
es entonces que se deben agregar barras de desplazamiento 
para poder acceder al contenido que quedaría oculto.\break
Uno de los elementos en Tkinter más sencillos que 
cumplen con este cometido es un Canvas. Luego entonces 
esa es la razón de tal fusión.
\end{DUlineblock}

\begin{quote}\begin{description}
\item[{Parameters}] \leavevmode\begin{itemize}
\item \textbf{\texttt{parent}} (\emph{\texttt{Tkinter.Frame}}) -- Frame padre al que pertenece.
\item \textbf{\texttt{renamed\_objective\_functions}} (\emph{\texttt{Dictionary}}) -- Diccionario de funciones objetivo renombradas \textbf{(véase View/Additional/ResultsGrapher/ResultsGrapherToplevel.py)}.
\item \textbf{\texttt{renamed\_decision\_variables}} (\emph{\texttt{Dictionary}}) -- Diccionario de variables de decisión renombradas  
\textbf{(véase View/Additional/ResultsGrapher/ResultsGrapherToplevel.py)}.
\item \textbf{\texttt{main\_features}} (\emph{\texttt{Dictionary}}) -- Diccionario que contiene, entre otras cosas, los nombres de los
parámetros asociados a cada técnica.
\item \textbf{\texttt{gathered\_information}} (\emph{\texttt{Dictionary}}) -- Diccionario que contiene todas las configuraciones 
recabadas ingresadas por el usuario \textbf{(véase View/Main/MainWindow.py)}.
\end{itemize}

\item[{Returns}] \leavevmode
Tkinter.Frame
\item[{Return type}] \leavevmode
Instance
\end{description}\end{quote}

%******* Termina descripción *******

%******* Empieza función *******
\begin{fulllineitems}

\pysiglinewithargsret{\sphinxbfcode{update\_scrollbar}}{\emph{event}}{}~

\begin{notice}{note}{Note:}
Este método es privado.
\end{notice}

Actualiza la barra de desplazamiento de acuerdo al 
número de elementos existentes en el Frame, esto 
para poder hacer un recorrido apropiado de la barra.

\begin{quote}\begin{description}
\item[{Parameters}] \leavevmode\begin{itemize}
\item \textbf{\texttt{event}} (\emph{\texttt{String}}) -- Elemento que ejecutó esta función.
\end{itemize}
\end{description}\end{quote}

\end{fulllineitems}
%******* Termina función *******

\end{fulllineitems}
%******* Termina clase *******

La clase actual toma como base el siguiente elemento:

%******* Empieza clase *******
\subparagraph{ContentFrame (clase)}
%Se coloca el vínculo interno procedente de esta misma sección (a_3_3_3_1_2_1).
\label{sec:a_3_3_3_1_2_1}
%******* Empieza descripción *******
\begin{fulllineitems}

\begin{DUlineblock}{0em}
\item[] Recaba el contenido de todas las funciones 
objetivo, variables de decisión y demás parámetros 
que el usuario ingresó para poder ejecutar un Task 
determinado.\break
Es entonces que plasma toda esta información en un 
Frame para que el usuario pueda cotejar los datos 
ingresados con los resultados obtenidos \textbf{(véase View/Additional/}\break\textbf{ResultsGrapher/GraphFrame.py)}.
\end{DUlineblock}

\begin{quote}\begin{description}
\item[{Parameters}] \leavevmode\begin{itemize}
\item \textbf{\texttt{parent}} (\emph{\texttt{Tkinter.Frame}}) -- Frame padre al que pertenece.
\item \textbf{\texttt{renamed\_objective\_functions}} (\emph{\texttt{Dictionary}}) -- Diccionario de funciones objetivo renombradas \textbf{(véase View/Additional/ResultsGrapher/ResultsGrapherToplevel.py)}.
\item \textbf{\texttt{renamed\_decision\_variables}} (\emph{\texttt{Dictionary}}) -- Diccionario de variables de decisión renombradas \textbf{(véase View/Additional/ResultsGrapher/ResultsGrapherToplevel.py)}.
\item \textbf{\texttt{main\_features}} (\emph{\texttt{Dictionary}}) -- Diccionario que contiene, entre otras cosas, los nombres de los parámetros asociados a cada técnica.
\item \textbf{\texttt{gathered\_information}} (\emph{\texttt{Dictionary}}) -- Diccionario que contiene todas las configuraciones  recabadas ingresadas por el usuario \textbf{(véase View/Main/MainWindow.py)}.
\end{itemize}

\item[{Returns}] \leavevmode
Tkinter.Frame
\item[{Return type}] \leavevmode
Instance
\end{description}\end{quote}

\end{fulllineitems}
%******* Termina clase *******

%******* Empieza clase *******
\subparagraph{ErrorFrame (clase)}
%Se coloca el vínculo interno procedente de esta misma sección (a_3_3_3_1_3).
\label{sec:a_3_3_3_1_3}
%******* Empieza descripción *******
\begin{fulllineitems}

\begin{DUlineblock}{0em}
\item[] Este Frame surge si durante el proceso 
interno en el Modelo \textbf{(véase Model/MOEA)} se 
suscita algún error del cual el método no se pueda 
recuperar.\break
Entonces aquí se desplegará toda la información 
relativa a la falla, asímismo funciona como medida 
de contingencia para darle una salida al programa 
y evitar que se quede atorado.
\end{DUlineblock}

\begin{quote}\begin{description}
\item[{Parameters}] \leavevmode\begin{itemize}
\item \textbf{\texttt{parent}} (\emph{\texttt{Tkinter.Frame}}) -- Frame padre al que pertenece.
\item \textbf{\texttt{final\_results}} (\emph{\texttt{Dictionary}}) -- Diccionario que contiene en este caso las características alusivas a la falla \textbf{(véase Model/MOEA)}.
\end{itemize}

\item[{Returns}] \leavevmode
Tkinter.Frame
\item[{Return type}] \leavevmode
Instance
\end{description}\end{quote}

\end{fulllineitems}
%******* Termina clase *******
%******* Termina módulo *******
%******* Termina módulo *******

%******* Termina documento *******
\end{document}

%Autor: Aarón Martín Castillo Medina.
%Asesora: Dra. Katya Rodríguez Vázquez
%Contacto: katya.rodriguez@iimas.unam.mx; amcm329@hotmail.com

%Este archivo contiene información relacionada con la capa Controlador
%(ó Controller), la cual representa tanto física como lógicamente a uno 
%de los componentes que conforman el producto de software y por tanto al 
%Manual Técnico. 


%Se indica que el documento es de tipo reporte bajo el paquete standalone.
\documentclass[class=report, crop=false]{standalone}

%Se cargan los paquetes relacionados con los subapéndices (elementos que
%conforman el Apéndice en su totalidad).
\usepackage{packages_used_section}

%Comienza el documento.
\begin{document}

\section{Controller (sección)}
%Se coloca el vínculo interno procedente de esta misma sección (a_4).
\label{sec:a_4}
Su función principal es la de establecer medidas de comunicación entre 
la Vista \textbf{(ó View)} y el Modelo \textbf{(ó Model)} de tal manera 
que el Controller \textbf{(ó Controlador)} recibe los datos recabados en la Vista 
y los transfiere al Modelo para que se puedan llevar a cabo las operaciones pertinentes 
y una vez concluidas dichas labores los resultados pasan por éste para llegar a la 
Vista y desde ahí graficarse apropiadamente.\break
De manera secundaria el Controlador ofrece métodos de saneamiento de los datos 
recabados en la Vista, con la finalidad de evitar al máximo disturbios indeseables 
en la sección Modelo y que éste opere con total eficiencia, además de alimentar a la 
Vista con las técnicas \textbf{(y sus respectivos parámetros)} disponibles en la 
sección Modelo para así permitirle al usuario operar con éstas de manera expedita.\break
Dicho almacén se encuentra en la sección \textbf{Controller/XML}, donde se deduce 
que las técnicas y características secundarias se encuentran plasmadas en archivos 
.xml.\medskip\break
El proyecto contempla métodos para operar con dichos archivos y el usuario entonces 
sólo tendrá que preocuparse por dar de alta la técnica pertinente en el archivo .xml 
adecuado \textbf{(además de implementarla en Modelo)} para que ésta sea reconocida en 
la sección Vista y se pueda hacer uso de ella.\medskip\break
A continuación se muestran los componentes principales de la sección Controller:

%******* Empieza clase *******
\subsection{Controller (clase)}
%Se coloca el vínculo interno procedente de esta misma sección (a_4_1).
\label{sec:a_4_1}

\begin{fulllineitems}

\begin{DUlineblock}{0em}
\item[] Proporciona la infraestructura adecuada para poder comunicar la 
sección Vista \textbf{(ó View)} con la sección Modelo \textbf{(ó Model)}, 
apoyándose de las clases XMLParser y Verifier.\medskip\break
El ciclo normal consiste en otorgar a la capa Vista \textbf{(ó View)} la 
información recabada en los archivos .xml con ayuda de la clase XMLParser 
con la  finalidad de notificar al usuario de todas las técnicas 
disponibles.\break
Una vez ejecutada la opción de iniciar un proceso genético por el 
usuario, se recaban los datos ingresados por el usuario, los cuales 
pasan por un proceso de verificación y transformación empleando para ello 
los métodos de la clase Verifier.\break
En caso de haber al menos una falla en alguno de los procedimientos mencionados 
anteriormente se regresa un mensaje de error, en otro caso se pasa la información 
respectiva a la capa Model para que pueda operar con ésta.\medskip\break
En cualquiera de los dos casos anteriores se regresa la información resultante a 
la Vista.
\end{DUlineblock}

\begin{quote}\begin{description}
\item[{Returns}] \leavevmode
Controller.Controller
\item[{Return type}] \leavevmode
Instance
\end{description}\end{quote}

%******* Empieza descripción *******
\begin{fulllineitems}

\pysiglinewithargsret{\sphinxbfcode{execute\_procedure}}{\emph{execution\_task\_count}, \emph{generations\_queue}, \emph{sanitized\_information}}{}
Realiza la ejecución de algún algoritmo M.O.E.A. \textbf{(Multi-Objective Evolutionary Algorithm)} 
y se encarga de obtener los resultados apropiadamente.

\begin{quote}\begin{description}
\item[{Parameters}] \leavevmode\begin{itemize}
\item \textbf{\texttt{execution\_task\_count}} (\emph{\texttt{Integer}}) -- Una característica numérica que identifica inequívocamente a esta función que será ejecutada de las demás, ya que el objetivo del proyecto es poder ejecutar varios de estos métodos de manera concurrente \textbf{(véase View/Additional/}\break\textbf{ResultsGrapher/ResultsGrapherToplevel.py)}.
\item \textbf{\texttt{generations\_queue}} (\emph{\texttt{Instance}}) -- Una instancia a una cola \textbf{(Queue)}, la cual servirá para escribir a esa estructura el número actual de generación por el que cursa el algoritmo. Esta acción es para fines de  concurrencia \textbf{(véase View/MainWindow.py)}.
\item \textbf{\texttt{sanitized\_information}} (\emph{\texttt{Dictionary}}) -- Los parámetros que ingresó el usuario debidamente verificados y saneados.
\end{itemize}
\item[{Returns}] \leavevmode
Un diccionario con información de los resultados de haber ejecutado el M.O.E.A. seleccionado por el usuario, la estructura del mismo puede verse en \textbf{Model/Community/}\break\textbf{Community.py}.
\item[{Return type}] \leavevmode
Dictionary
\end{description}\end{quote}

\end{fulllineitems}
%******* Termina descripción *******

%******* Empieza función *******
\begin{fulllineitems}

\pysiglinewithargsret{\sphinxbfcode{load\_features}}{}{}
Regresa los datos correspondientes \textbf{(debidamente verificados)}
a las \break técnicas disponibles para el usuario, los cuales se mostrarán en \break 
\textbf{View/MainWindow.py} \textbf{(véase View/Additional/MenuInternalOption/}\break
\textbf{InternalOptionTab/FeatureFrame.py)}.\break
Esta técnica tiene como base los símiles que se encuentran en 
\textbf{Controller/}\break\textbf{XMLParser.py} y \textbf{Controller/Verifier.py}.

\begin{quote}\begin{description}
\item[{Returns}] \leavevmode
Una estructura con los métodos disponibles para el usuario.
\item[{Return type}] \leavevmode
Dictionary
\end{description}\end{quote}

\end{fulllineitems}
%******* Termina función *******

%******* Empieza función *******
\begin{fulllineitems}

\pysiglinewithargsret{\sphinxbfcode{load\_mop\_examples}}{}{}
Obtiene los datos correspondientes \textbf{(previamente verificados)}
a los M.O.P.'s \textbf{(Multi-Objective Problems)} que se utilizan en 
\textbf{View/}\break\textbf{MainWindow.py} \textbf{(véase View/Additional/MenuInternalOption/}\break
\textbf{InternalOptionTab/MOPExampleFrame.py)}.\break
Esta técnica tiene como base las análogas que se encuentran en 
\textbf{Controller/XMLParser.py} y \textbf{Controller/Verifier.py}.

\begin{quote}\begin{description}
\item[{Returns}] \leavevmode
Una estructura con los M.O.P.'s disponibles para el usuario.
\item[{Return type}] \leavevmode
Dictionary
\end{description}\end{quote}

\end{fulllineitems}
%******* Termina función *******

%******* Empieza función *******
\begin{fulllineitems}

\pysiglinewithargsret{\sphinxbfcode{load\_python\_expressions}}{}{}
Obtiene los datos correspondientes \textbf{(previamente verificados)}
a las expresiones de Python, las cuales se usan para evaluar 
funciones objetivo más eficientemente \textbf{(véase View/Additional/MenuInternalOption/}\break
\textbf{InternalOptionTab/PythonExpressionFrame.py)}.\break
Esta función se apoya de las homónimas localizadas en \break
\textbf{Controller/XMLParser.py} y \textbf{Controller/Verifier.py}.

\begin{quote}\begin{description}
\item[{Returns}] \leavevmode
Una estructura con las expresiones de Python disponibles.
\item[{Return type}] \leavevmode
Dictionary
\end{description}\end{quote}

\end{fulllineitems}
%******* Termina función *******

%******* Empieza función *******
\begin{fulllineitems}

\pysiglinewithargsret{\sphinxbfcode{sanitize\_settings}}{\emph{general\_information}, \emph{features}}{}
Lleva a cabo la verificación y saneamiento de todos los datos
que ha ingresado el usuario en la sección View \textbf{(véase View/MainWindow)}.

\begin{quote}\begin{description}
\item[{Parameters}] \leavevmode\begin{itemize}
\item \textbf{\texttt{general\_information}} (\emph{\texttt{Dictionary}}) -- El conjunto de datos que el usuario ha ingresado o seleccionado.
\item \textbf{\texttt{features}} (\emph{\texttt{Dictionary}}) -- Una colección de todos los elementos con sus características disponibles para el usuario.
\end{itemize}
\item[{Returns}] \leavevmode
El diccionario que contiene todos los datos debidamente saneados.
\item[{Return type}] \leavevmode
Dictionary
\end{description}\end{quote}

\end{fulllineitems}
%******* Termina función *******

%******* Empieza función *******
\begin{fulllineitems}

\pysiglinewithargsret{\sphinxbfcode{save\_python\_expressions}}{\emph{data}}{}
Inserta las expresiones de Python que ha ingresado
el usuario en el archivo .xml correspondiente.

\begin{quote}\begin{description}
\item[{Parameters}] \leavevmode\begin{itemize}
\item \textbf{\texttt{data}} (\emph{\texttt{List}}) -- Un conjunto de las expresiones que ha ingresado el usuario. Cada elemento es a su vez una lista con dos elementos, el primero es la expresión original \textbf{(la que es comprensible por el usuario)}, mientras que la segunda es la expresión equivalente en Python.
\end{itemize}
\item[{Returns}] \leavevmode
Mensaje ``OK'' si la inserción ha sido exitosa, mientras que en caso de que haya habido un error entonces el mensaje es ``ERROR''.
\item[{Return type}] \leavevmode
String
\end{description}\end{quote}

\end{fulllineitems}
%******* Termina función *******

\end{fulllineitems}
%******* Termina clase *******

%******* Empieza clase *******
\subsection{XMLParser (clase)}
%Se coloca el vínculo interno procedente de esta misma sección (a_4_2).
\label{sec:a_4_2}
%******* Empieza descripción *******
\begin{fulllineitems}

\begin{DUlineblock}{0em}
\item[] Permite leer y escribir a archivos .xml 
\textbf{(los que se localizan en Controller/XML)}, los cuales tienen 
almacenados: 

\begin{itemize}
\item Los nombres de las técnicas con sus parámetros que se encuentran 
disponibles en la sección Model \textbf{(Features.xml)}.
\item La colección de palabras reservadas para poder emplear funciones y 
constantes auxiliares en las funciones objetivo \break\textbf{(PythonExpressions.xml)}.
\item El conjunto de M.O.P.'s \textbf{(Multi-Objective Problems, localizados en MOPExamples.xml)}.
\end{itemize}

Dichos archivos proporcionan la información necesaria a la interfaz 
gráfica \textbf{(véase View/MainWindow.py)}.
\end{DUlineblock}

\begin{quote}\begin{description}
\item[{Returns}] \leavevmode
Controller.XMLParser
\item[{Return type}] \leavevmode
Instance
\end{description}\end{quote}

%******* Termina descripción *******

%******* Empieza función *******
\begin{fulllineitems}

\pysiglinewithargsret{\sphinxbfcode{indent}}{\emph{element}, \emph{level=0}}{}
Indenta \textbf{(coloca espacios)} apropiadamente
en un documento .xml para poder distinguir más
rápidamente los distintos niveles que existen en éste.

\begin{quote}\begin{description}
\item[{Parameters}] \leavevmode\begin{itemize}
\item \textbf{\texttt{element}} (\emph{\texttt{String}}) -- Una línea del archivo .xml
\item \textbf{\texttt{level}} (\emph{\texttt{Integer}}) -- El nivel en el que se está haciendo el proceso de identado.
\end{itemize}
\end{description}\end{quote}

\end{fulllineitems}
%******* Termina función *******

%******* Empieza función *******
\begin{fulllineitems}

\pysiglinewithargsret{\sphinxbfcode{load\_xml\_features}}{\emph{features\_filename}}{}
Realiza la lectura del archivo que contenga el listado de técnicas 
y sus parámetros disponibles \textbf{(véase Model)} y 
carga todos los elementos que se encuentran en éste.

\begin{quote}\begin{description}
\item[{Parameters}] \leavevmode\begin{itemize}
\item \textbf{\texttt{features\_filename}} (\emph{\texttt{String}}) -- Nombre del archivo en cuestión.
\end{itemize}
\item[{Returns}] \leavevmode
Un diccionario que contiene todos los elementos del archivo.
\item[{Return type}] \leavevmode
Dictionary
\end{description}\end{quote}

\end{fulllineitems}
%******* Termina función *******

%******* Empieza función *******
\begin{fulllineitems}

\pysiglinewithargsret{\sphinxbfcode{load\_xml\_mop\_examples}}{\emph{features\_filename}}{}
Lleva a cabo la lectura del archivo que contenga el listado de M.O.P.'s 
\textbf{(Multi-Objective Problems)} y carga todos los elementos 
que se encuentran en éste.\break
Un M.O.P es una mezcla de variables de decisión y funciones 
objetivo ya estudiadas, se utilizan para reproducir su comportamiento 
y así garantizar, además de un correcto funcionamiento del programa, 
una opción rápida para probar las técnicas que se ofrecen.

\begin{quote}\begin{description}
\item[{Parameters}] \leavevmode\begin{itemize}
\item \textbf{\texttt{features\_filename}} (\emph{\texttt{String}}) -- Nombre del archivo en cuestión.
\end{itemize}
\item[{Returns}] \leavevmode
Un diccionario que contiene todos los elementos del archivo.
\item[{Return type}] \leavevmode
Dictionary
\end{description}\end{quote}

\end{fulllineitems}
%******* Termina función *******

%******* Empieza función *******
\begin{fulllineitems}

\pysiglinewithargsret{\sphinxbfcode{load\_xml\_python\_expressions}}{\emph{features\_filename}}{}
Realiza la lectura del archivo que contenga el listado de 
expresiones en Python y carga todos los elementos que se 
encuentran en éste.\break
La idea detrás de esto es que, al momento de crear y/o evaluar
funciones objetivo existen algunas palabras reservadas que no pueden 
ser usadas directamente como son las funciones trigonométricas, por eso
es que estas expresiones sirven como intermediarias entre el usuario y
el intérprete de Python.\break
En ocasiones a este tipo de expresiones, no sólo en el ámbito actual
sino en general, se les conoce como azúcar sintáctica.

\begin{quote}\begin{description}
\item[{Parameters}] \leavevmode\begin{itemize}
\item \textbf{\texttt{features\_filename}} (\emph{\texttt{String}}) -- Nombre del archivo en cuestión.
\end{itemize}
\item[{Returns}] \leavevmode
Un diccionario que contiene todos los elementos del archivo.
\item[{Return type}] \leavevmode
Dictionary
\end{description}\end{quote}

\end{fulllineitems}
%******* Termina función *******

%******* Empieza función *******
\begin{fulllineitems}
\pysiglinewithargsret{\sphinxbfcode{write\_xml\_python\_expressions}}{\emph{features\_filename}, \emph{features}}{}
Sobreescribe el archivo donde se encuentra el listado de expresiones
en Python.\break
El objetivo es que, una vez ejecutándose el programa y a través del menú
pertinente \textbf{(véase View/Additional/MenuInternalOption/}\break
\textbf{InternalOptionTab/PythonExpressionFrame.py)}, el usuario pueda 
añadir o eliminar las expresiones de Python que desee.\break
En ocasiones a este tipo de expresiones, no sólo en el ámbito actual 
sino en general, se les conoce como azúcar sintáctica.

\begin{quote}\begin{description}
\item[{Parameters}] \leavevmode\begin{itemize}
\item \textbf{\texttt{features\_filename}} (\emph{\texttt{String}}) -- Nombre del archivo en cuestión.
\item \textbf{\texttt{features}} (\emph{\texttt{List}}) -- La estructura que contiene las expresiones para ser guardadas
en el archivo .xml.
\end{itemize}
\end{description}\end{quote}

\end{fulllineitems}
%******* Termina función *******

\end{fulllineitems}
%******* Termina clase *******

%******* Empieza clase *******
\subsection{Verifier (clase)}
%Se coloca el vínculo interno procedente de esta misma sección (a_4_3).
\label{sec:a_4_3}
%******* Empieza descripción *******
\begin{fulllineitems}

\begin{DUlineblock}{0em}
\item[] Realiza principalmente la verificación y transformación adecuada 
de los datos que el usuario introduce en \textbf{View/MainWindow.py} para 
alimentar a los algoritmos que se encuentran en la sección Model 
\textbf{(más en concreto Model/MOEA)}.\break
En caso de haber algún error regresa los mensajes de error adecuados para 
que puedan ser interpretados por la capa Vista y precisar al usuario el 
acontecimiento ocurrido.\break
Por otra parte si se ha llevado a cabo la verificación correctamente se 
obtiene la información transformada apropiadamente.\medskip\break
De manera secundaria también ofrece métodos de verificación para la 
extracción y colocación de datos en los archivos .xml 
\textbf{(véase XMLParser y el directorio Controller/XML)}.
\end{DUlineblock}

\begin{quote}\begin{description}
\item[{Returns}] \leavevmode
Controller.Verifier
\item[{Return type}] \leavevmode
Instance
\end{description}\end{quote}

%******* Termina descripción *******

%******* Empieza función *******
\begin{fulllineitems}

\pysiglinewithargsret{\sphinxbfcode{cast\_parameter}}{\emph{parameter\_value}, \emph{parameter\_settings}}{}

\begin{notice}{note}{Note:}
Este método es privado.
\end{notice}

Verifica un parámetro asociado a alguna técnica.
Primero asegura que el parámetro se pueda evaluar correctamente,
posteriormente convierte apropiadamente el tipo de dato pasando 
de String a Boolean, Integer ó Float según corresponda.

\begin{quote}\begin{description}
\item[{Parameters}] \leavevmode\begin{itemize}
\item \textbf{\texttt{parameter\_value}} (\emph{\texttt{Float}}) -- El valor actual del parámetro.
\item \textbf{\texttt{parameter\_settings}} (\emph{\texttt{Dictionary}}) -- Un diccionario que contiene el tipo del parámetro \textbf{(bool, integer ó float)} y el rango que debe tomar tanto inferior como superior.
\end{itemize}
\item[{Returns}] \leavevmode
El valor saneado del parámetro si no hay fallas, pero si se encuentra algún desperfecto entonces se regresa un diccionario con la información detallada del desperfecto.
\item[{Return type}] \leavevmode
(Boolean, Integer, Float)/Dictionary
\end{description}\end{quote}

\end{fulllineitems}
%******* Termina función *******

%******* Empieza función *******
\begin{fulllineitems}
\pysiglinewithargsret{\sphinxbfcode{verify\_instance}}{\emph{name\_class}}{}

\begin{notice}{note}{Note:}
Este método es privado.
\end{notice}

Devuelve una instancia del nombre de la clase que se le pase
como parámetro.\break
Esta funcionalidad es útil sobre todo para la sección Model ya que
uno de los objetivos es proporcionar al usuario de una infraestructura 
rápida con técnicas fácilmente intercambiables sin necesidad de estar
importando explícitamente cada una de éstas.\break
De esta forma con base en una instancia se puede ejecutar 
cualquier método de manera dinámica.

\begin{quote}\begin{description}
\item[{Parameters}] \leavevmode\begin{itemize}
\item \textbf{\texttt{name\_class}} (\emph{\texttt{String}}) -- el nombre de la clase \textbf{(con su ruta)} de la cual se desea obtener una instancia.
\end{itemize}
\item[{Returns}] \leavevmode
Una instancia de la clase solicitada si el proceso es exitoso, en otro caso se obtiene un diccionario con los detalles de la falla.
\item[{Return type}] \leavevmode
Instance/Dictionary
\end{description}\end{quote}

\end{fulllineitems}
%******* Termina función *******

%******* Empieza función *******
\begin{fulllineitems}

\pysiglinewithargsret{\sphinxbfcode{get\_dynamic\_function}}{\emph{complete\_function}}{}
Obtiene una instancia de una función en un String 
de la forma \textbf{biblioteca.función}.
Este método se usa para convertir las expresiones de Python
en instancias que serán utilizadas al momento de evaluar  
funciones objetivo \textbf{(véase View/}\break
\textbf{Additional/MenuInternalOption/InternalOptionTab/}\break\textbf{PythonExpressionFrame.py, Controller/XML/PythonExpressions.xml)}

\begin{quote}\begin{description}
\item[{Parameters}] \leavevmode\begin{itemize}
\item \textbf{\texttt{complete\_function}} (\emph{\texttt{String}}) -- un String preferentemente de la forma \textbf{biblioteca.función} \textbf{(el punto debe ir incluido)}.
\end{itemize}
\item[{Returns}] \leavevmode
Una instancia de la función asociada a la biblioteca.
\item[{Return type}] \leavevmode
Instance
\end{description}\end{quote}

\end{fulllineitems}
%******* Termina función *******

%******* Empieza función *******
\begin{fulllineitems}

\pysiglinewithargsret{\sphinxbfcode{sanitize\_decision\_variables}}{\emph{vector\_variables}}{}
Verifica el conjunto de elementos de la categoría ``Decision Variables''
\textbf{(véase View/Main/DecisionVariable/DecisionVariableFrame.py)}, 
los cuales son precisamente las variables de decisión.\break
Primero se asegura que cada variable de decisión se pueda evaluar
correctamente, posteriormente convierte apropiadamente el tipo
de dato de sus respectivos rangos, pasando de String a Float.

\begin{quote}\begin{description}
\item[{Parameters}] \leavevmode\begin{itemize}
\item \textbf{\texttt{vector\_variables}} (\emph{\texttt{Dictionary}}) -- El vector que contiene las variables de decisión con sus correspondientes rangos.
\end{itemize}
\item[{Returns}] \leavevmode
Un diccionario con las variables de decisión y sus rangos debidamente saneados.
\item[{Return type}] \leavevmode
Dictionary
\end{description}\end{quote}

\end{fulllineitems}
%******* Termina función *******

%******* Empieza función *******
\begin{fulllineitems}

\pysiglinewithargsret{\sphinxbfcode{sanitize\_genetic\_operators\_settings}}{\emph{genetic\_operators\_settings},\emph{features}, \emph{vector\_variables}, \emph{number\_of\_decimals}}{}
Revisa la integridad y sanea los datos que ingresó el usuario 
concernientes a la sección ``Genetic Operators Settings''
\textbf{(véase View/Main}\break\textbf{/GeneticOperator/GeneticOperatorFrame.py)}.

\begin{quote}\begin{description}
\item[{Parameters}] \leavevmode\begin{itemize}
\item \textbf{\texttt{genetic\_operators\_settings}} (\emph{\texttt{Dictionary}}) -- El listado de técnicas y sus parámetros que el usuario
eligió en la sección correspondiente.
\item \textbf{\texttt{features}} (\emph{\texttt{Dictionary}}) -- El conjunto de las opciones disponibles para esta sección, así como sus características.
\item \textbf{\texttt{vector\_variables}} (\emph{\texttt{List}}) -- El vector de variables de decisión.
\item \textbf{\texttt{number\_of\_decimals}} (\emph{\texttt{Integer}}) -- El número de decimales que llevará cada solución en Population.
\end{itemize}

\item[{Returns}] \leavevmode
Un diccionario que, dependiendo de los resultados, puede contener
o información del error encontrado durante el procedimiento o 
todos los datos debidamente verificados y transformados.
\item[{Return type}] \leavevmode
Dictionary
\end{description}\end{quote}

\end{fulllineitems}
%******* Termina función *******

%******* Empieza función *******
\begin{fulllineitems}

\pysiglinewithargsret{\sphinxbfcode{sanitize\_moeas\_settings}}{\emph{moeas\_settings}, \emph{features}}{}
Verifica integridad y lleva a cabo el saneamiento de 
los datos que ingresó el usuario concernientes a la 
sección ``MOEAs Settings'' \textbf{(véase View/Main/}\break\textbf{MOEA/MOEAFrame.py)}.

\begin{quote}\begin{description}
\item[{Parameters}] \leavevmode\begin{itemize}
\item \textbf{\texttt{moeas\_settings}} (\emph{\texttt{Dictionary}}) -- El listado de técnicas y sus parámetros que el usuario eligió en la sección correspondiente.
\item \textbf{\texttt{features}} (\emph{\texttt{Dictionary}}) -- El conjunto de las opciones disponibles para esta sección, así como sus características.
\end{itemize}

\item[{Returns}] \leavevmode
Un diccionario que, dependiendo de los resultados, puede contener
o información del error encontrado durante el procedimiento o 
todos los datos debidamente verificados y transformados.
\item[{Return type}] \leavevmode
Dictionary
\end{description}\end{quote}

\end{fulllineitems}
%******* Termina función *******

%******* Empieza función *******
\begin{fulllineitems}

\pysiglinewithargsret{\sphinxbfcode{sanitize\_objective\_functions}}{\emph{vector\_variables},\emph{available\_expressions}, \emph{vector\_functions}}{}
Lleva a cabo el saneamiento de los elementos correspondientes a la categoría
``Objective Functions'' \textbf{(véase View/Main/ObjectiveFuncion/}\break
\textbf{ObjectiveFunctionFrame.py)}, los cuales son de hecho sólo 
las funciones objetivo.

\begin{quote}\begin{description}
\item[{Parameters}] \leavevmode\begin{itemize}
\item \textbf{\texttt{vector\_variables}} (\emph{\texttt{Dictionary}}) -- El vector de variables de decisión que el usuario ha ingresado.
\item \textbf{\texttt{available\_expressions}} (\emph{\texttt{Dictionary}}) -- Un listado con las expresiones de Python disponibles \textbf{(véase Controller/XML/PythonExpressions.xml,
View/Additional/MenuInternalOption/}\break\textbf{InternalOptionTab/PythonExpressionFrame.py)}.
\item \textbf{\texttt{vector\_functions}} (\emph{\texttt{Dictionary}}) -- El vector de funciones objetivo ingresados por el usuario.
\end{itemize}

\item[{Returns}] \leavevmode
Si el proceso fue exitoso, se obtiene el mismo vector\_functions,
en otro caso se regresa un diccionario con información detallada
sobre el errror encontrado.
\item[{Return type}] \leavevmode
List/Dictionary
\end{description}\end{quote}

\end{fulllineitems}
%******* Termina función *******

%******* Empieza función *******
\begin{fulllineitems}

\pysiglinewithargsret{\sphinxbfcode{sanitize\_population\_settings}}{\emph{population\_settings}, \emph{features}}{}
Verifica la consistencia y realiza el saneamiento de los datos
que ingresó el usuario concernientes a la sección ``Population Settings''
\textbf{(véase View/Main/}\break\textbf{Population/PopulationFrame.py)}.
\begin{quote}\begin{description}
\item[{Parameters}] \leavevmode\begin{itemize}
\item \textbf{\texttt{population\_settings}} (\emph{\texttt{Dictionary}}) -- El listado de técnicas y sus parámetros que el usuario eligió en la sección correspondiente.
\item \textbf{\texttt{features}} (\emph{\texttt{Dictionary}}) -- El conjunto de las opciones disponibles para esta sección, así como sus características.
\end{itemize}

\item[{Returns}] \leavevmode
Un diccionario que, dependiendo de los resultados, puede contener
o información del error encontrado durante el procedimiento o 
todos los datos debidamente verificados y transformados.
\item[{Return type}] \leavevmode
Dictionary
\end{description}\end{quote}

\end{fulllineitems}
%******* Termina función *******

%******* Empieza función *******
\begin{fulllineitems}

\pysiglinewithargsret{\sphinxbfcode{sanitize\_techniques}}{\emph{general\_information}, \emph{features}}{}
Realiza una verificación adicional concerniente al tipo \break
de representación de todas las técnicas seleccionadas.\break
Lo anterior significa que, usando la Representación Cromosómica \break
\textbf{(ó Chromosomal Representation, véase Model/}\break
\textbf{ChromosomalRepresentation, View/Main/Population/}\break
\textbf{PopulationFrame.py)}, todas las técnicas deben concordar 
con el mismo tipo de representación cromosómica que se haya 
seleccionado.\break
Para esta versión sólo están disponibles las representaciones binaria 
y de punto flotante.

\begin{quote}\begin{description}
\item[{Parameters}] \leavevmode\begin{itemize}
\item \textbf{\texttt{general\_information}} (\emph{\texttt{Dictionary}}) -- El listado de características disponibles \textbf{(véase XMLParser.py)}.
\item \textbf{\texttt{features}} (\emph{\texttt{Dictionary}}) -- La colección de datos que seleccionó el usuario en la sección View.
\end{itemize}

\item[{Returns}] \leavevmode
Un diccionario el cual, si la verificación es exitosa, es el mismo general\_informacion,
si por el contrario falla, entonces es un diccionario que contiene detalles del error.
\item[{Return type}] \leavevmode
Dictionary
\end{description}\end{quote}

\end{fulllineitems}
%******* Termina función *******

%******* Empieza función *******
\begin{fulllineitems}

\pysiglinewithargsret{\sphinxbfcode{verify\_load\_xml\_features}}{\emph{data}}{}
Verifica que los datos obtenidos de las técnicas disponibles que alimentan 
a la Ventana Principal \textbf{(véase View/MainWindow.py)} no tengan defectos.
Este método se apoya de \textbf{load\_xml\_features} localizado en 
\textbf{Controller/}\break\textbf{XMLParser.py}.

\begin{quote}\begin{description}
\item[{Parameters}] \leavevmode\begin{itemize}
\item \textbf{\texttt{data}} (\emph{\texttt{Dictionary}}) -- Los datos que son leídos por el método \textbf{load\_xml\_features} mencionado previamente.
\end{itemize}
\item[{Returns}] \leavevmode
Si los datos contienen algún error, un diccionario con las características de la falla, en otro caso los datos mismos.
\item[{Return type}] \leavevmode
Dictionary
\end{description}\end{quote}

\end{fulllineitems}
%******* Termina función *******

%******* Empieza función *******
\begin{fulllineitems}

\pysiglinewithargsret{\sphinxbfcode{verify\_load\_xml\_mop\_examples}}{\emph{data}}{}
Revisa que los M.O.P.'s \textbf{(Multi-Objective Problems)}
que se muestran en \textbf{View/MainWindow.py} a través de
\textbf{View/Additional/MenuInternalOption/}\break\textbf{InternalOptionTab/MOPExampleFrame.py} 
estén libres de errores.\break
Este método se apoya de \textbf{load\_mop\_examples} localizado en 
\textbf{Controller/}\break\textbf{XMLParser.py}.

\begin{quote}\begin{description}
\item[{Parameters}] \leavevmode\begin{itemize}
\item \textbf{\texttt{data}} (\emph{\texttt{Dictionary}}) -- Los datos que son leídos por el método \textbf{load\_mop\_examples} mencionado previamente.
\end{itemize}
\item[{Returns}] \leavevmode
Si los datos contienen algún error, un diccionario con las características de la falla, en otro caso los datos mismos.
\item[{Return type}] \leavevmode
Dictionary
\end{description}\end{quote}

\end{fulllineitems}
%******* Termina función *******

%******* Empieza función *******
\begin{fulllineitems}

\pysiglinewithargsret{\sphinxbfcode{verify\_load\_xml\_python\_expressions}}{\emph{data}}{}
Revisa que las expresiones de Python estén libres de errores.\break
Este método se apoya de \textbf{load\_python\_expressions} localizado en 
\textbf{Controller/XMLParser.py}.

\begin{quote}\begin{description}
\item[{Parameters}] \leavevmode\begin{itemize}
\item \textbf{\texttt{data}} (\emph{\texttt{Dictionary}}) -- Los datos que son leídos por el método \textbf{load\_python\_expressions} mencionado previamente.
\end{itemize}
\item[{Returns}] \leavevmode
Si los datos contienen algún error, un diccionario con las características de la falla, en otro caso los datos mismos.
\item[{Return type}] \leavevmode
Dictionary
\end{description}\end{quote}

\end{fulllineitems}
%******* Termina función *******

%******* Empieza función *******
\begin{fulllineitems}

\pysiglinewithargsret{\sphinxbfcode{verify\_write\_xml\_python\_expressions}}{\emph{data}}{}
Verifica la consistencia de los datos relativos a las
expresiones de Python antes de ser escritos en el 
archivo .xml correspondiente.

\begin{quote}\begin{description}
\item[{Parameters}] \leavevmode\begin{itemize}
\item \textbf{\texttt{data}} (\emph{\texttt{List}}) -- El conjunto de expresiones que serán almacenadas.
\end{itemize}
\item[{Returns}] \leavevmode
Un mensaje ``OK'' si la verificación fue satisfactoria, y ``ERROR'' en caso de aparecer alguna falla.
\item[{Return type}] \leavevmode
String
\end{description}\end{quote}

\end{fulllineitems}
%******* Termina función *******

\end{fulllineitems}
%******* Termina clase *******

%Termina el documento
\end{document}


%Termina el documento.
\end{document}
